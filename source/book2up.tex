% $Id: book.tex,v 1.2 2005/04/13 08:28:24 loreti Exp $
%
% *** Add at the very end: \clearpage \enlargethispage
%
% Fonts: Lucida with expert collection [expert], scaled
%        [lucidasmallscale].

\documentclass[12pt,a4paper,twoside]{book}
\usepackage[english,italian]{babel}
\usepackage[T1]{fontenc}
\usepackage[dvips]{graphicx,color}
\usepackage[hmarginratio=3:5,width=137mm,nomarginpar,ignoreheadfoot]{geometry}
\usepackage{amsmath,amssymb}
\usepackage[expert,lucidasmallscale]{lucidabr}
\usepackage{bsheaders}
\usepackage{indentfirst}
\usepackage{float}
\usepackage{lscape}
\usepackage{makeidx}
\usepackage{booktabs}

\floatstyle{ruled}                             % Declare a new
\restylefloat{table}                           % style for floats
\restylefloat{figure}                          % (tables and figures).

%%%%%%%%%% Redefine some style parameters (start) %%%%%%%%%%

\makeatletter

% 1) Supersedes a wrong definition of 'oldstylenums' for
%    the Lucida fonts

\@ifpackageloaded{lucidabr}{
  \renewcommand{\oldstylenums}[1]{%
    \begingroup
    \spaceskip\fontdimen\tw@\font
    \usefont{OML}{hlcm}{\f@series}{it}%
    \mathgroup\symletters #1%
    \endgroup
  }
}{}

% 2) [float.sty] Figure/table name in small-caps

\renewcommand\fs@ruled{\def\@fs@cfont{\scshape}\let\@fs@capt\floatc@ruled
  \def\@fs@pre{\hrule height.8pt depth0pt \kern2pt}%
  \def\@fs@post{\kern2pt\hrule\relax}%
  \def\@fs@mid{\kern2pt\hrule\kern2pt}%
  \let\@fs@iftopcapt\iftrue}
\renewcommand\floatc@ruled[2]{{\@fs@cfont #1}\ -\ #2\par}

% 3) [float.sty] Figure/table number with normal digits

\renewcommand{\thefigure}{{\normalfont\arabic{chapter}\alph{figure}}}
\renewcommand{\thetable}{{\normalfont\arabic{chapter}.\arabic{table}}}

% 4) [book.cls] Headers:
% - chapter and section names in small-caps;
% - page number with old-style digits;
% - chapter and section numbers with normal digits.
%
% The headers are defined with ps@headings (general); and
% explicitly redefined for the "Table of Contents", "List
% of Figures",  "List of Tables" and "Index" environments.

\def\ps@headings{%
  \let\@oddfoot\@empty\let\@evenfoot\@empty
  \def\@evenhead{\oldstylenums{\thepage}\hfil\scshape\leftmark}%
  \def\@oddhead{\scshape\rightmark\hfil\oldstylenums{\thepage}}%
  \let\@mkboth\markboth
  \def\chaptermark##1{%
    \markboth {%
      \ifnum \c@secnumdepth >\m@ne
      \if@mainmatter
      \@chapapp\ {\normalfont\thechapter}\ -\ %
      \fi
      \fi
      ##1}{}}%
  \def\sectionmark##1{%
    \markright {%
      \ifnum \c@secnumdepth >\z@
      {\normalfont\thesection}\ -\ %
      \fi
      ##1}}}

\renewcommand{\tableofcontents}{%
  \if@twocolumn
  \@restonecoltrue\onecolumn
  \else
  \@restonecolfalse
  \fi
  \chapter*{\contentsname
    \@mkboth{\scshape\contentsname}{\scshape\contentsname}}%
  \@starttoc{toc}%
  \if@restonecol\twocolumn\fi}

\renewcommand{\listoffigures}{%
  \if@twocolumn
  \@restonecoltrue\onecolumn
  \else
  \@restonecolfalse
  \fi
  \chapter*{\listfigurename
    \@mkboth{\scshape\listfigurename}{\scshape\listfigurename}}%
  \@starttoc{lof}%
  \if@restonecol\twocolumn\fi}

\renewcommand{\listoftables}{%
  \if@twocolumn
  \@restonecoltrue\onecolumn
  \else
  \@restonecolfalse
  \fi
  \chapter*{\listtablename
    \@mkboth{\scshape\listtablename}{\scshape\listtablename}}%
  \@starttoc{lot}%
  \if@restonecol\twocolumn\fi}

\renewenvironment{theindex}
{\if@twocolumn
  \@restonecolfalse
  \else
  \@restonecoltrue
  \fi
  \columnseprule \z@
  \columnsep 35\p@
  \twocolumn[\@makeschapterhead{\indexname}]%
  \@mkboth{\scshape\indexname}{\scshape\indexname}%
  \thispagestyle{plain}\parindent\z@
  \parskip\z@ \@plus .3\p@\relax
  \let\item\@idxitem}
{\if@restonecol\onecolumn\else\clearpage\fi}

% 5) [book.cls] Footers:
% - to use everywhere old-style digits for the page numbers,
%   we need to redefine the "plain" page style too...

\def\ps@plain{\let\@mkboth\@gobbletwo
  \let\@oddhead\@empty\def\@oddfoot{\reset@font\hfil
    \oldstylenums{\thepage}
    \hfil}\let\@evenhead\@empty\let\@evenfoot\@oddfoot}

% 6) [book.cls] \@pnumwidth is redefined (the default is
%    1.55em) to host larger digits for the page numbers in
%    the TOC index.  Also \l@chapter, l@section and so on
%    are redefined with respect to the defaults, in order to
%    host larger digits in the section/subsection numbers.

\renewcommand{\@pnumwidth}{2em}
\renewcommand*\l@chapter[2]{%
  \ifnum \c@tocdepth >\m@ne
  \addpenalty{-\@highpenalty}%
  \vskip 1.0em \@plus\p@
  \setlength\@tempdima{2.0em}%
  \begingroup
  \parindent \z@ \rightskip \@pnumwidth
  \parfillskip -\@pnumwidth
  \leavevmode \bfseries
  \advance\leftskip\@tempdima
  \hskip -\leftskip
  #1\nobreak\hfil \nobreak\hb@xt@\@pnumwidth{\hss #2}\par
  \penalty\@highpenalty
  \endgroup
  \fi}
\renewcommand*{\l@section}{\@dottedtocline{1}{2.0em}{2.8em}}
\renewcommand*{\l@subsection}{\@dottedtocline{2}{4.8em}{3.7em}}
\renewcommand*{\l@subsubsection}{\@dottedtocline{3}{8.5em}{4.1em}}
\renewcommand*{\l@paragraph}{\@dottedtocline{4}{12.6em}{5em}}
\renewcommand*{\l@subparagraph}{\@dottedtocline{5}{17.6em}{6em}}
\renewcommand*{\l@figure}{\@dottedtocline{1}{2.0em}{2.8em}} % default 1.5 2.3

% 7) [book.cls] Index parameters

\renewcommand{\subitem}{\@idxitem\quad}
\renewcommand{\subsubitem}{\@idxitem\quad\quad}

\makeatother

%%%%%%%%%% Redefine some style parameters (end) %%%%%%%%%%

% All other definitions: from "definitions.tex"

\input definitions

\raggedbottom
\pagestyle{empty}
\makeindex

\begin{document}
\frontmatter
\null\clearpage
% $Id: preambolo.tex,v 1.5 2006/05/05 08:55:16 loreti Exp $

\enlargethispage{20mm}
{\vspace*{-20mm}\centering%
  {\fontfamily{psv}\fontseries{b}\fontsize{18}{22}\selectfont
    Maurizio Loreti\par}\vspace*{5mm}
  {\fontfamily{psv}\fontseries{m}\fontsize{14}{17}\selectfont
    Dipartimento di Fisica \\
    Universit\`a degli Studi di Padova\par}\vspace*{7mm}
  \makebox[\textwidth][c]{\rule{1.2\textwidth}{3pt}}\par\vspace*{7mm}
  {\fontfamily{psv}\fontseries{b}\fontsize{37}{42}\selectfont
    Teoria degli Errori \\ e Fondamenti di \\
    Statistica\par}\vspace*{10mm}
  {\fontfamily{psv}\fontseries{b}\fontsize{20}{24}\selectfont
    Introduzione alla Fisica Sperimentale\par}\vspace*{10mm}
  \makebox[\textwidth][c]{\rule{1.1\textwidth}{3pt}}\par\vfill
  \includegraphics[width=55mm]{bo.eps.gz} \\[5mm]
  {\fontfamily{psv}\fontseries{m}\fontsize{14}{17}\selectfont
    % Gennaio 1998 \\ (Quinta Edizione)\par
    \thismonth\ \number\year \\
    (Edizione privata fuori commercio)\par}}
\clearpage
\begin{quote}
  \footnotesize\itshape Questo libro \`e stato completamente
  composto e stampato dall'autore.  Sono stati adoperati i
  programmi \TeX\ di Donald~E.~Knuth e \LaTeX\ di Leslie
  Lamport (nella versione \LaTeXe); i caratteri tipografici
  sono quelli della famiglia Lucida, disegnati da Bigelow \&
  Holmes e distribuiti dalla Y\&Y Inc.\ in versione
  PostScript\textsuperscript{\textregistered}.  Il libro
  viene distribuito sotto la licenza GNU GPL, contenuta
  nell'appendice \ref{ch:licgpl}.
\end{quote}
\vfill
\begin{center}
  {\bfseries
    {\huge IMPORTANTE:\par}\vspace*{5mm}
    {\Large questo testo non \`e ancora
      definitivo.\par}\vspace*{5mm}
    {\large Viene mantenuto sotto \verb|CVS|, e questo
      mi permette di risalire a tutti i cambiamenti
      effettuati a partire dalla data e dall'ora in
      cui \TeX\ \`e stato eseguito:\par}\vspace*{5mm}
    {\Large il \today\ alle \daytime.\par}}\vspace*{5mm}
  {\Large\itshape Maurizio Loreti\par}
  \vspace*{30mm}
  {\small\textcopyright\ Copyright 1987--2005 Maurizio
    Loreti \\
    \copyleft\ Copyleft 2005--$\infty$ (a free book
    distributed under the GNU GPL) \\[1ex]
    available at \texttt{http://wwwcdf.pd.infn.it/labo/INDEX.html}
  }
\end{center}
\clearpage
\setcounter{page}{1}
\pagestyle{headings}
\thispagestyle{headings}
\tableofcontents
\cleardoublepage
\addcontentsline{toc}{chapter}{\numberline{}Elenco delle figure}
\listoffigures

\endinput

% $Id: prefazione.tex,v 1.2 2005/04/13 08:28:24 loreti Exp $

\chapter*{Prefazione}
\addcontentsline{toc}{chapter}{\numberline{}Prefazione}
\chaptermark{Prefazione}

{\itshape Quando ho cominciato a tenere uno dei due corsi di
  Esperimentazioni di Fisica I (per il primo anno del Corso
  di Laurea in Fisica), ormai molti anni fa, non sono
  riuscito a trovare un libro di testo in cui fosse
  contenuto, della materia, tutto quello che io ritenevo
  fosse necessario per la formazione di uno studente che si
  supponeva destinato alla carriera del fisico; e, dopo aver
  usato per qualche tempo varie dispense manoscritte, mi
  sono deciso a riunirle in un testo completo in cui la
  materia fosse organicamente esposta.

  Per giungere a questo \`e stato fondamentale l'aiuto
  datomi dal docente dell'altro corso parallelo di
  Esperimentazioni, il Prof.\ Sergio Ciampolillo, senza il
  quale questo libro non sarebbe probabilmente mai venuto
  alla luce; e tanto pi\`u caro mi \`e stato questo suo
  aiuto in quanto lui stesso mi aveva insegnato nel passato
  la statistica, completa di tutti i crismi del rigore
  matematico ma esposta con la mentalit\`a di un fisico e
  mirata ai problemi dei fisici, nei lontani anni in cui
  frequentavo la Scuola di Perfezionamento (a quel tempo il
  Dottorato di Ricerca non era ancora nato).

  Assieme abbiamo deciso l'impostazione da dare al testo e,
  nel 1987, preparata la prima edizione; che era
  ciclostilata, e che veniva stampata e distribuita a prezzo
  di costo agli studenti a cura dello stesso Dipartimento di
  Fisica.  Il contenuto \`e stato poi pi\`u volte ampliato e
  rimaneggiato da me (e, all'inizio, ancora dal Prof.\
  Ciampolillo: che ha partecipato a tutte le successive
  edizioni fino alla quarta compresa); dalla seconda alla
  quarta edizione, poi, il testo \`e stato edito a cura
  della Libreria Progetto.

  L'esposizione della materia \`e vincolata dalla struttura
  del corso: un testo organico dovrebbe ad esempio
  presentare dapprima la probabilit\`a e poi la statistica;
  ma gli studenti entrano in laboratorio sin dal primo
  giorno, e fanno delle misure che devono pur sapere come
  organizzare e come trattare per estrarne delle
  informazioni significative.  Cos\`\i\ si \`e preferito
  parlare subito degli errori di misura e
  dell'organizzazione dei dati, per poi dimostrare soltanto
  alla fine (quando tutta la matematica necessaria \`e stata
  alfine esposta) alcune delle assunzioni fatte; si veda a
  tale proposito l'esempio della media aritmetica, che gli
  studenti adoperano fin dal primo giorno ma il cui uso
  viene del tutto giustificato soltanto nel paragrafo
  \ref{ch:11.mepeted} di questo libro.

  Questo testo non contiene soltanto materia oggetto di
  studio nel primo anno del Corso di Laurea: su richiesta di
  docenti degli anni successivi, nel passato erano state
  aggiunte alcune parti (collocate tra le appendici) che
  potessero loro servire come riferimento.  Ho poi
  largamente approfittato, sia dell'occasione offertami
  dall'uscita di questa quinta edizione che del fatto di
  dover tenere anche un corso di Statistica per la Scuola di
  Dottorato in Fisica, per includere nel testo degli
  argomenti di teoria assiomatica della probabilit\`a e di
  statistica teorica che vanno assai pi\`u avanti delle
  esigenze di uno studente del primo anno: questo perch\'e
  se, negli anni successivi, le necessit\`a dell'analisi
  spingeranno dei fisici ad approfondire dei particolari
  argomenti di statistica, queste aggiunte sono sicuramente
  le basi da cui partire.

  Ho cercato insomma di trasformare un testo ad uso
  specifico del corso di Esperimentazioni di Fisica I in una
  specie di riferimento base per la statistica e l'analisi
  dei dati: per questo anche il titolo \`e cambiato, e
  ``Introduzione alle Esperimentazioni di Fisica I'' \`e
  diventato un pi\`u ambizioso ``Introduzione alla Fisica
  Sperimentale''; e ``Teoria degli errori e analisi dei
  dati'' un pi\`u veritiero ``Teoria degli Errori e
  Fondamenti di Statistica''.  Ma, anche se le nuove
  aggiunte (addirittura per un raddoppio complessivo del
  contenuto originale) sono mescolate alle parti che
  utilizzo nel corso, ho cercato di far s\`\i\ che queste
  ultime possano essere svolte indipendentemente dalla
  conoscenza delle prime.

  Pu\`o stupire invece che manchi una parte di descrizione e
  discussione organica delle esperienze svolte e degli
  strumenti usati: ma gli studenti, come parte integrante
  del corso, devono stendere delle relazioni scritte sul
  loro operato che contengano appunto questi argomenti; e si
  \`e preferito evitare che trovassero gi\`a pronto un
  riferimento a cui potessero, per cos\`\i\ dire, ispirarsi.

  \vfill\begin{flushright}
    \mbox{
      \begin{tabular}{c}
        Maurizio Loreti \\[3mm]
        Gennaio 1998 \\
        (Quinta edizione)
      \end{tabular}
      }\quad
  \end{flushright}
}

\chapter*{Prefazione alla sesta edizione}
\addcontentsline{toc}{chapter}{\numberline{}Prefazione alla
  sesta edizione}
\chaptermark{Prefazione alla sesta edizione}

{\itshape Gli anni sono passati, e non ho mai smesso di
  modificare questo testo: per prima cosa ho corretto gli
  errori segnalati dai colleghi e dagli studenti, ed ho
  sfruttato parecchi dei loro consigli.  Poi ho aggiunto
  nuovi paragrafi: sia per descrivere distribuzioni teoriche
  che, sia pur poco usate nella Fisica, ogni tanto vi
  compaiono; sia per introdurre piccoli esempi ed
  applicazioni giocattolo usati nel tenere per la seconda
  volta un corso di Statistica alla Scuola di Dottorato in
  Fisica.

  Come conseguenza, questo testo \`e a parer mio parecchio
  migliorato rispetto a quello pubblicato nel 1998 dalla
  Decibel-Zanichelli; purtroppo le scarse vendite hanno
  indotto la casa editrice a rinunciare ad una seconda
  edizione che seguisse la prima.  Conseguentemente ho
  deciso di mettere nella sua forma attuale (la ``sesta
  edizione'') questo libro a disposizione della comunit\`a
  su Internet: sperando che possa ancora servire a qualcuno.
  La licenza \`e quella GPL, inclusa nella appendice
  \ref{ch:licgpl}; in sostanza \`e permesso modificare a
  piacimento e ridistribuire in qualunque forma questo
  libro, purch\'e la ridistribuzione comprenda il sorgente.

  Un'ultima considerazione personale: adesso il Corso di
  Laurea in Fisica \`e articolato in ``tre pi\`u due'' anni
  di studio; ed \`e con un certo piacere che ho visto come
  tutti gli studenti che hanno preparato con noi, sui dati
  dell'esperimento CDF, le loro tesi di laurea di primo
  livello abbiano potuto trovare aiuto in queste pagine per
  completarle, usando ad esempio sia metodi di verifica
  delle ipotesi basati sul rapporto delle funzioni di
  verosimiglianza che il test di Kolmogorov e Smirnov.

  \vfill\begin{flushright}
    \mbox{
      \begin{tabular}{c}
        Maurizio Loreti \\[3mm]
        \thismonth\ \number\year \\
        (Sesta edizione)
      \end{tabular}
    }\quad
  \end{flushright}
}

\endinput

\cleardoublepage
{\markboth{}{}
  \pagestyle{empty}
  ~\vspace*{20mm}
  \begin{flushright}
    \fontfamily{hlcn}\fontseries{m}\fontshape{it}
    \fontsize{12}{16}\selectfont
    ``Where shall I begin, please, your Majesty?'' he asked. \\
    ``Begin at the beginning,'' the King said, gravely, \\
    ``and go on till you come to the end: then stop.'' \\[2ex]
    \fontsize{10}{12}\selectfont
    Charles L.\ Dodgson (Lewis Carroll) \\
    Alice in Wonderland (1865) \par
  \end{flushright}}
\cleardoublepage
\mainmatter
% $Id: chapter1.tex,v 1.1 2005/03/01 10:06:08 loreti Exp $

\chapter{Introduzione}
Scopo della Fisica \`e lo studio dei fenomeni naturali, dei
quali essa cerca per prima cosa di dare una descrizione; che
deve essere non solo qualitativa, ma soprattutto
\emph{quantitativa}.  Questo richiede di individuare,
all'interno del fenomeno, quelle grandezze fisiche in grado
di caratterizzarlo univocamente; e di ottenere, per ognuna
di esse, i valori che si sono presentati in un insieme
significativo di casi reali.  Lo studio si estende poi
\emph{oltre} la semplice descrizione, e deve comprendere
l'indagine sulle relazioni reciproche tra pi\`u fenomeni,
sulle cause che li producono e su quelle che ne determinano
le modalit\`a di presentazione.  Fine ultimo di tale ricerca
\`e quello di formulare delle \emph{leggi fisiche} che siano
in grado di dare, del fenomeno in esame, una descrizione
razionale, quantitativa e (per quanto possibile) completa; e
che permettano di dedurre univocamente le caratteristiche
con cui esso si verificher\`a dalla conoscenza delle
caratteristiche degli altri fenomeni che lo hanno causato (o
che comunque con esso interagiscono).

Oggetto quindi della ricerca fisica devono essere delle
\emph{grandezze misurabili}; enti che cio\`e possano essere
caratterizzati dalla valutazione quantitativa di alcune loro
caratteristiche, suscettibili di variare da caso a caso a
seconda delle particolari modalit\`a con cui il fenomeno
studiato si svolge\/\footnote{Sono \emph{grandezze
    misurabili} anche quelle connesse a oggetti non
  direttamente osservabili, ma su cui possiamo indagare
  attraverso lo studio delle influenze prodotte dalla loro
  presenza sull'ambiente che li circonda.  Ad esempio i
  \emph{quarks}, costituenti delle particelle elementari
  dotate di interazione forte, secondo le attuali teorie per
  loro stessa natura non potrebbero esistere isolati allo
  stato libero; le loro caratteristiche (carica, spin etc.)
  non sono quindi direttamente suscettibili di misura: ma
  sono ugualmente oggetto della ricerca fisica, in quanto
  sono osservabili e misurabili i loro effetti al di fuori
  della particella entro la quale i quarks sono relegati.}.

\section{Il metodo scientifico}%
\index{metodo!scientifico|(}%
Il linguaggio usato dal ricercatore per la formulazione
delle leggi fisiche \`e il \emph{linguaggio matematico}, che
in modo naturale si presta a descrivere le relazioni tra i
dati numerici che individuano i fenomeni, le loro variazioni
ed i loro rapporti reciproci; il procedimento usato per
giungere a tale formulazione \`e il \emph{metodo
  scientifico}, la cui introduzione si fa storicamente
risalire a Galileo Galilei.  Esso pu\`o essere descritto
distinguendone alcune fasi successive:
\begin{itemize}
\item Una fase \emph{preliminare} in cui, basandosi sul
  bagaglio delle conoscenze precedentemente acquisite, si
  determinano sia le grandezze rilevanti per la descrizione
  del fenomeno che quelle che presumibilmente influenzano le
  modalit\`a con cui esso si presenter\`a.
\item Una fase \emph{sperimentale} in cui si compiono
  osservazioni accurate del fenomeno, controllando e
  misurando sia le grandezze che lo possono influenzare sia
  quelle caratteristiche quantitative che lo individuano e
  lo descrivono, mentre esso viene causato in maniera (per
  quanto possibile) esattamente riproducibile; ed in questo
  consiste specificatamente il lavoro dei fisici
  sperimentali.
\item Una fase di \emph{sintesi} o congettura in cui,
  partendo dai dati numerici raccolti nella fase precedente,
  si inducono delle relazioni matematiche tra le grandezze
  misurate che siano in grado di render conto delle
  osservazioni stesse; si formulano cio\`e delle leggi
  fisiche ipotetiche, controllando se esse sono in grado di
  spiegare il fenomeno.
\item Una fase \emph{deduttiva}, in cui dalle ipotesi
  formulate si traggono tutte le immaginabili conseguenze:
  particolarmente la previsione di fenomeni non ancora
  osservati (almeno non con la necessaria precisione); e
  questo \`e specificatamente il compito dei fisici teorici.
\item Infine una fase di \emph{verifica} delle ipotesi prima
  congetturate e poi sviluppate nei due passi precedenti, in
  cui si compiono ulteriori osservazioni sulle nuove
  speculazioni della teoria per accertarne l'esattezza.
\end{itemize}

Se nella fase di verifica si trova rispondenza con la
realt\`a, l'ipotesi diviene una legge fisica accettata; se
d'altra parte alcune conseguenze della teoria non risultano
confermate, e non si trovano spiegazioni delle discrepanze
tra quanto previsto e quanto osservato nell'ambito delle
conoscenze acquisite, la legge dovr\`a essere modificata in
parte, o rivoluzionata completamente per essere sostituita
da una nuova congettura; si ritorna cio\`e alla fase di
sintesi, e l'evoluzione della scienza comincia un nuovo
ciclo.

Naturalmente, anche se non si trovano contraddizioni con la
realt\`a ci\`o non vuol dire che la legge formulata sia
esatta: \`e possibile che esperimenti effettuati in
condizioni cui non si \`e pensato (o con strumenti di misura
pi\`u accurati di quelli di cui si dispone ora) dimostrino
in futuro che le nostre congetture erano in realt\`a
sbagliate; come esempio, basti pensare alla legge galileiana
del moto dei corpi ed alla moderna teoria della
relativit\`a.

\`E evidente quindi come le fasi di indagine e di verifica
sperimentale costituiscano parte fondamentale dello studio
dei fenomeni fisici; scopo di questo corso \`e quello di
presentare la teoria delle misure e degli
errori ad esse connessi.%
\index{metodo!scientifico|)}

\endinput

% $Id: chapter2.tex,v 1.1 2005/03/01 10:06:08 loreti Exp $

\chapter{La misura}
Ad ogni grandezza fisica si deve, almeno in linea di
principio, poter associare un valore numerico in modo
\emph{univoco} ed \emph{oggettivo}, cio\`e riproducibile
nelle stesse condizioni da qualsiasi osservatore; valore
pari al rapporto fra la grandezza stessa e l'unit\`a di
misura per essa prescelta.

Per eseguire tale associazione dobbiamo disporre di
strumenti e metodi che ci permettano di mettere in relazione
da una parte la grandezza da misurare, e dall'altra
l'unit\`a di misura (oppure suoi multipli o sottomultipli);
e ci dicano se esse sono uguali o, altrimenti, quale delle
due \`e maggiore.

\section{Misure dirette e misure indirette}
La misura si dice \emph{diretta}%
\index{misure!dirette}
quando si confronta direttamente la grandezza misurata con
l'unit\`a di misura (\emph{campione}) o suoi multipli o
sottomultipli; come esempio, la misura di una lunghezza
mediante un regolo graduato \`e una misura diretta.

\`E una misura diretta anche quella effettuata mediante
l'uso di \emph{strumenti pretarati} (ad esempio la misura
della temperatura mediante un termometro), che si basa sulla
propriet\`a dello strumento di reagire sempre nella stessa
maniera quando viene sottoposto alla medesima
sollecitazione.

Misure \emph{indirette}%
\index{misure!indirette}
sono invece quelle in cui non si misura la grandezza che
interessa, ma altre che risultino ad essa legate da una
qualche relazione funzionale; cos\`\i\ la velocit\`a di
un'automobile pu\`o essere valutata sia direttamente (con il
tachimetro) sia indirettamente: misurando spazi percorsi e
tempi impiegati, dai quali si risale poi alla velocit\`a
(media) con una operazione matematica.

\section{Le unit\`a di misura}%
\index{unit\`a di misura!fondamentali e derivate|(}
Le grandezze fisiche si sogliono dividere in
\emph{fondamentali} e \emph{derivate}.  Con il primo di
questi nomi si indicavano, originariamente, quelle grandezze
misurate con strumenti e metodi sperimentali che
richiedessero un confronto diretto con un campione, scelto
arbitrariamente come unit\`a di misura; mentre le seconde
venivano generalmente determinate in modo indiretto,
ovverosia (come appena detto) attraverso misure dirette di
altre grandezze ad esse legate da relazioni algebriche: che
permettevano non solo di calcolarne i valori, ma ne
fissavano nel contempo anche le unit\`a di misura.

In questo modo si sono definiti vari sistemi di misura
coerenti, come il Sistema Internazionale%
\index{Sistema Internazionale}
(\textbf{SI}) attualmente in uso: esso assume come grandezze
fondamentali lunghezza, massa, tempo, intensit\`a di
corrente elettrica, temperatura, intensit\`a luminosa e
quantit\`a di materia; con le rispettive unit\`a metro,
chilogrammo, secondo, Amp\`ere, grado Kelvin, candela e
mole.  Le unit\`a per la misura delle altre grandezze sono
poi univocamente determinate dalle relazioni algebriche che
le legano a quelle fondamentali.

\index{dimensioni (delle grandezze fisiche)|(}%
Se ciascuna unit\`a fondamentale viene ridotta di un certo
fattore, il valore della grandezza espresso nelle nuove
unit\`a dovr\`a essere moltiplicato per un prodotto di
potenze dei medesimi fattori.  Cos\`\i, per restare
nell'ambito della meccanica, se riduciamo l'unit\`a di
lunghezza di un fattore $L$, l'unit\`a di massa di un
fattore $M$ e quella di tempo di un fattore $T$, ed il
valore di una grandezza fisica ne risultasse in conseguenza
moltiplicato per
\begin{equation*}
  L^\lambda \, M^\mu \, T^\tau \peq ,
\end{equation*}
si dir\`a che la grandezza in questione ha le
\emph{dimensioni} di una lunghezza elevata alla potenza
$\lambda$ per una massa elevata alla potenza $\mu$ per un
tempo elevato alla potenza $\tau$.

Pensiamo alla velocit\`a (media) di un corpo in movimento,
che \`e definita come il rapporto tra lo spazio da esso
percorso in un certo intervallo di tempo e la durata di tale
intervallo: ed \`e dunque una grandezza derivata.  Una volta
scelte le unit\`a di misura delle lunghezze e dei tempi (per
esempio il metro ed il secondo), l'unit\`a di misura delle
velocit\`a risulta fissata univocamente (metro al secondo).

Se si alterano ad esempio l'unit\`a di lunghezza
moltiplicandola per un fattore $1 / L = 1000$ (chilometro),
quella di tempo moltiplicandola per un fattore $1 / T =
3600$ (ora) e quella di massa moltiplicandola per un fattore
$1 / M = 1000$ (tonnellata), il valore di qualunque
velocit\`a nella nuova unit\`a (chilometro all'ora)
risulter\`a alterato rispetto al precedente di un fattore
\begin{equation*}
  L^1 M^0 T^{-1} = L T^{-1}
\end{equation*}
e si dice pertanto che le dimensioni fisiche di una
velocit\`a sono quelle di una lunghezza divisa per un tempo.

Come altro esempio si consideri l'energia cinetica di un
corpo, definita come il lavoro compiuto dalla forza che si
deve applicare per arrestarlo; e che \`e pari numericamente
alla met\`a del prodotto della massa per il quadrato della
velocit\`a del corpo stesso:
\begin{equation*}
  K = \frac{1}{2} \, m v^2 \peq .
\end{equation*}

Essa \`e pertanto una grandezza derivata, la cui unit\`a di
misura nel Sistema Internazionale \`e l'energia cinetica di
un corpo avente massa di 2\un{Kg} ed in moto traslatorio con
velocit\`a di 1\un{m/s} (unit\`a detta \emph{joule}).
Passando al nuovo sistema di unit\`a prima definito (assai
inconsueto per un'energia), il valore di $K$ risulta
moltiplicato per il fattore $M^1 L^2 T^{-2}$; si dice dunque
che un'energia ha le dimensioni di una massa, moltiplicata
per il quadrato di una lunghezza e divisa per il quadrato di
un tempo.

Queste propriet\`a di trasformazione sono legate alla
cosiddetta \emph{analisi dimensionale} ed alla
\emph{similitudine meccanica}, argomenti che esulano da
questo corso.%
\index{dimensioni (delle grandezze fisiche)|)}
Basti qui osservare che il numero di unit\`a indipendenti
non coincide necessariamente con quello delle grandezze
assunte come ``fondamentali''; cos\`\i\ l'angolo piano e
l'angolo solido sono entrambi privi di dimensioni in termini
di grandezze fisiche fondamentali, e come tali dovrebbero
avere come unit\`a di misura derivata ($1\un{m} / 1\un{m}$ e
rispettivamente $1\un{m^2} / 1\un{m^2}$) lo stesso ``numero
puro'' 1, mentre esistono per essi due diverse unit\`a: il
radiante e lo steradiante, quasi essi avessero dimensioni
proprie e distinte.

N\'e vi \`e alcunch\'e di necessario nella scelta delle
grandezze fondamentali quale si \`e venuta configurando
storicamente nel Sistema Internazionale, potendosi definire
un sistema coerente anche con l'assegnazione di valori
convenzionali alle costanti universali delle leggi fisiche
(come proposto agli inizi del secolo da
Max Planck):%
\index{Planck, Max Karl Ernst Ludwig}
cos\`\i\ un sistema di unit\`a ``naturali''%
\index{unit\`a di misura!naturali}
si potrebbe fondare, in linea di principio, ponendo uguali
ad 1 la velocit\`a della luce nel vuoto, il quanto d'azione
(o costante di Planck), la costante di gravitazione
universale, la costante di Boltzmann ed il quanto elementare
di carica elettrica (ovverosia la carica dell'elettrone).
Ma, a parte considerazioni di opportunit\`a e consuetudine,
ci\`o che determina in ultima analisi fino a che punto si
possa tradurre in pratica un simile programma, e quali
grandezze siano quindi da considerare fondamentali, \`e la
riproducibilit\`a dei campioni e la precisione con cui \`e
possibile il confronto diretto tra grandezze omogenee.

\`E emblematica a questo riguardo la storia dell'evoluzione
delle unit\`a di misura delle lunghezze: queste anticamente
erano riferite a parti del corpo umano quali il braccio, il
cubito (gi\`a usato dagli Egizi), il piede e la larghezza
del pollice; ovvero delle medie di tali lunghezze su di un
numero limitato di individui.  L'ovvio vantaggio di una
simile definizione \`e la disponibilit\`a del campione in
ogni tempo e luogo; l'altrettanto ovvio svantaggio \`e la
grande variabilit\`a del campione stesso, donde il ricorso
dapprima a valori medi ed infine a campioni artificiali
costruiti con materiali e accorgimenti che garantissero una
minima variabilit\`a della loro lunghezza, col tempo e con
le condizioni esterne pi\`u o meno controllabili.

Cos\`\i, dopo la parentesi illuministica che port\`o
all'adozione della quarantamilionesima parte del meridiano
terrestre quale unit\`a di lunghezza (metro), e fino al
1960, il metro campione fu la distanza tra due tacche
tracciate su di un'opportuna sezione di una sbarra costruita
usando una lega metallica molto stabile; tuttavia le
alterazioni spontanee della struttura microcristallina della
sbarra fanno s\`\i\ che diversi campioni, aventi la medesima
lunghezza alla costruzione, presentino con l'andar del tempo
differenze apprezzabili dai moderni metodi di misura.
Inoltre l'uso di metodi ottici interferenziali fin\`\i \
per consentire un confronto pi\`u preciso delle lunghezze, e
condusse nel 1960 (come suggerito da Babinet gi\`a nel
1829!) a svincolare la definizione del metro dalla
necessit\`a di un supporto materiale macroscopico, col porlo
uguale a $1 \updot 650 \updot 763.73$ volte l'effettivo
campione: cio\`e la lunghezza d'onda nel vuoto della luce
emessa, in opportune condizioni, da una sorgente atomica
(riga arancione dell'isotopo del Kripton $^{86}
\makebox{Kr}$).

L'ulteriore rapido sviluppo della tecnologia, con l'avvento
di laser molto stabili e di misure accuratissime delle
distanze planetarie col metodo del radar, ha condotto
recentemente (1984) ad una nuova definizione del metro, come
distanza percorsa nel vuoto dalla luce in una determinata
frazione ($1 / 299 \updot 792 \updot 458$) dell'unit\`a di
tempo (secondo); il che equivale ad assumere un valore
convenzionale per il campione di velocit\`a (la velocit\`a
della luce nel vuoto) ed a ridurre la misura della lunghezza
fondamentale ad una misura di tempo.  \`E implicita nella
definizione anche la fiducia nell'indipendenza della
velocit\`a della luce nel vuoto sia dal sistema di
riferimento dell'osservatore che dal ``tipo'' di luce
(frequenza, stato di polarizzazione e cos\`\i\ via); ipotesi
queste che sono necessarie conseguenze delle moderne teorie
della fisica.

Le misure di lunghezza hanno dunque percorso l'intero arco
evolutivo, ed appare evidente come la complessa realt\`a
metrologica odierna non sia pi\`u riflessa esattamente nella
classificazione tradizionale di grandezze ``fondamentali'' e
``derivate''.  Infatti la velocit\`a assume ora in un certo
senso il ruolo di grandezza fondamentale, e tuttavia una
velocit\`a non si misura praticamente mai per confronto
diretto col campione (la velocit\`a della luce nel vuoto);
per converso le lunghezze sono spesso ancora oggi misurate
per confronto con campioni, ma la lunghezza del campione
primario (il metro) \`e a sua volta determinata da una
misura di tempo.

Per quanto riguarda l'unit\`a di durata temporale, essa fu
svincolata da un supporto macroscopico (il moto diurno della
terra o i moti planetari) nel 1964 con l'adozione di un
campione di frequenza atomico (in termini imprecisi il
cosiddetto ``orologio atomico al Cesio''), assegnando il
valore convenzionale di $9 \updot 192 \updot 631 \updot 770$
cicli al secondo (hertz) alla frequenza della radiazione
elettromagnetica emessa in una particolare transizione tra
due stati quantici dell'atomo di $^{133} \makebox{Cs}$.

Questa definizione del minuto secondo consente il confronto
di intervalli di tempo con un errore
relativo\/\footnote{Vedi il paragrafo \ref{ch:2.errel} alla
  fine del corrente capitolo.} inferiore ad una parte su
$10^{13}$.  Se si considera che il quanto d'azione $\hbar$,
che \`e la costante universale della meccanica (quantistica)
determinata con maggior precisione dopo la velocit\`a della
luce nel vuoto e che sia da essa indipendente, \`e noto
soltanto con una incertezza dell'ordine di 0.2 parti per
milione\/\footnote{Attualmente (2004), l'errore relativo sul
  valore comunemente usato di $\hbar$ (e che vale $1.054
  \updot 571 \updot 68 \times 10^{-34} \un{J s}$) \`e di
  $1.7 \times 10^{-7}$.}, si comprende quale iato si
dovrebbe colmare per portare a compimento il programma di
Planck anche con il tempo, cos\`\i\ come lo si \`e
realizzato per la lunghezza.

Ad uno stadio ancora meno avanzato \`e giunta l'evoluzione
delle misure di massa, il cui campione \`e tuttora
costituito da un particolare oggetto macroscopico detto
``chilogrammo-campione''.  Anche qui la precisione con cui
si possono confrontare le masse supera di vari ordini di
grandezza quella con cui \`e nota la costante di
gravitazione universale, cui l'attribuzione di un valore
convenzionale consentirebbe di ridurre le misure di massa a
quelle di tempo e di lunghezza.%
\index{unit\`a di misura!fondamentali e derivate|)}

\section{Gli strumenti di misura}%
\index{strumenti di misura|(}
Lo strumento di misura \`e un apparato che permette il
confronto tra la grandezza misurata e l'unit\`a prescelta.
Esso \`e costituito da un oggetto sensibile in qualche modo
alla grandezza da misurare, che si pu\`o chiamare
\emph{rivelatore}; eventualmente da un dispositivo
\emph{trasduttore}, che traduce le variazioni della
grandezza caratteristica del rivelatore in quelle di
un'altra grandezza pi\`u facilmente accessibile allo
sperimentatore; e da un dispositivo \emph{indicatore} che
presenta il risultato della misura ai sensi (generalmente
alla vista) dello sperimentatore: o direttamente o mediante
una registrazione, grafica o di altro genere.

Cos\`\i\ in un \emph{calibro}, strumento per la misura di
spessori, il rivelatore \`e costituito dalla ganascia mobile
col cursore ad essa solidale, e che pu\`o scorrere nella
guida facente corpo unico con la ganascia fissa; mentre
l'elemento indicatore \`e costituito dalla scala graduata in
millimetri tracciata sulla guida e dal segno di fede inciso
sul cursore, a sua volta generalmente collegato ad una scala
graduata ausiliaria (\emph{nonio}) per la lettura delle
frazioni di millimetro.  La grandezza letta sulla scala \`e
qui direttamente lo spessore oggetto della misura.

In un \emph{termometro} a liquido l'elemento sensibile alla
temperatura \`e il liquido contenuto nel bulbo; esso funge
almeno in parte anche da trasduttore, perch\'e la
propriet\`a termometrica che viene usata \`e il volume del
rivelatore stesso.  Il tubo capillare a sezione costante
traduce le variazioni di volume del rivelatore in variazioni
di lunghezza della colonna di liquido ivi contenuta; il
menisco che separa il liquido dal suo vapore nel capillare
funge da indicatore, assieme con la scala tracciata sulla
superficie esterna del tubo stesso o sopra un regolo ad essa
solidale.  La grandezza letta sulla scala \`e la distanza
del menisco da un segno di riferimento che pu\`o essere
messa in corrispondenza con la temperatura per mezzo di una
tabella di conversione; oppure, pi\`u spesso e comodamente,
le temperature corrispondenti sono scritte sulla scala
stessa accanto alle tacche della graduazione.

\index{strumenti di misura!caratteristiche|(}%
Le caratteristiche pi\`u importanti di uno strumento sono le
seguenti:
\begin{itemize}
\item La \emph{prontezza}: \`e determinata dal tempo
  necessario perch\'e lo strumento risponda in modo completo
  ad una variazione della sollecitazione; ad esempio, per
  avere una risposta corretta da un termometro si deve
  attendere che si raggiunga l'equilibrio termico tra il
  rivelatore e l'oggetto di cui si misura la temperatura.
\item L'\emph{intervallo d'uso}: \`e definito come l'insieme
  dei valori compresi tra la \emph{soglia} e la
  \emph{portata} dello strumento, cio\`e tra il minimo ed il
  massimo valore della grandezza che lo strumento pu\`o
  apprezzare in un singolo atto di misura.
\item La \emph{sensibilit\`a}: si pu\`o definire come il
  reciproco della incertezza di lettura propria dello
  strumento, cio\`e della pi\`u piccola variazione della
  grandezza che pu\`o essere letta sulla scala, e che si
  assume generalmente corrispondente alla pi\`u piccola
  divisione della scala stessa (o ad una frazione
  apprezzabile di questa).  La sensibilit\`a pu\`o essere
  diversa in differenti punti della scala, o per diversi
  valori della grandezza; \`e un fattore che limita
  l'intervallo d'uso dello strumento, potendo divenire
  insufficiente al di sotto della soglia od al di sopra
  della portata.
\item La \emph{precisione} dello strumento: \`e legata alla
  riproducibilit\`a del risultato della misura di una stessa
  grandezza.  Esso pu\`o variare da una parte per difetti
  dello strumento dovuti alla costruzione, che non pu\`o mai
  essere perfetta, e per il logoramento di alcune componenti
  in conseguenza dell'uso prolungato o improprio, o
  dell'invecchiamento; e, inoltre, per la presenza di varie
  cause di disturbo ineliminabili anche in condizioni
  normali d'uso dello strumento stesso.

  Tutto questo fa s\`\i\ che misure ripetute di una stessa
  grandezza fisica si distribuiscano in un intervallo pi\`u
  o meno ampio; la precisione si pu\`o definire come il
  reciproco dell'incertezza sul valore della grandezza che
  viene determinata dall'insieme di questi fattori: ed \`e
  sostanzialmente legata all'entit\`a degli \emph{errori
    casuali}, di cui parleremo tra poco nel paragrafo
  \ref{ch:2.errmis}.

\item L'\emph{accuratezza} dello strumento; ossia la sua
  capacit\`a di fornire valori corrispondenti a quello
  realmente posseduto dalla grandezza in esame.  In altre
  parole, se lo strumento \`e accurato ci si aspetta che i
  risultati di misure ripetute della stessa grandezza fisica
  siano equamente distribuiti in un intorno del valore vero;
  questa caratteristica degli strumenti sar\`a, come
  vedremo, legata alla presenza di \emph{errori sistematici}
  da essi introdotti (di questi, e delle loro possibili
  cause parleremo sempre nel paragrafo \ref{ch:2.errmis}).

  Ci si attende da uno sperimentatore serio che sappia
  individuare le cause di scarsa accuratezza nei suoi
  strumenti (ad esempio un'errata taratura dello zero della
  scala) ed in qualche modo neutralizzarle; cos\`\i\ da
  ricondursi, in ogni caso, a \emph{risultati accurati}.
\end{itemize}%
\index{strumenti di misura!caratteristiche|)}

Per sfruttare a pieno le possibilit\`a di uno strumento di
misura, \`e opportuno che la sensibilit\`a non sia troppo
inferiore alla precisione; gli strumenti di uso corrente
sono costruiti con una sensibilit\`a circa uguale alla
precisione in condizioni normali d'uso.

Anche se, per questo motivo, generalmente la sensibilit\`a e
la precisione in uno strumento hanno valori simili, fate
attenzione a non confondere i due concetti: la sensibilit\`a
\`e una caratteristica intrinseca degli strumenti, e rimane
perci\`o costante in ogni situazione; mentre la precisione
delle nostre misure dipende, \`e vero, dal tipo di strumento
usato (e quindi dalla sua sensibilit\`a) --- ma anche dalle
modalit\`a contestuali di impiego e dal tipo di grandezza
misurata.

Cos\`\i\ su un orologio nella cui scala non siano
riportate che poche divisioni (l'inverso della sensibilit\`a
sia ad esempio di 60 o 15 minuti) non \`e difficile stimare
l'ora con una approssimazione che invece \`e dell'ordine di
pochi minuti; mentre un cronometro in grado di apprezzare il
decimillesimo di secondo, se azionato a mano, difficilmente
pu\`o raggiungere una precisione inferiore al decimo.

Similmente, un regolo lungo un metro e graduato al
millimetro pu\`o essere usato per valutare le dimensioni di
un quaderno (con un singolo atto di misura); oppure
(riportandolo varie volte di seguito a se stesso) le
dimensioni di un edificio.  \`E evidente come, pur essendo
lo strumento lo stesso (quindi la sensibilit\`a non varia)
la precisione delle misure debba essere completamente
diversa nei due casi.%
\index{strumenti di misura|)}

\section{Errori di misura}
\label{ch:2.errmis}
Come gi\`a accennato in relazione alla precisione di uno
strumento, se si esegue una misura di una qualsiasi
grandezza fisica si commettono inevitabilmente errori;
conseguentemente il valore ottenuto per la grandezza
misurata non \`e mai esattamente uguale al suo vero valore,
che non ci potr\`a perci\`o mai essere noto con precisione
arbitrariamente grande (diversamente da quanto accade con
una costante matematica, come ad esempio $\pi$).

Quando si ripete la misura della stessa grandezza col
medesimo strumento, nelle medesime condizioni e seguendo la
medesima procedura, la presenza delle varie cause di errore
(che andremo tra poco ad esaminare) produce delle differenze
casuali tra il valore misurato ed il valore vero; differenze
variabili da una misura all'altra, ed in modo imprevedibile
singolarmente.  In conseguenza di ci\`o, i risultati di
queste misure ripetute (se lo strumento \`e abbastanza
sensibile) fluttueranno apprezzabilmente in maniera casuale
in un certo intervallo: la cui ampiezza definir\`a la
precisione delle misure stesse.  Gli errori di questo tipo
si dicono \emph{errori casuali}%
\index{errori di misura!casuali},
e la loro esistenza \`e facilmente accertabile con l'uso di
un qualsiasi strumento sensibile.

Tuttavia, certe cause di errore possono dar luogo a una
discrepanza tra valore misurato e valore vero che si
riproduce inalterata in una serie di misure ripetute: e la
inosservabilit\`a delle fluttuazioni non garantisce affatto
che tale discrepanza sia inferiore all'incertezza di lettura
dello strumento; n\'e si pu\`o esser certi che essa sia
contenuta entro l'intervallo di variabilit\`a degli errori
casuali (quando esso sia maggiore dell'incertezza di
lettura).

Gli errori di questo secondo tipo si dicono
\emph{errori sistematici}%
\index{errori di misura!sistematici|(emidx}
e sono i pi\`u insidiosi, perch\'e non risultano
immediatamente identificabili.  Cause di errori sistematici
possono essere quelle elencate nel seguito (ma la lista non
\`e necessariamente completa):
\begin{enumerate}
\item \emph{Difetti dello strumento, risalenti alla
    costruzione o conseguenti al suo deterioramento}.  Ad
  esempio, in una bilancia con bracci di lunghezza diversa,
  l'uguaglianza dei momenti applicati ai due bracci ed
  assicurata dall'equilibrio del giogo non implica
  l'uguaglianza delle masse ad essi sospese: perch\'e una
  massa minore sospesa al braccio pi\`u lungo produrr\`a una
  azione atta ad equilibrare quella esercitata da una massa
  maggiore sospesa all'altro (questo errore si potrebbe
  anche classificare nel tipo \ref{li:2.forapp}, cio\`e come
  errore di interpretazione del risultato).

  Un altro esempio \`e quello di un goniometro
  \emph{eccentrico}, cio\`e avente la croce centrale o
  l'asse di rotazione in posizione diversa da quella del
  centro del cerchio recante la graduazione: ci\`o determina
  come conseguenza misure di angoli acuti sistematicamente
  errate per difetto o per eccesso a seconda della posizione
  del centro presunto rispetto agli assi 0\gra--180\gra\ e
  90\gra--270\gra\ del goniometro.

  Lo zero di una scala (ad esempio di un termometro) pu\`o
  essere spostato dalla posizione corretta di taratura, per
  cui tutte le letture saranno in difetto o in eccesso a
  seconda del verso di tale spostamento.  Oppure la scala
  stessa dello strumento pu\`o essere difettosa: cos\`\i, se
  il capillare di un termometro non ha sezione costante,
  anche se le posizioni corrispondenti a due punti fissi
  come 0\gra{}C e 100\gra{}C fossero esatte, le temperature
  lette risulterebbero in difetto in un tratto della scala
  ed in eccesso in un altro tratto.
\item \emph{Uso dello strumento in condizioni errate},
  cio\`e diverse da quelle previste per il suo uso corretto.
  Tale \`e l'uso di regoli, calibri e simili strumenti per
  misurare le lunghezze, o di recipienti tarati per la
  misura dei volumi, a temperature diverse da quella di
  taratura (generalmente fissata a 20\gra{}C); infatti la
  dilatazione termica far\`a s\`\i\ che lunghezze e volumi
  risultino alterati, in difetto o in eccesso a seconda che
  si operi a temperatura superiore o inferiore.

  Si pu\`o naturalmente commettere un errore anche usando lo
  strumento a 20\gra{}C, quando ci\`o che interessa in
  realt\`a \`e conoscere il valore di una grandezza
  dipendente dalla temperatura (la lunghezza di un oggetto,
  il volume di un corpo, la resistenza elettrica di un filo
  o qualsiasi altra) ad una temperatura diversa da
  20\gra{}C.
\item \emph{Errori di stima da parte dello sperimentatore}:
  un esempio di questo tipo di errore si ha quando, nello
  stimare una certa frazione di divisione di una scala
  graduata, lo sperimentatore tende a valutarla sempre in
  difetto o sempre in eccesso; oppure quando, nel leggere la
  posizione di un indice mobile posto di fronte ad una scala
  graduata (non sullo stesso piano), lo sperimentatore tenga
  il proprio occhio sistematicamente alla sinistra o alla
  destra del piano passante per l'indice ed ortogonale alla
  scala stessa (\emph{errore di parallasse}).  Proprio per
  evitare questi errori di parallasse, dietro gli indici
  mobili degli strumenti pi\`u precisi si pone uno specchio
  che aiuta l'osservatore a posizionarsi esattamente davanti
  ad esso.
\item \emph{Perturbazioni esterne}; un esempio di errori di
  questo tipo \`e la presenza di corpi estranei, come la
  polvere, interposti tra le ganasce di un calibro e
  l'oggetto da misurare: questo porta a sovrastimarne lo
  spessore.

  Un altro esempio \`e la misura della profondit\`a del
  fondo marino o fluviale con uno scandaglio (filo a piombo)
  in presenza di corrente; questa fa deviare il filo dalla
  verticale e porta sempre a sovrastimare la profondit\`a se
  il fondo \`e approssimativamente orizzontale.
\item \emph{Perturbazione del fenomeno osservato da parte
    dell'operazione di misura}.  Tra gli errori di questo
  tipo si pu\`o citare la misura dello spessore di un
  oggetto con un calibro a cursore, o col pi\`u sensibile
  calibro a vite micrometrica (Palmer); l'operazione
  richiede l'accostamento delle ganasce dello strumento
  all'oggetto, ed effettuandola si comprime inevitabilmente
  quest'ultimo con una forza sia pur piccola: e se ne
  provoca perci\`o una deformazione, con leggera riduzione
  dello spessore.
\item \label{li:2.forapp} \emph{Uso di formule errate o
    approssimate nelle misure indirette}.  Un esempio \`e
  offerto dalla misura indiretta dell'accelerazione di
  gravit\`a $g$, ottenuta dalla misura della lunghezza
  (cosiddetta ridotta) $l$ di un apposito tipo di pendolo
  (di Kater) e dalla misura del suo periodo di oscillazione
  $T_0$, utilizzando la formula
  \begin{gather}%
  \index{pendolo, periodo del|(}
    g = 4 \pi^2 \, \frac{l}{{T_0}^2} \label{eq:2.g} \\
    \intertext{ottenuta dalla nota espressione del periodo}
    T_0 = 2 \pi \sqrt{ \frac{l}{g} } \peq . \label{eq:2.t0}
  \end{gather}

  Ma questa formula vale solo, al limite, per oscillazioni
  di ampiezza infinitesima; mentre una formula che meglio
  approssima la realt\`a \`e\/\footnote{Riguardo a questo
    punto ed al successivo, per una discussione approfondita
    del moto del pendolo si pu\`o consultare: G.~Bruhat -
    Cours de M\'ecanique Physique - Ed.\ Masson, pagg.
    311--321.}
  \begin{equation*}
    T \; = \;T(\theta) \; = \; 2 \pi \sqrt{ \frac{l}{g} }
    \left( 1 + \frac{\theta^2}{16} \right) \; = \;
    T_0 \left( 1 + \frac{\theta^2}{16} \right)
  \end{equation*}
  ed essa mostra come il periodo $T$ sia una funzione
  leggermente crescente dell'ampiezza massima $\theta$ delle
  oscillazioni (qui espressa in radianti).  L'uso della
  formula \eqref{eq:2.g} di prima approssimazione per
  determinare $g$ comporta dunque una sua sottostima, che
  diviene tanto pi\`u sensibile quanto maggiore \`e
  $\theta$: questo in quanto si usa in luogo di $T_0$ la
  durata $T$ di una oscillazione reale avente ampiezza non
  nulla --- e perci\`o sempre superiore a $T_0$.

  La medesima misura \`e affetta anche da un'altra causa di
  errore sistematico, originata dal fatto che il pendolo non
  ruota oscillando attorno al filo orizzontale del coltello
  di sospensione; ma compie un moto in cui il profilo del
  taglio del coltello (che \`e approssimativamente un
  cilindro con raggio di curvatura minimo dell'ordine dei
  centesimi di millimetro) rotola sul piano di appoggio.  A
  causa dell'impossibilit\`a di una perfetta realizzazione
  meccanica dell'apparato, il fenomeno osservato \`e diverso
  da quello supposto che si intendeva produrre: e la sua
  errata interpretazione comporta una sovrastima di $g$.

  Infatti la formula del periodo, corretta \emph{per questo
    solo effetto}, risulta essere
  \begin{equation*}
    T = T_0 \, \sqrt{ 1 - \frac{r}{a} }
  \end{equation*}
  (in cui $r$ \`e il raggio di curvatura del filo del
  coltello ed $a$ la distanza del centro di massa dal punto
  di appoggio) ed il $T$ reale \`e sempre inferiore al $T_0$
  definito nell'equazione \eqref{eq:2.t0}.
\end{enumerate}%
\index{pendolo, periodo del|)}

Un modo per rivelare la presenza di errori sistematici
insospettati pu\`o essere quello di misurare, se possibile,
la stessa grandezza con strumenti e metodi diversi; questi
presumibilmente sono affetti da errori aventi cause diverse
e possono fornire perci\`o risultati differenti.  Tuttavia
neppure l'assenza di questo effetto d\`a la certezza che la
misura sia esente da errori sistematici, ed essi sono
generalmente individuati solo da una attenta e minuziosa
analisi critica: sia dello strumento usato, sia della
procedura seguita nella misura.

Una volta scoperto, un errore sistematico pu\`o essere
eliminato: modificando o lo strumento o la procedura, oppure
ancora apportando una opportuna correzione al risultato
della misura (sebbene questo comporti generalmente un
aumento dell'errore casuale: il fattore di correzione deve
essere ricavato sperimentalmente, e quindi sar\`a
affetto da un suo errore intrinseco).%
\index{errori di misura!sistematici|)}

\index{errori di misura!casuali|(emidx}%
Le prime cinque categorie sopra citate come possibili cause
di errori sistematici, possono produrre anche errori
casuali: cos\`\i, per il primo tipo, gli inevitabili giochi
meccanici e gli attriti tra parti dello strumento in moto
relativo possono dar luogo a risultati fluttuanti; per
quanto riguarda il secondo tipo, condizioni ambientali
variabili e non del tutto controllabili (come temperatura e
pressione) possono produrre variazioni imprevedibili del
risultato.

Lo sperimentatore non ha un comportamento fisso e costante
sia nelle valutazioni che nelle azioni compiute durante
l'operazione di misura; come un esempio di questo terzo tipo
di errori si consideri l'imprevedibile variabilit\`a del
tempo di reazione nell'avvio e nell'arresto di un cronometro
a comando manuale.

Anche i disturbi esterni (quarto tipo), potendo essere di
natura e intensit\`a variabile, produrranno errori di un
segno determinato (sistematici), ma di entit\`a variabile ed
imprevedibile; dunque, in parte, anche casuali.

Si aggiunga a ci\`o che disturbi casuali possono essere
presenti nello strumento stesso per la costituzione
corpuscolare della materia e per la natura fondamentalmente
statistica di certe grandezze fisiche.  Cos\`\i\
l'equipaggio mobile, sospeso ad un filo lungo e sottile, di
una bilancia a torsione di estrema sensibilit\`a, avr\`a
posizioni fluttuanti attorno a quella di equilibrio: non
solo a causa del bombardamento incessante cui esso \`e
sottoposto da parte delle molecole del gas circostante; ma
anche nel vuoto assoluto, per l'agitazione termica dei suoi
stessi costituenti.

Infine, anche le cause del quinto tipo possono dar luogo ad
errori casuali se il disturbo del fenomeno o dell'oggetto
prodotto dall'operazione di misura \`e di entit\`a variabile
e non controllata.

Alle cause comuni con gli errori sistematici si deve qui
aggiungerne una ulteriore e tipica degli errori casuali, e
consistente nella \emph{imperfetta definizione della
  grandezza} che si intende misurare.  Anche restando
nell'ambito della fisica classica (e come accennato in
relazione ai disturbi delle misure), certe grandezze, quali
la pressione e la temperatura, sono in realt\`a legate a
delle medie statistiche, come l'energia cinetica media
molecolare; in quanto tali esse hanno un'indeterminazione
intrinseca, che tuttavia non si manifesta nelle misure
relative ad oggetti e fenomeni macroscopici se non in casi
eccezionali.

Ad un livello meno fondamentale, se si misura pi\`u volte
con un calibro il diametro di un oggetto sferico pu\`o
avvenire che i risultati siano leggermente diversi di misura
in misura; questo perch\'e l'oggetto non pu\`o essere
\emph{perfettamente} sferico, ed ogni suo diametro ha una
lunghezza generalmente diversa da quella di un altro.

Per concludere, gli errori casuali:
\begin{itemize}
\item Sono osservabili solo con uno strumento
  sufficientemente sensibile, cio\`e quando sono di entit\`a
  maggiore dell'incertezza di lettura della scala.
\item Possono essere ridotti; ad esempio migliorando le
  caratteristiche dello strumento, o controllando pi\`u
  strettamente le condizioni del suo uso e dell'ambiente e
  precisando meglio la procedura di esecuzione della misura:
  ma ci\`o con difficolt\`a crescente sempre pi\`u con la
  precisione.  \emph{Non possono quindi \textbf{mai} essere
    eliminati}.
\item Posseggono tuttavia certe propriet\`a statistiche, che
  studieremo nell'ambito di una teoria matematica che
  verr\`a affrontata nei prossimi capitoli; la loro entit\`a
  pu\`o pertanto essere \emph{stimata}.
\end{itemize}%
\index{errori di misura!casuali|)}

Compito della \emph{teoria dell'errore} \`e appunto quello
di stimare l'errore presumibilmente commesso nell'atto della
misura, a partire dai dati sperimentali stessi.
Riassumendo:
\begin{quote}
  \textit{Scopo della misura di una grandezza fisica \`e il
    valutare sia il rapporto della grandezza stessa con una
    certa unit\`a di misura, sia l'errore da cui tale
    rapporto \`e presumibilmente affetto.}
\end{quote}

Il risultato delle misure dovr\`a quindi \emph{sempre}
essere espresso in una forma del tipo
\begin{equation*}
  l \, = \, 12.34 \pm 0.01 \; \un{m}
\end{equation*}
in cui compaiano le tre parti \emph{valore}, \emph{errore}
ed \emph{unit\`a di misura}.

\section{Cifre significative ed arrotondamenti}%
\index{cifre significative|(emidx}%
\index{arrotondamenti|see{cifre significative}}

Sempre per quanto riguarda il modo di esprimere il risultato
delle nostre misure, \emph{\`e un errore} spingere la
valutazione del risultato stesso al di l\`a della precisione
sperimentale; in altre parole, se il calcolo dell'errore per
la misura di una lunghezza indica incertezza sulla cifra, ad
esempio, dei centimetri, \`e un errore dare nel risultato la
cifra dei millimetri, o (peggio) dei decimi o centesimi di
millimetro.  Nei risultati intermedi possiamo tenere per i
successivi calcoli tutte le cifre che vogliamo; ma, giunti
\emph{al risultato finale}, e solo una volta che l'errore
sia stato calcolato, bisogna troncare il risultato stesso al
livello dell'errore da noi stimato ed arrotondare.
Cos\`\i\/\footnote{Come vedremo nelle ultime righe
  dell'appendice \ref{ch:b.errvar}, normalmente per l'errore
  si d\`a una sola cifra significativa; o al massimo due, se
  le misure sono state veramente molte --- o anche per
  diminuire il disagio psicologico legato al ``buttare via
  qualcosa'' del frutto delle proprie fatiche\ldots}
\begin{center}
  \begin{math}
    \begin{array}{ccccc}
      12.34567 \pm 0.231   & \makebox{diventa} & 12.3   \pm 0.2   &
      \makebox{o} & 12.34 \pm 0.23 \peq ; \\*
      12.34567 \pm 0.00789 & \makebox{diventa} & 12.346 \pm 0.008 &
      \makebox{o} & 12.3457 \pm 0.0079 \peq .
    \end{array}
  \end{math}
\end{center}%
\index{cifre significative|)}

\section{Errore relativo}%
\index{errore!relativo|(}
\label{ch:2.errel}
Una volta valutato l'errore presumibile $\Delta x$ (errore
\emph{assoluto}) da cui \`e affetta la misura $x_0$ di una
grandezza fisica $x$, il rapporto
\begin{equation} \label{eq:2.errel}
 \epsilon = \frac{\Delta x}{\left| x_0 \right|}
\end{equation}
(indicato in \emph{valore} od in \emph{percentuale}) prende
il nome di \emph{errore relativo}; essendo definito
attraverso il modulo del valore stimato della grandezza in
esame, l'errore relativo \`e una quantit\`a sicuramente
positiva.

L'errore relativo \`e importante perch\'e, in un certo
senso, esprime la \emph{qualit\`a} della misura di una
grandezza: \`e evidente come un errore assoluto stimato in
1\un{cm} assuma ben diverso significato se riferito alla
misura di un tavolo o di una distanza astronomica --- ed \`e
appunto la differenza fra gli errori relativi a suggerirci
tale interpretazione.

\`E opportuno tuttavia osservare che l'errore relativo
definito nella \eqref{eq:2.errel} \`e privo di senso quando
il valore vero della grandezza che si misura \`e nullo;
pertanto si potr\`a parlare di errore relativo solo quando
si possa escludere tale eventualit\`a con pratica certezza:
nel caso cio\`e che sia $ \left| x_0 \right| \gg \Delta x $,
ovvero che $\epsilon$ sia di almeno un ordine di grandezza
inferiore all'unit\`a.%
\index{errore!relativo|)}

\endinput

% $Id: chapter3.tex,v 1.1 2005/03/01 10:06:08 loreti Exp $

\chapter{Elementi di teoria della probabilit\`a}
Abbiamo gi\`a notato come, per la ineliminabile presenza
degli errori di misura, quello che otteniamo come risultato
della stima del valore di una grandezza fisica non sia
praticamente mai il valore vero della grandezza stessa;
inoltre, se ripetiamo pi\`u volte la misura, non otteniamo
mai, in generale, nemmeno lo stesso risultato.

Da questo si deduce che, sulla base di misure ripetute
comunque effettuate, non si potr\`a mai affermare che un
qualsiasi numero reale sia (o non sia) il valore vero della
grandezza stessa.  \`E per\`o evidente come tutti gli
infiniti numeri reali non debbano essere posti sullo stesso
piano: alcuni di essi saranno pi\`u verosimili
(intuitivamente i numeri vicini ai risultati delle nostre
misure ripetute), altri (pi\`u lontani) saranno meno
verosimili.

Il problema della misura va dunque impostato \emph{in
  termini probabilistici}; e potremo dire di averlo risolto
quando, a partire dai dati sperimentali, saremo in grado di
determinare un intervallo di valori avente una assegnata
probabilit\`a di contenere il valore vero.  Prima di
proseguire, introduciamo dunque alcuni elementi della
\emph{teoria della probabilit\`a}.

\section{La probabilit\`a: eventi e variabili casuali}%
\index{casuali!eventi|(}
\label{ch:3.varcas}
Oggetto della teoria delle probabilit\`a \`e lo studio dei
fenomeni \emph{casuali} o \emph{aleatori}: cio\`e fenomeni
ripetibili (almeno in teoria) infinite volte e che possono
manifestarsi in pi\`u modalit\`a, imprevedibili
singolarmente, che si escludono a vicenda l'una con l'altra;
esempi tipici di fenomeni casuali sono il lancio di un dado
o di una moneta, o l'estrazione di una carta da un mazzo.
Come risultato del lancio della moneta o del dado, essi
cadranno e si ridurranno in quiete con una determinata
faccia rivolta verso l'alto; per la moneta le possibilit\`a
sono due, mentre per il dado sono sei.

Il complesso delle possibili modalit\`a con cui un fenomeno
casuale si pu\`o verificare costituisce l'insieme (o
\emph{spazio}) dei \emph{risultati}, $\mathcal{S}$; esso
pu\`o essere costituito da un numero finito o infinito di
elementi.

Definiremo poi come \emph{evento casuale} l'associazione di
una o pi\`u di queste possibili modalit\`a: ad esempio, lo
spazio dei risultati per il fenomeno ``lancio di un dado''
\`e un insieme composto da sei elementi; ed uno degli eventi
casuali che \`e possibile definire (e che corrisponde al
realizzarsi dell'uno o dell'altro di tre dei sei possibili
risultati) consiste nell'uscita di un numero dispari.
L'insieme di tutti i possibili eventi (o \emph{spazio degli
  eventi}) $\mathcal{E}$ \`e dunque l'insieme di tutti i
sottoinsiemi di $\mathcal{S}$ (\emph{insieme potenza} o
\emph{insieme delle parti} di $\mathcal{S}$); compresi
l'insieme vuoto $\emptyset$ ed $\mathcal{S}$ stesso, che si
chiamano anche rispettivamente \emph{evento impossibile} ed
\emph{evento certo}.%
\index{casuali!eventi|)}

\index{casuali!variabili|(emidx}%
Se si \`e in grado di fissare una legge di corrispondenza
che permetta di associare ad ogni modalit\`a di un fenomeno
casuale scelta nell'insieme $\mathcal{S}$ uno ed un solo
numero reale $x$, questo numero prende il nome di
\emph{variabile casuale} definita su $\mathcal{S}$.  Le
variabili casuali possono assumere un numero finito od
infinito di valori, e possono essere discrete o continue;
\`e da notare che, per la presenza degli errori, la misura
di una grandezza fisica pu\`o essere considerata come un
evento casuale --- ed il risultato numerico che da tale
misura otteniamo \`e una variabile casuale che possiamo
associare all'evento stesso.%
\index{casuali!variabili|)}

\section{La probabilit\`a: definizioni}
La definizione ``classica'' di probabilit\`a \`e la
seguente:
\begin{quote}
  \index{probabilit\`a!definizione!classica}%
  \textit{Si definisce come probabilit\`a di un evento
    casuale il rapporto tra il numero di casi favorevoli al
    presentarsi dell'evento stesso ed il numero totale di
    casi possibili, purch\'e tutti questi casi possibili
    siano ugualmente probabili.}
\end{quote}
e se ne ricava immediatamente il seguente
\begin{quote}
  \textsc{Corollario:} \textit{la probabilit\`a di un evento
    casuale \`e un numero compreso tra zero e uno, che
    assume il valore zero per gli eventi impossibili ed uno
    per quelli certi.}
\end{quote}

La definizione ``classica'' sembra sufficiente a permetterci
di calcolare le probabilit\`a di semplici eventi casuali che
possano manifestarsi in un numero finito di modalit\`a
equiprobabili (ad esempio per i giochi d'azzardo), ma \`e
intrinsecamente insoddisfacente perch\'e racchiude in s\'e
stessa una \emph{tautologia}: si nota immediatamente come,
per definire la probabilit\`a, essa presupponga che si sia
gi\`a in grado di valutare l'equiprobabilit\`a delle varie
modalit\`a con cui pu\`o manifestarsi l'evento considerato.
Nel caso di una variabile casuale continua, ci\`o si traduce
nell'indeterminazione di quale tra le variabili
topologicamente equivalenti (ossia legate da trasformazioni
continue) sia quella equiprobabile, cio\`e con probabilit\`a
per ogni intervallo proporzionale all'ampiezza
dell'intervallo stesso.

Si possono dare della probabilit\`a definizioni pi\`u
soddisfacenti dal punto di vista logico, ad esempio la
seguente (definizione
\emph{empirica}\/\thinspace\footnote{Anche questa
  definizione non \`e completamente soddisfacente dal punto
  di vista concettuale (come vedremo pi\`u in dettaglio nel
  paragrafo \ref{ch:3.convstat}); ma \`e tra le pi\`u
  intuitive, perch\'e tra le pi\`u vicine all'uso pratico.},
teorizzata da von%
\index{von Mises, Richard}
Mises\/\footnote{Richard von Mises fu un matematico che
  visse dal 1883 al 1953; comp\`\i\ ricerche nei campi della
  probabilit\`a e della statistica, ma soprattutto in quello
  della matematica applicata alla meccanica dei fluidi (nel
  1913 istitu\`\i\ all'Universit\`a di Vienna il primo corso
  al mondo sul volo, e nel 1915 progett\`o un aereo che
  pilot\`o personalmente nel corso della I guerra
  mondiale).}):%
\index{probabilit\`a!definizione!empirica|(}
definiamo la \emph{frequenza relativa}%
\index{frequenza!relativa}
$f(E)$ con cui un evento casuale $E$ si \`e presentato in un
numero totale $N$ di casi reali come il rapporto tra il
numero $n$ di volte in cui l'evento si \`e effettivamente
prodotto (\emph{frequenza assoluta})%
\index{frequenza!assoluta}
ed il numero $N$ delle prove effettuate; la probabilit\`a di
$E$ si definisce euristicamente come l'estensione del
concetto di frequenza relativa su un numero grandissimo di
prove, cio\`e
\begin{equation*}
  p(E) \; \approx \; \lim_{N \rightarrow \infty} f(E)
  \; = \; \lim_{N \rightarrow \infty}
  \left( \frac{n}{N} \right) \peq .
\end{equation*}%
\index{probabilit\`a!definizione!empirica|)}

\section{Propriet\`a della probabilit\`a}
Proseguendo in questa nostra esposizione, useremo ora la
definizione empirica per ricavare alcune propriet\`a delle
probabilit\`a di eventi casuali: queste stesse propriet\`a,
come vedremo nel paragrafo \ref{ch:3.lpdeas}, possono essere
ricavate a partire dalla \emph{definizione assiomatica}
(matematicamente soddisfacente, e che verr\`a presentata nel
paragrafo \ref{ch:3.deaspr}).  Il motivo per cui ci basiamo
sulla definizione empirica \`e sia la maggiore semplicit\`a
delle dimostrazioni che la concretezza e l'intuitivit\`a dei
ragionamenti, che si possono facilmente esemplificare con
semplici procedure pratiche come il lancio di monete e dadi.

\subsection{L'evento complementare}
La mancata realizzazione dell'evento $E$ costituisce
l'\emph{evento complementare}%
\index{complementare, evento}
ad $E$, che indicheremo con \ob{E}; i due eventi $E$ ed
\ob{E}\ si escludono mutuamente, ed esauriscono l'insieme di
tutti i possibili risultati di una prova od esperimento
elementare del tipo considerato.  La frequenza relativa di
\ob{E}\ su $N$ prove \`e
\begin{equation*}
  f \left( \ob{E} \right)
  \; = \; \frac{N-n}{N}
  \; = \; 1-\frac{n}{N} \; = \; 1-f(E)
\end{equation*}
da cui si ricava
\begin{equation*}
  p \left( \ob{E} \right) = 1-p(E)
  \makebox[50mm]{o anche}
  p ( E ) + p \left( \ob{E} \right) = 1 \peq .
\end{equation*}

Analogamente si pu\`o dimostrare che, se $A,B,\ldots,Z$ sono
eventi casuali \emph{mutuamente esclusivi} e che
\emph{esauriscono l'insieme di tutti i possibili risultati},
vale la
\begin{equation} \label{eq:3.norpro}
  p(A) + p(B) +\cdots+ p(Z) = 1 \peq .
\end{equation}

\subsection{Probabilit\`a totale}
Il risultato di una prova o esperimento pi\`u complesso
pu\`o essere costituito dal verificarsi di due eventi
simultanei in luogo di uno solo; come esempio, si consideri
il lancio di una moneta e l'estrazione contemporanea di una
carta da un mazzo.  Se $E$ indica l'apparizione della testa
(\ob{E}\ allora sar\`a l'apparizione della croce) ed $F$
l'estrazione di una carta nera (\ob{F}\ di una carta rossa),
esistono quattro eventi fondamentali non ulteriormente
decomponibili e che si escludono vicendevolmente: $E F$, $E
\ob{F}$, $\ob{E} F$ e $\ob{E} \ob{F}$.

Il simbolo $E F$ indica qui l'evento composto \emph{prodotto
  logico} dei due eventi semplici $E$ ed $F$, cio\`e quello
consistente nel verificarsi \emph{sia dell'uno che
  dell'altro}.  Se ora, su $N$ prove effettuate, la
frequenza assoluta con cui i quattro eventi fondamentali si
sono verificati \`e quella indicata nella seguente tabella:
\medskip
\begin{center}
  \begin{tabular}{c|c|c|}
    \multicolumn{1}{c}{\tabbot} &
    \multicolumn{1}{c}{$F$} &
    \multicolumn{1}{c}{\ob{F}} \\
    \cline{2-3}
    $E$\tabtop\tabbot & $n_{11}$ & $n_{12}$ \\
    \cline{2-3}
    \ob{E}\tabtop\tabbot & $n_{21}$ & $n_{22}$ \\
    \cline{2-3}
  \end{tabular}
\end{center}
\medskip le rispettive frequenze relative saranno
\begin{align*}
  f \left( EF \right) &= \frac{n_{11}}{N} &
    f \left( E \ob{F} \right) &= \frac{n_{12}}{N} \\[2ex]
  f \left( \ob{E} F \right) &= \frac{n_{21}}{N} &
    f \left( \ob{E} \ob{F} \right) &= \frac{n_{22}}{N}
    \peq .
\end{align*}

Facendo uso della definizione empirica di probabilit\`a si
trova, partendo dalle seguenti identit\`a:
\begin{gather*}
  f(E) \; = \; \frac{n_{11}+n_{12}}{N} \; = \;
    f(EF) + f \left( E \ob{F} \right) \\[2ex]
  f(F) \; = \; \frac{n_{11}+n_{21}}{N} \; = \;
    f(EF) + f \left( \ob{E} F \right)
\end{gather*}
che devono valere le
\begin{gather*}
  p(E) = p (EF) + p \left( E \ob{F} \right) \peq , \\[1ex]
  p(F) = p (EF) + p \left( \ob{E} F \right) \peq ,
\end{gather*}
ed altre due simili per \ob{E}\ e \ob{F}.

Se ora si applica la definizione empirica all'evento
complesso $E+F$ \emph{somma logica} degli eventi semplici
$E$ ed $F$, definito come l'evento casuale consistente nel
verificarsi \emph{o dell'uno o dell'altro di essi o di
  entrambi}, otteniamo
\begin{align*}
  f(E+F) &= \frac{n_{11}+n_{12}+n_{21}}{N} \\[1ex]
  &= \frac{(n_{11}+n_{12})+(n_{11}+n_{21})-n_{11}}{N} \\[1ex]
  &= f(E)+f(F)-f(EF)
\end{align*}
da cui, passando al limite,
\begin{equation*}%
\index{probabilit\`a!totale (teorema della)|(}
  p(E+F) = p(E)+p(F)-p(EF) \peq .
\end{equation*}

Nel caso particolare di due eventi $E$ ed $F$ che si
escludano mutuamente (cio\`e per cui sia $p(EF) = 0$ e
$n_{11} \equiv 0$) vale la cosiddetta \emph{legge della
  probabilit\`a totale}:
\begin{equation*}
  p(E+F) = p(E)+p(F)
\end{equation*}
Questa si generalizza poi per induzione completa al caso di
pi\`u eventi (sempre per\`o \emph{mutuamente esclusivi}),
per la cui somma logica la probabilit\`a \`e uguale alla
somma delle probabilit\`a degli eventi semplici:
\begin{equation} \label{eq:3.protot}
  p(A+B+\cdots+Z) \: = \: p(A)+p(B)+\cdots+p(Z) \peq .
\end{equation}%
\index{probabilit\`a!totale (teorema della)|)}

\subsection{Probabilit\`a condizionata e probabilit\`a
  composta}
La probabilit\`a che si verifichi l'evento $E$ nel caso in
cui si sa gi\`a che si \`e verificato l'evento $F$ si indica
con il simbolo $p(E|F)$ e si chiama
\emph{probabilit\`a condizionata}:%
\index{probabilit\`a!condizionata}
si ricava per essa facilmente, usando la terminologia
dell'esempio precedente, l'identit\`a
\begin{equation*}
  f(E|F) \; = \; \frac{n_{11}}{n_{11}+n_{21}}
  \; = \; \frac{n_{11}}{N} \: \frac{N}{n_{11}+n_{21}}
  \; = \; \frac{f(EF)}{f(F)}
\end{equation*}
con l'analoga
\begin{equation*}
  f(F|E) \; = \; \frac{f(EF)}{f(E)} \peq ;
\end{equation*}
e vale quindi, passando al limite, la
\begin{equation}%
\index{probabilit\`a!composta (teorema della)|(}
\label{eq:3.leprco}
  p(EF) \; = \; p(F) \cdot p(E|F)
  \; = \; p(E) \cdot p(F|E) \peq .
\end{equation}

\index{statistica!indipendenza|(}%
Nel caso particolare di due eventi casuali tali che il
verificarsi o meno dell'uno non alteri la probabilit\`a di
presentarsi dell'altro, ovverosia per cui risulti $p(E|F) =
p(E)$ e $p(F|E) = p(F)$, questi si dicono tra loro
\emph{statisticamente indipendenti}\/\thinspace\footnote{Il
  concetto di indipendenza statistica tra eventi casuali fu
  definito per la prima volta nel 1718 da Abraham de Moivre%
  \index{de Moivre!Abraham}
  (purtroppo noto al grosso pubblico solo per aver
  correttamente predetto il giorno della propria morte
  servendosi di una formula matematica), nel suo libro ``The
  Doctrine of Chance''.};%
\index{statistica!indipendenza|)}
e per essi vale la seguente legge (della \emph{probabilit\`a
  composta}):
\begin{equation*}
  p(EF) = p(E) \cdot p(F) \peq .
\end{equation*}

Questa si generalizza facilmente (sempre per induzione
completa) ad un evento complesso costituito dal verificarsi
contemporaneo di un numero qualsiasi di eventi semplici
(sempre per\`o tutti statisticamente indipendenti tra loro);
per il quale vale la
\begin{equation} \label{eq:3.procom}
  p(A \cdot B\cdots Z) = p(A) \cdot p(B)\cdots p(Z) \peq .
\end{equation}%
\index{probabilit\`a!composta (teorema della)|)}

\index{statistica!indipendenza|(}%
Pi\`u in particolare, gli eventi casuali appartenenti ad un
insieme di dimensione $N$ (con $N>2$) si dicono \emph{tutti}
statisticamente indipendenti tra loro quando la
probabilit\`a del verificarsi di uno qualsiasi di essi non
\`e alterata dal fatto che uno \emph{o pi\`u d'uno} degli
altri si sia gi\`a presentato.

Come esempio si consideri il lancio indipendente di due
dadi, ed i seguenti tre eventi casuali: $A$, consistente
nell'uscita di un numero dispari sul primo dado; $B$,
consistente nell'uscita di un numero dispari sul secondo
dado; e $C$, consistente nell'uscita di un punteggio
complessivo dispari.  \`E facile vedere che questi eventi
casuali sono, se considerati a due a due, statisticamente
indipendenti: $A$ e $B$ per ipotesi, $A$ e $C$ perch\'e
$p(C|A) = \frac{1}{2} = p(C)$, ed infine $B$ e $C$ perch\'e
anche $p(C|B) = \frac{1}{2} = p(C)$; ma gli stessi tre
eventi, se vengono considerati nel loro complesso,
\emph{non} sono \emph{tutti} statisticamente indipendenti
--- perch\'e il verificarsi di $A$ assieme a $B$ rende poi
impossibile il verificarsi di $C$.%
\index{statistica!indipendenza|)}

\subsection{Il teorema di Bayes}%
\index{Bayes, teorema di|(}
Supponiamo che un dato fenomeno casuale $A$ possa dare luogo
a $N$ eventualit\`a mutuamente esclusive $A_j$, che
esauriscano inoltre la totalit\`a delle possibilit\`a; e sia
poi un differente fenomeno casuale che possa condurre o al
verificarsi o al non verificarsi di un evento $E$.
Osservando la realizzazione di entrambi questi fenomeni, se
$E$ si verifica, assieme ad esso si dovr\`a verificare anche
una ed una sola delle eventualit\`a $A_j$; applicando prima
la legge della probabilit\`a totale \eqref{eq:3.protot} e
poi l'equazione \eqref{eq:3.leprco}, si ottiene
\begin{equation} \label{eq:3.peaj}
  p(E) \; = \; \sum_{j=1}^N p(E \cdot A_j) \; = \;
    \sum_{j=1}^N p(A_j) \cdot p(E | A_j) \peq .
\end{equation}

Ora, riprendendo la legge fondamentale delle probabilit\`a
condizionate \eqref{eq:3.leprco}, ne ricaviamo
\begin{equation*}
  p(A_i | E) = \frac{p(A_i) \cdot p(E | A_i)} { p(E) }
\end{equation*}
e, sostituendovi la \eqref{eq:3.peaj}, si giunge alla
\begin{equation} \label{eq:3.tbayes}
  \boxed{ \rule[-6mm]{0mm}{14mm} \quad
    p(A_i | E) = \frac{p(A_i) \cdot p(E | A_i)}
    {\sum_j \left[ p(A_j) \cdot p(E | A_j) \right]}
    \quad }
\end{equation}
L'equazione \eqref{eq:3.tbayes} \`e nota con il nome di
\emph{teorema di Bayes}, e viene spesso usata nel calcolo
delle probabilit\`a; talvolta anche, come adesso vedremo,
quando le $A_j$ non siano tanto eventi casuali in senso
stretto, quanto piuttosto \emph{ipotesi} da discutere per
capire se esse siano o meno rispondenti alla realt\`a.

Facendo un esempio concreto, si abbiano due monete: una
``buona'', che presenti come risultato la testa e la croce
con uguale probabilit\`a (dunque pari a $0.5$); ed una
``cattiva'', con due teste sulle due facce.  Inizialmente si
sceglie una delle due monete; quindi avremo due
eventualit\`a mutuamente esclusive: $A_1$ (\`e stata scelta
la moneta ``buona'') e $A_2$ (\`e stata scelta la moneta
``cattiva'') con probabilit\`a rispettive $p(A_1) = p(A_2) =
0.5$.  Se l'evento casuale $E$ consiste nell'uscita di una
testa, ovviamente $p(E|A_1) = 0.5$ e $P(E|A_2) = 1$.

Se ora facciamo un esperimento, lanciando la moneta una
volta e ottenendo una testa, quale \`e la probabilit\`a che
nell'effettuare la scelta iniziale si sia presa quella
``buona''?  La risposta \`e data dal teorema di Bayes, da
cui si ottiene:
\begin{align*}
  p(A_1|E) &= \frac{p(A_1) \cdot p(E|A_1)}{p(A_1)
    \cdot p(E|A_1) + p(A_2) \cdot p(E|A_2)} \\[1ex]
  &= \frac{0.5 \cdot 0.5}{0.5 \cdot 0.5 + 0.5 \cdot
    1} \\[1ex]
  &= \frac{0.25}{0.75} \\[1ex]
  &= \frac{1}{3} \peq .
\end{align*}

Ovviamente, se si volesse progettare un esperimento reale,
sarebbe meglio associarlo al lanciare la moneta $N$ volte
(con $N > 1$): o si ottiene almeno una croce, ed allora \`e
sicuramente vera $A_1$; o, invece, si presenta l'evento $E$
consistente nell'ottenere $N$ teste in $N$ lanci.  In
quest'ultimo caso, $p(E|A_2) = 1$ e $p(E|A_1) = 1/2^N$ se i
lanci sono indipendenti tra loro; utilizzando ancora
l'equazione \eqref{eq:3.tbayes}, si ricava che la
probabilit\`a di aver scelto la moneta ``buona'', $p(A_1)$,
\`e data da $1/(1+2^N)$ --- e di conseguenza $p(A_2) =
2^N/(1+2^N)$ \`e la probabilit\`a che si sia scelta la
moneta ``cattiva''.

Qui il teorema di Bayes viene utilizzato per
\emph{verificare una ipotesi statistica}: ovvero per
calcolare la probabilit\`a che l'una o l'altra di un insieme
di condizioni $A_j$ che si escludono a vicenda sia vera,
sulla base di osservazioni sperimentali riassunte dal
verificarsi di $E$; ma questo ci risulta possibile solo
perch\'e si conoscono \emph{a priori} le probabilit\`a di
\emph{tutte} le condizioni stesse $p(A_j)$.

Se, viceversa, queste non sono note, la \eqref{eq:3.tbayes}
ci d\`a ancora la probabilit\`a che sia vera l'una o l'altra
delle ipotesi $A_j$ se sappiamo che si \`e verificata la
condizione sperimentale $E$; ma essa non si pu\`o ovviamente
calcolare, a meno di fare opportune ipotesi sui valori delle
$p(A_j)$: ad esempio assumendole tutte uguali, il che \`e
chiaramente arbitrario.  Per essere pi\`u specifici, non
potremmo servirci di un esperimento analogo a quelli
delineati e del teorema di Bayes per calcolare la
probabilit\`a che una particolare moneta da 1 euro ricevuta
in resto sia o non sia ``buona'': a meno di non conoscere a
priori $p(A_1)$ e $p(A_2)$, le probabilit\`a che una moneta
da 1 euro scelta a caso tra tutte quelle circolanti nella
nostra zona sia ``buona'' o ``cattiva''.%
\index{Bayes, teorema di|)}

\section{Definizione assiomatica della probabilit\`a}%
\index{probabilit\`a!definizione!assiomatica|(}
\label{ch:3.deaspr}
Per completezza, accenniamo infine alla cosiddetta
\emph{definizione assiomatica della
  probabilit\`a}\thinspace\footnote{Questa definizione \`e
  dovuta all'eminente matematico russo Andrei Nikolaevich
  Kolmogorov%
  \index{Kolmogorov, Andrei Nikolaevich};
  vissuto dal 1903 al 1987, si occup\`o principalmente di
  statistica e di topologia.  Fu enunciata nel suo libro del
  1933 \textit{Grundbegriffe der
    Wahrscheinlichkeitsrechnung}.}, che \`e matematicamente
consistente:
\begin{quote}
  \textit{Sia $\mathcal{S}$ l'insieme di tutti i possibili
    risultati di un fenomeno casuale, ed $E$ un qualsiasi
    evento casuale definito su $\mathcal{S}$ (ossia un
    qualsiasi sottoinsieme $E \subseteq \mathcal{S}$).  Si
    definisce come ``probabilit\`a'' di $E$ un numero,
    $p(E)$, associato univocamente all'evento stesso, che
    soddisfi alle seguenti tre propriet\`a:}
  \begin{enumerate}
  \item\label{def:3.dap1} $p(E) \ge 0$ \textit{per ogni}
    $E$;
  \item\label{def:3.dap2} $p(\mathcal{S}) = 1$;
  \item\label{def:3.dap3} \itshape $p(E_1 \cup E_2 \cup
    \cdots) = p(E_1) + p(E_2) +\cdots$ per qualsiasi insieme
    di eventi $E_1, E_2,\ldots$, in numero finito od
    infinito e a due a due senza alcun elemento in comune
    (ossia tali che $E_i \cap E_j = \emptyset$ per ogni $i
    \neq j$).
  \end{enumerate}
\end{quote}%
\index{probabilit\`a!definizione!assiomatica|)}

Questa definizione, pur matematicamente
consistente\/\footnote{Volendo essere del tutto rigorosi,
  questa definizione risulta valida solo se l'insieme dei
  possibili risultati \`e composto da un numero finito o da
  un'infinit\`a numerabile di elementi; la reale definizione
  assiomatica della probabilit\`a \`e leggermente differente
  (ed ancora pi\`u astratta).}, non dice nulla su come
assegnare dei valori alla probabilit\`a; tuttavia su tali
valori si possono fare delle ipotesi, verificabili poi
analizzando gli eventi reali osservati.

\subsection{Le leggi della probabilit\`a e la
  definizione assiomatica}
\label{ch:3.lpdeas}
Dalla definizione assiomatica \`e possibile ricavare, come
abbiamo gi\`a prima accennato, le stesse leggi cui siamo
giunti a partire dalla definizione empirica.  Infatti:
\begin{itemize}
\item Essendo $\mathcal{S} \cup \emptyset = \mathcal{S}$, la
  propriet\`a \ref{def:3.dap3} (applicabile perch\'e
  $\mathcal{S} \cap \emptyset = \emptyset$) implica
  $p(\mathcal{S}) + p(\emptyset) = p(\mathcal{S})$; da cui
  ricaviamo, vista la propriet\`a \ref{def:3.dap2},
  \begin{equation*}
    p(\emptyset) = 0 \peq .
  \end{equation*}
\item Se $A \supset B$, essendo in questo caso $A = B \cup
  \left( A \cap \ob{B} \right)$, applicando la propriet\`a
  \ref{def:3.dap3} (il che \`e lecito dato che $B \cap
  \left( A \cap \ob{B} \right) = \emptyset$) si ottiene
  $p(A) = p(B) + p \left( A \cap \ob{B} \right)$; e, vista
  la propriet\`a \ref{def:3.dap1},
  \begin{equation*}
    A \supset B \quad \Rightarrow \quad p(A) \geq p(B) \peq .
  \end{equation*}
\item Dati due insiemi $A$ e $B$, visto che qualunque essi
  siano valgono le seguenti identit\`a:
  \begin{align*}
    A &= (A \cap B) \cup \left( A \cap \ob{B} \right) \\
    B &= (A \cap B) \cup \left( \ob{A} \cap B \right) \\
    (A \cup B) &= (A \cap B) \cup \left( A \cap \ob{B} \right)
      \cup \left( \ob{A} \cap B \right) \\
  \end{align*}
  e applicando a queste tre relazioni (dopo aver verificato
  che gli insiemi a secondo membro sono tutti disgiunti) la
  propriet\`a \ref{def:3.dap3} e sommando e sottraendo
  opportunamente i risultati, si ottiene la \emph{legge
    della probabilit\`a totale} nella sua forma pi\`u
  generale:
  \begin{equation*}%
  \index{probabilit\`a!totale (teorema della)}
    p(A \cup B) = p(A) + p(B) - p(A \cap B) \peq .
  \end{equation*}
\end{itemize}

Definendo poi $p(E|A)$ (con $p(A) \ne 0$) come
\begin{equation}%
\label{eq:3.procas}%
\index{probabilit\`a!condizionata}
  p(E|A) = \frac{p(E \cap A)}{p(A)} \peq ,
\end{equation}
\`e facile riconoscere che anche essa rappresenta una
probabilit\`a: essendo $p(E \cap A) \geq 0$ e $p(A) > 0$,
$p(E|A)$ soddisfa alla propriet\`a \ref{def:3.dap1}; essendo
$\mathcal{S} \cap A = A$, $p(\mathcal{S}|A) = p(A)/p(A) =
1$, e $p(E|A)$ soddisfa alla propriet\`a \ref{def:3.dap2};
infine, se $E_1, E_2,\ldots$ sono insiemi a due a due
disgiunti,
\begin{align*}
  p(E_1 \cup E_2 \cup\cdots|A) &= \frac{p[(E_1 \cup
    E_2 \cup\cdots) \cap A]}{p(A)} \\[1ex]
  &= \frac{p[(E_1 \cap A) \cup (E_2 \cap A)
    \cup\cdots]}{p(A)} \\[1ex]
  &= \frac{p(E_1 \cap A)}{p(A)} + \frac{p(E_2 \cap A)}{
    p(A) } +\cdots \\[1.5ex]
  &= p(E_1|A) + p(E_2|A) +\cdots
\end{align*}
e $p(E|A)$ soddisfa anche alla propriet\`a \ref{def:3.dap3}.
Dalla \eqref{eq:3.procas} si ottiene infine la \emph{legge
  della probabilit\`a composta} nella sua forma pi\`u
generale,
\begin{equation*}%
\index{probabilit\`a!composta (teorema della)}
  p(A \cap B) \; = \; p(A|B) \cdot p(B) \; = \; p(B|A)
    \cdot  p(A) \peq .
\end{equation*}

\section{La convergenza statistica}%
\index{statistica!convergenza|(}%
\index{limite debole|see{statistica, convergenza}}%
\label{ch:3.convstat}
Difetto della definizione empirica di probabilit\`a, oltre a
quello di essere basata su di un esperimento, \`e quello di
presupporre a priori una convergenza della frequenza
relativa $f$, al crescere di $N$, verso un valore ben
definito: valore che si assume poi come probabilit\`a
dell'evento.

\index{grandi numeri, legge dei|(}%
Qualora si assuma come definizione di probabilit\`a quella
assiomatica, \`e effettivamente possibile dimostrare (come
vedremo pi\`u avanti nel paragrafo \ref{ch:5.granum}, ed in
particolare nel sottoparagrafo \ref{ch:5.teober}) come, al
crescere del numero di prove, la frequenza relativa di un
\emph{qualunque} evento casuale converga verso la
probabilit\`a dell'evento stesso.

\`E tuttavia assai importante sottolineare come questa legge
(\emph{legge dei grandi numeri}, o \emph{teorema di
  Bernoulli}) non implichi una convergenza esatta nel senso
dell'analisi: non implichi cio\`e che, scelto un qualunque
numero positivo $\epsilon$, sia possibile determinare in
conseguenza un intero $M$ tale che, se si effettuano $N$
prove, per ogni $N>M$ risulti \emph{sicuramente} $|f(E) -
p(E)| < \epsilon$.  Si pensi in proposito alla chiara
impossibilit\`a di fissare un numero $M$ tale che, quando si
lanci un dado pi\`u di $M$ volte, si sia \emph{certi} di
ottenere almeno un sei: al crescere di $M$ crescer\`a la
\emph{probabilit\`a} del verificarsi di questo evento, ma
non si potr\`a mai raggiungere la certezza.

Nella legge dei grandi numeri il concetto di convergenza va
inteso invece in senso \emph{statistico} (o \emph{debole}, o
\emph{stocastico}); si dice che all'aumentare del numero di
prove $N$ una grandezza $x$ tende statisticamente al limite
$X$ quando, scelta una qualsiasi coppia di numeri positivi
$\epsilon$ e $\delta$, si pu\`o in conseguenza determinare
un numero intero $M$ tale che, se si effettua un numero di
prove $N$ maggiore di $M$, la probabilit\`a che $x$
differisca da $X$ per pi\`u di $\epsilon$ risulti minore di
$\delta$.  Indicando col simbolo $\Pr(E)$ la probabilit\`a
di un evento $E$, la definizione di convergenza statistica
\`e
\begin{equation} \label{eq:3.limsta}
  \forall \epsilon, \delta > 0 \quad \rightarrow
  \quad \exists M : \quad N > M \; \Rightarrow \;
  \Pr \Bigl( | x - X | \geq \epsilon \Bigr) \leq
  \delta \peq .
\end{equation}

Nel paragrafo \ref{ch:5.granum} vedremo che, dato un
qualunque evento casuale $E$ avente probabilit\`a $\Pr(E)$
di manifestarsi, si pu\`o dimostrare che la sua frequenza
relativa $f(E)$ su $N$ prove converge statisticamente a
$\Pr(E)$ all'aumentare di $N$; o, in altre parole, come
aumentando il numero di prove si possa rendere tanto
improbabile quanto si vuole che la frequenza relativa e la
probabilit\`a di un qualunque evento casuale $E$
differiscano pi\`u di una quantit\`a
prefissata.%
\index{grandi numeri, legge dei|)}%
\index{statistica!convergenza|)}

\endinput

% $Id: chapter4.tex,v 1.1 2005/03/01 10:06:08 loreti Exp $

\chapter{Elaborazione dei dati}
In questo capitolo si discute dell'organizzazione da dare ai
dati sperimentali, e su come si possano da essi ricavare
quantit\`a significative.

\section{Istogrammi}%
\index{istogrammi|(emidx}
Una volta che si disponga di un insieme di pi\`u misure
della stessa grandezza fisica (nella statistica si parla in
genere di un \emph{campione} di misure), \`e opportuno
cercare di organizzarle in modo che il loro significato
risulti a colpo d'occhio evidente; la maniera pi\`u consueta
di rappresentare graficamente le misure \`e quella di
disporle in un \emph{istogramma}.

Essendovi una corrispondenza biunivoca tra i numeri reali ed
i punti di una retta orientata, ognuna delle nostre misure
pu\`o essere rappresentata su di essa da un punto;
l'istogramma \`e un particolare tipo di diagramma cartesiano
in cui l'asse delle ascisse \`e dedicato a tale
rappresentazione.  Tuttavia \`e facile rendersi conto del
fatto che non tutti i valori della variabile sono in
realt\`a permessi, perch\'e gli strumenti forniscono per
loro natura un insieme discreto di valori essendo limitati
ad un numero finito di cifre significative.

Conviene allora mettere in evidenza sull'asse delle ascisse
tutti i possibili valori che possono essere ottenuti da una
misura reale; cio\`e punti separati da un intervallo che
corrisponde alla cifra significativa pi\`u bassa dello
strumento, o comunque alla pi\`u piccola differenza
apprezzabile con esso se l'ultima cifra deve essere stimata
dall'osservatore (ad esempio il decimo di grado stimato ad
occhio su un goniometro avente scala al mezzo grado).

Nelle ordinate del diagramma si rappresenta poi la frequenza
assoluta con la quale i diversi valori si sono presentati;
questo si fa associando ad ognuna delle misure un rettangolo
avente area unitaria, che viene riportato con la base al di
sopra dell'intervallo appropriato ogni volta che uno dei
possibili valori \`e stato ottenuto.

Nel caso consueto in cui l'asse delle ascisse venga diviso
in intervalli aventi tutti la stessa ampiezza, tutti questi
rettangoli avranno ovviamente la stessa altezza: di modo che
\`e possibile, dall'altezza di una colonna di rettangoli
unitari sovrapposti, risalire al numero di dati del campione
aventi un determinato valore.
\begin{figure}[htbp]
  \vspace*{2ex}
  \begin{center} {
    \input{isto.pstex_t}
  } \end{center}
  \caption[Istogramma di un campione di misure]
  {Esempio di istogramma (100 misure ripetute della somma
    degli angoli interni di un triangolo).}
  \label{fig:4.istri}
\end{figure}

Se le frequenze assolute risultassero troppo piccole, pu\`o
essere opportuno raggruppare le misure in \emph{classi di
  frequenza};%
\index{classi di frequenza|emidx}
ciascuna classe corrispondendo ad un intervallo multiplo
opportuno del pi\`u piccolo rappresentabile discusso sopra.

Anzich\'e costruire l'istogramma riportandovi un risultato
per volta, si possono contare prima le frequenze in ciascuna
classe e disegnare sopra ognuna di esse un rettangolo avente
area corrispondente alla frequenza ivi osservata.  L'area
dell'istogramma sopra ad un qualsiasi intervallo \`e
proporzionale alla frequenza assoluta con cui si \`e
osservato un valore che cade entro di esso; uguale, se si
assume come unit\`a di misura per le aree quella del
rettangolo di altezza unitaria.  L'area totale sottesa
dall'istogramma \`e, sempre rispetto a tale unit\`a, pari al
numero di osservazioni $N$.%
\index{istogrammi|)}

\index{frequenza!cumulativa|(}%
Un'altra rappresentazione, che \`e poco usata ma vantaggiosa
perch\'e non richiede la previa (e in qualche misura
arbitraria) definizione delle classi di frequenza, \`e
quella della \emph{frequenza cumulativa}, assoluta o
relativa.%
\label{def:4.frcure}
Essa \`e definita, per ogni valore dell'ascissa $x$, dal
numero (assoluto o relativo) di volte per cui il risultato
della misura \`e stato minore o uguale a $x$: si tratta
dunque di una funzione monotona non decrescente con uno
scalino pari rispettivamente ad 1 o a $1 / N$ in
corrispondenza di ognuno degli $N$ valori osservati.
Risulta inoltre
\begin{equation*}
  0 \; = \; F(-\infty) \; \le \; F(x) \; \le \; F(+\infty)
  \; = \;
  \begin{cases}
      N & \text{(ass.)} \\[4mm]
      1 & \text{(rel.)}
  \end{cases}
\end{equation*}%
\index{frequenza!cumulativa|)}
\begin{figure}[htbp]
  \vspace*{2ex}
  \begin{center} {
    \input{cumul.pstex_t}
  } \end{center}
  \caption[Frequenza cumulativa relativa di un campione
    di misure]{Frequenza cumulativa relativa per le stesse
    misure della figura \ref{fig:4.istri}.}
\end{figure}

\section{Stime di tendenza centrale}%
\index{stime!di tendenza centrale|(}%
\label{ch:4.tecen}
In presenza di $N$ valori osservati di una grandezza fisica
(che non siano tutti coincidenti), si pone il problema di
definire un algoritmo che fornisca la stima migliore del
valore vero della grandezza osservata; cio\`e di determinare
quale, tra le infinite funzioni dei dati, ha la maggiore
probabilit\`a di darci il valore vero.

Ora, se supponiamo di avere eliminato tutti gli errori
sistematici, \`e intuitivo come il valore di tale stima
debba corrispondere ad una ascissa in posizione centrale
rispetto alla distribuzione dei valori osservati; sappiamo
infatti che gli errori casuali hanno uguale probabilit\`a di
presentarsi in difetto ed in eccesso rispetto al valore vero
e, \emph{se il numero di misure \`e sufficientemente
  elevato}, ci aspettiamo (sulla base della legge dei grandi
numeri)%
\index{grandi numeri, legge dei}
che la distribuzione effettiva delle frequenze non si
discosti troppo da quella teorica delle probabilit\`a.
Dunque ci si attende che i valori osservati si
distribuiscano simmetricamente rispetto al valore vero.

\subsection{La moda}%
\index{moda|(}
Nella statistica esistono varie stime della cosiddetta
\emph{tendenza centrale} di un campione; una di queste stime
\`e il valore corrispondente al massimo della frequenza,
cio\`e il valore che si \`e presentato il maggior numero di
volte (ovvero la media dei valori \emph{contigui} che
presentassero tutti la medesima massima frequenza): tale
stima (se esiste) si chiama \emph{moda} del campione, e si
indica con il simbolo $\widehat x$.

In generale per\`o la distribuzione potrebbe non avere
massimo (distribuzioni \emph{amodali}), oppure averne pi\`u
d'uno in intervalli non contigui (distribuzioni
\emph{multimodali}); anche se questo non dovrebbe essere il
caso per le distribuzioni di misure ripetute.
\begin{figure}[htbp]
  \vspace*{2ex}
  \begin{center} {
    \input{modes.pstex_t}
  } \end{center}
  \caption[Distribuzioni unimodali, bimodali e amodali]
  {Due distribuzioni unimodali (in alto), una bimodale (in
    basso a sinistra), una senza moda (in basso a destra);
    quest'ultima distribuzione simula il campionamento a
    istanti casuali dell'elongazione di un pendolo.}
\end{figure}
Talvolta si dice che la distribuzione non ha moda anche se
il massimo esiste, ma si presenta ad uno degli estremi
dell'intervallo che contiene le misure; non essendo in tal
caso la moda, ovviamente, una stima di tendenza centrale.

Per tutti questi motivi la moda non \`e di uso molto
frequente, e non \`e opportuna in questo contesto anche
per ragioni che saranno esaminate pi\`u avanti.%
\index{moda|)}

\subsection{La mediana}%
\index{mediana|(}
Un'altra stima di tendenza centrale di uso frequente nella
statistica (anche se non nella fisica) \`e la \emph{mediana}
di un campione: indicata col simbolo $\widetilde x$, \`e
definita come quel valore che divide l'istogramma dei dati
in due parti di uguale area\/\footnote{Il valore della
  mediana di un insieme di dati, cos\`\i\ definito, dipende
  dalla scelta delle classi si frequenza; per questo motivo
  la mediana in genere non si adopera tanto per i campioni
  sperimentali di dati, quanto per le distribuzioni
  teoriche.}; in termini meno precisi, la mediana lascia un
uguale numero di dati alla propria sinistra ed alla propria
destra\/\footnote{Basta applicare le due definizioni ad un
  insieme di dati composto dai tre valori $\{ 0, 1, 1 \}$
  per rendersi conto della differenza.}.  Usando questa
forma della definizione, per trovare la mediana di un
insieme di valori tutti distinti basta disporli in ordine
crescente e prendere il valore centrale (per un numero
dispari di misure; si prende la semisomma dei due valori
centrali se le misure sono in numero pari).

Al contrario della moda, la mediana esiste sempre; nel
diagramma della frequenza cumulativa relativa \`e definita
dall'ascissa corrispondente all'ordinata del 50\%.  Si pu\`o
dimostrare anche che la mediana $\widetilde x$ \`e quel
valore di $x$ che rende minima la somma dei valori assoluti
degli scarti delle nostre misure $x_{i}$ da $x$; cio\`e tale
che
\begin{equation*}
  \min \left\{ \sum_{i=1}^N | x_i - x | \right\}
  \: = \: \sum_{i=1}^N | x_i - \widetilde x | \peq .
\end{equation*}%
\index{mediana|)}

\subsection{La media aritmetica}%
\index{media!aritmetica!come stima di tendenza centrale|(}
\label{ch:4.medari}
La stima di gran lunga pi\`u usata della tendenza centrale
di un campione \`e la \emph{media aritmetica} $\bar x$ dei
valori osservati, definita attraverso la
\begin{equation} \label{eq:4.mediar}
  \bar x = \frac{1}{N} \sum_{i=1}^N x_i \peq .
\end{equation}
Propriet\`a matematiche della media aritmetica sono le
seguenti:%
\index{media!aritmetica!propriet\`a matematiche|(}
\begin{quote}
  \textsc{Propriet\`a 1:} \textit{la somma degli scarti di
    un insieme di valori dalla loro media aritmetica \`e
    identicamente nulla.}
\end{quote}
Infatti dalla definizione risulta
\begin{align}
  \sum_{i=1}^N ( x_i - \bar x ) &=
    \sum_{i=1}^N x_i - \sum_{i=1}^N  \bar x
    \notag \\[1ex]
  &= \sum_{i=1}^N x_i - N \bar x \notag \\[1ex]
  &= N \bar x - N \bar x \notag \\
  \intertext{ed infine}
  \sum_{i=1}^N ( x_i - \bar x ) &\equiv 0 \peq
  . \label{eq:4.mprop1}
\end{align}

\begin{quote}
  \textsc{Propriet\`a 2:} \textit{la media aritmetica $\bar
    x$ di un insieme di dati numerici $x_1, x_2,\ldots, x_N$
    \`e quel valore di $x$ rispetto al quale risulta minima
    la somma dei quadrati degli scarti dalle $x_i$; cio\`e
    quel numero per il quale \`e verificata la}
  \begin{equation*}
    \min \left \{ \sum_{i=1}^N (x_i - x)^2 \right \}
    \; = \; \sum_{i=1}^N ( x_i - \bar x ) ^2 \peq .
  \end{equation*}
\end{quote}

Infatti abbiamo
\begin{align*}
  \sum_{i=1}^N (x_i-x) ^2 &=
    \sum_{i=1}^N \bigl[ (x_i-\bar x) +
    (\bar x-x) \bigr] ^2 \\[1ex]
  &= \sum_{i=1}^N \left[ (x_i-\bar x) ^2 +
    (\bar x - x) ^2
    + 2 (x_i-\bar x) (\bar x - x) \right]
    \\[1ex]
  &= \sum_{i=1}^N (x_i-\bar x) ^2
    \;+\; \sum_{i=1}^N (\bar x-x) ^2
    \;+\; 2(\bar x-x)\sum_{i=1}^N (x_i-\bar x) \peq ;
\end{align*}
da qui, sfruttando l'equazione \eqref{eq:4.mprop1}, si
ottiene
\begin{equation} \label{eq:4.mprop2}
  \sum_{i=1}^N (x_i-x) ^2 = \sum_{i=1}^N
    (x_i-\bar x)^2 \;+\; N(\bar x - x)^2
\end{equation}
e finalmente
\begin{equation*}
  \sum_{i=1}^N (x_i-x) ^2 \ge \sum_{i=1}^N
    (x_i-\bar x)^2 \peq .
\end{equation*}%
\index{media!aritmetica!propriet\`a matematiche|)}%
\index{media!aritmetica!come stima di tendenza centrale|)}%
\index{stime!di tendenza centrale|)}

\subsection{Considerazioni complessive}
Oltre le tre stime citate di tendenza centrale ne esistono
altre, di uso per\`o limitato a casi particolari e che non
hanno frequente uso n\'e nella statistica n\'e nella fisica;
per soli motivi di completezza citiamo qui:
\begin{itemize}
\item la \emph{media geometrica},%
  \index{media!geometrica}
  $g$, definita come la radice $N$-esima del prodotto degli
  $N$ valori rappresentati nel campione:
  \begin{equation*}
    g^N = \prod_{i=1}^N x_i \peq ;
  \end{equation*}
\item la \emph{media armonica},%
  \index{media!armonica}
  $h$, definita come il reciproco del valore medio dei
  reciproci dei dati:
  \begin{equation*}
    \frac{1}{h} = \frac{1}{N} \sum_{i=1}^N \frac{1}{x_i} \peq
    ;
  \end{equation*}
\item la \emph{media quadratica},%
  \index{media!quadratica}
  $q$, definita come la radice quadrata del valore medio dei
  quadrati dei dati:
  \begin{equation*}
    q = \sqrt{ \frac{1}{N} \sum_{i=1}^N {x_i}^2 } \peq .
  \end{equation*}
\end{itemize}

Se la distribuzione dei dati non \`e troppo irregolare, le
prime tre stime citate per la tendenza centrale (moda,
mediana e media aritmetica) non sono molto lontane; esiste
una relazione empirica che le lega e che \`e valida per
distribuzioni non troppo asimmetriche:
\begin{equation*}
  \left| \bar x - \widehat x \right| \; \approx \; 3
  \left| \bar x - \widetilde x \right| \peq ,
\end{equation*}
cio\`e la differenza tra media aritmetica e moda \`e circa
il triplo della differenza tra media aritmetica e mediana.
\begin{figure}[htbp]
  \vspace*{2ex}
  \begin{center} {
    \input{maxbol.pstex_t}
  } \end{center}
  \caption[La distribuzione di Maxwell--Boltzmann]
  {I valori delle tre principali stime di tendenza centrale
    per la distribuzione di Maxwell--Boltzmann; l'unit\`a
    per le ascisse \`e il parametro $\alpha$ che compare
    nell'equazione \eqref{eq:4.maxbol}.}
  \label{fig:4.maxbol}
\end{figure}

Come esempio, nella figura \ref{fig:4.maxbol} \`e mostrato
l'andamento di una distribuzione di probabilit\`a per una
variabile (continua) che \`e di interesse per la fisica; e
nel grafico sono messi in evidenza i valori per essa assunti
dalle tre stime di tendenza centrale considerate.  Si tratta
della funzione di frequenza detta di
\emph{Maxwell--Boltzmann},%
\index{distribuzione!di Maxwell--Boltzmann}
e secondo essa sono ad esempio distribuiti, in un gas
perfetto, i moduli delle velocit\`a delle molecole:
l'equazione della curva \`e
\begin{equation} \label{eq:4.maxbol}
  y \; = \; f(v) \; = \; \frac{4}{\sqrt{\pi}} \:
  \alpha^\frac{3}{2} \, v^2 \, e^{-\alpha \, v^2}
\end{equation}
(in cui $\alpha$ \`e una costante dipendente dalla massa
delle molecole e dalla temperatura del gas).

La scelta dell'uso dell'una o dell'altra stima statistica
per determinare la tendenza centrale di un campione di
misure ripetute andr\`a decisa sulla base delle propriet\`a
delle stime stesse; pi\`u precisamente sulla base dello
studio di come si distribuiscono varie stime che si
riferiscano a campioni analoghi, cio\`e ad insiemi di misure
della stessa grandezza fisica presumibilmente affette dallo
stesso errore (eseguite  insomma in condizioni simili) e
composti da uno stesso numero di dati.  La stima che
sceglieremo dovrebbe essere la migliore, nel senso gi\`a
usato all'inizio di questo paragrafo \ref{ch:4.tecen}:
quella che ha la maggiore probabilit\`a di darci il valore
vero della grandezza misurata.

\subsection{Prima giustificazione della media}%
\index{media!aritmetica!come stima del valore vero|(} La
stima di tendenza centrale che \`e di uso generale per le
misure ripetute \`e la \emph{media aritmetica}: i motivi
sono svariati e sostanzialmente legati alle propriet\`a
statistiche della media stessa; di essi ci occuperemo ancora
pi\`u avanti.  In particolare vedremo nel paragrafo
\ref{ch:11.mepeted} che la media aritmetica \`e
effettivamente la stima \emph{migliore}, nel senso or ora
chiarito di questa frase.

A questo punto possiamo gi\`a comunque renderci conto (anche
se in maniera \emph{non rigorosa}) che la media aritmetica
di pi\`u misure dovrebbe avere un errore inferiore a quello
delle misure stesse; indichiamo con $x^*$ il valore vero
della grandezza $x$, e con $x_i$ ($i = 1, 2, \ldots, N$) le
$N$ determinazioni sperimentali di $x$: l'errore assoluto
commesso in ognuna delle misure $x_i$ sar\`a dato da
$\epsilon_i = x_i - x^*$.  L'errore assoluto della media
aritmetica \`e allora dato da
\begin{equation*}
  \bar \epsilon = \bar x - x^*
\end{equation*}
e, sfruttando la \eqref{eq:4.mediar},
\begin{equation*}
  \bar \epsilon \; = \; \frac{1}{N}\sum_{i=1}^N x_i \; - \;
  x^* \; = \; \frac{1}{N}\sum_{i=1}^N \left( x_i-x^* \right)
  \; = \; \frac{1}{N}\sum_{i=1}^N \epsilon_i \peq .
\end{equation*}

Se gli errori sono solo casuali, saranno ugualmente
probabili in difetto e in eccesso rispetto al valore vero; e
se le misure sono numerose gli $\epsilon_{i}$ tenderanno
quindi ad eliminarsi a vicenda nella sommatoria, che inoltre
\`e moltiplicata per un fattore
$1/N$.%
\index{media!aritmetica!come stima del valore vero|)}

\subsection{La media aritmetica espressa tramite le
  frequenze}
\label{ch:4.medpes}
Siano $x_i$, con $i = 1, \ldots, N$, gli $N$ valori del
campione di cui vogliamo calcolare la media aritmetica;
supponiamo che qualcuno dei valori ottenuti sia
\emph{ripetuto}, ad esempio che il valore $x_1$ si sia
presentato $n_1$ volte, $x_2$ si sia presentato $n_2$ volte
e cos\`\i\ via: la media aritmetica si pu\`o calcolare
come
\begin{align*}
  \bar x &= \frac{n_1 x_1 + n_2 x_2 +\cdots}{N}
  & (N &= n_1 + n_2 +\cdots) \peq .
\end{align*}

Indichiamo con $x_j$ ($j = 1, 2, \ldots, M$) gli $M$ valori
distinti di $x$ presenti nel campione; $n_j$ \`e la
frequenza assoluta con cui abbiamo ottenuto il valore $x_j$
nel corso delle nostre misure, ed il rapporto $n_j / N$ \`e
la frequenza relativa $f_j$ dello stesso evento casuale:
allora possiamo scrivere
\begin{equation*}
  \bar x \; = \; \frac{1}{N} \sum_{i=1}^N x_i \; = \;
  \frac{1}{N} \sum_{j=1}^M n_j x_j \; = \;
  \sum_{j=1}^M \frac{n_j}{N} \, x_j \; = \;
  \sum_{j=1}^M f_j x_j \peq .
\end{equation*}

\index{media!pesata|(}%
Formule in cui si sommano valori numerici (qui gli $x_j$)
moltiplicati ciascuno per un fattore specifico ($f_j$) vanno
sotto il nome generico di \emph{formule di media pesata}:
ogni valore distinto dagli altri contribuisce infatti al
risultato finale con un \emph{peso} relativo dato dal numero
$f_j$.

\`E bene osservare come si possano definire infinite medie
pesate dei valori numerici $x_j$, corrispondenti alle
infinite differenti maniere di attribuire ad ognuno di essi
un peso; ed anche che, in genere, con il nome di ``media
pesata'' ci si riferisce a quella particolare formula che
permette di calcolare la migliore stima del valore vero di
una grandezza fisica sulla base di pi\`u misure aventi
differente precisione (l'equazione \eqref{eq:11.medpes}, che
incontreremo pi\`u avanti nel paragrafo
\ref{ch:11.mepeted}), e non alla formula precedente.%
\index{media!pesata|)}

\index{media!aritmetica!propriet\`a matematiche|(}%
Fin qui tale formula si presenta solo come un artificio per
calcolare la media aritmetica di un insieme di valori
risparmiando alcune operazioni; ma pensiamo di far tendere
all'infinito il numero di misure effettuate.  In tal caso,
se assumiamo che la frequenza relativa con cui ogni valore
si \`e presentato tenda stocasticamente alla probabilit\`a
rispettiva, in definitiva otteniamo che la media aritmetica
delle misure deve anch'essa tendere ad un limite
determinato:
\begin{equation*}%
\index{media!aritmetica!come stima del valore vero}
  \lim_{N \rightarrow \infty} \bar x \; = \;
  \sum \nolimits_j p_j x_j \peq .
\end{equation*}

In definitiva, se siamo in grado di assegnare in qualche
modo una probabilit\`a al presentarsi di ognuno dei
possibili valori di una misura, siamo anche in grado di
calcolare il valore assunto dalla media aritmetica di un
campione di quei valori nel limite di infinite misure
effettuate.%
\index{media!aritmetica!propriet\`a matematiche|)}
Di questa formula ci serviremo pi\`u avanti, una volta
ricavata appunto (sotto opportune ipotesi) la probabilit\`a
di ottenere un certo risultato dalle misure di una grandezza
fisica.

\section{Stime di dispersione}%
\index{stime!di dispersione|(}
Abbiamo sviluppato il paragrafo \ref{ch:4.tecen} partendo
dall'intuizione (giustificata con l'aiuto delle
caratteristiche degli errori casuali e della legge dei
grandi numeri) che la tendenza centrale di un insieme di
misure \`e legata al valore vero della grandezza misurata.
Cos\`\i, similmente, si intuisce che agli errori introdotti
nell'eseguire le nostre misure \`e legata un'altra grandezza
caratteristica del campione, cio\`e la sua
\emph{dispersione}: ovvero la valutazione della larghezza
dell'intervallo in $x$ in cui le misure stesse sono
distribuite attorno al valore centrale.

\subsection{Semidispersione massima e quantili}%
\index{semidispersione massima|(}%
\index{dispersione massima|see{semidispersione massima}}
La pi\`u grossolana delle stime statistiche di dispersione
si effettua trovando il massimo ed il minimo valore
osservato: la \emph{semidispersione massima} \`e definita
come la semidifferenza tra questi due valori,
\begin{equation*}
  \frac{x_{\max} - x_{\min}}{2} \peq .
\end{equation*}

Essa ha il difetto di ignorare la maggior parte dei dati e
particolarmente quelli, generalmente preponderanti, prossimi
al centro della distribuzione; inoltre normalmente aumenta
all'aumentare del numero di misure, invece di tendere ad un
valore determinato.  Il doppio della semidispersione massima
\begin{equation*}
  R = x_{\max} - x_{\min}
\end{equation*}
\`e anch'esso usato come stima della dispersione di un
campione, e viene chiamato \emph{range}.%
\index{range}%
\index{semidispersione massima|)}

\index{quantili|(}%
\index{quartili|see{quantili}}%
\index{decili|see{quantili}}%
\index{percentili|see{quantili}}%
Grandezze frequentemente usate per caratterizzare una
distribuzione nella statistica (non nella fisica) sono i
\emph{quartili}, i \emph{decili} ed i \emph{percentili}
(collettivamente \emph{quantili}), indicati con $Q_i$ ($i=1,
2, 3$); con $D_i$ ($i=1, \ldots, 9$); e con $P_i$
($i=1,\ldots, 99$) rispettivamente.  Essi sono definiti
(analogamente alla mediana) come quei valori della $x$ che
dividono la distribuzione rispettivamente in 4, 10 e 100
parti di uguale area; ovviamente vale la
\begin{equation*}
  Q_2 \: \equiv \: D_5 \: \equiv \: P_{50}
  \: \equiv \: \widetilde x \peq .
\end{equation*}

Come stima della dispersione di una distribuzione \`e usato
dagli statistici l'\emph{intervallo semiinterquartilico} $Q
= (Q_3 - Q_1) / 2$, come pure la differenza $P_{90} -
P_{10}$ tra il novantesimo ed il decimo percentile; tali
intervalli esistono sempre, ma non sono padroneggiabili
agevolmente negli sviluppi teorici.%
\index{quantili|)}

\subsection{Deviazione media assoluta (errore medio)}%
\index{errore!medio|(}%
\index{deviazione media assoluta|see{errore medio}}
Altra stima di dispersione \`e la \emph{deviazione media
  assoluta} (o \emph{errore medio}), definita come
\begin{equation*}
  \overline{ \rule{0pt}{2.0ex} | x - \bar x | } \: =
  \: \frac{1}{N}\sum_{i=1}^N \left| x_i - \bar x \right| \peq
  ,
\end{equation*}
oppure, meno frequentemente, come
\begin{equation*}
  \overline{ \rule{0pt}{2.0ex} | x - \widetilde x | } \: = \:
  \frac{1}{N}\sum_{i=1}^N \left| x_i - \widetilde x \right|
  \peq ;
\end{equation*}
ma anch'essa non \`e facile da trattare a ragione della
operazione non lineare costituita dal valore assoluto.%
\index{errore!medio|)}

\subsection{Varianza e deviazione standard}%
\index{varianza|(}%
\index{errore!quadratico medio|(}
La pi\`u importante (e pi\`u usata, non solo in fisica)
stima di dispersione \`e la \emph{deviazione standard}
(oppure \emph{scarto} o \emph{deviazione quadratica media})
$s$; che si definisce come la radice quadrata della
\emph{varianza}, $s^2$:
\begin{equation*}
  s^2 = \frac{1}{N} \sum_{i=1}^N \left( x_i
    - \bar x \right) ^2 \peq .
\end{equation*}

Per distribuzioni non troppo asimmetriche la deviazione
media assoluta \`e circa i $\frac{4}{5}$ della deviazione
standard, e l'intervallo semiinterquartilico \`e circa i
$\frac{2}{3}$ della stessa.

Per calcolare lo scarto quadratico medio di un campione
senza l'aiuto di un calcolatore appositamente programmato,
pu\`o risultare utile sfruttare la sua seguente propriet\`a:

\begin{align*}%
\index{varianza!propriet\`a matematiche|(}
  N \, s^2 &= \sum_{i=1}^N ( x_i - \bar x ) ^2 \\[1ex]
  &= \sum_{i=1}^N \bigl( {x_i}^2 +
    \bar x ^2 - 2 \, \bar x \, x_i \bigr) \\[1ex]
  &= \sum_{i=1}^N {x_i}^2 \; + \; N \, \bar x ^2 \; - \;
    2 \, \bar x \sum_{i=1}^N x_i \\[1ex]
  &= \sum_{i=1}^N {x_i}^2 \; + \; N \, \bar x ^2 \; - \;
    2 \, N \, \bar x ^2 \\[1ex]
  &= \sum_{i=1}^N {x_i}^2 \; - \; N \, \bar x ^2 \\
\intertext{da cui la formula}
  s^2 &= \frac{1}{N} \sum_{i=1}^N {x_i}^2 \; - \; \bar x ^2
\end{align*}
che permette un calcolo pi\`u agevole di $s^2$ accumulando
successivamente i quadrati dei valori
osservati anzich\'e quelli dei loro scarti dalla media.%
\index{varianza!propriet\`a matematiche|)}%
\index{errore!quadratico medio|)}%
\index{varianza|)}%
\index{stime!di dispersione|)}

\section{Giustificazione della media}%
\index{media!aritmetica!come stima del valore vero|(}%
\label{ch:4.giumed}
Stabiliamo quindi per convenzione che il nostro metodo per
misurare la dispersione di un campione di dati \`e quello
del calcolo della deviazione standard; accettiamo anche che
\emph{in qualche modo} questo numero sia legato all'errore
presumibilmente commesso nel corso delle misure.  Una
definizione pi\`u precisa di ci\`o che si intende con le
parole ``errore commesso'', ovverosia l'interpretazione
probabilistica dello scarto quadratico medio nei riguardi
delle misure ripetute, verr\`a data pi\`u avanti nel
paragrafo \ref{ch:9.scanor}.

Comunque, una volta assunto questo, possiamo approfondire il
discorso gi\`a cominciato sulla media aritmetica come stima
del centro della distribuzione e quindi del valore vero di
una grandezza, nel caso di misure ripetute ed in assenza di
errori sistematici.  \`E infatti possibile
provare\/\footnote{La   dimostrazione risale a Gauss%
  \index{Gauss, Karl Friedrich}
  se ci si limita alle sole operazioni lineari sui dati, e
  solo ad anni recenti per un qualsiasi algoritmo operante
  su di essi; vedi in proposito
  l'appendice~\ref{ch:e.maxlik}.} che la media aritmetica
\`e la stima del valore vero affetta \emph{dal minimo errore
  casuale},%
\index{errore!della media|(}
cio\`e avente la pi\`u piccola deviazione standard.

Riferendosi a quanto prima accennato, ci\`o significa che le
medie aritmetiche di molti campioni analoghi di $N$ misure
avranno un istogramma pi\`u stretto delle mode, delle
mediane e di qualsiasi altra misura di tendenza centrale
desumibile dagli stessi campioni; la larghezza di tale
istogramma (misurata, come abbiamo assunto, dal suo scarto
quadratico medio) sar\`a messa in relazione con lo scarto
quadratico medio delle misure da un teorema di cui ci
occuperemo nel seguito.  Da esso discender\`a anche che
l'errore statistico della media aritmetica converge a zero
al crescere indefinito del numero di dati $N$.%
\index{errore!della media|)}
Per concludere:
\begin{quote}
  \begin{enumerate}
  \item \textit{Disponendo di pi\`u misure ripetute della
      stessa grandezza fisica, si assume come migliore stima
      del valore vero di quella grandezza la loro media
      aritmetica.}
  \item \textit{Questa stima \`e pi\`u precisa di quanto non
      lo siano le singole misure, ed \`e tanto pi\`u
      attendibile quanto maggiore \`e il numero delle
      stesse.}
  \item \textit{Come valutazione dell'errore commesso nelle
      singole misure si assume il loro scarto quadratico
      medio; o meglio, per motivi che verranno chiariti in
      seguito, la quantit\`a}\/\thinspace\footnote{La
      differenza tra questa formula e quella prima citata
      non \`e praticamente avvertibile se $N$ non \`e troppo
      piccolo.}
    \begin{equation*}
      \mu = \sqrt{ \frac{\sum \limits_{i=1}^N
          ( x_i - \bar x )^2 }{N-1} } \peq .
    \end{equation*}
  \end{enumerate}
\end{quote}%
\index{media!aritmetica!come stima del valore vero|)}

\endinput

% $Id: chapter5.tex,v 1.1 2005/03/01 10:06:08 loreti Exp $

\chapter{Variabili casuali unidimensionali discrete}%
\label{ch:5.varcun}
Gi\`a sappiamo (come osservato nel paragrafo
\ref{ch:3.varcas}) che, a causa degli inevitabili errori, la
misura di una grandezza fisica pu\`o essere considerata un
evento casuale; e che il numero reale da noi ottenuto in
conseguenza della misura stessa pu\`o essere considerato una
variabile casuale definita sull'insieme di tutti i possibili
risultati.

Un insieme finito di operazioni di misura, i cui risultati
costituiscono quello che in linguaggio statistico si dice
\emph{campione},%
\index{campione}
si pu\`o pensare come un particolare sottoinsieme formato da
elementi estratti a caso dall'insieme di tutte le infinite
possibili operazioni di misura che potrebbero essere
effettuate sulla stessa grandezza fisica, eseguite col
medesimo strumento e sfruttando le medesime procedure.

Quest'ultimo insieme nella terminologia della statistica si
dice \emph{universo} o \emph{popolazione},%
\index{popolazione}
ed \`e in effetti una finzione (si pensi all'universo di
tutti i possibili lanci di un dado nella teoria dei giochi
d'azzardo), nel senso che in realt\`a esso non \`e
un'entit\`a preesistente alle operazioni effettivamente
eseguite; a differenza dell'insieme di tutti gli individui
di una vera popolazione, dalla quale si estrae realmente un
campione per eseguire una ricerca demografica.  Sebbene sia
una finzione, questo concetto \`e tuttavia utile per poter
applicare la teoria della probabilit\`a alle caratteristiche
di un campione.

In questo capitolo esamineremo il comportamento delle
variabili casuali in generale (ed in particolare quello dei
risultati delle misure): tra le altre cose, metteremo in
evidenza i rapporti tra grandezze statistiche che si
riferiscano ad un campione limitato e grandezze analoghe che
siano invece riferite all'intera popolazione (\emph{teoria
  del campionamento}); e dimostreremo la validit\`a della
legge dei grandi numeri.

\section{Generalit\`a}%
\index{casuali!variabili|(}
Riprendiamo ora il concetto di variabile casuale gi\`a
introdotto in precedenza nel paragrafo \ref{ch:3.varcas}, e
consideriamo alcuni esempi: se si associa ad ogni faccia di
un dado un numero compreso tra 1 e 6 (il punteggio inciso
sulla faccia stessa), si definisce una variabile casuale
discreta; se l'evento casuale consiste invece nel lancio di
due monete, indicando con $E$ l'apparizione della testa nel
lancio della prima e con $F$ l'apparizione della testa nel
lancio della seconda, il numero $x$ di teste osservate
nell'evento \`e ancora una variabile casuale discreta, la
cui definizione \`e data dalla tabella seguente: \medskip
\begin{center}
  \begin{tabular}{rc}
     & $x$ \\
    \midrule
    $EF$ & 2 \\
    $E \ob{F}$ & 1 \\
    $\ob{E} F$ & 1 \\
    $\ob{E} \ob{F}$ & 0 \\
    \bottomrule
  \end{tabular}
\end{center}
\medskip e, come si pu\`o notare, la corrispondenza tra la
variabile casuale e l'insieme dei possibili risultati non
\`e in questo caso biunivoca.

Se l'insieme di definizione \`e continuo, la variabile
casuale $x(E)$ pu\`o essere continua; \`e questo il caso
pi\`u frequente nella fisica, ad esempio per le misure: ma
anche in tal caso, a causa della sensibilit\`a limitata
degli strumenti, l'intervallo continuo di definizione della
variabile $x$ viene in pratica suddiviso in un numero finito
$M$ di intervalli, che vengono rappresentati dai valori
centrali $x_j$ della variabile casuale.%
\index{casuali!variabili|)}

Detta $\nu_j$ la frequenza assoluta con cui si \`e
presentato il risultato $x_{j}$ nelle $N$ prove complessive,
sar\`a
\begin{equation*}
  \sum_{j=1}^M \nu_j = N
\end{equation*}
(potendo alcune frequenze $\nu_j$ risultare nulle perch\'e i
corrispondenti valori $x_j$ non sono stati osservati nelle
prove).  Indicata con
\begin{equation*}
  f_j = \frac{\nu_j}{N}
\end{equation*}
la frequenza relativa del valore $x_j$ nelle $N$ prove,
dalla prima relazione segue
\begin{equation*}
  \sum_{j=1}^M f_j \; = \; \sum_{j=1}^M
  \frac{\nu_j}{N} \; = \;
  \frac{1}{N} \sum_{j=1}^M \nu_j \; \equiv \; 1
\end{equation*}
esaurendo gli $M$ valori $x_j$ tutti i possibili risultati
della misura.

Se il numero delle prove $N$ \`e molto grande e viene fatto
crescere a piacere, ciascuna $f_j$ deve tendere
statisticamente al valore $p_j$ (probabilit\`a di osservare
il valore $x_j$), e sar\`a ancora
\begin{equation*}
  \sum_{j=1}^{M} p_{j} \equiv 1
\end{equation*}
come dovevamo ovviamente attenderci ricordando l'equazione
\eqref{eq:3.norpro}.

\section{Speranza matematica}%
\index{speranza matematica|(}%
\label{ch:5.medcl}
Come sappiamo dal paragrafo \ref{ch:4.medpes}, il valore
medio della variabile $x$ su di un campione finito \`e dato
dall'equazione
\begin{equation*}
  \bar x = \sum\nolimits_i f_i x_i
\end{equation*}
dove la sommatoria si intende estesa a tutti i valori che la
$x$ pu\`o assumere, essendo nulle le frequenze di quelli che
non si sono effettivamente presentati; definiamo in maniera
analoga una nuova grandezza $E(x)$, relativa all'intera
popolazione, mediante la
\begin{equation} \label{eq:5.spermat}
  E(x) = \sum\nolimits_i p_i x_i \peq .
\end{equation}
$E(x)$ (che si chiama \emph{speranza matematica} della
variabile casuale $x$) ci appare quindi come una
generalizzazione alla popolazione del concetto di media
aritmetica e, se si assumesse come definizione di
probabilit\`a quella empirica, sarebbe in base ad essa il
limite (statistico) del valore medio del campione
all'aumentare della sua dimensione; per cui lo chiameremo
anche, meno appropriatamente, \emph{valore medio di $x$
  sull'intera popolazione}.

\`E da notare come non ci sia alcuna garanzia dell'esistenza
di $E(x)$ se l'insieme dei possibili valori $x_i$ non \`e
finito (in particolare se $x$ \`e una variabile continua);
in effetti esistono delle distribuzioni di probabilit\`a
usate anche in fisica (ad esempio la \emph{distribuzione di
  Cauchy},%
\index{distribuzione!di Cauchy} che studieremo pi\`u avanti
nel paragrafo \ref{ch:8.cauchy}) per le quali la sommatoria
della \eqref{eq:5.spermat} non converge, e che non ammettono
quindi speranza matematica.%
\index{speranza matematica|)}

\index{varianza!della popolazione|(emidx}%
La speranza matematica per la variabile casuale $\bigl[ x -
E(x) \bigr]^2$ (ossia la generalizzazione alla popolazione
della varianza di un campione) si indica poi col simbolo
$\var(x)$:
\begin{equation*}
  \var(x) \; = \; E \left\{ \bigl[ x - E(x) \bigr]^2
    \right\} \; = \; \sum\nolimits_i p_i \bigl[ x_i -
    E(x) \bigr]^2 \peq ,
\end{equation*}
e ad essa ci riferiremo come \emph{varianza della
  popolazione della variabile casuale} $x$; come $E(x)$, e
per gli stessi motivi, anch'essa potrebbe non esistere per
quelle variabili che assumono un numero infinito di
possibili valori.%
\index{varianza!della popolazione|)}

Le considerazioni dei paragrafi seguenti si applicano
ovviamente solo a popolazioni di variabili casuali per le
quali esista finita la speranza matematica e, qualora la si
consideri, la varianza.  Inoltre non useremo mai la
definizione empirica di probabilit\`a, ma quella
assiomatica; e vedremo come, partendo da essa, si possa
dimostrare la legge detta ``dei grandi numeri''%
\index{grandi numeri, legge dei}
gi\`a enunciata nel paragrafo \ref{ch:3.convstat}: ossia la
convergenza, all'aumentare del numero di prove effettuate,
della frequenza di un qualsiasi evento casuale alla sua
probabilit\`a.

\section{Il valore medio delle combinazioni lineari}%
\index{speranza matematica!di combinazioni lineari|(}%
\index{combinazioni lineari!speranza matematica|(}%
\label{ch:5.vmclc}
Consideriamo due variabili casuali $x$ e $y$, aventi
speranza matematica $E(x)$ ed $E(y)$ rispettivamente; ed una
loro qualsiasi combinazione lineare a coefficienti costanti
$z = ax + by$.  Vogliamo dimostrare ora che la speranza
matematica della nuova variabile $z$ esiste, ed \`e data
dalla combinazione lineare delle speranze matematiche di $x$
e di $y$ con gli stessi coefficienti $a$ e $b$.

Indichiamo con $x_j$ i possibili valori della prima
variabile, e con $y_k$ quelli della seconda; indichiamo poi
con $p_j$ e $q_k$ le probabilit\`a di ottenere un
determinato valore rispettivamente per la $x$ e per la $y$.
Chiamiamo poi $P_{jk}$ la probabilit\`a che simultaneamente
si abbia $x = x_j$ ed $y = y_k$; un particolare valore per
la $x$ potr\`a essere associato ad uno qualsiasi dei diversi
valori della $y$, che sono tra loro mutuamente esclusivi: in
definitiva, applicando la legge della probabilit\`a totale
(equazione \eqref{eq:3.protot}) risulter\`a
\begin{align*}
    p_j &= \sum \nolimits_k P_{jk} & &\text{e} &
    q_k &= \sum \nolimits_j P_{jk} \peq .
\end{align*}

Per la speranza matematica $E(z)$ di $z$ avremo poi
\begin{align*}
  E(ax + by) &= \sum \nolimits_{jk} P_{jk} \left(
    a \, x_j + b \, y_k \right) \\[1ex]
  &= \sum \nolimits_{jk} a \, P_{jk} \, x_j \; + \;
    \sum \nolimits_{jk} b \, P_{jk} \, y_k
    \\[1ex]
  &= a \sum \nolimits_j \left( \sum \nolimits_k
    P_{jk} \right) x_j \; + \; b \sum \nolimits_k
    \left( \sum \nolimits_j P_{jk} \right) y_k
    \\[1ex]
  &= a \sum \nolimits_j p_j \, x_j \; + \;
    b \sum \nolimits_k q_k \, y_k \notag \\[1ex]
  &= a \, E(x) + b \, E(y) \peq .
\end{align*}

\`E immediato poi estendere, per induzione completa, questa
dimostrazione alla combinazione lineare di un numero
qualsiasi di variabili casuali: se abbiamo
\begin{equation*}
  F = a x + b y + c z +\cdots
\end{equation*}
allora
\begin{equation} \label{eq:5.medcol}
  E(F) = a \, E(x) + b \, E(y) + c \, E(z) + \cdots \peq .
\end{equation}%
\index{combinazioni lineari!speranza matematica|)}%
\index{speranza matematica!di combinazioni lineari|)}

\index{speranza matematica!della media aritmetica|(}%
Una importante conseguenza pu\`o subito essere ricavata
applicando l'equazione \eqref{eq:5.medcol} alla media
aritmetica $\bar x$ di un campione di $N$ misure: essa
infatti si pu\`o considerare come una particolare
combinazione lineare delle misure stesse, con coefficienti
tutti uguali tra loro e pari ad $1/N$.

Prendendo dalla popolazione un differente campione di $N$
misure, la loro media aritmetica $\bar x$ sar\`a anch'essa
in generale diversa: quale sar\`a la speranza matematica di
$\bar x$, ovverosia il valore medio delle varie $\bar x$ su
un numero molto elevato di campioni di $N$ misure estratti a
caso dalla popolazione --- e, al limite, su tutti i campioni
(aventi la stessa dimensione fissa $N$) che dalla
popolazione \`e possibile ricavare?
\begin{align}
  E \left( \bar x \right) &= E \left( \frac{1}{N}
    \sum_{i=1}^N x_i \right) \notag \\[1ex]
  &= \frac{1}{N} \sum_{i=1}^N E \left( x_i \right)
    \notag \\[1ex]
  &= \frac{1}{N} \cdot N \, E(x) \notag \\
\intertext{ed infine}
  E \left( \bar x \right) &= E(x) \label{eq:5.ebarx}
\end{align}
cio\`e:
\begin{quote}
  \textit{Il valore medio della popolazione delle medie
    aritmetiche dei campioni di dimensione finita $N$
    estratti da una popolazione coincide con il valore medio
    della popolazione stessa}.
\end{quote}%
\index{speranza matematica!della media aritmetica|)}

\section{La varianza delle combinazioni lineari}%
\index{varianza!di combinazioni lineari!di variabili indipendenti|(}%
\index{combinazioni lineari!varianza!di variabili indipendenti|(}
Dimostriamo ora un altro teorema generale che riguarda la
varianza di una combinazione lineare di pi\`u variabili
casuali, che supporremo per\`o \emph{statisticamente
  indipendenti}.  Usando gli stessi simboli gi\`a introdotti
nel paragrafo \ref{ch:5.vmclc}, e dette $x$ ed $y$ due
variabili casuali che godano di tale propriet\`a, sappiamo
dall'equazione \eqref{eq:3.procom} che la probabilit\`a
$P_{jk}$ che contemporaneamente risulti sia $x = x_j$ che $y
= y_k$ \`e data dal prodotto delle probabilit\`a rispettive
$p_j$ e $q_k$.

Per semplificare i calcoli, dimostriamo questo teorema
dapprima nel caso particolare di due popolazioni $x$ e $y$
che abbiano \emph{speranza matematica nulla}; estenderemo
poi il risultato a due variabili (sempre statisticamente
indipendenti) aventi speranza matematica qualunque.  Ci\`o
premesso, la combinazione lineare
\begin{equation*}
  z = a x + b y
\end{equation*}
ha anch'essa speranza matematica zero: infatti applicando
l'equazione \eqref{eq:5.medcol} risulta
\begin{equation*}
  E(z) \; = \; E(ax+by) \; = \; a \, E(x) +
  b \, E(y) \; = \; 0
\end{equation*}
e si pu\`o allora ricavare (indicando con i simboli
${\sigma_x}^2$, ${\sigma_y}^2$ e ${\sigma_z}^2$ le varianze
di $x$, $y$ e $z$ rispettivamente):
\begin{align*}
  {\sigma_z}^2 &= E \left \{ \bigl[ z -
      E(z) \bigr] ^2 \right \} \\[1ex]
    &= E \left\{ z^2 \right\} \\[1ex]
    &= E \left \{ \left( ax+by \right) ^2 \right
      \} \\[1ex]
    &= \sum \nolimits_{jk} P_{jk} \left( a \,
      x_j + b \, y_k \right) ^2 \\[1ex]
    &= \sum \nolimits_{jk} p_j q_k \bigl(
      a^2 {x_j}^2 + b^2 {y_k}^2 + 2 a \, b \,
      x_j \, y_k \bigr) \\[1ex]
    &= a^2 \sum \nolimits_k q_k
      \sum \nolimits_j p_j {x_j}^2 \; + \;
      b^2 \sum \nolimits_j p_j \sum \nolimits_k
      q_k {y_k}^2 \; + \; 2ab \sum \nolimits_j
      p_j x_j \sum \nolimits_k q_k y_k \\[1ex]
    &= a^2 {\sigma_x}^2 \, \sum \nolimits_k q_k
      \; + \; b^2 {\sigma_y}^2 \, \sum
      \nolimits_j p_j \; + \; 2 a b \, E(x) \,
      E(y)
\end{align*}
ed infine
\begin{equation} \label{eq:5.vcl}
  {\sigma_z}^2 \; = \; a^2 {\sigma_x}^2
    \; + \; b^2 {\sigma_y}^2 \peq .
\end{equation}

Allo scopo di estendere la validit\`a dell'equazione
\eqref{eq:5.vcl} appena dimostrata a due variabili casuali
$x$ e $y$ aventi speranza matematica anche differente da
zero, dimostriamo ora il seguente
\begin{quote}
  \textsc{Teorema:}\label{def:5.varind} \textit{due
    variabili casuali che differiscano per un fattore
    costante hanno la stessa varianza.}
\end{quote}

Infatti, se le due variabili casuali $x$ e $\xi$ soddisfano
questa ipotesi, allora deve risultare:
\begin{align*}
  \xi &= x + K \\[1ex]
  E(\xi) &= E(x) + K \\[1ex]
  {\sigma_\xi}^2 &=
    E \left \{ \bigl[ \xi - E(\xi) \bigr] ^2
    \right \} \\[1ex]
  &= E \left \{ \bigl[ x + K - E(x) - K
    \bigr] ^2 \right \} \\[1ex]
  &= E \left \{ \bigl[ x - E(x) \bigr] ^2
    \right \} \\[1ex]
  &= {\sigma_x}^2 \peq .
\end{align*}

Ora, date due variabili casuali $x$ e $y$ qualsiasi, ed una
loro generica combinazione lineare $z = a x + b y$, basta
definire altre due variabili casuali ausiliarie
\begin{align*}
  \xi &= x - E(x) & &\text{ed} &
  \eta &= y - E(y)
\end{align*}
(che ovviamente soddisfano l'ipotesi di avere speranza
matematica zero): pertanto la loro combinazione lineare $
\zeta = a \xi + b \eta $, che differisce anch'essa da $z$
per un fattore costante e pari ad $a E(x) + b E(y)$, avr\`a
varianza che, in conseguenza della \eqref{eq:5.vcl}, sar\`a
data dalla
\begin{equation*}
  {\sigma_\zeta}^2 = a^2 {\sigma_\xi}^2 + b^2
  {\sigma_\eta}^2 \peq .
\end{equation*}

Ma per quanto detto, $x$ e $\xi$ hanno la stessa varianza;
cos\`\i\ $y$ ed $\eta$, e $z$ e $\zeta$.  Ne consegue come
per \emph{qualsiasi} coppia di variabili casuali (purch\'e
per\`o statisticamente indipendenti) vale la relazione
\eqref{eq:5.vcl}, che possiamo enunciare nel modo seguente:
\begin{quote}
  \textit{Una combinazione lineare, a coefficienti costanti,
    di due variabili casuali statisticamente indipendenti ha
    varianza uguale alla combinazione lineare delle
    rispettive varianze, con coefficienti pari ai quadrati
    dei coefficienti rispettivi\thinspace\/\footnote{O, come
      si usa dire in sintesi, \emph{gli errori si combinano
        quadraticamente}.  Una formula pi\`u generale, che
      si pu\`o applicare a coppie di variabili casuali
      qualunque, verr\`a dimostrata
      nell'appendice~\ref{ch:c.covcor}.}.}
\end{quote}

\`E ovvio poi estendere (per induzione completa) questo
risultato alla combinazione lineare di un numero finito
qualsivoglia di variabili casuali, che siano per\`o sempre
tra loro tutte statisticamente indipendenti: se
\begin{equation*}
  F = ax + by + cz +\cdots
\end{equation*}
allora
\begin{equation} \label{eq:5.varcol}
  {\sigma_F}^2 = a^2 {\sigma_x}^2 + b^2 {\sigma_y}^2 +
  c^2 {\sigma_z}^2 +\cdots \peq .
\end{equation}%
\index{combinazioni lineari!varianza!di variabili indipendenti|)}%
\index{varianza!di combinazioni lineari!di variabili indipendenti|)}

\section{L'errore della media dei campioni}%
\index{varianza!della media aritmetica|(}
Torniamo ora ad occuparci dello studio delle propriet\`a
statistiche della media aritmetica di un campione di $N$
misure indipendenti estratto da una popolazione,
\begin{equation*}
  \bar x = \frac{1}{N} \sum_{i=1}^{N} x_{i} \peq ;
\end{equation*}
e cerchiamo in particolare di determinarne la varianza.
Applicando l'equazione \eqref{eq:5.varcol} appena
dimostrata, risulta
\begin{align*}
  {\sigma_{\bar x}}^2 &= \frac{1}{N^2}
     \sum_{i=1}^N {\sigma_{x_i}}^2 \\[1ex]
  &= \frac{1}{N^{2}} \cdot N {\sigma_x}^2
\end{align*}
ed infine
\begin{equation} \label{eq:5.sbarx}
  \boxed{ \rule[-6mm]{0mm}{14mm} \quad
    {\sigma_{\bar x}}^2 = \frac{ {\sigma_x}^2 }{N}
    \quad }
\end{equation}

In definitiva abbiamo dimostrato che
\begin{quote}
  \begin{itemize}
  \item \textit{Le medie aritmetiche di campioni di $N$
      misure hanno varianza pari alla varianza della
      popolazione da cui le misure provengono, divisa per la
      dimensione dei campioni.}
  \end{itemize}
\end{quote}
e conseguentemente
\begin{quote}
  \begin{itemize}
  \item \textit{L'errore quadratico medio della media di un
      campione \`e minore dell'analogo errore delle singole
      misure, e tende a zero al crescere del numero di
      misure effettuato.}
  \end{itemize}
\end{quote}%
\index{varianza!della media aritmetica|)}

\section{La legge dei grandi numeri}%
\index{grandi numeri, legge dei|(emidx}%
\label{ch:5.granum}
Le relazioni \eqref{eq:5.ebarx} e \eqref{eq:5.sbarx} sono
state dimostrate sulla base della definizione di speranza
matematica, e senza presupporre la convergenza verso di essa
della media dei campioni (n\'e quella delle frequenze verso
la probabilit\`a); vediamo ora come la legge dei grandi
numeri (cui abbiamo gi\`a accennato nel paragrafo
\ref{ch:3.convstat}) si possa da esse dedurre.

\subsection{La disuguaglianza di Bienaym\'e--\v Ceby\v
  sef}%
\index{Bienaym\'e--\v Ceby\v sef, disuguaglianza di|(}
Sia una variabile casuale $x$, e siano $E(x)$ e $\sigma^2$
la speranza matematica e la varianza della sua popolazione;
vogliamo ora determinare la probabilit\`a che un valore di
$x$ scelto a caso differisca (in valore assoluto) da $E(x)$
per pi\`u di una assegnata quantit\`a (positiva) $\epsilon$.
Questa \`e ovviamente data, in base alla legge della
probabilit\`a totale \eqref{eq:3.protot}, dalla
\begin{equation*}
  \Pr \Bigl( \bigl| x - E(x) \bigr| \geq \epsilon \Bigl) \;
    = \; \sum_{|x_i - E(x)| \geq \epsilon} p_i
\end{equation*}
dove la sommatoria \`e estesa solo a quei valori $x_i$ che
soddisfano a tale condizione.  Ora, sappiamo che
\begin{equation*}
  \sigma^2 \; = \;
    E \left\{ \bigl[ x - E(x) \bigr] ^2 \right\}
    \; = \; \sum\nolimits_i p_i \, \bigl[ x_i -
    E(x) \bigr] ^2 \peq ;
\end{equation*}
se si restringe la sommatoria ai soli termini $x_i$ che
differiscono (in modulo) da $E(x)$ per pi\`u di $\epsilon$,
il suo valore diminuir\`a o, al massimo, rimarr\`a
invariato: deve risultare insomma
\begin{equation*}
  \sigma^2 \; \geq \; \sum_{|x_i - E(x)| \geq
    \epsilon} p_i \, \bigl[ x_i - E(x) \bigr] ^2
    \; \geq \; \sum_{|x_i - E(x)| \geq \epsilon}
    p_i \: \epsilon^2 \; = \; \epsilon^2
    \sum_{|x_i - E(x)| \geq \epsilon} p_i
\end{equation*}
e da questa relazione si ottiene la
\emph{disuguaglianza di Bienaym\'e--\v Ceby\v
sef}\/\thinspace\footnote{Ir\'en\'ee-Jules Bienaym\'e,%
\index{Bienaym\'e, Ir\'en\'ee-Jules}
  francese, fu un matematico e statistico vissuto dal
  1796 al 1878; Pafnuty Lvovi\v c \v Ceby\v sef,%
  \index{Ceby@\v Ceby\v sef!Pafnuty Lvovi\v c}
  matematico russo vissuto dal 1821 al 1894, si
  occup\`o di analisi, teoria dei numeri,
  probabilit\`a, meccanica razionale e topologia.}
\begin{gather}
  \boxed{ \rule[-6mm]{0mm}{14mm} \quad \Pr \Bigl(
    \bigl| x - E(x) \bigr| \geq \epsilon \Bigr) \; \leq \;
    \frac{\sigma^2}{\epsilon^2} \quad }
    \label{eq:5.bieceb} \\
  \intertext{e, se si pone $\epsilon = k \, \sigma$,}
  \Pr \Bigl( \bigl| x - E(x) \bigr| \geq k \, \sigma \Bigr)
    \; \leq \frac{1}{k^2} \label{eq:5.bieceb1}
\end{gather}
(se nella dimostrazione si sostituissero le frequenze
relative alle probabilit\`a e la media aritmetica ad $E(x)$,
si troverebbe che una analoga relazione vale anche per ogni
campione di valori sperimentali $x_i$ rispetto alla media
aritmetica $\bar x$ ed alla
varianza del campione $s^2$).%
\index{Bienaym\'e--\v Ceby\v sef, disuguaglianza di|)}

La \eqref{eq:5.bieceb1} fissa un \emph{limite superiore} per
la probabilit\`a esaminata, limite che deve valere per
\emph{qualsiasi} variabile casuale; con $k \leq 1$ non si
ottiene alcuna informazione significativa da essa, ma con $k
> 1$ si vede che il maggiorante della probabilit\`a tende a
zero all'aumentare di $k$.  In particolare, per
\emph{qualsiasi} variabile casuale la probabilit\`a di uno
scarto dal valore medio non inferiore in valore assoluto a
$2 \sigma$ non pu\`o superare $\frac{1}{4} = 25\%$; e quella
di uno scarto non inferiore in valore assoluto a $3 \sigma$
non pu\`o superare $\frac{1}{9} \approx 11.1\%$.

Si deve notare che non si \`e fatta alcuna ipotesi sulla
distribuzione, a parte l'esistenza della sua varianza
$\sigma^2$ e della sua speranza matematica $E(x)$; in
termini cos\`\i\ generali il limite superiore
\eqref{eq:5.bieceb1} non pu\`o essere ridotto, ma non \`e
escluso che (per una particolare distribuzione) la
probabilit\`a per la variabile da essa descritta di
differire dal suo valore medio sia pi\`u piccola ancora di
quella fissata dalla disuguaglianza di Bienaym\'e--\v Ceby\v
sef.  Ad esempio, se esiste finita la quantit\`a
\begin{gather*}
  \mu_4 = E \left\{ \bigl[ x - E(x) \bigr]^4
    \right\} \\
  \intertext{(momento del quarto ordine rispetto
    alla media), con passaggi analoghi si troverebbe
    che}
  \Pr \left\{ \bigl[ x - E(x) \bigr]^4 \geq
    \epsilon \right\} \leq \frac{\mu_4}{\epsilon^4} \\
  \intertext{e, quindi, che}
  \Pr \left\{ \bigl[ x - E(x) \bigr]^4 \geq
    k \, \sigma \right\} \leq \frac{\mu_4}{k^4 \,
   \sigma^4} \peq .
\end{gather*}

Imponendo altre condizioni (anche non molto restrittive)
alla distribuzione di probabilit\`a, si potrebbe ridurre
ulteriormente (in quantit\`a anche notevole) il limite
superiore stabilito in generale dalla \eqref{eq:5.bieceb1};
e stimare cos\`\i\ anche la probabilit\`a di uno scarto
della variabile casuale dal suo valore medio inferiore a
$\sigma$.  Risale ad esempio a Gauss%
\index{Gauss, Karl Friedrich}
(1821) la dimostrazione che per una variabile continua
avente distribuzione unimodale (con massimo in $x_0$), e per
la quale esista finita la quantit\`a ${\sigma_0}^2 = E
\bigl[ ( x - x_0 )^2 \bigr]$, la probabilit\`a di uno scarto
dalla moda $x_0$ non inferiore in valore assoluto ad una
quantit\`a prefissata non pu\`o superare la frazione
$\frac{4}{9}$ del limite di Bienaym\'e--\v Ceby\v sef:
\begin{equation*}
  \Pr \Bigl\{ \left| x - x_0 \right| \geq k \,
    \sigma \Bigr\} \leq \frac{4}{9 \, k^2} \peq .
\end{equation*}

Se la distribuzione \`e anche \emph{simmetrica}, moda e
media coincidono entrambe col centro di simmetria; e
$\sigma_0$ \`e uguale alla deviazione standard $\sigma$.
Per distribuzioni di questo genere, quindi, il limite
superiore per la probabilit\`a di uno scarto che non sia
inferiore a $k$ volte l'errore quadratico medio scende a
$\frac{4}{9} \approx 44.4\%$ per $k=1$; a $\frac{1}{9}
\approx 11.1\%$ per $k=2$; ed a $\frac{4}{81} \approx 4.9\%$
per $k=3$ (e vedremo poi nel paragrafo \ref{ch:9.scanor} che
per le misure affette da errori puramente casuali i limiti
superiori sono ancora pi\`u stringenti di questi).

\subsection{Il teorema di \v Ceby\v sef}%
\index{Ceby@\v Ceby\v sef!teorema di|(emidx}
Adesso applichiamo la \eqref{eq:5.bieceb} alla variabile
casuale $\bar x$, media aritmetica di un campione di
dimensione $N$ di valori che supponiamo essere
statisticamente indipendenti:
\begin{equation} \label{eq:5.bichmed}
  \Pr \Bigl( \bigl| \bar x - E( \bar x ) \bigr| \geq
  \epsilon \Bigr) \; \leq \; \frac{ {\sigma_{\bar x}}^2 }{
    \epsilon^2 } \peq ;
\end{equation}
ma valendo, per questa variabile casuale, le
\begin{align*}
  E(\bar x) &= E(x) & &\text{e} & \var \left( \bar x
  \right) &= \frac{\sigma^2}{N} \peq ,
\end{align*}
sostituendo nella \eqref{eq:5.bichmed} otteniamo
\begin{equation} \label{eq:5.teoceb1}
  \Pr \Bigl( \bigl| \bar x - E(x) \bigr| \geq \epsilon
  \Bigr) \; \leq \; \frac{\sigma^2}{N \, \epsilon^2} \peq .
\end{equation}

Ora, scelti comunque due numeri positivi $\epsilon$ e
$\delta$, si pu\`o trovare in conseguenza un valore di $N$
per cui il secondo membro della \eqref{eq:5.teoceb1} risulti
sicuramente minore di $\delta$: basta prendere $N > M =
\lceil \sigma^2/(\delta \, \epsilon^2) \rceil $.  In base
alla definizione \eqref{eq:3.limsta}, questo significa che
vale il
\begin{quote}
  \textsc{Teorema (di \v Ceby\v sef):}
  \label{th:5.teoceb}
  \textit{il valore medio di un campione finito di valori di
    una variabile casuale qualunque converge
    statisticamente, all'aumentare della dimensione del
    campione, alla speranza matematica per quella
    variabile.}
\end{quote}%
\index{Ceby@\v Ceby\v sef!teorema di|)}

\subsection{Il teorema di Bernoulli}%
\index{Bernoulli!teorema di|(}%
\label{ch:5.teober}
Sia un qualsiasi evento casuale $E$ avente probabilit\`a $p$
di verificarsi; indichiamo con $q = 1 - p$ la probabilit\`a
del non verificarsi di $E$ (cio\`e la probabilit\`a
dell'evento complementare \ob{E}\,).

Consideriamo poi un insieme di $N$ prove nelle quali si
osserva se $E$ si \`e o no verificato; ed introduciamo una
variabile casuale $y$, definita come il numero di volte in
cui $E$ si \`e verificato in una di tali prove.  Ovviamente
$y$ pu\`o assumere i due soli valori 1 (con probabilit\`a
$p$) e 0 (con probabilit\`a $q$); la sua speranza matematica
\`e perci\`o data da
\begin{equation}
  E( y ) \; = \; 1 \cdot p + 0 \cdot q \; = \; p \peq
  \label{eq:5.speber} .
\end{equation}

La frequenza relativa $f$ dell'evento $E$ nelle $N$ prove si
pu\`o chiaramente esprimere (indicando con $y_i$ il valore
assunto dalla variabile casuale $y$ nella $i$-esima di esse)
come
\begin{equation*}
  f \; = \; \frac{1}{N} \sum_{i=1}^N y_i \peq ,
\end{equation*}
ossia \`e data \emph{dal valore medio della $y$ sul campione
  di prove}, $\bar y$; ma quest'ultimo (per il teorema di \v
Ceby\v sef\footnote{Il teorema di \v Ceby\v sef vale per
  tutte le variabili casuali per le quali esistano sia la
  speranza matematica che la varianza: la prima \`e espressa
  dall'equazione \eqref{eq:5.speber}, la seconda sar\`a
  ricavata pi\`u tardi nell'equazione \eqref{eq:8.varber} a
  pagina \pageref{eq:8.varber}.}) deve convergere
statisticamente, all'aumentare di $N$, alla speranza
matematica per $y$: che vale proprio $p$.  Riassumendo,
abbiamo cos\`\i\ dimostrato il
\begin{quote}
  \textsc{Teorema (di Bernoulli, o legge ``dei grandi
    numeri''):} \textit{la frequenza relativa di qualunque
    evento casuale converge (statisticamente) alla sua
    probabilit\`a all'aumentare del numero delle prove.}
\end{quote}%
\index{Bernoulli!teorema di|)}%
\index{grandi numeri, legge dei|)}

\section{Valore medio e valore vero}%
\index{media!aritmetica!come stima del valore vero|(}
Anche se non ci basiamo sulla definizione empirica di
probabilit\`a, ma su quella assiomatica, possiamo ora
presupporre la convergenza della media aritmetica dei
campioni di misure alla speranza matematica della grandezza
misurata, che ora a buon diritto possiamo chiamare ``valore
medio del risultato della misura sull'intera popolazione''.

Si pu\`o ora meglio precisare la distinzione fra errori
casuali ed errori sistematici: i primi,%
\index{errori di misura!casuali}
visto che possono verificarsi con uguale probabilit\`a in
difetto ed in eccesso rispetto al valore vero, avranno
valore medio nullo; mentre errori sistematici%
\index{errori di misura!sistematici}
causeranno invece per definizione una differenza tra il
valore medio delle misure $E(x)$ ed il valore vero.  In
assenza di errori sistematici assumiamo allora che valore
medio e valore vero coincidano: ammettiamo insomma (lo
proveremo pi\`u avanti per la distribuzione normale) che in
tal caso $E(x)$ esista e sia uguale a $x^*$.  Sappiamo
insomma che risulta
\begin{gather*}
  \bar x \; = \; \frac{1}{N} \sum_{i=1}^N x_i
    \; = \; \frac{1}{N} \sum_{j=1}^M \nu_j \, x_j
    \; = \; \sum_{j=1}^M f_j \, x_j \\[1ex]
  \lim_{N \rightarrow \infty} \bar x \; \equiv \;
    E(x) \; = \; \sum\nolimits_j p_j \, x_j \\[1ex]
  \intertext{e postuliamo che}
  E(x) \equiv x^* \peq ;
\end{gather*}
inoltre sappiamo che anche
\begin{equation*}
  E \left( \bar x \right) \; = \; E(x) \; \equiv \; x^* \peq
  .
\end{equation*}
Ossia, non solo $\bar x$ converge ad $E(x)$ all'aumentare
della dimensione del campione; ma, qualunque sia il valore
di quest'ultima grandezza, \textbf{mediamente} $\bar x$
\emph{coincide} con $E(x)$.  Ripetendo varie volte la misura
ed ottenendo cos\`\i\ pi\`u campioni con differenti medie
aritmetiche, dando come stima di $E(x)$ la media di uno dei
nostri campioni avremo insomma la stessa probabilit\`a di
sbagliare per difetto o per eccesso\/\footnote{Questo nella
  terminologia statistica si esprime dicendo che la media
  dei campioni \`e una stima \emph{imparziale}%
  \index{stima!imparziale}
  della media della popolazione; al contrario della varianza
  del campione che, come vedremo nel prossimo paragrafo, \`e
  una stima \emph{parziale} (o \emph{distorta}) della
  varianza della popolazione (il concetto verr\`a poi
  approfondito nel paragrafo \ref{ch:11.sticar}).}.%
\index{media!aritmetica!come stima del valore vero|)}

\section{Scarto ed errore quadratico medio}%
\index{varianza!della popolazione e di campioni|(}%
\label{ch:5.scedeqm}
L'ultimo punto da approfondire riguarda la relazione tra la
varianza $s^2$ di un campione di $N$ misure e quella
$\sigma^2$ della popolazione da cui il campione proviene.
Ora, $s^2$ si pu\`o esprimere come
\begin{equation*}
  s^2 \; = \; \frac{1}{N} \sum_{i=1}^N \left( x_i -
    \bar x \right) ^2
\end{equation*}
e possiamo osservare che (per qualsiasi numero $x^*$ e
quindi anche per l'incognito valore vero) vale la seguente
relazione matematica:

\begin{equation} \label{eq:5.mprop2}
  \begin{split}
    \quad \frac{1}{N} \sum_{i=1}^N &\left( x_i - x^*
      \right) ^2 \; = \; \frac{1}{N} \sum_{i=1}^N
      \bigl[ \left( x_i - \bar x \right)
      + \left( \bar x - x^* \right) \bigr] ^2 \\
    &= \frac{1}{N} \left[ \,
      \sum_{i=1}^N \left( x_i - \bar x \right) ^2
      \; + \; \sum_{i=1}^N \left( \bar x - x^* \right)
      ^2 \; + \; 2 \left( \bar x - x^* \right)
      \sum_{i=1}^N \left( x_i - \bar x \right)
      \right] \\[1ex]
    &= \frac{1}{N} \left[ \,
      \sum_{i=1}^N \left( x_i - \bar x \right) ^2
      \; + \; N \left( \bar x - x^* \right) ^2
      \right] \\[1ex]
    &= s^2 \; + \; \left( \bar x - x^* \right) ^2
  \end{split} \raisetag{15pt}
\end{equation}
(si \`e sfruttata qui l'equazione \eqref{eq:4.mprop1},
secondo la quale la somma algebrica degli scarti delle
misure dalla loro media aritmetica \`e identicamente nulla;
vedi anche l'analoga formula \eqref{eq:4.mprop2} nel
paragrafo \ref{ch:4.medari}).

Cerchiamo ora di capire come le varianze $s^2$ dei campioni
di dimensione $N$ siano legate all'analoga grandezza,
$\sigma^{2}$ o $\var(x)$, definita sull'intera popolazione,
e per\`o calcolata rispetto al valore medio di essa, $E(x) =
x^{*}$:
\begin{equation*}
  \sigma^{2} \; = \; E \left \{ \bigl[ x - E(x)
    \bigr]^2 \right\} \; = \; E \left \{ \left( x -
    x^{*} \right) ^{2} \right \} \peq .
\end{equation*}
Sfruttando la relazione \eqref{eq:5.mprop2} in precedenza
trovata, si ha
\begin{equation*}
  s^2 = \frac{1}{N} \sum_{i=1}^N \left( x_i -
    x^* \right) ^2 \; - \; \left( \bar x
    - x^* \right) ^2
\end{equation*}
e prendendo i valori medi di entrambi i membri (sugli
infiniti campioni di dimensione $N$ che si possono pensare
estratti in modo casuale dalla popolazione originaria),
otteniamo
\begin{equation*}
  E ( s^2 ) \; = \; \sigma^2 - E
  \left \{ \left( \bar x - x^* \right) ^2
  \right \} \peq .
\end{equation*}

Ricordando come il valore medio del quadrato degli scarti di
una variabile (qui $\bar x$) dal suo valore medio (che \`e
$E(\bar x) = E(x) = x^*$) sia per definizione la varianza
della variabile stessa (che indicheremo qui come quadrato
dell'errore quadratico medio $\sigma_{\bar x}$), si ricava
infine:
\begin{equation} \label{eq:5.esssss}
  E ( s^2 ) \; = \; \sigma^2 - \,
  {\sigma_{\bar x}}^2 \; < \; \sigma^2 \peq .
\end{equation}

Insomma:
\begin{quote}
  \begin{itemize}
  \item \textit{Il valore medio della varianza $s^2$ di un
      campione \`e sistematicamente inferiore all'analoga
      grandezza $\sigma^2$ che si riferisce all'intera
      popolazione.}
  \item \textit{La differenza tra la varianza della
      popolazione $\sigma^{2}$ e la varianza di un campione
      di $N$ misure da essa estratto \`e \textbf{in media}
      pari alla varianza della media del campione.}
  \end{itemize}
\end{quote}

\section{Stima della varianza della popolazione}
Vediamo ora come si pu\`o stimare la varianza dell'insieme
delle infinite misure che possono essere fatte di una
grandezza fisica a partire da un particolare campione di $N$
misure.  Riprendiamo l'equazione \eqref{eq:5.esssss}; in
essa abbiamo gi\`a dimostrato che
\begin{equation*}
  E(s^2) = \sigma^2 - {\sigma_{\bar x}}^2
\end{equation*}
e sappiamo dalla \eqref{eq:5.sbarx} che la varianza della
media del campione vale
\begin{equation*}
  {\sigma_{\bar x}}^2 = \frac{\sigma^2}{N} \peq .
\end{equation*}

Risulta pertanto
\begin{equation*}
  \boxed{ \rule[-6mm]{0mm}{14mm} \quad
    E(s^{2}) = \frac{N-1}{N} \: \sigma^{2}
    \quad }
\end{equation*}
e quindi
\begin{quote}
  \textit{\textbf{Mediamente} la varianza di un
  campione di $N$ misure \`e inferiore alla varianza
  della intera popolazione per un fattore $(N-1)/N$.}
\end{quote}

Questo \`e il motivo per cui, per avere una stima
\emph{imparziale}%
\index{stima!imparziale}
(ossia \emph{mediamente corretta}) di $\sigma$, si usa (come
gi\`a anticipato) la quantit\`a $\mu$ definita attraverso la
\begin{equation*}
  \mu ^2 \; = \; \frac{N}{N-1} \: s^2 \; = \;
  \frac{\sum\limits_{i=1}^N \left( x_i - \bar x
    \right) ^2 }{N-1} \peq ,
\end{equation*}
quantit\`a il cui valore medio su infiniti campioni risulta
proprio $\sigma^2$.

\section{Ancora sull'errore quadratico medio}
Diamo qui un'altra dimostrazione del teorema riguardante la
stima corretta dell'errore quadratico medio di una
popolazione a partire da un campione, seguendo una linea
diversa e pi\`u vicina alle verifiche sperimentali che si
possono fare avendo a disposizione numerosi dati.

Si supponga di avere $M$ campioni contrassegnati dall'indice
$j$ (con $j$ che assume i valori $1,\ldots,M$); ciascuno di
essi sia poi costituito da $N$ misure ripetute della stessa
grandezza $x$, contrassegnate a loro volta dall'indice $i$
($i=1,\ldots,N$): il valore osservato nella misura $i$-esima
del campione $j$-esimo sia indicato insomma dal simbolo
$x_{ij}$.

Indicando con $x^{*}$ il valore vero di $x$, e con $\bar
x_{j}$ la media aritmetica del campione $j$-esimo, vale la
\begin{align*}
  \left( x_{ij} - x^* \right) ^2 &= \Bigl[
    \left( x_{ij} - \bar x_j \right) + \left(
    \bar x_j - x^* \right) \Bigr] ^2 \\[1ex]
  &= \left( x_{ij} - \bar x_j \right) ^2 \; + \;
    \left( \bar x_j - x^* \right) ^2 \; + \;
    2 \left( \bar x_j - x^* \right) \left( x_{ij}
    - \bar x_j \right) \peq .
\end{align*}

Ora sommiamo su $i$ tutte le $N$ uguaglianze che si hanno
per i valori dell'indice $i=1,2,\ldots,N$ e dividiamo per
$N$; se indichiamo con $ {s_j}^2 $ la varianza del campione
$j$-esimo, data da
\begin{equation*}
  {s_j}^2 = \frac{1}{N} \sum_{i=1}^{N}
  \left( x_{ij} - \bar x_{j} \right) ^{2}
\end{equation*}
otteniamo alla fine
\begin{equation*}
  \frac{1}{N} \sum_{i=1}^N \left( x_{ij} - x^* \right) ^2 \;
  = \; {s_j}^2 \; + \; \left( \bar x_j - x^* \right) ^2 \; +
  \; \frac{2}{N} \left( \bar x_j - x^* \right)
  \sum_{i=1}^N \left( x_{ij} - \bar x_j \right) \peq .
\end{equation*}

L'ultima sommatoria a destra \`e la somma algebrica degli
scarti delle misure del campione $j$-esimo dalla loro media
aritmetica $\bar x_j$ che sappiamo essere identicamente
nulla. Dunque, per ogni $j$ vale la
\begin{equation*}
  \frac{1}{N} \sum_{i=1}^N \left( x_{ij} - x^* \right) ^2 =
  {s_j}^2 \; + \; \left( \bar x_j - x^* \right) ^2
\end{equation*}
e se sommiamo membro a membro tutte le $M$ uguaglianze che
abbiamo per $j=1,2,\ldots,M$ e dividiamo per $M$, risulta
\begin{equation*}
  \frac{1}{M} \sum_{j=1}^M \: \frac{1}{N} \sum_{i=1}^N
  \left( x_{ij} - x^* \right) ^2 \; = \; \frac{1}{M}
  \sum_{j=1}^M {s_j}^2 \: + \: \frac{1}{M} \sum_{j=1}^M
  \left( \bar x_j - x^* \right) ^2 \peq .
\end{equation*}

Ora supponiamo di avere a disposizione moltissimi campioni e
passiamo al limite per $M \rightarrow \infty$.  Il primo
membro (che rappresenta il valore medio, su tutti i dati e
tutti gli infiniti campioni, del quadrato degli scarti dal
valore \emph{vero}) converge stocasticamente alla varianza
della variabile casuale $x$; il secondo termine a destra
(valore medio, su tutti gli infiniti campioni, del quadrato
degli scarti della media aritmetica del campione dal proprio
valore vero) converge alla varianza delle medie dei campioni
di $N$ misure ${\sigma_{\bar x}}^2$.

Il primo termine a destra \`e il valore medio della varianza
dei campioni di $N$ misure e, sostituendo, infine si trova
\begin{align*}
  \sigma ^{2} &= \lim_{M \rightarrow \infty} \,
    \frac{1}{NM} \, \sum \nolimits_{ij}
    \left( x_{ij} - x^{*} \right) ^{2} \\[1ex]
  &= \lim_{M \rightarrow \infty} \,
    \frac{1}{M} \, \sum_{j=1}^{M} {s_j}^2
    \; + \; \lim_{M \rightarrow \infty} \,
    \frac{1}{M} \, \sum_{j=1}^{M}
    \left( \bar x_{j} - x^{*} \right) ^{2} \\[1ex]
  &= E ( s^{2} ) \; + \; {\sigma_{\bar x}}^2 \peq .
\end{align*}

Ora, avendo gi\`a dimostrato che
\begin{equation*}
  {\sigma_{\bar x}}^2 = \frac{\sigma^{2}}{N} \peq ,
\end{equation*}
si ricava facilmente
\begin{equation*}
  \sigma^{2} = E ( s^{2} ) + \frac{\sigma^{2}}{N}
\end{equation*}
ovvero
\begin{equation*}
  E ( s^{2} ) = \frac{N-1}{N} \, \sigma^{2}
\end{equation*}
che \`e il risultato gi\`a ottenuto.

Si noti che mentre molti teoremi della statistica sono
validi solo \emph{asintoticamente}, cio\`e per campioni
numerosi o per un numero molto grande di variabili, questo
teorema vale per ogni $N$
($ \ge 2$).%
\index{varianza!della popolazione e di campioni|)}

\endinput

% $Id: chapter6.tex,v 1.1 2005/03/01 10:06:08 loreti Exp $

\chapter{Variabili casuali unidimensionali continue}
Le definizioni di probabilit\`a che abbiamo finora usato
sono adatte solo per una variabile casuale che possa
assumere solo valori discreti; vediamo innanzi tutto come il
concetto di probabilit\`a si possa generalizzare a variabili
casuali continue, variabili che possono cio\`e assumere
tutti gli infiniti valori appartenenti ad un insieme
continuo: tali si suppone generalmente siano i risultati
delle misure delle grandezze fisiche, per poter applicare ad
essi il calcolo differenziale ed integrale.

\section{La densit\`a di probabilit\`a}%
\index{probabilit\`a!densit\`a di|(emidx}
Definiamo arbitrariamente delle classi di frequenza,
suddividendo l'asse delle $x$ in intervalli di ampiezze che,
per semplicit\`a, supponiamo siano tutte uguali; ed
immaginiamo di fare un certo numero $N$ di misure della
grandezza fisica $x$.  Come sappiamo, possiamo riportare le
misure ottenute
in istogramma%
\index{istogrammi|(}
tracciando, al di sopra dell'intervallo che rappresenta ogni
classe, un rettangolo avente area uguale alla frequenza
relativa\/\footnote{Non vi \`e alcuna differenza nell'usare
  frequenze relative o assolute: essendo esse proporzionali
  l'una all'altra, l'aspetto dell'istogramma \`e il medesimo
  --- cambia solo la scala dell'asse delle ordinate.} con
cui una misura \`e caduta in essa; l'altezza dell'istogramma
in ogni intervallo \`e data quindi da tale frequenza divisa
per l'ampiezza dell'intervallo di base, e l'area totale
dell'istogramma stesso vale uno.
\begin{figure}[htbp]
  \vspace*{2ex}
  \begin{center} {
    \input{istgau.pstex_t}
  } \end{center}
  \caption[Il comportamento limite delle frequenze
    relative]{Nella prima figura, l'istogramma della
    grandezza $x$ per un numero piccolo di misure; nella
    seconda, lo stesso istogramma per un numero molto
    grande di misure; nell'ultima, l'istogramma si
    approssima alla curva limite quando l'intervallo di
    base tende a zero.}
\end{figure}

Se immaginiamo di far tendere all'infinito il numero di
misure effettuate, in base alla legge dei grandi numeri ci
aspettiamo un ``aggiustamento'' dell'istogramma in modo che
l'area rappresentata sopra ogni intervallo tenda alla
\emph{probabilit\`a} che il valore misurato cada entro di
esso; le altezze tenderanno quindi al rapporto tra questa
probabilit\`a e l'ampiezza dell'intervallo di base
dell'istogramma.

Disponendo di un numero infinitamente grande di misure, ha
senso diminuire l'ampiezza degli intervalli in cui l'asse
delle $x$ \`e stato diviso, e renderla piccola a piacere.%
\index{istogrammi|)}
Se l'intervallo corrispondente ad una data classe di
frequenza tende a zero, la probabilit\`a che una misura cada
in esso tende ugualmente a zero; ma se esiste ed \`e finito
il limite del rapporto tra probabilit\`a $\de p$ ed ampiezza
$\de x$ dell'intervallo, l'istogramma tender\`a ad una curva
continua la cui ordinata sar\`a in ogni punto data da tale
limite.

L'ordinata di questa curva al di sopra di un intervallo
infinitesimo $\de x$ vale quindi
\begin{equation*}
  y \; = \; f(x) \; = \; \frac{\de p}{\de x}
\end{equation*}
e le dimensioni della grandezza $y$ sono quelle di una
probabilit\`a (un numero puro) divise per quelle della
grandezza $x$; la $y$ prende il nome di \emph{densit\`a di
  probabilit\`a}, o di \emph{funzione di frequenza}, della
$x$.

La variabile continua schematizza il caso in cui i valori
osservabili (sempre discreti per la sensibilit\`a limitata
degli strumenti) sono molto densi, separati cio\`e da
intervalli molto piccoli, e assai numerosi.  In questa
situazione la probabilit\`a di osservare uno solo di tali
valori \`e anch'essa estremamente piccola --- ed ha
interesse soltanto la probabilit\`a che venga osservato uno
tra i molti possibili valori della $x$ che cadono in un dato
intervallo $ [ x_1 , x_2 ] $ di ampiezza grande rispetto
alla risoluzione sperimentale.

Se dividiamo tale intervallo in un numero molto grande di
sottointervalli infinitesimi di ampiezza $\de x$, gli eventi
casuali consistenti nell'appartenere il risultato della
misura ad una delle classi di frequenza relative sono
mutuamente esclusivi; di conseguenza, vista l'equazione
\eqref{eq:3.protot}, la probabilit\`a che $x$ appartenga
all'intervallo finito $ [ x_1, x_2 ] $ \`e data dalla somma
delle probabilit\`a (infinitesime) rispettive $ \de p = f(x)
\, \de x $: e questa, per definizione, \`e l'integrale di
$f(x)$ rispetto ad $x$ nell'intervallo $ [ x_1, x_2 ] $.

Insomma, qualunque sia l'intervallo $ [ x_1, x_2 ] $ vale la
\begin{equation*}
  \Pr \Bigl(x \in  [x_1,x_2] \Bigr) =
  \int_{x_1}^{x_2} \! f(x) \, \de x \peq ;
\end{equation*}
e, in definitiva:
\begin{quote}
  \textit{Per le variabili continue non si pu\`o parlare di
    probabilit\`a attraverso le definizioni gi\`a esaminate.
    \`E invece possibile associare ad ogni variabile
    continua $x$ una funzione ``densit\`a di probabilit\`a''
    $f(x)$, da cui si pu\`o dedurre la probabilit\`a che la
    $x$ cada in un qualsiasi intervallo finito prefissato:
    questa \`e data semplicemente dall'area sottesa dalla
    curva nell'intervallo in questione.}
\end{quote}%
\index{probabilit\`a!densit\`a di|)}

\index{funzione!di distribuzione|(emidx}%
\label{def:6.fundis}
Analogamente al concetto sperimentale di frequenza
cumulativa relativa, introdotto a pagina
\pageref{def:4.frcure} nel paragrafo \ref{def:4.frcure}, si
pu\`o definire la \emph{funzione di distribuzione} per una
variabile continua $x$ come
\begin{equation*}
  F(x) = \int_{- \infty}^x \! f(t) \, \de t \peq .
\end{equation*}
Essa rappresenta la probabilit\`a di osservare un valore non
superiore ad $x$, e dovr\`a necessariamente soddisfare la
$F(+ \infty) \equiv 1$.  Quindi deve valere la cosiddetta
\begin{quote}
  \textsc{Condizione di normalizzazione:}%
  \index{normalizzazione!condizione di|emidx}
  \textit{l'integrale di una qualunque funzione che
    rappresenti una densit\`a di probabilit\`a,
    nell'intervallo $ \left[ -\infty, +\infty \right]$ vale
    1.}
\end{quote}
\begin{equation} \label{eq:6.connor}
  \int_{- \infty}^{+ \infty} \! f(x) \, \de x \: = \: 1 \peq
  .
\end{equation}%
\index{funzione!di distribuzione|)}

\`E da enfatizzare come il solo fatto che valga la
condizione di normalizzazione, ossia che converga
l'integrale \eqref{eq:6.connor}, \`e sufficiente a garantire
che una qualsiasi funzione che rappresenti una densit\`a di
probabilit\`a \emph{debba tendere a zero} quando la
variabile indipendente tende a pi\`u o meno infinito; e
questo senza alcun riferimento alla particolare natura del
fenomeno casuale cui essa \`e collegata.  Questo non \`e
sorprendente, visto che la disuguaglianza
\eqref{eq:5.bieceb} di Bienaym\'e--\v Ceby\v sef implica che
a distanze via via crescenti dal valore medio di una
qualsiasi variabile casuale corrispondano probabilit\`a via
via decrescenti, e che si annullano asintoticamente.

Al lettore attento non sar\`a sfuggito il fatto che, per
introdurre il concetto di densit\`a di probabilit\`a, ci si
\`e ancora una volta basati sul risultato di un esperimento
reale (l'istogramma delle frequenze relative in un
campione); e si \`e ipotizzato poi che la rappresentazione
di tale esperimento si comporti in un determinato modo
quando alcuni parametri (il numero di misure e la
sensibilit\`a sperimentale) vengono fatti tendere a limiti
che, nella pratica, sono irraggiungibili.

Questo \`e in un certo senso analogo all'enfasi che abbiamo
prima posto sulla definizione empirica della probabilit\`a,
in quanto pi\`u vicina all'esperienza reale di una
definizione totalmente astratta come quella assiomatica; per
un matematico la densit\`a di probabilit\`a di una variabile
casuale continua \`e invece definita semplicemente come una
funzione non negativa, integrabile su tutto l'asse reale e
che obbedisca alla condizione di normalizzazione.  Il passo
successivo consiste nell'associare ad ogni intervallo
infinitesimo $\de x$ la quantit\`a $\de p = f(x) \de x$, e
ad ogni intervallo finito $ [ x_1, x_2 ] $ il corrispondente
integrale: integrale che, come si pu\`o facilmente
controllare, soddisfa la definizione assiomatica di
probabilit\`a.

\section{La speranza matematica per le variabili
  continue}%
\index{speranza matematica!per variabili continue|(emidx}%
\label{ch:6.mevaco}
Possiamo ora determinare l'espressione della speranza
matematica di una generica variabile casuale continua $x$;
questa grandezza, che avevamo gi\`a definito nell'equazione
\eqref{eq:5.spermat} come
\begin{equation*}
  E(x) = \sum \nolimits_i p_i \, x_i
\end{equation*}
per una variabile discreta, si dovr\`a ora scrivere per una
variabile continua
\begin{equation*}
  E(x) = \int_{- \infty}^{+ \infty} \! x \cdot f(x) \,
    \de x \peq ;
\end{equation*}
dove per $f(x)$ si intende la funzione densit\`a di
probabilit\`a della variabile casuale $x$.

Per ricavare questa formula, basta pensare di aver suddiviso
l'asse delle $x$ in un numero grandissimo di intervalli
estremamente piccoli di ampiezza $\de x$, ad ognuno dei
quali \`e associata una probabilit\`a anch'essa estremamente
piccola che vale $\de p = f(x) \, \de x$; e sostituire poi
nella formula per variabili discrete.  In base al teorema di
pagina \pageref{th:5.teoceb} (il teorema di \v Ceby\v sef),%
\index{Ceby@\v Ceby\v sef!teorema di}
le medie aritmetiche dei campioni finiti di valori della
grandezza $x$ tendono proprio a questo $E(x)$ all'aumentare
indefinito di $N$.

La speranza matematica di una qualsiasi grandezza $W(x)$
funzione della variabile casuale $x$ sar\`a poi
\begin{equation} \label{eq:6.mevaco}
  E \bigl[ W(x) \bigr] \; = \; \int_{- \infty}^{+
    \infty} \! W(x) \cdot f(x) \, \de x \peq .
\end{equation}%
\index{speranza matematica!per variabili continue|)}

\section{I momenti}%
\index{momenti|(}
Per qualunque variabile casuale $x$ si possono definire,
sempre sulla popolazione, i cosiddetti \emph{momenti}: il
momento di ordine $k$ rispetto all'origine, $\lambda_k$, \`e
la speranza matematica di $x^k$; ed il momento di ordine $k$
rispetto alla media, $\mu_k$, \`e la speranza matematica di
$\bigl[ x - E(x) \bigr]^k$.  In formula (con ovvio
significato dei simboli):
\begin{gather*}
  \lambda_k \; = \; E \bigl( x^k \bigr) \; = \;
    \sum\nolimits_i p_i \, {x_i}^k \\
  \intertext{e}
  \mu_k \; = \; E \left\{ \bigl[ x - E(x) \bigr]^k
    \right\} \; = \; \sum\nolimits_i p_i \, \bigl[ x_i -
    E(x) \bigr]^k
    \intertext{per una variabile discreta (analogamente,
      usando le frequenze, si possono definire i momenti
      rispetto all'origine ed alla media aritmetica di un
      campione); oppure}
  \lambda_k \; = \; E \bigl( x^k \bigr) \; = \;
    \int_{-\infty}^{+\infty} \! x^k \, f(x) \, \de x \\
  \intertext{e}
  \mu_k \; = \; E \left\{ \bigl[ x - E(x) \bigr]^k
    \right\} \; = \; \int_{-\infty}^{+\infty} \! \bigl[
    x - E(x) \bigr]^k \, f(x) \, \de x
\end{gather*}
per una variabile continua.

Chiaramente, se la popolazione \`e costituita da un numero
infinito di elementi (quindi, in particolare, per le
variabili continue), non \`e detto che i momenti esistano;
inoltre $E(x) \equiv \lambda_1$ e $\var(x) \equiv \mu_2
\equiv \lambda_2 - {\lambda_1}^2 $.  Dalla definizione
consegue immediatamente che, per qualsiasi popolazione per
cui esista $E(x)$,
\begin{align*}
  \mu_1 &= \int_{-\infty}^{+\infty} \! \bigl[ x -
    E(x) \bigr] \, f(x) \, \de x \\[1ex]
  &= \int_{-\infty}^{+\infty} \! x \, f(x) \, \de x
    - E(x) \int_{-\infty}^{+\infty} \! f(x) \, \de x
    \\[1ex]
  &= E(x) - E(x) \\[1ex]
  &\equiv 0 \peq .
\end{align*}

\`E poi facile dimostrare che, per popolazioni simmetriche
rispetto alla media, tutti i momenti di ordine dispari
rispetto ad essa, se esistono, valgono zero: basta
considerare come, negli integrali, i contributi infinitesimi
di ognuno degli intervallini si possano associare a due a
due in modo che si annullino vicendevolmente.  Il valore del
momento del terzo ordine rispetto alla media aritmetica
pu\`o quindi essere considerato una sorta di misura
dell'asimmetria di una distribuzione.

In pratica per\`o si preferisce usare, in luogo di $\mu_3$,
un parametro adimensionale; definendo il cosiddetto
\emph{coefficiente di asimmetria}%
\index{coefficiente!di asimmetria}%
\index{asimmetria|see{coefficiente di asimmetria}}
(o \emph{skewness}, in inglese) come
\begin{equation*}
  \gamma_1 \; = \; \frac{\mu_3}{\left( \sqrt{\mu_2}
    \right)^3} \; = \; \frac{\mu_3}{\sigma^3}
\end{equation*}
(dove $\sigma = \sqrt{\mu_2}$ \`e la radice quadrata della
varianza); $\gamma_1$ \`e nullo per densit\`a di
probabilit\`a simmetriche rispetto alla media, oppure ha
segno positivo (o negativo) a seconda che i valori della
funzione di frequenza per la variabile casuale in questione
si trovino ``sbilanciati'' verso la destra (o verso la
sinistra) rispetto al valore medio.

Dal momento del quarto ordine rispetto alla media si pu\`o
ricavare un altro parametro adimensionale talvolta usato per
caratterizzare una distribuzione: il
\emph{coefficiente di curt\`osi}%
\index{coefficiente!di curtosi|(}%
\index{curtosi|see{coefficiente di curtosi}}
 $\gamma'_2$, definito come
\begin{equation} \label{eq:6.curtosi}
  \gamma'_2 \; = \; \frac{\mu_4}{{\mu_2}^2} \; = \;
    \frac{\mu_4}{\sigma^4}
\end{equation}
e che \`e ovviamente sempre positivo.  Esso misura in un
certo senso la ``rapidit\`a'' con cui una distribuzione di
probabilit\`a converge a zero quando ci si allontana dalla
zona centrale in cui essa assume i valori pi\`u alti
(individuata dal valore di $E(x) \equiv \lambda_1$): o, se
si preferisce, l'importanza delle sue ``code'' laterali;
infatti, quanto pi\`u rapidamente la funzione converge a
zero in queste code, tanto pi\`u piccolo sar\`a il valore di
$\gamma'_2$.  Come si potrebbe ricavare integrandone la
funzione di frequenza (che troveremo pi\`u avanti nel
paragrafo \ref{ch:8.gauss}), il coefficiente di curtosi
della distribuzione normale calcolato usando la
\eqref{eq:6.curtosi} vale 3; per questo motivo si preferisce
generalmente definirlo in modo differente, usando la
\begin{equation*}
  \gamma_2 \; = \; \frac{\mu_4}{\sigma^4} - 3 \peq .
\end{equation*}
Questo fa s\`\i\ che esso valga zero per la funzione di
Gauss, e che assuma poi valori di segno negativo o positivo
per funzioni che convergano a zero nelle code in maniera
rispettivamente pi\`u ``rapida'' o pi\`u ``lenta'' della
distribuzione normale.%
\index{coefficiente!di curtosi|)}%
\index{momenti|)}

\section{Funzione generatrice e funzione   caratteristica}%
\index{funzione!generatrice dei momenti|(}%
\index{momenti!funzione generatrice|see{funzione generatrice dei momenti}}%
\label{ch:6.fugeca}
La speranza matematica della funzione $e^{tx}$ per una
variabile casuale continua $x$ prende il nome di
\emph{funzione generatrice dei momenti} della variabile
stessa; la indicheremo nel seguito col simbolo $M_x(t)$.  Il
motivo del nome \`e che risulta, indicando con $f(x)$ la
densit\`a di probabilit\`a di $x$:
\begin{gather}
  \boxed{ \rule[-6mm]{0mm}{14mm} \quad
    M_x(t) \; = \; E \bigl( e^{tx} \bigr) \; = \;
      \int_{-\infty}^{+\infty} \! e^{t x} f(x) \, \de x
      \quad } \label{eq:6.fugemo} \\
  \intertext{(per una variabile continua, oppure}
  M_x(t) = \sum\nolimits_i p_i \, e^{t x_i} \notag \\
  \intertext{per una variabile discreta); e, ricordando sia
    lo sviluppo in serie di McLaurin della funzione
    esponenziale}
  e^{tx} \; = \; \sum_{k=0}^\infty \frac{(tx)^k}{k!}
    \notag \\
  \intertext{che la definizione dei momenti rispetto
    all'origine, \emph{se questi esistono tutti fino a
    qualsiasi ordine} risulta anche}
  M_x(t) \; = \; \sum_{k=0}^\infty \frac{t^k}{k!} \,
    \lambda_k \notag \\
  \intertext{da cui}
  \left. \frac{\de^k M_x(t)}{\de t^k} \right|_{t=0} \; =
    \; \lambda_k \notag
\end{gather}
e, in definitiva, derivando successivamente la funzione
generatrice si possono ricavare tutti i momenti della
funzione di frequenza da cui essa discende.  Se interessa
invece uno dei momenti non rispetto all'origine, ma rispetto
al valore medio $\lambda$, basta considerare l'altra
funzione
\begin{gather}
  \ob{M}_x(t) \; = \; E \left[ e^{t(x - \lambda)}
    \right] \; = \; e^{-t \lambda} M_x(t)
    \label{eq:6.fugemm} \\
  \intertext{e si trova facilmente che risulta}
  \left. \frac{\de^k \ob{M}_x(t)}{\de t^k}
    \right|_{t=0} \; = \;\mu_k \peq . \notag
\end{gather}%
\index{funzione!generatrice dei momenti|)}

\index{funzione!caratteristica|(}%
La speranza matematica della funzione $e^{itx}$ si chiama
invece \emph{funzione caratteristica} della variabile
casuale $x$, e si indica con $\phi_x(t)$:
\begin{gather}
  \boxed{ \rule[-6mm]{0mm}{14mm} \quad
    \phi_x(t) \; = \; E \bigl( e^{itx} \bigr) \; = \;
      \int_{-\infty}^{+\infty} \! e^{itx} f(x) \, \de x
    \quad } \label{eq:6.funcar} \\
  \intertext{(per una variabile continua, e}
  \phi_x(t) \; = \; \sum\nolimits_k p_k \, e^{i t x_k}
    \label{eq:6.fucadi} \\
  \intertext{per una variabile discreta); e, se esistono i
    momenti di qualsiasi ordine rispetto all'origine,
    risulta anche}
  \phi_x(t) \; = \; \sum_{k=0}^\infty \frac{(it)^k}{k!}
    \, \lambda_k \label{eq:6.funcar1} \\
  \intertext{dalla quale si ricava}
  \left. \frac{\de^k \phi_x(t)}{\de t^k} \right|_{t=0}
    \; = \; i^k \lambda_k \peq . \label{eq:6.fucamo}
\end{gather}%
\index{funzione!caratteristica|)}

Queste funzioni sono importanti in virt\`u di una serie di
teoremi, che citeremo qui senza dimostrarli:
\begin{itemize}
\item I momenti (se esistono fino a qualunque ordine)
  caratterizzano univocamente una variabile casuale; se due
  variabili casuali hanno gli stessi momenti fino a
  qualsiasi ordine, la loro densit\`a di probabilit\`a \`e
  identica.
\item La funzione generatrice esiste solo se esistono i
  momenti fino a qualsiasi ordine; e anch'essa caratterizza
  univocamente una variabile casuale, nel senso che se due
  variabili hanno la stessa funzione generatrice la loro
  densit\`a di probabilit\`a \`e identica.
\item La $\phi_x(t)$ prima definita si chiama anche
  \emph{trasformata di Fourier}%
  \index{Fourier, trasformata di}
  della funzione $f(x)$; anch'essa caratterizza univocamente
  una variabile casuale nel senso su detto.  Le propriet\`a
  che contraddistinguono una funzione che rappresenti una
  densit\`a di probabilit\`a implicano poi che la funzione
  caratteristica, a differenza della funzione generatrice
  dei momenti, \emph{esista sempre} per qualsiasi variabile
  casuale; la \eqref{eq:6.fucamo} \`e per\`o valida solo se
  i momenti esistono fino a qualsiasi ordine.  Inoltre, se
  \`e nota la $\phi_x(t)$, \emph{la si pu\`o sempre
    invertire} (riottenendo da essa la $f$) attraverso la
  \begin{equation} \label{eq:6.trinfo}
    \boxed{ \rule[-6mm]{0mm}{14mm} \quad
      f(x) \; = \; \frac{1}{2\pi}
        \int_{-\infty}^{+\infty} \! e^{-ixt} \phi_x(t)
        \, \de t \quad }
  \end{equation}
  (\emph{trasformata inversa di Fourier}).%
  \index{Fourier, trasformata di}
\end{itemize}

Vogliamo infine ricavare una relazione che ci sar\`a utile
pi\`u avanti: siano le $N$ variabili casuali continue $x_k$
(che supponiamo tutte statisticamente indipendenti tra
loro), ognuna delle quali sia associata ad una particolare
funzione caratteristica $\phi_k(t)$; il problema che
vogliamo affrontare consiste nel determinare la funzione
caratteristica della nuova variabile casuale $S$, definita
come loro somma:
\begin{equation*}
  S = \sum_{k=1}^N x_k \peq .
\end{equation*}

Il valore di ogni $x_k$ sar\`a univocamente definito dai
possibili risultati di un qualche evento casuale $E_k$; per
cui la $S$ si pu\`o pensare univocamente definita dalle
possibili \emph{associazioni} di tutti i risultati di questi
$N$ eventi --- associazioni che, in sostanza, corrispondono
alle possibili posizioni di un punto in uno spazio
cartesiano $N$-dimensionale, in cui ognuna delle variabili
$x_k$ sia rappresentata su uno degli assi.

\index{funzione!caratteristica!di somme di variabili|(}%
Visto che i valori $x_k$ sono (per ipotesi) tra loro tutti
statisticamente indipendenti, la probabilit\`a di ottenere
una particolare $N$-pla \`e data dal prodotto delle
probabilit\`a relative ad ogni singolo valore: e, se
indichiamo con $f_k(x_k)$ la funzione densit\`a di
probabilit\`a della generica $x_k$, la probabilit\`a di
ottenere un determinato valore per la $S$ \`e data da
\begin{equation*}
  \de P \; \equiv \; g(S) \, \de S \; = \;
    \prod_{k=1}^N f_k(x_k) \, \de x_k
\end{equation*}
($\de S$ rappresenta un intorno (ipercubico) infinitesimo
del punto $S$, di coordinate cartesiane $\{ x_k \}$ nello
spazio $N$-dimensionale prima descritto, corrispondente agli
$N$ intorni unidimensionali $\de x_k$ dei valori assunti
dalle $N$ variabili casuali $x_k$); e la densit\`a di
probabilit\`a per la $S$ vale quindi
\begin{equation*}
  g(S) = \prod_{k=1}^N f_k(x_k) \peq .
\end{equation*}

La funzione caratteristica di $S$ \`e, dall'equazione di
definizione \eqref{eq:6.funcar},
\begin{align*}
  \phi_S(t) &= \int_{-\infty}^{+\infty} \! e^{itS} g(S)
    \, \de S \\[1ex]
  &= \int_{-\infty}^{+\infty} \prod_{k=1}^N
    e^{itx_k} f_k(x_k) \, \de x_k
\end{align*}
ed infine
\begin{equation} \label{eq:6.fucacl}
  \boxed{ \rule[-6mm]{0mm}{14mm} \quad
    \phi_S(t) = \prod_{k=1}^N \phi_k(t)
    \quad }
\end{equation}
Quindi \emph{la funzione caratteristica della somma di $N$
  variabili casuali statisticamente indipendenti \`e pari al
  prodotto delle loro funzioni caratteristiche}.%
\index{funzione!caratteristica!di somme di variabili|)}

\subsection{Funzioni caratteristiche di variabili discrete}%
\index{funzione!caratteristica!per variabili discrete|(}
Invece della funzione caratteristica definita attraverso la
\eqref{eq:6.fucadi}, e che \`e una funzione complessa di
variabile reale, talvolta, per variabili casuali discrete,
viene usata una rappresentazione equivalente ricorrendo alla
variabile complessa
\begin{gather*}
  z = e^{it} \peq . \\
  \intertext{Sostituendo questa definizione di $z$ nella
    \eqref{eq:6.fucadi} si ottiene la \emph{funzione
      caratteristica di variabile complessa}}
  \phi_x (z) \; = \; \sum\nolimits_k p_k \, z^{x_k} \; = \;
  E \bigl( z^x \bigr) \peq ,
\end{gather*}
che ha propriet\`a analoghe a quelle della funzione
caratteristica di variabile reale $\phi_x(t)$.  In
particolare, definendo una variabile casuale $w$ come somma
di due altre variabili $x$ e $y$ discrete e tra loro
indipendenti, la funzione caratteristica di variabile
complessa $\phi_w(z)$ \`e ancora il prodotto delle due
funzioni caratteristiche $\phi_x(z)$ e $\phi_y(z)$: infatti
\begin{align*}
  \phi_w (z) &= \sum\nolimits_{jk} \Pr ( x_j ) \,
  \Pr ( y_k ) \, z^{\left( x_j + y_k \right)}
  \\[1ex]
  &= \sum\nolimits_j \Pr ( x_j ) \, z^{x_j} \cdot
  \sum\nolimits_k \Pr ( y_k ) \, z^{y_k} \\[1ex]
  &= \phi_x(z) \cdot \phi_y(z) \peq ;
\end{align*}
e, generalizzando per induzione completa, la somma $S$ di un
numero prefissato $N$ di variabili casuali discrete e tutte
tra loro indipendenti
\begin{gather*}
  S = \sum_{k=1}^N x_k \\
  \intertext{\`e anch'essa associata alla funzione
    caratteristica di variabile complessa}
  \phi_S (z) = \prod_{k=1}^N \phi_{x_k} (z) \peq .
\end{gather*}
Nel caso particolare, poi, in cui le $N$ variabili
provengano dalla stessa popolazione,
\begin{equation} \label{eq:6.fixedn}
  \phi_S (z) = \bigl[ \phi_x (z) \bigr] ^N \peq .
\end{equation}

\index{somma di un numero casuale di variabili discrete|(}%
Cosa accade se il numero $N$ di variabili casuali da sommare
non \`e costante, \emph{ma \`e anch'esso una variabile
  casuale} (ovviamente discreta)?  In altre parole, vogliamo
qui di seguito trovare la rappresentazione analitica della
funzione caratteristica della \emph{somma di un numero
  casuale di variabili casuali discrete, indipendenti ed
  aventi tutte la stessa distribuzione}.  Supponiamo che la
$N$ sia associata ad una funzione caratteristica
\begin{equation} \label{eq:6.nonly}
  \phi_N (z) \; = \; E \bigl( z^N \bigr) \; = \;
  \sum\nolimits_N \Pr (N) \, z^N \peq ;
\end{equation}
la probabilit\`a di ottenere un determinato valore per la
$S$ vale
\begin{gather*}
  \Pr (S) = \sum\nolimits_N \Pr (N) \, \Pr (S | N) \\
  \intertext{e di conseguenza la funzione caratteristica
    di variabile complessa associata alla $S$ che, per
    definizione, \`e data dalla}
  \phi_S (z) \; = \; E \bigl( z^S \bigr) \; = \;
  \sum\nolimits_S \Pr (S) \, z^S
\end{gather*}
si potr\`a scrivere anche
\begin{align*}
  \phi_S (z) &= \sum\nolimits_S z^S \cdot \sum\nolimits_N
  \Pr (N) \, \Pr (S | N) \\[1ex]
  &= \sum\nolimits_N \Pr (N) \cdot \sum\nolimits_S \Pr
  (S | N) \, z^S \\[1ex]
  &= \sum\nolimits_N \Pr (N) \cdot \bigl[ \phi_x (z)
  \bigr]^N \peq .
\end{align*}
Nell'ultimo passaggio si \`e sfruttato il fatto che la
sommatoria su $S$ rappresenta la speranza matematica di
$z^S$ \emph{condizionata dall'avere assunto $N$ un
  determinato valore}; rappresenta quindi la funzione
caratteristica della $S$ quando $N$ ha un valore costante
prefissato, che appunto \`e data dalla \eqref{eq:6.fixedn}.

Ricordando poi la \eqref{eq:6.nonly}, la funzione
caratteristica cercata \`e infine data dalla \emph{funzione
  di funzione}
\begin{equation} \label{eq:6.varn}
  \boxed{ \rule[-5mm]{0mm}{12mm} \quad
    \phi_S (z) = \phi_N \bigl[ \phi_x (z) \bigr] \quad }
 \end{equation}
 \`E immediato riconoscere che, se $N$ non \`e propriamente
 una variabile casuale e pu\`o assumere un unico valore
 $N_0$, essendo tutte le $\Pr(N)$ nulle meno $\Pr(N_0) = 1$,
\begin{gather*}
  \phi_N (z) = z^{N_0} \\
  \intertext{e}
  \phi_S (z) \; = \; \phi_N \bigl[ \phi_x (z) \bigr] \; =
  \; \bigl[ \phi_x (z) \bigr]^{N_0}
\end{gather*}
e la \eqref{eq:6.varn} ridiventa la meno generale
\eqref{eq:6.fixedn}.%
\index{somma di un numero casuale di variabili discrete|)}%
\index{funzione!caratteristica!per variabili discrete|)}

\section{Cambiamento di variabile casuale}%
\index{cambiamento di variabile casuale|(}
Supponiamo sia nota la funzione $f(x)$ densit\`a di
probabilit\`a della variabile casuale $x$; e sia $y$ una
nuova variabile casuale definita in funzione della $x$
attraverso una qualche relazione matematica $y=y(x)$.  Ci
proponiamo di vedere come, da queste ipotesi, si possa
ricavare la densit\`a di probabilit\`a $g(y)$ della nuova
variabile $y$.

Supponiamo dapprima che la corrispondenza tra le due
variabili continue sia \emph{biunivoca}: ossia che la
$y=y(x)$ sia una funzione monotona in senso stretto,
crescente o decrescente, e di cui quindi esista la funzione
inversa che indicheremo con $x = x(y)$; ed inoltre
supponiamo che la $y(x)$ sia derivabile.  Questo, dovendo
risultare $y'(x) \ne 0$ in conseguenza dell'ipotesi fatta,
implica che sia derivabile anche la $x(y)$ e che risulti
\begin{equation*}
  x'(y) \; = \; \frac{1}{y' \left[ x(y) \right]} \peq .
\end{equation*}

L'asserita biunivocit\`a della corrispondenza tra le due
variabili assicura che, se la prima \`e compresa in un
intervallo infinitesimo di ampiezza $\de x$ centrato sul
generico valore $x$, allora e solo allora la seconda \`e
compresa in un intervallo di ampiezza $\de y = \left| y'(x)
\right| \de x$ (il valore assoluto tiene conto del fatto che
la $y(x)$ pu\`o essere sia crescente che decrescente)
centrato attorno al valore $y=y(x)$.  Questo a sua volta
implica che le probabilit\`a (infinitesime) degli eventi
casuali consistenti nell'essere la $x$ o la $y$ appartenenti
a tali intervalli debbano essere uguali: ossia che risulti
\begin{gather}
  f(x) \, \de x \; = \; g(y) \, \de y \; = \; g(y)
    \left| y'(x) \right| \de x \notag \\
  \intertext{identicamente per ogni $x$, il che \`e
    possibile soltanto se}
  g(y) \; = \; \frac{f(x)}{\left| y'(x) \right|} \; =
    \; \frac{f \left[ x(y) \right]}{\left| y' \left[
    x(y) \right] \right|} \; = \; f \left[ x(y) \right]
    \cdot \left| x'(y) \right| \peq . \label{eq:6.cavaun}
\end{gather}

Se la relazione che lega $y$ ad $x$ non \`e invece
biunivoca, i ragionamenti sono pi\`u complicati e devono
essere fatti tenendo conto della natura della particolare
funzione in esame; ad esempio, se
\begin{gather*}
  y \; = \; x^2 \makebox[50mm]{e quindi} x \; = \; \pm
    \sqrt{y}\\
  \intertext{un particolare valore per la $y$
    corrisponde a \emph{due} eventualit\`a (mutuamente
    esclusive) per la $x$; perci\`o}
  g(y) \, \de y \; = \; \bigl[ f ( -\sqrt{y} ) + f( \sqrt{y}
    ) \bigr] \, \de x \\
  \intertext{e quindi}
  g(y) \; = \; \bigl[ f \left( - \sqrt{y} \right) + f
    \left( \sqrt{y} \right) \bigr] \cdot \bigl| x'(y)
    \bigr| \; = \; \frac{ f \left( - \sqrt{y} \right)
    + f \left( \sqrt{y} \right) }{2 \sqrt{y}} \peq .
\end{gather*}%
\index{cambiamento di variabile casuale|)}

\index{funzione!generatrice dei momenti!per trasformazioni lineari|(}%
\index{funzione!caratteristica!per trasformazioni lineari|(}%
Per quello che riguarda la funzione generatrice dei momenti
e la funzione caratteristica associate a variabili casuali
definite l'una in funzione dell'altra, se ci limitiamo a
considerare una \emph{trasformazione lineare} del tipo
$y=ax+b$, vale la
\begin{align*}
  M_y(t) &= E \left( e^{ty} \right) \\[1ex]
  &= E \left[ e^{t \left( ax+b \right)} \right] \\[1ex]
  &= e^{tb} \, E \bigl( e^{tax} \bigr)
\end{align*}
da cui infine ricaviamo la
\begin{gather}
  \boxed{ \rule[-6mm]{0mm}{14mm} \quad
    M_y(t) = e^{tb} \, M_x(at) \quad }
    \label{eq:6.fgmcav} \\
  \intertext{per la funzione generatrice dei momenti; e
    potremmo ricavare l'analoga}
  \boxed{ \rule[-6mm]{0mm}{14mm} \quad
    \phi_y(t) = e^{itb} \, \phi_x(at) \quad }
    \label{eq:6.fuccav}
\end{gather}
per la funzione caratteristica (si confronti anche la
funzione \eqref{eq:6.fugemm}, prima usata per ricavare i
momenti rispetto alla media, e che si pu\`o pensare ottenuta
dalla \eqref{eq:6.fugemo} applicando alla variabile casuale
una traslazione che ne porti il valore medio nell'origine).%
\index{funzione!caratteristica!per trasformazioni lineari|)}%
\index{funzione!generatrice dei momenti!per trasformazioni lineari|)}

\section{I valori estremi di un campione}%
\index{campione!valori estremi|(emidx}%
\label{ch:6.estremi}
Sia $x$ una variabile casuale continua, di cui siano note
sia la funzione di frequenza $f(x)$ che la funzione di
distribuzione $F(x)$; e sia disponibile un campione di
dimensione $N$ di valori \emph{indipendenti} di questa
variabile casuale.  Supponiamo inoltre, una volta ottenuti
tali valori, di averli disposti \emph{in ordine crescente}:
ovvero in modo che risulti $x_1 \le x_2 \le \cdots \le x_N$.
Vogliamo qui, come esercizio, determinare la funzione di
frequenza \emph{del generico di questi valori ordinati},
$x_i$: funzione che verr\`a nel seguito identificata dal
simbolo $f_i(x)$.

Supponiamo che $x_i$ sia compreso nell'intervallo
infinitesimo $ [ x, x+\de x ] $; la scelta di un certo $i$
divide naturalmente il campione (ordinato) in tre
sottoinsiemi, ovvero:
\begin{enumerate}
\item $x_i$ stesso, che pu\`o essere ottenuto (dall'insieme
  non ordinato dei valori originariamente a disposizione) in
  $N$ maniere differenti; si sa inoltre che \`e compreso
  nell'intervallo $[x, x+\de x]$ --- evento, questo, che
  avviene con probabilit\`a $f(x) \, \de x$.
\item I primi $(i-1)$ valori: questi possono essere
  ottenuti, dagli $N-1$ elementi restanti dall'insieme non
  ordinato dei valori originari, in $C^{N-1}_{i-1}$ modi
  distinti\/\footnote{$C^N_K$ \`e il numero delle
    combinazioni di classe $K$ di $N$ oggetti; si veda in
    proposito il paragrafo \ref{ch:a.combina}.}; ognuno di
  essi \`e inoltre minore di $x$, e questo avviene con
  probabilit\`a data da $F(x)$.
\item I residui $(N-i)$ valori: questi sono univocamente
  determinati dalle due scelte precedenti; inoltre ognuno di
  essi \`e maggiore di $x$, e questo avviene con
  probabilit\`a $\bigl[ 1 - F(x) \bigr]$.
\end{enumerate}
In definitiva, applicando i teoremi della probabilit\`a
totale e della probabilit\`a composta, possiamo affermare
che risulta
\begin{equation} \label{eq:6.iesimo}
  f_i(x) \, \de x \; = \; N \: \binom{N-1}{i-1} \: \bigl[
  F(x) \bigr]^{i - 1} \, \bigl[ 1 - F(x) \bigr]^{N - i} \,
  f(x) \,  \de x \peq ;
\end{equation}
in particolare, i valori estremi $x_1$ e $x_N$ hanno
densit\`a di probabilit\`a date da
\begin{gather*}
  f_1(x) \; =  \;N \, \bigl[ 1 - F(x) \bigr]^{N - 1} \, f(x)
  \\
  \intertext{e da}
  f_N(x) \; = \; N \, \bigl[ F(x) \bigr]^{N - 1} \, f(x)
  \peq .
\end{gather*}%
\index{campione!valori estremi|)}

\endinput

% $Id: chapter7.tex,v 1.1 2005/03/01 10:06:08 loreti Exp $

\chapter{Variabili casuali pluridimensionali}
Pu\`o avvenire che un evento casuale complesso $E$ sia
decomponibile in $N$ eventi semplici $E_i$, ognuno dei quali
a sua volta sia descrivibile mediante una variabile casuale
$x_i$ (che supporremo continua); le differenti modalit\`a
dell'evento $E$ si possono allora associare univocamente
alla $N$-pla dei valori delle $x_i$, ossia alla posizione di
un punto in uno spazio cartesiano $N$-dimensionale.

\section{Variabili casuali bidimensionali}%
\label{ch:7.bidim}
Nel caso multidimensionale pi\`u semplice, $N = 2$, se
supponiamo che la probabilit\`a $\de P$ per la coppia di
variabili casuali $x$ ed $y$ di trovarsi nell'intorno
(infinitesimo) di una certo punto dello spazio
bidimensionale sia proporzionale all'ampiezza dell'intorno
stesso e dipenda dalla sua posizione, possiamo definire la
\emph{densit\`a di
  probabilit\`a}%
\index{probabilit\`a!densit\`a di}
(o \emph{funzione di frequenza}) \emph{congiunta}, $f(x,y)$,
attraverso la
\begin{equation*}
  \de P = f(x,y) \: \de x \, \de y \peq ;
\end{equation*}
e, analogamente a quanto fatto nel caso unidimensionale,
definire poi attraverso di essa altre funzioni.  Ad esempio
la \emph{funzione di distribuzione congiunta}%
\index{funzione!di distribuzione},
\begin{gather*}
  F(x,y) = \int_{-\infty}^x \! \! \de u \int_{-\infty}^y \!
  \de v \: f(u,v) \\
    \intertext{che d\`a la probabilit\`a di ottenere valori
    delle due variabili non superiori a quantit\`a
    prefissate; le \emph{funzioni di frequenza marginali}%
    \index{probabilit\`a!funzione marginale}}
  g(x) = \int_{-\infty}^{+\infty} \! f(x,y) \, \de y
  \makebox[20mm]{e}
  h(y) = \int_{-\infty}^{+\infty} \! f(x,y) \, \de x
  \intertext{che rappresentano la densit\`a di probabilit\`a
    di ottenere un dato valore per una delle due variabili
    \emph{qualunque sia il valore assunto dall'altra}; ed
    infine le \emph{funzioni di distribuzione marginali}}
  G(x) = \int_{-\infty}^x \! \! g(t) \, \de t = F(x,
  +\infty)
  \makebox[20mm]{e}
  H(y) = \int_{-\infty}^y \! \! h(t) \, \de t = F(+\infty,
  y) \peq . \\
  \intertext{La \emph{condizione di normalizzazione}%
    \index{normalizzazione!condizione di}
    si potr\`a poi scrivere}
  F(+\infty, +\infty) = 1 \peq .
\end{gather*}

\index{probabilit\`a!condizionata|(}%
Per un insieme di due variabili si possono poi definire le
funzioni di frequenza \emph{condizionate}, $\pi(x|y)$ e
$\pi(y|x)$; esse rappresentano la densit\`a di probabilit\`a
dei valori di una variabile quando gi\`a si conosce il
valore dell'altra.  Per definizione deve valere la
\begin{gather*}
  f(x,y) \: \de x \, \de y \; = \; g(x) \, \de x \cdot
  \pi(y|x) \, \de y \; = \; h(y) \, \de y \cdot \pi(x|y) \,
  \de x \\
  \intertext{per cui tra probabilit\`a condizionate,
    marginali e congiunte valgono la}
  \pi(y|x) = \frac{ f(x,y) }{ g(x) }
  \makebox[35mm]{e la}
  \pi(x|y) = \frac{ f(x,y) }{ h(y) } \peq .
\end{gather*}
Due variabili casuali sono, come sappiamo, statisticamente
indipendenti tra loro quando il fatto che una di esse abbia
un determinato valore non altera le probabilit\`a relative
ai valori dell'altra: ovvero quando
\begin{gather}
  \pi(x|y) = g(x)
  \makebox[30mm]{e}
  \pi(y|x) = h(y) \peq ; \label{eq:7.istat1} \\
  \intertext{e questo a sua volta implica che}
  f(x,y) = g(x) \cdot h(y) \label{eq:7.istat2}
\end{gather}
Non \`e difficile poi, assunta vera la \eqref{eq:7.istat2},
giungere alla \eqref{eq:7.istat1}; in definitiva:
\index{statistica!indipendenza|(}%
\begin{quote}
  \textit{Due variabili casuali continue sono
    statisticamente indipendenti tra loro se e solo se la
    densit\`a di probabilit\`a congiunta \`e fattorizzabile
    nel prodotto delle funzioni marginali.}
\end{quote}%
\index{statistica!indipendenza|)}%
\index{probabilit\`a!condizionata|)}

\subsection{Momenti, funzione caratteristica e funzione
  generatrice}
Analogamente a quanto fatto per le variabili casuali
unidimensionali, in uno spazio degli eventi bidimensionale
in cui rappresentiamo le due variabili $\{x,y\}$ aventi
densit\`a di probabilit\`a congiunta $f(x,y)$, si
pu\`o definire la speranza matematica%
\index{speranza matematica!per variabili continue}
(o valore medio) di una qualunque funzione $\psi(x,y)$ come
\begin{gather*}
  E \bigl[ \psi(x,y) \bigr] = \int_{-\infty}^{+\infty} \!
  \! \de x \int_{-\infty}^{+\infty} \! \! \de y \: \psi(x,y)
  \, f(x,y) \peq ; \\
  \intertext{i momenti%
    \index{momenti|(}
    rispetto all'origine come}
  \lambda_{mn} = E \left( x^m \, y^n \right) \\
  \intertext{e quelli rispetto alla media come}
  \mu_{mn} = E \bigl[ ( x - \lambda_{10} )^m ( y -
  \lambda_{01} )^n \bigr] \peq .
\end{gather*}%
\index{momenti|)}

Risulta ovviamente:
\begin{align*}
  \lambda_{00} \; &\equiv \; 1 \\[1ex]
  \lambda_{10} \; &= \; E(x) \\[1ex]
  \lambda_{01} \; &= \; E(y) \\[1ex]
  \mu_{20} \; &= \; E \left\{ \bigl[ x - E(x) \bigr]^2
  \right\} \; = \; \var (x) \\[1ex]
  \mu_{02} \; &= \; E \left\{ \bigl[ y - E(y) \bigr]^2
  \right\} \; = \; \var (y) \\[1ex]
  \mu_{11} \; &= \; E \Bigl\{ \bigl[ x - E(x) \bigr] \bigl[
    y - E(y) \bigr] \Bigr\}
\end{align*}
La quantit\`a $\mu_{11}$ si chiama anche \emph{covarianza}%
\index{covarianza}
di $x$ ed $y$; si indica generalmente col simbolo
$\cov(x,y)$, e di essa ci occuperemo pi\`u in dettaglio
nell'appendice \ref{ch:c.covcor} (almeno per quel che
riguarda le variabili discrete).  Un'altra grandezza
collegata alla covarianza \`e il cosiddetto
\emph{coefficiente di correlazione lineare},%
\index{correlazione lineare, coefficiente di}
che si indica col simbolo $r_{xy}$ (o, semplicemente, con
$r$): \`e definito come
\begin{equation*}
  r_{xy} \; = \; \frac{ \mu_{11} }{ \sqrt{\mu_{20} \:
      \mu_{02}} } \; = \; \frac{ \cov(x,y) }{ \sigma_x \,
    \sigma_y } \peq ,
\end{equation*}
e si tratta di una grandezza adimensionale compresa, come
vedremo, nell'intervallo $[ -1, +1 ]$.  Anche del
coefficiente di correlazione lineare ci occuperemo
estesamente pi\`u avanti, e sempre nell'appendice
\ref{ch:c.covcor}.

La funzione caratteristica per due variabili,%
\index{funzione!caratteristica}
che esiste sempre, \`e la
\begin{gather*}
  \phi_{xy} (u,v) = E \left[ e^{ i (ux+vy) } \right] \peq ;
  \\
  \intertext{se poi esistono tutti i momenti, vale anche la}
  \left. \frac{ \partial^{m+n} \phi_{xy} }{ \partial u^m \,
      \partial v^n } \right|_{\substack{u=0\\ v=0}} =
  (i)^{ m+n } \, \lambda_{mn} \peq . \\
  \intertext{La funzione generatrice,%
    \index{funzione!generatrice dei momenti}
    che esiste solo se tutti i momenti esistono, \`e poi
    definita come}
  M_{xy} (u,v) = E \left[ e^{ ( ux + vy ) } \right] \\
  \intertext{e per essa vale la}
  \left. \frac{ \partial^{m+n} M_{xy} }{ \partial u^m \,
      \partial v^n } \right|_{\substack{u=0\\ v=0}} =
  \lambda_{mn} \peq .
\end{gather*}

\subsection{Cambiamento di variabile casuale}%
\index{cambiamento di variabile casuale|(}
Supponiamo di definire due nuove variabili casuali $u$ e $v$
per descrivere un evento casuale collegato a due variabili
continue $x$ ed $y$; e questo attraverso due funzioni
\begin{gather*}
  u = u(x,y) \makebox[30mm]{e} v = v(x,y) \peq . \\
  \intertext{\emph{Se} la corrispondenza tra le due coppie
    di variabili \`e biunivoca, esistono le funzioni
    inverse}
  x = x(u,v) \makebox[30mm]{e} y = y(u,v) \peq ; \\
  \intertext{\emph{se} inoltre esistono anche le derivate
    parziali prime della $x$ e della $y$ rispetto alla $u$
    ed alla $v$, esiste anche non nullo il
    \emph{determinante Jacobiano}%
    \index{Jacobiano!determinante|(}}
  \frac{ \partial (x,y) }{ \partial (u,v) } = \det \left\|
    \begin{array}{cc}
      \dfrac{ \partial x }{ \partial u } & \dfrac{ \partial
        x }{ \partial v } \\[3ex]
      \dfrac{ \partial y }{ \partial u } & \dfrac{ \partial
        y }{ \partial v }
    \end{array} \right\| \\
  \intertext{dotato della propriet\`a che}
  \frac{ \partial (x,y) }{ \partial (u,v) } = \left[ \frac{
      \partial (u,v) }{ \partial (x,y) } \right]^{-1}
\end{gather*}%
\index{Jacobiano!determinante|)}

In tal caso, dalla richiesta di invarianza della
probabilit\`a sotto il cambiamento di variabili,
\begin{gather}
  f(x,y) \: \de x \, \de y = g(u,v) \: \de u \, \de v \notag
  \\
  \intertext{si ottiene la funzione densit\`a di
    probabilit\`a congiunta per $u$ e $v$, che \`e legata
    alla $f(x,y)$ dalla}
  g(u,v) = f \left[ x(u,v), y(u,v) \right] \cdot \left|
    \frac{\partial (x,y)}{\partial (u,v)} \right|
  \label{eq:7.cavar2}
\end{gather}%
\index{cambiamento di variabile casuale|)}

\subsection{Applicazione: il rapporto di due variabili
  casuali indipendenti}%
\index{rapporto di variabili|(}
Come esempio, consideriamo due variabili casuali $x$ ed $y$
indipendenti tra loro e di cui si conoscano le funzioni di
frequenza, rispettivamente $f(x)$ e $g(y)$; e si sappia
inoltre che la $y$ non possa essere nulla.  Fatte queste
ipotesi, useremo la formula precedente per calcolare la
funzione di frequenza $\varphi(u)$ della variabile casuale
$u$ rapporto tra $x$ ed $y$.  Definite
\begin{gather*}
  u = \dfrac{x}{y} \makebox[30mm]{e} v = y \peq , \\
  \intertext{la corrispondenza tra le coppie di variabili
    \`e biunivoca; e le funzioni inverse sono la}
  x = uv \makebox[35mm]{e la} y = v \peq . \\
  \intertext{Le funzioni di frequenza congiunte delle due
    coppie di variabili sono, ricordando la
    \eqref{eq:7.istat2} e la \eqref{eq:7.cavar2}}
  f(x,y) = f(x) \, g(y) \makebox[30mm]{e}
  \varphi(u,v) =  f(x) \, g(y) \left| \frac{ \partial (x,y) }{
      \partial (u,v) } \right| \\
  \intertext{rispettivamente; e, calcolando le derivate
    parziali,}
  \frac{ \partial (x,y) }{ \partial (u,v) } = \det \left\|
    \begin{array}{cc}
      v & u \\[1ex]
      0 & 1
    \end{array} \right\| \\
  \intertext{per cui}
  \varphi(u,v) = f(uv) \, g(v) \, |v| \peq .
\end{gather*}
In conclusione, la funzione di distribuzione della sola $u$
(la funzione marginale) \`e la
\begin{equation} \label{eq:7.rapvar}
  \varphi(u) \; = \; \int_{-\infty}^{+\infty} \!
  \varphi(u,v) \, \de v \; = \; \int_{-\infty}^{+\infty} \!
  f(uv) \, g(v) \, |v| \, \de v
\end{equation}%
\index{rapporto di variabili|)}

\subsection[Applicazione: il decadimento debole della
  $\Lambda^0$]{Applicazione: il decadimento debole della
  $\boldsymbol{\Lambda}^{\boldsymbol{0}}$}
La particella elementare $\Lambda^0$ decade, attraverso
processi governati dalle interazioni deboli, nei due canali
\begin{align*}
  \Lambda^0 &\to p + \pi^- &&\text{e} & \Lambda^0 &\to n +
  \pi^0 \peq ;
\end{align*}
il suo decadimento \`e quindi un evento casuale che pu\`o
essere descritto dalle due variabili $c$ (carica del
nucleone nello stato finale, 1 o 0 rispettivamente) e $t$
(tempo di vita della $\Lambda^0$).

La teoria (confermata dagli esperimenti) richiede che la
legge di decadimento sia la stessa per entrambi gli stati
finali, ovvero esponenziale con la stessa vita media%
\index{vita media}
$\tau$; e che il cosiddetto \emph{branching ratio},%
\index{branching ratio}
cio\`e il rapporto delle probabilit\`a di decadimento nei
due canali citati, sia indipendente dal tempo di vita e
valga
\begin{equation*}
  \frac{\Pr \left( \Lambda^0 \to p + \pi^- \right)}{\Pr
    \left( \Lambda^0 \to n + \pi^0 \right)} = 2 \peq .
\end{equation*}

In altre parole, le probabilit\`a marginali e condizionate
per le due variabili (una discreta, l'altra continua) devono
essere: per la $c$
\begin{align*}
  g(1) &= g(1|t) = \frac{2}{3} &&\text{e} &
  g(0) &= g(0|t) = \frac{1}{3}
\end{align*}
o, in maniera compatta,
\begin{gather*}
  g(c) = g(c|t) = \frac{c + 1}{3} \peq ; \\
  \intertext{per il tempo di vita $t$,}
  h(t) = h(t|0) = h(t|1) = \frac{1}{\tau} \, e^{-
    \frac{t}{\tau} } \peq .
  \intertext{La probabilit\`a congiunta delle due variabili
    casuali \`e, infine,}
  f(c, t) = g(c) \cdot h(t) = \frac{c + 1}{3 \tau} \, e^{-
    \frac{t}{\tau} } \peq .
\end{gather*}

\subsection[Applicazione: il decadimento debole
  $K^0_{e3}$]{Applicazione: il decadimento debole
  $\boldsymbol{K}^{\boldsymbol{0}}_{\boldsymbol{e3}}$}
I decadimenti $K^0_{e3}$ consistono nei due processi deboli
di decadimento del mesone $K^0$
\begin{align*}
  K^0 &\to e^- + \pi^+ + \bar \nu_e &&\text{e} & K^0 &\to
  e^+ + \pi^- + \nu_e \peq ;
\end{align*}
essi possono essere descritti dalle due variabili casuali
$c$ (carica dell'elettrone nello stato finale, $c = \mp 1$)
e $t$ (tempo di vita del $K^0$).  La teoria, sulla base
della cosiddetta ``ipotesi $\Delta Q = \Delta S$''), prevede
che la funzione di frequenza congiunta sia
\begin{gather}
  f(t, c) = \frac{ N(t, c) }{ \sum_c \int_0^{+\infty}  N(t,
    c) \, \de t} \notag \\
  \intertext{ove si \`e indicato con $N(t,c)$ la funzione}
  N(t, c) = e^{- \lambda_1 t} + e^{- \lambda_2 t} + 2 c
  \cos(\omega t) \, e^{- \frac{\lambda_1 + \lambda_2}{2} \,
    t} \peq : \label{eq:7.kappae3}
\end{gather}
nella \eqref{eq:7.kappae3}, le costanti $\lambda_1$ e
$\lambda_2$ rappresentano gli inversi delle vite medie dei
mesoni $K^0_1$ e $K^0_2$, mentre $\omega$ corrisponde alla
differenza tra le loro masse.

Si vede immediatamente che la \eqref{eq:7.kappae3} non \`e
fattorizzabile: quindi le due variabili \emph{non sono} tra
loro indipendenti.  In particolare, le probabilit\`a
marginali sono date dalla
\begin{align*}
  h(t) &= \frac{\lambda_1 \, \lambda_2}{\lambda_1 +
    \lambda_2} \, \left( e^{- \lambda_1 t} + e^{- \lambda_2
      t} \right) \\
  \intertext{e dalla}
  g(c) &=  \frac{1}{2} + \frac{4 \, c \, \lambda_1
    \lambda_2}{( \lambda_1 + \lambda_2 )^2 + 4 \omega^2}
  \peq ;
\end{align*}
mentre le probabilit\`a condizionate sono, da definizione,
la
\begin{align*}
  h(t|c) &= \frac{ f(t, c) }{ g(c) } &&\text{e la} &
  g(c|t) &= \frac{ f(t, c) }{ h(t) } \peq .
\end{align*}

\subsection{Ancora sui valori estremi di un campione}%
\index{campione!valori estremi|(}%
\label{ch:7.estremi}
Come esempio, e ricordando il paragrafo \ref{ch:6.estremi},
calcoliamo la densit\`a di probabilit\`a congiunta dei due
valori estremi $x_1$ e $x_N$ di un campione \emph{ordinato}
e di dimensione $N$; questo sotto l'ipotesi che i dati
appartengano a una popolazione avente funzione di frequenza
$f(x)$ e funzione di distribuzione $F(x)$ entrambe note.

$x_1$ pu\`o provenire dal campione a disposizione in $N$
maniere distinte; una volta noto $x_1$, poi, $x_N$ pu\`o
essere scelto in $(N-1)$ modi diversi; e, infine, ognuno dei
dati restanti \`e compreso tra $x_1$ e $x_N$: e questo
avviene con probabilit\`a $\bigl[ F(x_N) - F(x_1) \bigr]$.
Ripetendo i ragionamenti del paragrafo \ref{ch:6.estremi},
si ricava
\begin{equation} \label{eq:7.estremi}
  f(x_1, x_N) \; = \; N \, (N - 1) \, \bigl[ F(x_N) - F(x_1)
  \bigr]^{N - 2} \, f(x_1) \, f(x_N)
\end{equation}
che \emph{non \`e fattorizzabile}: quindi i valori minimo e
massimo di un campione \emph{non} sono indipendenti tra
loro.  Introducendo le variabili ausiliarie
\begin{align*}
  &\begin{cases}
    \displaystyle \xi = N \cdot F(x_1) \\[1ex]
    \displaystyle \eta = N \cdot \bigl[ 1 - F(x_N) \bigr]
  \end{cases}
  &&\text{con}
  &&\begin{cases}
    \displaystyle \de \xi = N \, f(x_1) \: \de x_1
    \\[1ex]
    \displaystyle \de \eta = - N \, f(x_N) \: \de x_N
  \end{cases}
\end{align*}
ed essendo $F(x_N) - F(x_1)$ identicamente uguale a $1 -
\bigl[ 1 - F(x_N) \bigr] - F(x_1)$, dalla
\eqref{eq:7.estremi} si ricava
\begin{gather*}
  f(\xi, \eta) \; = \; \frac{N-1}{N} \, \left( 1 -
    \frac{\xi + \eta}{N} \right)^{N - 2} \\
  \intertext{che, ricordando il limite notevole}
  \lim_{x \to +\infty} \left( 1 + \frac{k}{x} \right)^x \; =
  \; e^k \peq , \\
  \intertext{asintoticamente diventa}
  f(\xi, \eta) \quad \xrightarrow{\quad N \to \infty \quad}
  \quad e^{- \left( \xi + \eta \right)} \; \equiv \; e^{-
    \xi} \, e^{- \eta} \peq .
\end{gather*}
Quindi $\xi$ ed $\eta$ (come anche di conseguenza $x_1$ e
$x_N$) sono statisticamente indipendenti \emph{solo
  asintoticamente}, all'aumentare indefinito della
dimensione del campione.%
\index{campione!valori estremi|)}

\section{Cenni sulle variabili casuali in pi\`u di due
  dimensioni}%
\index{probabilit\`a!densit\`a di|(}
Estendendo a spazi cartesiani a pi\`u di due dimensioni il
concetto di densit\`a di probabilit\`a, possiamo pensare di
associare ad un evento casuale $E$ descritto da $N$
variabili continue $x_1, x_2, \ldots, x_N$ una funzione $f$
di tutte queste variabili; la probabilit\`a che,
simultaneamente, ognuna di esse cada in un intervallo
infinitesimo attorno ad un determinato valore sar\`a poi
data da
\begin{equation*}
  \de P = f(x_1, x_2,\ldots, x_N) \, \de x_1 \, \de x_2
    \cdots \de x_N \peq .
\end{equation*}%
\index{probabilit\`a!densit\`a di|)}

Usando la legge della probabilit\`a totale e la definizione
dell'operazione di integrazione, \`e poi immediato
riconoscere che la probabilit\`a dell'evento casuale
consistente nell'essere ognuna delle $x_i$ compresa in un
determinato intervallo finito $[a_i, b_i]$ \`e data da
\begin{equation*}
  P = \int_{a_1}^{b_1} \! \de x_1 \int_{a_2}^{b_2} \!
    \de x_2 \, \cdots \int_{a_N}^{b_N} \! \de x_N \cdot
    f(x_1, x_2,\ldots, x_N) \peq .
\end{equation*}

\index{probabilit\`a!funzione marginale|(emidx}%
Similmente, poi, se consideriamo il sottoinsieme delle prime
$M$ variabili $x_i$ (con $M < N$), la probabilit\`a che
ognuna di esse cada all'interno di intervallini infinitesimi
attorno ad una $M$-pla di valori prefissati,
\emph{indipendentemente} dal valore assunto dalle altre
$N-M$ variabili, \`e data da
\begin{align*}
  \de P &\equiv f^M (x_1,\ldots,x_M) \, \de x_1 \,
    \cdots \de x_M \\[1ex]
  &= \de x_1 \cdots \de x_M
    \int_{-\infty}^{+\infty} \! \de x_{M+1}
    \int_{-\infty}^{+\infty} \! \de x_{M+2} \cdots
    \int_{-\infty}^{+\infty} \! \de x_N \cdot
    f(x_1, x_2,\ldots, x_N)
\end{align*}
dove gli integrali definiti sulle $N-M$ variabili che non
interessano si intendono estesi a tutto l'asse reale;
potendosi senza perdere in generalit\`a assumere che tale
sia il loro dominio di esistenza, definendo eventualmente la
$f$ come identicamente nulla al di fuori del reale
intervallo di variabilit\`a se esse fossero limitate.

La $f^M$ definita dalla equazione precedente prende il nome
di \emph{densit\`a di probabilit\`a marginale} delle $M$
variabili casuali $x_1,\ldots,x_M$;%
\index{probabilit\`a!funzione marginale|)}
infine la condizione di normalizzazione si scriver\`a
\begin{equation*}%
\index{normalizzazione!condizione di}
  \int_{-\infty}^{+\infty} \! \de x_1
    \int_{-\infty}^{+\infty} \! \de x_2 \, \cdots
    \int_{-\infty}^{+\infty} \! \de x_N \cdot
    f(x_1, x_2,\ldots, x_N) = 1 \peq .
\end{equation*}

\index{probabilit\`a!condizionata|(}%
Definendo, analogamente a quanto fatto nel paragrafo
\ref{ch:7.bidim}, la densit\`a di probabilit\`a delle $M$
variabili casuali $x_j$ (con $j=1, 2,\ldots ,M$ e $M<N$)
\emph{condizionata} dai valori assunti dalle restanti $N-M$
variabili attraverso la
\begin{equation} \label{eq:6.procas}
  f(x_1, x_2,\ldots,x_M | x_{M+1}, x_{M+2},\ldots,x_N)
  = \frac{f(x_1, x_2,\ldots,x_N)}{f^M(x_{M+1},
  x_{M+2},\ldots,x_N)}
\end{equation}%
\index{probabilit\`a!condizionata|)}%
\index{statistica!indipendenza|(}%
il concetto di indipendenza statistica pu\`o facilmente
essere generalizzato a \emph{sottogruppi} di variabili:
diremo che le $M$ variabili $x_j$ sono statisticamente
indipendenti dalle restanti $N - M$ quando la probabilit\`a
che le $x_1, x_2, \ldots, x_M$ assumano determinati valori
non dipende dai valori assunti dalle $x_{M+1}, x_{M+2},
\ldots, x_N$ --- e dunque quando la densit\`a condizionata
\eqref{eq:6.procas} \`e identicamente uguale alla densit\`a
marginale $f^M(x_1, x_2,\ldots,x_M)$.

Esaminando la \eqref{eq:6.procas} si pu\`o facilmente capire
come, perch\'e questo avvenga, occorra e basti che la
densit\`a di probabilit\`a complessiva sia
\emph{fattorizzabile} nel prodotto di due termini: il primo
dei quali sia funzione solo delle prime $M$ variabili ed il
secondo dei quali dipenda soltanto dalle altre $N-M$;
ovviamente ognuno dei fattori coincide con le probabilit\`a
marginali, per cui la condizione \`e espressa
matematicamente dalla formula
\begin{gather}
  f(x_1,\ldots, x_N) = f^M (x_1,\ldots, x_M) \cdot
    f^M(x_{M+1},\ldots, x_N) \notag \\
  \intertext{e, in particolare, le variabili sono
    \emph{tutte} indipendenti tra loro se e solo se
    risulta}
  f(x_1, x_2,\ldots, x_N) = f^M (x_1) \cdot f^M (x_2)
    \cdots f^M (x_N) \peq . \label{eq:6.instmu}
\end{gather}%
\index{statistica!indipendenza|)}

\index{cambiamento di variabile casuale|(}%
Nel caso che esista un differente insieme di $N$ variabili
$y_i$ in grado di descrivere lo stesso fenomeno casuale $E$,
il requisito che la probabilit\`a di realizzarsi di un
qualunque sottoinsieme dei possibili risultati (l'integrale
definito, su una qualunque regione $\Omega$ dello spazio ad
$N$ dimensioni, della funzione densit\`a di probabilit\`a)
sia invariante per il cambiamento delle variabili di
integrazione, porta infine a ricavare la formula di
trasformazione delle densit\`a di probabilit\`a per il
cambiamento di variabili casuali nel caso multidimensionale:
\begin{equation} \label{eq:6.cavamu}
  f(y_1, y_2,\ldots ,y_N) = f(x_1, x_2,\ldots, x_N)
    \cdot \left| \frac{\partial (x_1, x_2,\ldots
    ,x_N)}{\partial (y_1, y_2,\ldots ,y_N)} \right|
\end{equation}
dove con il simbolo
\begin{gather*}
  \left| \frac{\partial (x_1, x_2,\ldots
    ,x_N)}{\partial (y_1, y_2,\ldots ,y_N)} \right| \\
  \intertext{si \`e indicato il valore assoluto del
    determinante Jacobiano%
    \index{Jacobiano, determinante|emidx}
    delle $x$ rispetto alle $y$:}
  \frac{\partial (x_1, x_2,\ldots
    ,x_N)}{\partial (y_1, y_2,\ldots ,y_N)} =
    \det \left\|
      \begin{array}{cccc}
        \dfrac{\partial x_1}{\partial y_1} &
          \dfrac{\partial x_1}{\partial y_2} & \cdots &
          \dfrac{\partial x_1}{\partial y_N} \\[2ex]
        \dfrac{\partial x_2}{\partial y_1} &
          \dfrac{\partial x_2}{\partial y_2} & \cdots &
          \dfrac{\partial x_2}{\partial y_N} \\[2ex]
        \cdots & \cdots & \cdots & \cdots \\[1ex]
        \dfrac{\partial x_N}{\partial y_1} &
          \dfrac{\partial x_N}{\partial y_2} & \cdots &
          \dfrac{\partial x_N}{\partial y_N}
      \end{array}
    \right\| \\
  \intertext{che esiste sempre non nullo se la
    trasformazione tra l'insieme delle funzioni $x_i$ e
    quello delle funzioni $y_i$ \`e biunivoca; e che
    gode, sempre in questa ipotesi, della propriet\`a che}
  \frac{\partial (y_1, y_2,\ldots
    ,y_N)}{\partial (x_1, x_2,\ldots ,x_N)} =
    \left[ \frac{\partial (x_1, x_2,\ldots
    ,x_N)}{\partial (y_1, y_2,\ldots ,y_N)} \right]^{-1} \peq
    .
\end{gather*}%
\index{cambiamento di variabile casuale|)}

\endinput

% $Id: chapter8.tex,v 1.2 2006/02/20 11:28:27 loreti Exp $

\chapter{Esempi di distribuzioni teoriche}
In questo capitolo presentiamo alcune funzioni teoriche che
rappresentano densit\`a di probabilit\`a di variabili
casuali unidimensionali (continue e discrete) che hanno
importanza per la fisica.

\section{La distribuzione uniforme}%
\index{distribuzione!uniforme|(emidx}%
\label{ch:8.distun}
Il caso pi\`u semplice, dal punto di vista teorico, \`e
quello di una variabile casuale $x$ che possa assumere solo
valori compresi in un intervallo finito avente estremi
costanti prefissati, $[ a, b ]$; e ivi con probabilit\`a
uguale per ogni punto\/\footnote{La frase \`e intuitiva, ma
  impropria; si intende qui che la probabilit\`a, per la
  variabile casuale, di cadere in un intervallino di
  ampiezza (infinitesima) prefissata $\de x$ e centrato su
  un qualsivoglia punto del dominio di definizione, ha
  sempre lo stesso valore.}.

Questo implica che la densit\`a di probabilit\`a $f(x)$ di
questa variabile debba essere definita come
\begin{equation*}
  \begin{cases}
    f(x) = 0 & \quad\text{per $x<a$ e per $x>b$;}
      \\[1.5ex]
    f(x) = \dfrac{1}{b-a} = \mathrm{cost.} &
      \quad\text{per $a \leq x \leq b$.}
  \end{cases}
\end{equation*}
(il valore costante di $f(x)$ quando $x \in [ a, b ]$ \`e
fissato dalla condizione di normalizzazione).  La funzione
di distribuzione $F(x)$ della $x$ \`e data da
\begin{equation*}
  F(x) \; = \; \int_{-\infty}^x \! f(t) \, \de t \; =
  \;
  \begin{cases}
    0 & \text{per $x<a$;} \\[2ex]
    \dfrac{x-a}{b-a} & \text{per $a \leq x \leq b$;}
    \\[2ex]
    1 & \text{per $x>b$.}
  \end{cases}
\end{equation*}
I valori della media e della varianza della variabile
casuale $x$, come si pu\`o facilmente calcolare, valgono
\begin{equation} \label{eq:8.mevaun}
  \begin{cases}
    E(x) \; = \; \dfrac{a+b}{2} \\[2ex]
    \var(x) \; = \; \dfrac{\left( b-a \right)^2}{12}
    \end{cases}
\end{equation}%
\index{distribuzione!uniforme|)}

Per vedere una prima applicazione pratica della
distribuzione uniforme, supponiamo di misurare una grandezza
fisica  usando uno strumento \emph{digitale}: ad esempio una
bilancia con sensibilit\`a inversa di  1 grammo.  Se, per
semplicit\`a, escludiamo la presenza di errori sistematici,
il fatto che il display digitale indichi (ad esempio) $10$
grammi significa solo che la massa dell'oggetto pesato \`e
maggiore o uguale a questo valore e minore di $11$
grammi\/\footnote{La maggior parte degli strumenti digitali
  \emph{tronca} il valore mostrato e si comporta appunto in
  questo modo; altri invece \emph{arrotondano} il risultato
  e, se questo fosse il caso, vorrebbe dire che la massa
  dell'oggetto pesato \`e maggiore o uguale a $9.5$\un{g} e
  minore di $10.5$\un{g}.}; e tutti i valori interni a
questo intervallo ci appaiono inoltre come ugualmente
plausibili.  Per questo motivo, viste le
\eqref{eq:8.mevaun}, in casi di questo genere si attribuisce
all'oggetto pesato una massa di $10.5$\un{g} con un errore
di $1 / \sqrt{12} \approx 0.3 \un{g}$.

\subsection[Applicazione: decadimento del
    $\pi^0$]{Applicazione: decadimento del
    $\boldsymbol{\pi}^{\boldsymbol{0}}$}
\begin{figure}[htbp]
  \vspace*{2ex}
  \begin{center} {
    \input{polar.pstex_t}
  } \end{center}
  \caption[Le aree elementari sulla superficie di una
    sfera di raggio $R$]{Le aree elementari sulla superficie
    di una sfera di raggio $R$ (in coordinate polari).}
  \label{fig:8.sfera}
\end{figure}

Esistono, nella fisica, variabili casuali che seguono la
distribuzione uniforme: ad esempio, se una particella
instabile non dotata di momento angolare intrinseco (come il
mesone $\pi^0$), originariamente in quiete in un punto (che
supporremo sia l'origine degli assi coordinati), decade, i
prodotti di decadimento si distribuiscono uniformemente tra
le varie direzioni possibili; sostanzialmente per motivi di
simmetria, perch\'e non esiste nessuna direzione
privilegiata nel sistema di riferimento considerato
(ovverosia nessuna caratteristica intrinseca del fenomeno
che possa servire per definire uno, o pi\`u d'uno, degli
assi coordinati).

Con riferimento alla figura \ref{fig:8.sfera}, pensiamo
introdotto un sistema di coordinate polari $\{ R, \theta,
\varphi \}$: l'elemento infinitesimo di area, $\de S$, sulla
sfera di raggio $R$, che corrisponde a valori della
\emph{colatitudine} compresi tra $\theta$ e $\theta + \de
\theta$, e dell'\emph{azimuth} compresi tra $\varphi$ e
$\varphi + \de \varphi$, \`e uno pseudorettangolo di lati $R
\, \de\theta$ ed $R \sin\theta \, \de\varphi$; quindi, a
meno del segno,
\begin{gather*}
  \left| \de S \right| \; = \; R^2 \sin\theta \,
    \de\theta \, \de\varphi \; = \; - R^2 \,
    \de(\cos\theta) \, \de\varphi \\
  \intertext{mentre l'angolo solido corrispondente
    vale}
  \de\Omega \; = \; \frac{\left| \de S \right|}{R^2} \;
    = \; \sin\theta \, \de\theta \, \de\varphi \; = \;
    - \de(\cos\theta) \, \de\varphi \peq .
\end{gather*}

L'asserita uniformit\`a nell'emissione dei prodotti di
decadimento si traduce nella condizione che la
probabilit\`a, per essi, di essere contenuti in un qualsiasi
angolo solido, sia proporzionale all'ampiezza di
quest'ultimo:
\begin{equation*}
  \de P \; = \; K \, \de \Omega \; = \; K' \, \de (\cos
  \theta) \, \de \varphi
\end{equation*}
(ove $K$ e $K'$ sono due opportune costanti); ovverosia
richiede che le due variabili casuali
\begin{align*}
  u &= \cos\theta &&\text{e} & v &= \varphi
\end{align*}
abbiano distribuzione uniforme, e siano inoltre
statisticamente indipendenti tra loro (questo in conseguenza
dell'equazione \eqref{eq:6.instmu}).

\subsection{Applicazione: generazione di numeri casuali
  con distribuzione data}%
\index{pseudo-casuali, numeri|(}
Supponiamo che la variabile casuale $x$ abbia densit\`a di
probabilit\`a $f(x)$ e funzione di distribuzione%
\index{funzione!di distribuzione|(}
$F(x)$: vogliamo ora dimostrare che la variabile casuale $y
= F(x)$ \`e distribuita uniformemente nell'intervallo $[ 0,
1 ]$ \emph{qualunque} siano $f(x)$ e $F(x)$.  Chiaramente
$y$ pu\`o appartenere solo a tale intervallo; ed inoltre,
essendo funzione integrale di $f(x)$, \`e dotata della
propriet\`a di essere continua e derivabile in tutto
l'insieme di definizione e con derivata prima data da
\begin{gather*}
  y' \; = \; F'(x) \; = \; f(x) \\
  \intertext{cos\`\i\ che, ricordando l'equazione
    \eqref{eq:6.cavaun}, la densit\`a di probabilit\`a
    della nuova variabile $y$ \`e data (ove $f(x)$ non
    sia nulla) dalla}
  g(y) \; = \; \frac{f(x)}{y'(x)} \; = \;
    \frac{f(x)}{f(x)} \; \equiv \; 1
\end{gather*}
come volevamo dimostrare.%
\index{funzione!di distribuzione|)}

Supponiamo sia nota la densit\`a di probabilit\`a $f(x)$ di
una qualche variabile casuale $x$; e che si vogliano
ottenere dei numeri che si presentino secondo una legge di
probabilit\`a data appunto da questa $f(x)$.  I moderni
calcolatori numerici sono in grado di generare sequenze di
numeri casuali\/\footnote{O meglio \emph{pseudo-casuali}:
  ovverosia prodotti da un algoritmo ripetibile, quindi non
  propriamente ``imprevedibili''; ma in modo tale che le
  loro propriet\`a statistiche siano indistinguibili da
  quelle di una sequenza casuale propriamente detta.}  che
hanno distribuzione uniforme in un intervallo dipendente
dall'implementazione dell'algoritmo, e che possono a loro
volta essere usati per produrre numeri casuali con
distribuzione uniforme nell'intervallo $[0,1]$; se $y$ \`e
uno di tali numeri, e se si \`e in grado di invertire,
numericamente od analiticamente, la funzione di
distribuzione $F(x)$ della variabile casuale $x$, i numeri
\begin{equation*}
  x \; = \; F^{-1}(y)
\end{equation*}
hanno densit\`a di probabilit\`a data da $f(x)$, come
appunto richiesto.

Generalmente le funzioni di distribuzione $F(x)$ non si
sanno invertire per via analitica; un metodo numerico spesso
impiegato, e che richiede la sola preventiva conoscenza
della $f(x)$ (quindi non bisogna nemmeno saper calcolare la
$F(x)$, per non parlare della sua inversa) \`e illustrato
qui di seguito (\emph{metodo
dei rigetti}).%
\index{metodo!dei rigetti|(}
\begin{figure}[htbp]
  \vspace*{2ex}
  \begin{center} {
    \input{riget.pstex_t}
  } \end{center}
  \caption[Il metodo dei rigetti --- esempio]
    {La scelta di un numero a caso con distribuzione
    prefissata mediante tecniche numeriche (la densit\`a di
    probabilit\`a \`e la stessa della figura
    \ref{fig:4.maxbol}); la funzione maggiorante \`e una
    spezzata (superiormente) o la retta $y=0.9$
    (inferiormente).}
  \label{fig:8.maxbol}
\end{figure}
Si faccia riferimento alla figura \ref{fig:8.maxbol}: sia
$x$ limitata in un intervallo chiuso $[x_{\min}, x_{\max}]$
(nella figura, $x_{\min} = 0$ e $x_{\max} = 3$); e si
conosca una funzione $y=\varphi(x)$ \emph{maggiorante} della
$f(x)$, ossia una funzione che risulti comunque non
inferiore alla $f$ per qualunque $x \in [x_{\min},
x_{\max}]$.

Nel caso si sappia scegliere, sul piano $\{ x,y \}$, un
punto con distribuzione uniforme nella parte di piano
limitata inferiormente dall'asse delle ascisse,
superiormente dalla funzione $y=\varphi(x)$, e,
lateralmente, dalle due rette di equazione $x = x_{\min}$ ed
$x = x_{\max}$, basta accettare tale punto se la sua
ordinata risulta non superiore alla corrispondente $f(x)$; e
rigettarlo in caso contrario, iterando il procedimento fino
a che la condizione precedente non \`e soddisfatta: le
ascisse $x$ dei punti accettati seguono la funzione di
distribuzione $f(x)$.

Infatti, i punti accettati saranno distribuiti uniformemente
nella parte di piano limitata dalla $y=f(x)$; quindi, in un
intervallino infinitesimo centrato su una particolare $x$,
vi sar\`a un numero di punti accettati proporzionale
all'altezza della curva sopra di esso --- ovverosia ogni
ascissa $x$ viene accettata con densit\`a di probabilit\`a
che \`e proprio $f(x)$.

La scelta, infine, di un punto che sia distribuito
uniformemente nella parte di piano limitata dalla funzione
$y=\varphi(x)$ si sa sicuramente effettuare se $\varphi(x)$
\`e stata scelta in modo che si sappia invertire la sua
funzione integrale
\begin{equation*}
  \Phi(x) = \int_{-\infty}^x \! \varphi(t) \, \de t
\end{equation*}
cos\`\i\ che si possa associare, a qualsiasi valore $A$
compreso tra 0 e $\Phi(+\infty)$, quella $x = \Phi^{-1}(A)$
che lascia alla propria sinistra un'area $A$ al di sotto
della funzione $y=\varphi(x)$ (una scelta banale \`e quella
di prendere come maggiorante una retta, o meglio una
spezzata --- come illustrato nella figura
\ref{fig:8.maxbol}).

In tal caso basta scegliere un numero $A$ con distribuzione
uniforme tra i limiti $\Phi(x_{\min})$ e $\Phi(x_{\max})$;
trovare la $x = \Phi^{-1}(A)$ che soddisfa la condizione
precedente; ed infine scegliere una $y$ con distribuzione
uniforme tra 0 e $\varphi(x)$.  Non \`e difficile rendersi
conto che il punto $(x,y)$ soddisfa alla condizione
richiesta di essere distribuito uniformemente nella parte
del semipiano $y>0$ limitata superiormente dalla funzione
$y=\varphi(x)$: a questo punto non rimane che calcolare la
$f(x)$ ed accettare $x$ se $y \leq f(x)$.

Se proprio non si \`e in grado di effettuare una scelta
migliore, anche una retta del tipo $y=\mathrm{cost.}$ pu\`o
andar bene; basta tener presente che l'algoritmo viene
sfruttato tanto pi\`u efficacemente quanto pi\`u
$y=\varphi(x)$ \`e vicina alla $f(x)$ (in tal caso il numero
di rigetti \`e minore).

Per questo motivo, una scelta del tipo
$\varphi(x)=\mathrm{cost.}$ \`e assolutamente da evitare se
la $f(x)$ \`e sensibilmente diversa da zero solo in una
parte ristretta dell'intervallo di definizione (perch\'e in
tal caso la scelta uniforme di $x$ all'interno dell'area su
detta ci farebbe trascorrere gran parte del tempo ad
esaminare valori poco probabili rispetto alla $\varphi(x)$,
che vengono in conseguenza quasi sempre rifiutati).%
\index{metodo!dei rigetti|)}%
\index{pseudo-casuali, numeri|)}

\subsection{Esempio: valori estremi di un campione di dati a
  distribuzione uniforme}
\index{campione!valori estremi|(}%
\label{ch:8.estremi}
Come ulteriore esempio, applichiamo le conclusioni dei
paragrafi \ref{ch:6.estremi} e \ref{ch:7.estremi} ad un
campione di valori proveniente da una distribuzione
uniforme.  Usando le espressioni per $f(x)$ e $F(x)$ che
conosciamo, ed essendo\/\footnote{All'interno
  dell'intervallo $[a,b]$; per brevit\`a ometteremo, qui e
  nel seguito, di specificare che, al di fuori di questo
  intervallo, le densit\`a di probabilit\`a sono
  identicamente nulle e le funzioni di distribuzione valgono
  o zero od uno.}
\begin{equation*}
  1 - F(x) \; = \; \frac{b - x}{b - a} \peq ,
\end{equation*}
la \eqref{eq:6.iesimo} diventa
\begin{equation*}
  f_i(x) \; = \; N \, \binom{N - 1}{i - 1} \: \frac{(x -
    a)^{i - 1} \, (b - x)^{N - i}}{(b - a)^N}
\end{equation*}
e, in particolare, per i due valori minimo e massimo
presenti nel campione le densit\`a di probabilit\`a si
scrivono
\begin{gather*}
  f_1(x) \; = \; N \, \frac{(b - x)^{N - 1}}{(b - a)^N} \\
  \intertext{e}
  f_N(x) \; = \; N \, \frac{(x - a)^{N - 1}}{(b - a)^N} \peq
  .
\end{gather*}
Come conseguenza, la speranza matematica di $x_N$ vale
\begin{align*}
  E ( x_N ) &= \int_a^b \! x \cdot f_N(x) \, \de x \\[1ex]
  &= \frac{N}{(b - a)^N} \int_a^b \bigl[ a + (x - a)
  \bigr] \, (x - a)^{N - 1} \, \de x \\[1ex]
  &= \frac{N}{(b - a)^N}\:  \left[ a \, \frac{(x - a)^N}{N}
    + \frac{(x - a)^{N + 1}}{N + 1} \right]_a^b \\[1ex]
  &= a + \frac{N}{N + 1} \, (b - a) \\[1ex]
  &= b - \frac{1}{N + 1} \, (b - a) \peq .
\end{align*}
Allo stesso modo si troverebbe
\begin{gather*}
  E ( x_1 ) \; = \; a + \frac{1}{N + 1} \, (b - a) \peq ; \\
  \intertext{e, per il generico $x_i$,}
  E ( x_i ) \; = \; a + \frac{i}{N + 1} \, (b - a) \peq . \\
  \intertext{Dopo gli opportuni calcoli, si potrebbero
    ricavare anche le varianze rispettive: che valgono}
  \var( x_1 ) \; = \; \var( x_N ) \; = \; \frac{N}{(N +
    1)^2 \, (N + 2)} \, (b - a)^2 \\
  \intertext{e}
  \var( x_i ) \; = \; \frac{i \cdot (N - i + 1)}{(N + 1)^2
    \, (N + 2)} \, (b - a)^2 \peq .
\end{gather*}

\`E immediato calcolare la speranza matematica della
semisomma del pi\`u piccolo e del pi\`u grande valore
presenti nel campione
\begin{gather*}
  d \; = \; \frac{x_1 + x_N}{2} \\
  \intertext{che vale}
  E(d) \; = \; \frac{ E(x_1) + E(x_N) }{2} \; = \;
  \frac{a+b}{2} \peq ; \\
  \intertext{come pure quella del cosiddetto \emph{range},%
    \index{distribuzione!uniforme!range|(}%
    }
  R \; = \; x_N - x_1 \\
  \intertext{per il quale}
  E(R) \; = \; E(x_N) - E(x_1) \; = \; (b - a) \left( 1 -
    \frac{2}{N + 1} \right) \peq . \\
  \intertext{Per il calcolo delle varianze, invece, si deve
    ricorrere alla distribuzione congiunta
    \eqref{eq:7.estremi}, dalla quale si pu\`o ricavare}
  \var(d) \; = \; \frac{(b - a)^2}{2 \, (N + 1) \, ( N + 2)}
  \\
  \intertext{e}
  \var(R) \; = \; \frac{2 \: (N - 1)}{(N + 1)^2 \, (N + 2)}
  \, (b - a)^2 \peq .
\end{gather*}%
\index{distribuzione!uniforme!range|)}%
\index{campione!valori estremi|)}

\section{La distribuzione normale}%
\index{distribuzione!normale|(emidx}%
\label{ch:8.gauss}
La \emph{funzione normale} (o \emph{funzione di Gauss}), che
esamineremo poi in dettaglio nel prossimo capitolo mettendo
l'accento sui suoi legami con le misure ripetute delle
grandezze fisiche, \`e una funzione di frequenza per la $x$
che dipende da due parametri $\mu$ e $\sigma$ (con la
condizione $\sigma > 0$) definita come
\begin{figure}[htbp]
  \vspace*{2ex}
  \begin{center} {
    \input{gauss.pstex_t}
  } \end{center}
  \caption[La distribuzione normale standardizzata]
    {L'andamento della funzione $N(x;0,1)$ per la
    \emph{variabile normale standardizzata} (ossia con media
    0 e varianza 1).}
  \label{fig:8.gauss}
\end{figure}
\begin{equation*}
  f(x) \; \equiv \; N(x; \mu, \sigma) \; = \;
    \frac{1}{\sigma \sqrt{2 \pi}} \, e^{ - \frac{1}{2}
    \left( \frac{x - \mu}{\sigma} \right)^2} \peq .
\end{equation*}

L'andamento della funzione normale \`e quello delineato
nella figura \ref{fig:8.gauss}: quando $x = \mu$ si ha un
punto di massimo, nel quale la funzione ha il valore
$\widehat y = ( \sigma \sqrt{2 \pi} )^{-1} \approx 0.4 /
\sigma$.  La \emph{larghezza a met\`a
  altezza}\thinspace\footnote{In genere indicata con la
  sigla FWHM, acronimo di \emph{full width at half maximum};
  \`e un parametro talvolta usato nella pratica per
  caratterizzare una curva, perch\'e facile da misurare su
  un oscilloscopio.}  \`e pari all'ampiezza dell'intervallo
che separa i due punti $x_1$ ed $x_2$ di ascissa $\mu \pm
\sigma \sqrt{2 \, \ln 2}$ e di ordinata $y_1 = y_2 =
\widehat y / 2$: e vale quindi $2 \sigma \sqrt{2 \, \ln 2}
\approx 2.35 \sigma$.

La funzione generatrice dei momenti \`e definita attraverso
l'equazione \eqref{eq:6.fugemo} e, nel caso della
distribuzione normale, abbiamo
\begin{align*}
  M_x(t) &= \int_{-\infty}^{+\infty} \! e^{
    tx} \frac{1}{\sigma \sqrt{2 \pi}} \,
    e^{ - \frac{1}{2} \left( \frac{x -
    \mu}{\sigma} \right)^2} \de x \\[2ex]
  &= \frac{e^{ t \mu}}{\sigma \sqrt{2
    \pi}} \int_{-\infty}^{+\infty} \! e^{
    t (x - \mu)} e^{ - \frac{1}{2} \left(
    \frac{x - \mu}{\sigma} \right)^2} \de x \\[2ex]
  &= \frac{e^{ t \mu}}{\sigma \sqrt{2
    \pi}} \int_{-\infty}^{+\infty} \! e^{
    \left[ \frac{\sigma^2 t^2}{2} - \frac{(x - \mu -
    \sigma^2 t)^2}{2 \sigma^2} \right]} \, \de x
    \\[2ex]
  &= e^{ \left( t \mu + \frac{\sigma^2
    t^2}{2} \right)} \int_{-\infty}^{+\infty}
    \frac{1}{\sigma \sqrt{2 \pi}} \, e^{ -
    \frac{1}{2} \bigl[ \frac{x - ( \mu + \sigma^2 t
    )}{\sigma} \bigr] ^2} \de x \peq .
\end{align*}

Riconoscendo nell'argomento dell'integrale la funzione
$N(x;\mu+\sigma^2 t,\sigma)$, ovverosia la funzione normale
relativa ai parametri $\mu + \sigma^2 t$ e $\sigma$, \`e
immediato capire che esso vale 1 in conseguenza della
condizione di normalizzazione; quindi la funzione
generatrice dei momenti, per la distribuzione normale, \`e
data da
\begin{equation} \label{eq:8.fgmodn}
  M_x(t) \; = \; e^{\left( t \mu + \frac{\sigma^2
    t^2}{2} \right)}
\end{equation}
e, con passaggi simili, si potrebbe trovare la funzione
caratteristica della distribuzione normale: che vale
\begin{equation} \label{eq:8.fucadn}
  \phi_x(t) \; = \; e^{\left( i t \mu - \frac{\sigma^2
    t^2}{2} \right)} \peq .
\end{equation}

Sfruttando la \eqref{eq:8.fgmodn} \`e facile calcolare la
speranza matematica della distribuzione normale:
\begin{equation*}
  E(x) \; = \; \left. \frac{\de \, M_x(t)}{\de t}
    \right|_{t=0} \; = \; \mu \peq ;
\end{equation*}
la funzione generatrice dei momenti rispetto alla media
$\mu$ vale allora
\begin{equation} \label{eq:8.fgmmdn}
  \ob{M}_x (t) \; = \; e^{- t \mu} M_x(t) \; = \;
    e^{ \frac{\sigma^2 t^2}{2} }
\end{equation}
e dalla \eqref{eq:8.fgmmdn} si ricava poi la varianza
della $x$,
\begin{equation*}
  \var(x) \; = \; \left. \frac{\de^2 \ob{M}_x
    (t)}{\de t^2} \right|_{t=0} \; = \; \sigma^2 \peq .
\end{equation*}

Vista la simmetria della funzione, tutti i suoi momenti di
ordine dispari rispetto alla media sono nulli; mentre quelli
di ordine pari soddisfano alla formula generale (valida per
qualsiasi intero $k$)
\begin{gather}
  \mu_{2k} \; = \; E \left\{ \bigl[ x - E(x)
    \bigr]^{2k} \right\} \; = \; \frac{(2k)!}{2^k \,
    k!} \, {\mu_2} ^k \label{eq:8.mopaga} \\
  \intertext{con}
  \mu_2 \; = \; E \left\{ \bigl[ x - E(x) \bigr]^2
    \right\} \; = \sigma^2 \peq . \notag
\end{gather}

Nel caso particolare di una variabile normale con valore
medio $\mu=0$ e varianza $\sigma^2=1$ (\emph{variabile
  normale standardizzata}), la funzione generatrice dei
momenti diventa
\begin{gather*}
  M_x(t) \; \equiv \;  \ob{M}_x(t) \; = \;
    e^{\frac{t^2}{2}} \\
  \intertext{e la funzione caratteristica}
  \phi_x(t) \; = \; e^{- \frac{t^2}{2}} \peq .
\end{gather*}

Dimostriamo ora il seguente importante
\begin{quote}
  \textsc{Teorema:}%
  \index{combinazioni lineari!di variabili normali|(}%
  \label{th:8.colino}
  \textit{combinazioni lineari di variabili casuali normali
    e tutte statisticamente indipendenti tra loro sono
    ancora distribuite secondo la legge normale.}
\end{quote}
Siano $N$ variabili normali $x_k$ (con $k=1,\ldots,N$), e
siano $\mu_k$ e ${\sigma_k}^2$ i loro valori medi e le loro
varianze rispettivamente; consideriamo poi la nuova
variabile casuale $y$ definita dalla
\begin{gather*}
  y = \sum_{k=1}^N a_k \, x_k \\
  \intertext{(ove le $a_k$ sono coefficienti costanti). La
    funzione caratteristica di ognuna delle $x_k$ \`e, dalla
    \eqref{eq:8.fucadn},}
  \phi_{x_k}(t) = e^{\left( i t \mu_k -
    \frac{{\sigma_k}^2 t^2}{2} \right)} \\
  \intertext{e quella della variabile ausiliaria $\xi_k =
    a_k x_k$, dall'equazione \eqref{eq:6.fuccav},}
  \phi_{\xi_k}(t) \; = \; \phi_{x_k}(a_k t) \; =
    \;e^{\left( i a_k t \mu_k - \frac{{\sigma_k}^2
    {a_k}^2 t^2}{2} \right)} \peq . \\
  \intertext{Infine, la funzione caratteristica della
    $y$ vale, essendo}
  y = \sum_{k=1}^N \xi_k
\end{gather*}
e ricordando l'equazione \eqref{eq:6.fucacl}, applicabile
perch\'e anche le $\xi_k$ sono indipendenti tra loro,
otteniamo
\begin{align*}
  \phi_y(t) &= \prod_{k=1}^N \phi_{\xi_k}(t) \\[1ex]
  &= \prod_{k=1}^N e^{\left( i t a_k \mu_k -
    \frac{1}{2} t^2 {a_k}^2 {\sigma_k}^2 \right)} \\[1ex]
  &= e^{\left[ i t \left( \sum_k a_k \mu_k \right) -
    \frac{1}{2} t^2 \left( \sum_k {a_k}^2 {\sigma_k}^2
    \right) \right]} \\[1ex]
  &= e^{\left( i t \mu - \frac{t^2 \sigma^2}{2}
    \right)}
\end{align*}
ove si \`e posto
\begin{align*}
  \mu &= \sum_{k=1}^N a_k \, \mu_k &&\text{e} &
    \sigma^2 &= \sum_{k=1}^N {a_k}^2 {\sigma_k}^2 \peq .
\end{align*}
Questa \`e appunto la funzione caratteristica di una nuova
distribuzione normale; e, in virt\`u di uno dei teoremi
enunciati nel paragrafo \ref{ch:6.fugeca}, quanto dimostrato
prova la tesi.%
\index{combinazioni lineari!di variabili normali|)}%
\index{distribuzione!normale|)}

\section{La distribuzione di Cauchy}%
\index{distribuzione!di Cauchy|(emidx}%
\index{distribuzione!di Breit--Wigner|see{distribuzione di Cauchy}}%
\label{ch:8.cauchy}
La distribuzione di Cauchy (o \emph{distribuzione di
  Breit--Wigner}, nome con il quale \`e pi\`u nota nel mondo
della fisica) \`e definita da una densit\`a di probabilit\`a
che corrisponde alla funzione, dipendente da due parametri
$\theta$ e $d$ (con la condizione $d > 0$),
\begin{equation} \label{eq:8.cauchy}
  f(x;\theta, d) \; = \; \frac{1}{\pi d} \, \frac{1}{1
    + \left( \frac{x - \theta}{d} \right)^2} \peq .
\end{equation}
\begin{figure}[htbp]
  \vspace*{2ex}
  \begin{center} {
    \input{cauchy.pstex_t}
  } \end{center}
  \caption[La distribuzione di Cauchy]
    {L'andamento della distribuzione di Cauchy,
    per $\theta = 0$ e $d = 1$.}
  \label{fig:8.cauchy}
\end{figure}
Anche se la \eqref{eq:8.cauchy} \`e integrabile, e la sua
funzione integrale, ovverosia la funzione di distribuzione
della $x$, vale
\begin{equation*}
  F(x;\theta, d) \; = \; \int_{-\infty}^x \! f(t) \,
    \de t \; = \; \frac{1}{2} + \frac{1}{\pi} \arctan
    \left( \frac{x - \theta}{d} \right)
\end{equation*}
\emph{nessuno dei momenti esiste}, nemmeno la media.

$\theta$ \`e la mediana della distribuzione e $d$ ne misura
la larghezza a met\`a altezza, come \`e rilevabile ad
esempio dalla figura \ref{fig:8.cauchy}.  La funzione
caratteristica della distribuzione di Cauchy \`e la
\begin{gather*}
  \phi_x (t; \theta, d) = e^{ i \theta t - |t| \, d } \peq ;
  \\
  \intertext{per la cosiddetta \emph{variabile
      standardizzata},}
  u = \frac{ x - \theta }{ d }
\end{gather*}
funzione di frequenza, funzione di distribuzione e funzione
caratteristica valgono rispettivamente
\begin{equation*}
  \begin{cases}
    f(u) \; = \; \dfrac{ 1 }{ \pi \left( 1 + u^2 \right) }
    \\[3ex]
    F(u) \; = \; \dfrac{1}{2} + \dfrac{1}{\pi} \, \arctan u
    \\[3ex]
    \phi_u (t) \; = \; \displaystyle e^{ - |t| }
  \end{cases}
\end{equation*}

Secondo la funzione \eqref{eq:8.cauchy} sono, ad esempio,
distribuite le intensit\`a nelle righe spettrali di
emissione e di assorbimento degli atomi (che hanno una
ampiezza non nulla); e la massa invariante delle risonanze
nella fisica delle particelle elementari.  \`E evidente
per\`o come nella fisica la distribuzione di Cauchy possa
descrivere questi fenomeni solo in prima approssimazione:
infatti essa si annulla solo per $x \to \pm \infty$, ed \`e
chiaramente priva di significato fisico una probabilit\`a
non nulla di emissione spettrale per frequenze negative, o
di masse invarianti anch'esse negative nel caso delle
risonanze.

Per la distribuzione di Cauchy \emph{troncata}, ossia quella
descritta dalla funzione di frequenza (per la variabile
standardizzata)
\begin{equation*}
  f(u | -K \le u \le K) =
  \begin{cases}
    0 & \quad |u| > K \\[1ex]
    \dfrac{1}{2 \, \arctan K} \, \dfrac{1}{\left( 1 + u^2
      \right)} & \quad |u| \le K
  \end{cases}
\end{equation*}
(\emph{discontinua} in $u = \pm K$), esistono invece i
momenti: i primi due valgono
\begin{align*}
  E(u | -K \le u \le K) &= 0 \\
  \intertext{e}
  \var(u | -K \le u \le K) &= \frac{K}{\arctan K} - 1
\end{align*}

\index{combinazioni lineari!di variabili di Cauchy|(}%
Se le $x_k$ sono $N$ variabili casuali indipendenti che
seguono la distribuzione di Cauchy con parametri $\theta_k$
e $d_k$, una generica loro combinazione lineare
\begin{gather*}
  y = \sum_{k=1}^N a_k \, x_k \\
  \intertext{segue la stessa distribuzione: infatti la
    funzione generatrice per le $x_k$ \`e}
  \phi_{x_k} (t) = e^{ i \theta_k t - |t| d_k } \\
  \intertext{e, definendo $\xi_k = a_k x_k$ e ricordando la
    \eqref{eq:6.fuccav},}
  \phi_{\xi_k} (t) \; = \; \phi_{x_k} (a_k t) \; = \; e^{ i
    a_k \theta_k t - |t| \cdot | a_k | \, d_k } \peq ; \\
  \intertext{infine, applicando la \eqref{eq:6.fucacl},}
  \phi_y (t) \; = \; \prod_{k=1}^N \phi_{\xi_k} (t) \; = \;
  e^{ i \theta_y t - |t| d_y } \\
  \intertext{ove si \`e posto}
  \theta_y = \sum_{k=1}^N a_k \theta_k
  \makebox[40mm]{e}
  d_y = \sum_{k=1}^N |a_k| d_k \peq .
\end{gather*}%
\index{combinazioni lineari!di variabili di Cauchy|)}

Una conseguenza importante \`e che il valore medio di un
campione di misure proveniente da una popolazione che segua
la distribuzione di Cauchy con certi parametri $\theta$ e
$d$ (in questo caso tutte le $a_k$ sono uguali e valgono
$1/N$) \`e distribuito anch'esso secondo Cauchy \emph{e con
  gli stessi parametri}; in altre parole \emph{non si
  guadagna nessuna informazione} accumulando pi\`u di una
misura (e calcolando la media aritmetica del
campione)\/\footnote{Esistono altre tecniche, basate per\`o
  sull'uso della mediana, che permettono di migliorare la
  conoscenza del valore di $\theta$ disponendo di pi\`u
  di una misura.}.%
\index{distribuzione!di Cauchy|)}

\subsection{Il rapporto di due variabili normali}%
\index{rapporto di variabili!normali|(}
Siano due variabili casuali $x$ ed $y$ che seguano la
distribuzione normale standardizzata $N(0,1)$; e sia inoltre
la $y$ definita su tutto l'asse reale ad eccezione
dell'origine ($y \ne 0$).  La densit\`a di probabilit\`a
congiunta di $x$ e $y$ \`e la
\begin{gather*}
  f(x,y) \; = \; N(x; 0, 1) \cdot N(y; 0, 1) \; = \;
  \frac{1}{2 \pi} \, e^{- \frac{1}{2} x^2 } e^{ -
    \frac{1}{2} y^2 } \peq ; \\
  \intertext{definendo}
  u = \frac{x}{y} \makebox[30mm]{e} v = y
\end{gather*}
e ricordando la \eqref{eq:7.rapvar}, la densit\`a di
probabilit\`a $\varphi(u)$ della $u$ \`e la
\begin{align*}
  \varphi(u) &= \frac{1}{2 \pi} \int_{-\infty}^{+\infty} \!
  e^{- \frac{1}{2} \, u^2 v^2} e^{- \frac{1}{2} \, v^2} \,
  |v| \, \de v \\[1ex]
  &= \frac{1}{\pi} \int_0^{+\infty} \! e^{- \frac{1}{2} \,
    v^2 \left( 1 + u^2 \right) } \, v \, \de v \\[1ex]
  &= \frac{1}{\pi \left( 1 + u^2 \right)} \int_0^{+\infty}
  \! e^{-t} \, \de t \\[1ex]
  &= \frac{1}{\pi \left( 1 + u^2 \right)} \left[ - e^{-t}
  \right]_0^{+\infty} \\[1ex]
  &= \frac{1}{\pi \left( 1 + u^2 \right)}
\end{align*}
Per eseguire l'integrazione si \`e effettuata la
sostituzione
\begin{equation*}
  t = \frac{1}{2} \, v^2 \left( 1 + u^2 \right)
  \makebox[40mm]{$\Longrightarrow$}
  \de t = \left( 1 + u^2 \right) v \, \de v
\end{equation*}
e si riconosce immediatamente nella $\varphi(u)$ la
densit\`a di probabilit\`a di una variabile (standardizzata)
di Cauchy: \emph{il rapporto tra due variabili normali segue
  la distribuzione di Cauchy}.%
\index{rapporto di variabili!normali|)}

\section{La distribuzione di Bernoulli}%
\index{distribuzione!di Bernoulli|(}%
\index{binomiale, distribuzione|see{distribuzione di Bernoulli}}%
\label{ch:8.binom}
Consideriamo un evento casuale ripetibile $E$, avente
probabilit\`a costante $p$ di verificarsi; indichiamo con $q
= 1 - p$ la probabilit\`a del non verificarsi di $E$ (cio\`e
la probabilit\`a dell'evento complementare \ob{E}\,).
Vogliamo ora determinare la probabilit\`a $P(x;N)$ che in
$N$ prove ripetute $E$ si verifichi esattamente $x$ volte
(deve necessariamente risultare $0 \le x \le N$).

L'evento casuale costituito dal presentarsi di $E$ per $x$
volte (e quindi dal presentarsi di \ob{E}\ per le restanti
$N - x$) \`e un evento complesso che pu\`o verificarsi in
diverse maniere, corrispondenti a tutte le diverse possibili
sequenze di successi e fallimenti; queste sono ovviamente
mutuamente esclusive, ed in numero pari a quello delle
possibili combinazioni di $N$ oggetti a $x$ a $x$, che vale
\begin{equation*}
  C^N_x \; = \; \binom{N}{x} \; = \; \frac{N!}{x! \,
    (N-x)!} \peq .
\end{equation*}

Essendo poi ognuna delle prove statisticamente indipendente
dalle altre (infatti la probabilit\`a di $E$ non cambia di
prova in prova), ognuna delle possibili sequenze di $x$
successi ed $N-x$ fallimenti ha una probabilit\`a di
presentarsi che vale $p^x q^{N-x}$; in definitiva
\begin{equation} \label{eq:8.binom}
  P(x;N) \; = \; \frac{N!}{x! \, (N-x)!} \,
    p^x q^{N-x} \peq .
\end{equation}

Questa distribuzione di probabilit\`a $ P(x;N) $ per una
variabile casuale discreta $x$ si chiama \emph{distribuzione
  binomiale} o \emph{di Bernoulli}\thinspace\footnote{I
  Bernoulli furono una famiglia originaria di Anversa poi
  trasferitasi a Basilea, numerosi membri della quale ebbero
  importanza per le scienze del diciassettesimo e del
  diciottesimo secolo; quello cui vanno attribuiti gli studi
  di statistica ebbe nome Jacob (o Jacques), visse dal 1654
  al 1705, e fu zio del pi\`u noto Daniel (cui si deve il
  teorema di Bernoulli della dinamica dei fluidi).};%
\index{Bernoulli!Jacob (o Jacques)}
vogliamo ora determinarne alcune costanti caratteristiche.

Verifichiamo per prima cosa che vale la condizione di
normalizzazione: sfruttando la formula per lo sviluppo delle
potenze del binomio, risulta
\begin{equation*}
  \sum_{x=0}^N P(x;N) \; = \;
    \sum_{x=0}^N \frac{N!}{x! \, (N-x)!}
    \, p^x q^{N-x} \; = \;
    {(p+q)}^N \; \equiv \; 1 \peq .
\end{equation*}

Vogliamo ora calcolare la speranza matematica della
variabile $x$ (ossia il numero di successi attesi, in media,
in $N$ prove): per questo useremo la stessa variabile
casuale ausiliaria gi\`a considerata nel paragrafo
\ref{ch:5.teober}, $y$, che rappresenta il numero di
successi nella generica delle $N$ prove eseguite.

Avevamo a suo tempo gi\`a calcolato, sempre nel paragrafo
\ref{ch:5.teober}, la speranza matematica della $y$
\begin{gather}
  E ( y ) \; = \; 1 \cdot p + 0 \cdot q \;
    = \; p \peq ; \notag \\
  \intertext{e, osservando che anche $y^2$
    pu\`o assumere i due soli valori 1 e 0, sempre
    con le probabilit\`a rispettive $p$ e $q$,}
  E \bigl( y^2 \bigr) \; = \; 1 \cdot p + 0
  \cdot q \; = \; p \notag \\
  \intertext{e quindi la varianza della $y$ esiste e vale}
  \var ( y ) \; = \; E \bigl( y^2
    \bigr) - \bigl[ E ( y ) \bigr] ^2
    \; = \; p - p^2 \; = \; p \, (1-p) \;= \; pq \peq
    \label{eq:8.varber} .
\end{gather}

Il numero totale $x$ di successi nelle $N$ prove \`e legato
ai valori $y_i$ della $y$ in ognuna di esse dalla
\begin{gather*}
  x = \sum_{i=1}^N y_i \\
  \intertext{e risulta quindi, per speranza matematica e
    varianza della distribuzione binomiale,}
  E(x) \; = \; E \left( \, \sum_{i=1}^N y_i
    \right) \; = \; \sum_{i=1}^N E \left( y_i
    \right) \; = \; Np \\[1ex]
  \var (x) \; = \; {\sigma_x}^2 \; = \;
    \var \left( \, \sum_{i=1}^N y_i \right) \; = \;
    \sum_{i=1}^N \var \left( y_i \right) \; = \;
    Npq \peq .
\end{gather*}

\begin{figure}[hbtp]
  \vspace*{2ex}
  \begin{center} {
    \input{binom.pstex_t}
  } \end{center}
  \caption[La distribuzione binomiale]
    {La distribuzione binomiale, per un numero
    di prove $N=50$ e due differenti valori della
    probabilit\`a $p$.}
  \label{fig:8.figbin}
\end{figure}

\index{distribuzione!di Bernoulli!e distribuzione normale|(}%
Come \`e evidente dalla figura \ref{fig:8.figbin}, la forma
della distribuzione binomiale \`e molto simile a quella di
una curva di Gauss; si pu\`o in effetti dimostrare che
quando $N$ tende all'infinito la distribuzione di
probabilit\`a dei possibili valori tende ad una
distribuzione normale avente la stessa media $Np$ e la
stessa varianza $Npq$.  Infatti la funzione generatrice dei
momenti della distribuzione di Bernoulli \`e
\begin{align*}
  M_x(t) &= E \bigl( e^{tx} \bigr) \\[1ex]
  &= \sum_{x=0}^N e^{tx} \, \binom{N}{x} \, p^x q^{N-x}
    \\[1ex]
  &= \sum_{x=0}^N \binom{N}{x} \! \bigl( p e^t
    \bigr)^x \! q^{N-x} \\[1ex]
  &= \bigl( p e^t + q \bigr)^N \\
  \intertext{o anche, ricordando che $q=1-p$,}
  M_x(t) &= \left[ 1 + p \bigl( e^t - 1 \bigr)
    \right]^N
\end{align*}
e se, per semplificare i calcoli, ci riferiamo alla
\emph{variabile standardizzata}
\begin{gather*}
  z \; = \; \frac{x - E(x)}{\sigma_x} \; = \; \frac{x -
    Np}{\sqrt{Npq}} \; = \; ax+b \\
  \intertext{ove si \`e posto}
  a \; = \; \frac{1}{\sqrt{Npq}} \makebox[35mm]{e} b \;
    = \; - \, \frac{Np}{\sqrt{Npq}} \\
\end{gather*}
applicando la \eqref{eq:6.fgmcav} si trova
\begin{align*}
  M_z(t) &= e^{tb} \, M_x(at) \\[1ex]
  &= e^{- \frac{Np}{\sqrt{Npq}} t} \left[ 1 + p
    \left( e^{\frac{t}{\sqrt{Npq}}} - 1 \right)
    \right]^N
\end{align*}
da cui, passando ai logaritmi naturali,

\begin{align*}
  \ln M_z(t) &= - \, \frac{Np}{\sqrt{Npq}} \, t + N \,
    \ln \left[ 1 + p \left( e^{\frac{t}{\sqrt{Npq}}} -
    1 \right) \right] \\[1ex]
  &= - \, \frac{Np}{\sqrt{Npq}} \, t + N \left[ p
    \left( e^{\frac{t}{\sqrt{Npq}}} - 1 \right) -
    \frac{p^2}{2} \left( e^{\frac{t}{\sqrt{Npq}}} - 1
    \right)^2 + \right. \\[1ex]
  & \qquad \left. + \frac{p^3}{3} \left(
      e^{\frac{t}{\sqrt{Npq}}} - 1 \right)^3
    +\cdots\right] \\[1ex]
  &= - \, \frac{Np}{\sqrt{Npq}} \, t + N \left\{ p
    \left[ \frac{t}{\sqrt{Npq}} + \frac{1}{2} \,
    \frac{t^2}{Npq} + \frac{1}{6} \,
    \frac{t^3}{(Npq)^\frac{3}{2}} +\cdots\right] -
    \right. \\[1ex]
  & \qquad - \frac{p^2}{2} \left[ \frac{t}{\sqrt{Npq}}
    + \frac{1}{2} \, \frac{t^2}{Npq} + \frac{1}{6} \,
    \frac{t^3}{(Npq)^\frac{3}{2}} +\cdots\right]^2 +
    \\[1ex]
  & \qquad \left. + \frac{p^3}{3} \left[
    \frac{t}{\sqrt{Npq}} + \frac{1}{2} \,
    \frac{t^2}{Npq} + \frac{1}{6} \,
    \frac{t^3}{(Npq)^\frac{3}{2}} +\cdots\right]^3
    +\cdots\right\} \\[1ex]
  &= N \left\{ p \left[ \frac{1}{2} \, \frac{t^2}{Npq}
    + \frac{1}{6} \, \frac{t^3}{(Npq)^\frac{3}{2}}
    +\cdots\right] - \right. \\[1ex]
  & \qquad \left. - \frac{p^2}{2} \left[
    \frac{t^2}{Npq} + \frac{t^3}{(Npq)^\frac{3}{2}}
    +\cdots\right] + \frac{p^3}{3} \left[
    \frac{t^3}{(Npq)^\frac{3}{2}} +\cdots\right]
    \right\} \\[1ex]
  &= N \left[ \frac{1}{2} \, \frac{t^2}{Npq} \, p(1-p)
    + \frac{t^3}{(Npq)^\frac{3}{2}} \left( \frac{p}{6}
    - \frac{p^2}{2} + \frac{p^3}{3} \right) +
    \mathcal{O} \bigl( t^4 N^{-2} \bigr) \right] \\[1ex]
  &= \frac{1}{2} \, t^2 + \mathcal{O} \bigl( t^3
    N^{-\frac{1}{2}} \bigr)
\end{align*}
ove si \`e sviluppato in serie di McLaurin prima
\begin{gather*}
  \ln(1+x) \; = \; x - \frac{x^2}{2} + \frac{x^3}{3}
    +\cdots \\
  \intertext{e poi}
  e^x \; = \; 1 + x + \frac{x^2}{2!} + \frac{x^3}{3!}
    +\cdots
\end{gather*}
e si sono svolti i prodotti tenendo solo i termini dei primi
ordini.

Chiaramente quando $N$ viene fatto tendere all'infinito
tutti i termini eccetto il primo tendono a zero, per cui
\begin{equation*}
  \lim_{N \to \infty} M_z(t) \; = \; e^\frac{t^2}{2}
\end{equation*}
e $M_z(t)$ tende quindi alla funzione generatrice dei
momenti di una distribuzione normale standardizzata; in
conseguenza si \`e effettivamente provato, visto uno dei
teoremi citati nel paragrafo \ref{ch:6.fugeca}, che la
distribuzione binomiale tende ad una distribuzione normale.

In pratica, quando il numero di prove $N$ \`e elevato e la
probabilit\`a $p$ non \`e troppo vicina ai valori estremi 0
ed 1, la distribuzione binomiale \`e bene approssimata da
una distribuzione normale; in generale si ritiene che
l'approssimazione sia accettabile quando entrambi i prodotti
$Np$ e $Nq$ hanno valore non inferiore a 5.%
\index{distribuzione!di Bernoulli!e distribuzione normale|)}%
\index{distribuzione!di Bernoulli|)}

\subsection{Applicazione: decadimenti radioattivi}
Se la probabilit\`a $\Lambda_t$ per un singolo nucleo
instabile di decadere in un intervallo di tempo $t$ \`e
costante, la probabilit\`a di avere un numero prefissato di
decadimenti nel tempo $t$ in un insieme di $N$ nuclei \`e
data dalla distribuzione binomiale; in particolare, il
numero medio di decadimenti in $N$ nuclei e nel tempo $t$
\`e $N \Lambda_t$.

Se si ammette poi che $\Lambda_t$ sia proporzionale al
tempo\/\footnote{Questa ipotesi pu\`o evidentemente essere
  soddisfatta solo in prima approssimazione: basta pensare
  al fatto che $\Lambda_t$ deve raggiungere l'unit\`a solo
  dopo un tempo infinito.  In particolare, la probabilit\`a
  per un nucleo di decadere in un tempo $2t$ vale $
  \Lambda_{2t} = \Lambda_t + \left( 1 - \Lambda_t \right)
  \cdot \Lambda_t = 2 \Lambda_t - {\Lambda_t}^2$; e
  l'ipotesi fatta \`e in effetti valida solo se $\Lambda_t$
  \`e infinitesimo, o (in pratica) se l'osservazione
  riguarda un lasso di tempo trascurabile rispetto alla vita
  media.} $t$, indicando con $\lambda$ la probabilit\`a di
decadimento nell'unit\`a di tempo avremo $\Lambda_t =
\lambda t$; in un tempo infinitesimo $\de t$, il numero di
atomi $N(t)$ presente al tempo $t$ varia mediamente di
\begin{equation*}
  \de N \; = \; - N \, \lambda \:
  \de t \peq .
\end{equation*}

Separando le variabili ed integrando, il numero \emph{medio}
di atomi presenti al tempo $t$, dopo il decadimento di una
parte di quelli $N_0$ presenti all'istante iniziale $t = 0$,
\`e dato dalla
\begin{gather}
  N(t) \; = \; N_0 \, e^{- \lambda t} \label{eq:8.nt1}
    \\
  \intertext{ed il numero medio di decadimenti dalla}
  N_0 - N(t) \; = \; N_0 \bigl( 1 - e^{- \lambda t}
    \bigr) \peq . \notag
\end{gather}

La \emph{vita media}%
\index{vita media}
$\tau$ di una sostanza radioattiva si pu\`o definire come il
tempo necessario perch\'e il numero originario di nuclei si
riduca mediamente di un fattore $1/e$; quindi $ \tau = 1 /
\lambda $, e la \eqref{eq:8.nt1} si pu\`o riscrivere
\begin{equation} \label{eq:8.nt2}
  N(t) \; = \; N_0 \, e^{- \frac{t}{\tau}} \peq .
\end{equation}

\subsection{Applicazione: il rapporto di asimmetria}%
\index{rapporto di asimmetria|see{asimmetria, rapporto di}}%
\index{asimmetria, rapporto di|(}%
\label{ch:8.rasim1}
Frequentemente, nella fisica, si devono considerare
esperimenti in cui si cerca di mettere in evidenza delle
\emph{asimmetrie}; ovvero, la non invarianza della funzione
di frequenza di una qualche variabile casuale per
riflessione rispetto ad un piano.  Supponiamo, come esempio,
di considerare la possibilit\`a che un dato fenomeno abbia
una diversa probabilit\`a di presentarsi \emph{in avanti} o
\emph{all'indietro} rispetto ad una opportuna superficie di
riferimento; e di raccogliere $N$ eventi sperimentali
dividendoli in due sottoinsiemi (mutuamente esclusivi ed
esaurienti) collegati a queste due categorie, indicando con
$F$ e $B$ (iniziali delle due parole inglesi \emph{forward}
e \emph{backward}) il loro numero: ovviamente dovr\`a
risultare $N = F + B$.

Il cosiddetto \emph{rapporto di asimmetria}, $R$, si
definisce come
\begin{equation} \label{eq:8.rapasi}
  R \; = \; \frac{F - B}{F + B} \; = \; \frac{F - B}{N} \; =
  \; \frac{2F}{N} - 1 \peq :
\end{equation}
\`e ovvio sia che $-1 \leq R \leq 1$, sia che soltanto due
dei quattro valori $N$, $F$, $B$ ed $R$ sono indipendenti;
e, volendo, dall'ultima forma della \eqref{eq:8.rapasi} si
possono ricavare le espressioni di $F$ e $B$ in funzione di
$N$ ed $R$, ovvero
\begin{align*}
  F &= \frac{N (1 + R)}{2} &&\text{e} & B &= \frac{N (1 -
    R)}{2} \peq .
\end{align*}

Se indichiamo con $p$ la probabilit\`a di un evento in
avanti (e con $q = 1 - p$ quella di uno all'indietro), il
trovare esattamente $F$ eventi in avanti su un totale di $N$
ha probabilit\`a data dalla distribuzione binomiale: ovvero
\begin{gather*}
  \Pr(F) = \binom{N}{F} \, p^F (1 - p)^{N - F} \\
  \intertext{con, inoltre,}
  E(F) = Np \makebox[40mm]{e} \var(F) = N p \, (1 - p) \peq
  . \\
  \intertext{Ma, per quanto detto, c'\`e una corrispondenza
    biunivoca tra i valori di $N$ ed $F$ da una parte, e
    quello di $R$; cos\`\i\ che}
  \Pr(R) = \binom{N}{\frac{N (1 + R)}{2}} \, p^{\frac{N (1
      + R)}{2}} \, (1 - p)^{\frac{N (1 - R)}{2}} \peq ,
  \\[1ex]
  E(R) = \frac{2}{N} \, E(F) - 1 = 2 p - 1 \\
  \intertext{e}
  \var(R) = \frac{4}{N^2} \, \var(F) = \frac{4 p ( 1 -
    p)}{N} \peq .
\end{gather*}

\emph{Se il numero di eventi nel campione \`e elevato} e $p$
lontano dai valori estremi, cos\`\i\ da potere sia sfruttare
l'approssimazione normale alla distribuzione di Bernoulli,
sia pensare che risulti
\begin{align*}
  p &\simeq \frac{F}{N} &&\text{che} & q &\simeq \frac{B}{N}
  \peq ,
\end{align*}
come conseguenza anche la distribuzione di $R$ sar\`a
approssimativamente normale; e con i primi due momenti dati
da
\begin{align*}
  E(R) &\simeq 2 \, \frac{F}{N} - 1 &&\text{e} & \var(R)
  &\simeq 4 \, \frac{F B}{N^3} \peq .
\end{align*}

Del rapporto di asimmetria parleremo ancora pi\`u avanti,
nel corso di questo stesso capitolo: pi\`u esattamente nel
paragrafo \ref{ch:8.rasim2}.%
\index{asimmetria, rapporto di|)}

\subsection{La distribuzione binomiale negativa}%
\index{distribuzione!binomiale negativa|(}%
\index{binomiale negativa,
  distribuzione|see{distribuzione binomiale negativa}}
Consideriamo ancora un evento casuale $E$ ripetibile, avente
probabilit\`a costante $p$ di presentarsi (e quindi
probabilit\`a $q = 1 - p$ di non presentarsi) in una singola
prova; in pi\`u prove successive l'evento seguir\`a dunque
la statistica di Bernoulli.  Vogliamo ora calcolare la
probabilit\`a $f(x; N, p)$ che, prima che si verifichi
l'$N$-esimo successo, si siano avuti esattamente $x$
insuccessi; o, se si preferisce, la probabilit\`a che
l'$N$-simo successo si presenti nella $(N+x)$-sima prova.

L'evento casuale considerato si realizza se e solo se nelle
prime $N + x - 1$ prove si \`e presentata una delle
possibili sequenze di $N - 1$ successi e $x$ fallimenti; e
se poi, nella prova successiva, si ha un ulteriore successo.
La prima condizione, applicando la \eqref{eq:8.binom}, ha
probabilit\`a
\begin{equation*}
  \binom{N + x - 1}{N - 1} \, p^{N - 1} \, q^x \peq ;
\end{equation*}
e, vista l'indipendenza statistica delle prove tra loro,
risulta dunque
\begin{equation} \label{eq:8.bineg}
  f(x; N, p) \; = \; \binom{N + x - 1}{N - 1} \, p^N \, q^x
  \; = \; \binom{N + x - 1}{x} \, p^N \, q^x \peq .
\end{equation}
(nell'ultimo passaggio si \`e sfruttata la propriet\`a dei
coefficienti binomiali $\binom{N}{K} \equiv \binom{N}{N -
  K}$; vedi in proposito il paragrafo \ref{ch:a.combina}).

Questa distribuzione di probabilit\`a prende il nome di
\emph{distribuzione binomiale
  negativa}\thinspace\footnote{La distribuzione binomiale
  negativa \`e talvolta chiamata anche \emph{distribuzione
    di Pascal} o \emph{di P\'olya}.}; il motivo di tale nome
\`e che l'equazione \eqref{eq:8.bineg} pu\`o essere
riscritta in forma compatta sfruttando una ``estensione''
dei coefficienti binomiali che permette di definirli anche
per valori negativi di $N$.  La funzione generatrice dei
momenti \`e
\begin{gather*}
  M_x(t) \; = \; E \bigl( e^{tx} \bigr) \; = \; \left(
    \frac{p}{1 - q e^t} \right)^N \peq ; \\
  \intertext{da questa si possono poi ricavare la speranza
    matematica}
  E(x) = \frac{N q}{p} \peq , \\
  \intertext{e la varianza}
  \var(x) = N \, \frac{q}{p^2} \peq .
\end{gather*}

La distribuzione binomiale negativa con $N = 1$ prende il
nome di \emph{distribuzione geometrica};%
\index{distribuzione!geometrica|(}
la probabilit\`a \eqref{eq:8.bineg} diventa
\begin{equation*}
  f(x; p) = p q^x \peq ,
\end{equation*}
ed \`e quella dell'evento casuale consistente nell'ottenere
il primo successo dopo esattamente $x$ insuccessi; ponendo
$N = 1$ nelle formule precedenti, speranza matematica e
varianza della distribuzione geometrica sono rispettivamente
\begin{equation*}
  E(x) = \frac{q}{p} \makebox[35mm]{e} \var(x) =
  \frac{q}{p^2} \peq .
\end{equation*}%
\index{distribuzione!geometrica|)}%
\index{distribuzione!binomiale negativa|)}

\section{La distribuzione di Poisson}%
\index{distribuzione!di Poisson|(}%
\index{Poisson!distribuzione di|see{distribuzione di Poisson}}%
\label{ch:8.poisson}
Sia $E$ un evento casuale che avvenga rispettando le
seguenti ipotesi:
\begin{enumerate}
\item La probabilit\`a del verificarsi dell'evento $E$ in un
  intervallo di tempo\/\footnote{Sebbene ci si riferisca,
    per esemplificare le nostre considerazioni, ad un
    processo \emph{temporale} (e si faccia poi l'esempio del
    numero di decadimenti in un intervallo costante di tempo
    per una sostanza radioattiva come quello di una
    variabile casuale che segue la distribuzione di
    Poisson), gli stessi ragionamenti naturalmente si
    applicano anche a fenomeni fisici riferiti ad intervalli
    di differente natura, per esempio di spazio.

    Cos\`\i\ anche il numero di urti per unit\`a di
    lunghezza delle molecole dei gas segue la distribuzione
    di Poisson (se si ammette che la probabilit\`a di un
    urto nel percorrere un intervallo infinitesimo di spazio
    sia proporzionale alla sua lunghezza, ed analogamente
    per le altre ipotesi).} molto piccolo (al limite
  infinitesimo) $\de t$ \`e proporzionale alla durata di
  tale intervallo;
\item\label{it:pois2} Il verificarsi o meno dell'evento in
  un certo intervallo temporale \`e indipendente dal
  verificarsi o meno dell'evento prima o dopo di esso;
\item La probabilit\`a che pi\`u di un evento si
  verifichi in un tempo infinitesimo $\de t$ \`e
  infinitesima di ordine superiore rispetto a $\de t$.
\end{enumerate}
vogliamo ora ricavare la probabilit\`a $P(x;t)$ che in un
intervallo di tempo finito, di durata $t$, si verifichi
esattamente un numero prefissato $x$ di eventi $E$.  Usando
questa simbologia, la prima ipotesi fatta sul processo
casuale in esame si scrive
\begin{align*}
  P (1; \de t) \; &= \; \lambda \, \de t
  && \Longrightarrow &
  P (0; \de t) \; &= \; 1 - \lambda \, \de t
\end{align*}
e, viste le altre ipotesi ed applicando in conseguenza i
teoremi delle probabilit\`a totali e composte, la
probabilit\`a di avere $x$ eventi in un intervallo di tempo
lungo $t+\de t$ \`e data, a meno di infinitesimi di ordine
superiore, da
\begin{align*}
  P(x; t+\de t) &= P(x-1;t) \cdot
    P(1;\de t) + P(x;t) \cdot P(0; \de t)
    \\[1ex]
  &= P(x-1; t) \, \lambda \, \de t +
    P(x;t) \, (1 - \lambda \, \de t)
\end{align*}
cio\`e
\begin{equation*}
  \frac{P(x;t+\de t) - P(x;t)}{\de t} \; \equiv \;
    \frac{\de}{\de t} \, P(x;t) \; = \;
    - \lambda \, P(x;t) \, + \, \lambda \, P(x-1;t) \peq .
\end{equation*}

Ora, quando $x=0$, essendo chiaramente nulla la
probabilit\`a di avere un numero negativo di eventi $E$ in
un tempo qualsiasi, risulta in particolare
\begin{gather*}
  \frac{\de}{\de t} \, P(0;t) =
    - \lambda \, P(0;t) \\
  \intertext{da cui}
  P(0;t) = e^{- \lambda t}
\end{gather*}
(la costante di integrazione si determina imponendo che
$P(0;0)=1 $).  Da questa relazione si pu\`o ricavare
$P(1;t)$ e, con una serie di integrazioni successive,
$P(x;t)$: risulta
\begin{equation} \label{eq:8.poisson}
  P(x;t) = \frac{ (\lambda t)^x }{ x! } \,
    e^{-\lambda t} \peq .
\end{equation}

In questa espressione $x$ \`e l'unica variabile casuale, e
$t$ funge da parametro: se introduciamo la nuova grandezza
$\alpha = \lambda t$, possiamo scrivere
\begin{equation} \label{eq:8.poiss2}
  P(x;\alpha) = \frac{\alpha^x}{x!} \, e^{-\alpha} \peq .
\end{equation}

Questa distribuzione di probabilit\`a per una variabile
casuale (discreta) $x$ prende il nome di \emph{distribuzione
  di Poisson}\thinspace\footnote{Sim\'eon Denis Poisson
  visse in Francia dal 1781 al 1840; matematico e fisico di
  valore, si occup\`o della teoria degli integrali e delle
  serie, di meccanica, elettricit\`a, magnetismo ed
  astronomia.  Gli studi sulla distribuzione che porta il
  suo nome compaiono nel trattato del 1837 ``Recherches sur
  la probabilit\'e des jugements\ldots''.};%
\index{Poisson!Sim\'eon Denis} da essa si pu\`o ottenere, ad
esempio, la probabilit\`a di ottenere $x$ decadimenti in una
massa nota di sostanza radioattiva nel tempo $t$: infatti
per questo tipo di processi fisici risultano soddisfatte le
tre ipotesi di partenza.

\index{distribuzione!di Poisson!e distribuzione di Bernoulli|(}%
Pi\`u esattamente, la probabilit\`a di avere precisamente
$x$ decadimenti radioattivi nel tempo $t$ \`e data dalla
distribuzione binomiale; la distribuzione di Poisson \`e una
approssimazione alla distribuzione binomiale che si pu\`o
ritenere valida qualora si considerino eventi casuali di
probabilit\`a estremamente piccola, e che ci \`e possibile
vedere solo perch\'e si compiono osservazioni su un numero
molto elevato di essi: in formula, quando
\begin{equation} \label{eq:8.poival}
  p^2 \ll p \makebox[30mm]{e} p \ll Np \ll N
\end{equation}
(\emph{eventi rari su larga base statistica}).

Anche se la distribuzione di Poisson \`e, come nel caso dei
decadimenti radioattivi, una approssimazione di quella
binomiale, si preferisce per\`o sempre usarla nella pratica
al posto di quest'ultima quando le \eqref{eq:8.poival} siano
approssimativamente verificate: infatti se $N$ \`e grande i
fattoriali e le potenze presenti nella \eqref{eq:8.binom}
rendono generalmente l'espressione difficile da calcolare.%
\index{distribuzione!di Poisson!e distribuzione di Bernoulli|)}
Verifichiamo ora la condizione di normalizzazione:
\begin{equation*}
  \sum_{x=0}^{+ \infty} P(x) \; = \;
     e^{-\alpha} \sum_{x=0}^{+ \infty}
     \frac{\alpha^x}{x!} \; = \;
     e^{-\alpha} e^{\alpha} \; \equiv \; 1
\end{equation*}
(riconoscendo nella sommatoria l'espressione di uno sviluppo
in serie di McLaurin della funzione esponenziale).
Calcoliamo poi la speranza matematica di $x $:
\begin{align*}
  E(x) &= \sum_{x=0}^{+ \infty} x \,
  \frac{\alpha^x}{x!} \, e^{-\alpha} \\[1ex]
  &= \sum_{x=1}^{+ \infty} x \,
  \frac{\alpha^x}{x!} \, e^{-\alpha} \\[1ex]
  &= \alpha \, e^{-\alpha} \sum_{x=1}^{+ \infty}
  \frac{\alpha^{x-1}}{(x-1)!} \\[1ex]
  &= \alpha \, e^{-\alpha} \sum_{y=0}^{+ \infty}
  \frac{\alpha^y}{y!} \\[1ex]
  &= \alpha \, e^{-\alpha} e^\alpha \\[1ex]
  &= \alpha \peq .
\end{align*}

Nei passaggi si \`e prima osservato che il primo termine
della sommatoria ($x = 0$) \`e nullo, e si \`e poi
introdotta la nuova variabile $y = x - 1$.  Troviamo ora la
speranza matematica di $x^2$: con passaggi analoghi, si
ottiene
\begin{align*}
  E(x^2) &= \sum_{x=0}^{+ \infty} x^2
    \frac{\alpha^x}{x!} \, e^{-\alpha} \\[1ex]
  &= \alpha \sum_{x=1}^{+ \infty} x \,
    \frac{\alpha^{x-1}}{(x-1)!}
    \, e^{-\alpha} \\[1ex]
  &= \alpha \sum_{x=1}^{+ \infty}
    \bigl[ (x-1)+1 \bigr]
    \, \frac{\alpha^{x-1}}{(x-1)!} \, e^{-\alpha}
    \\[1ex]
  &= \alpha \left[ \,
    \sum_{y=0}^{+ \infty} y \,
    \frac{\alpha^y}{y!} \, e^{-\alpha} \; + \;
    \sum_{y=0}^{+ \infty} \frac{\alpha^y}{y!} \,
    e^{-\alpha} \right] \\[1ex]
  &= \alpha \left[ \,
    \sum_{y=0}^{+ \infty} y \, P(y) \; + \;
    \sum_{y=0}^{+ \infty} P(y) \right] \\[1ex]
  &= \alpha \, (\alpha + 1)
\end{align*}
e la varianza di $x$ risulta allora anch'essa data da
\begin{equation*}
  \var (x) \; = \;
    E \bigl( x^2 \bigr) - \bigl[ E(x) \bigr]^2
    \; = \; \alpha \, (\alpha+1) - \alpha^2 \; = \;
    \alpha \peq .
\end{equation*}
\begin{figure}[hbtp]
  \vspace*{2ex}
  \begin{center} {
    \input{poisson.pstex_t}
  } \end{center}
  \caption[La distribuzione di Poisson]
    {La distribuzione di
    Poisson, per tre diversi valori del parametro
    $\alpha$.}
  \label{fig:8.poissn}
\end{figure}

La funzione generatrice dei momenti, come si potrebbe
facilmente ottenere dalla definizione, \`e la
\begin{gather}
  M_x(t) \; = \; e^{- \alpha} e^{\alpha e^t} \; = \;
  e^{\alpha \left( e^t - 1 \right)} \peq ;
  \label{eq:8.fgmpoi} \\
  \intertext{la funzione caratteristica di variabile reale}
  \phi_x (t) = e^{\alpha \left( e^{it} - 1
    \right)} \peq , \notag \\
  \intertext{e la funzione caratteristica di variabile
    complessa}
  \phi_x (z) = e^{\alpha (z - 1)} \peq .
  \label{eq:8.fcapoi}
\end{gather}
Da esse potrebbero essere ricavati tutti i momenti
successivi; i primi quattro valgono
\begin{align*}
  &\begin{cases}
    \lambda_1 \; \equiv \; E(x) \; = \; \alpha \\[1ex]
    \lambda_2 \; \equiv \; E \bigl( x^2 \bigr) \; = \;
    \alpha \, ( \alpha + 1 ) \\[1ex]
    \lambda_3 \; = \; \alpha \, \bigl[ (\alpha + 1)^2 +
    \alpha \bigr] \\[1ex]
    \lambda_4 \; =  \;\alpha \, \bigl[ \alpha^3 + 6 \alpha^2
    + 7 \alpha + 1 \bigr]
  \end{cases}
  &&\begin{cases}
    \mu_1 \; \equiv \; 0 \\[1ex]
    \mu_2 \; \equiv \; \var(x) \; = \; \alpha \\[1ex]
    \mu_3 \; = \; \alpha \\[1ex]
    \mu_4 \; = \; \alpha \, ( 3 \alpha + 1 )
  \end{cases}
\end{align*}

Un'altra conseguenza della \eqref{eq:8.fgmpoi} \`e che la
somma $w = x + y$ di due variabili casuali indipendenti che
seguano la distribuzione di Poisson (con valori medi $\xi$
ed $\eta$) segue anch'essa tale distribuzione (con valore
medio pari a $\xi + \eta$):
\begin{equation} \label{eq:8.sumpoi}
  M_w(t) \; = \; e^{ \xi \left( e^t - 1 \right) } \, e^{
    \eta \left( e^t - 1 \right) } \; = \; e^{ (\xi + \eta)
    \left( e^t - 1 \right) } \peq .
\end{equation}

\index{distribuzione!di Poisson!e distribuzione normale|(}%
Anche la distribuzione di Poisson, come si vede dai grafici
di figura \ref{fig:8.poissn}, \`e bene approssimata da una
distribuzione normale quando $\alpha$ \`e abbastanza
elevato; questo non deve stupire, visto lo stretto legame
che esiste tra la distribuzione di Poisson e quella di
Bernoulli --- il cui limite per grandi $N$ \`e appunto la
funzione di Gauss.  Volendo, si potrebbe ripetere per la
funzione generatrice \eqref{eq:8.fgmpoi} una analisi analoga
a quella a suo tempo compiuta per la distribuzione
binomiale; in questo modo si proverebbe rigorosamente il
fatto che anche la distribuzione di Poisson, per grandi
$\alpha$, tende a quella normale.

In genere si ritiene che, per valori medi $\alpha \gtrsim
8$, si possa ritenere soddisfacente l'approssimazione
normale alla distribuzione di Poisson.%
\index{distribuzione!di Poisson!e distribuzione normale|)}%
\index{distribuzione!di Poisson|)}

\subsection{Applicazione: esperimenti ``negativi''}
Si osserva un numero $N_0$ di protoni per un tempo $t$, e
non si registra alcun decadimento.  Quale \`e il limite
inferiore che si pu\`o dare sulla vita media del protone,
$\tau$, con una probabilit\`a (\emph{livello di confidenza})
del 95\%?

L'evento casuale consistente nel non avere osservato alcun
decadimento \`e somma logica di altri due eventi mutuamente
esclusivi: o il protone \`e stabile (e non pu\`o quindi
decadere); o il protone \`e instabile, e si sono inoltre
verificati 0 decadimenti nel tempo di osservazione
(supponiamo per semplicit\`a che ognuno di essi abbia poi
probabilit\`a 1 di essere osservato).

In questa seconda eventualit\`a, dalla \eqref{eq:8.nt2} si
pu\`o ricavare il numero medio di decadimenti attesi nel
tempo $t$, che \`e
\begin{equation*}
  \alpha \; = \; N_0 \left( 1 -
    e^{-\frac{t}{\tau}} \right) \; \approx
    \; N_0 \, \frac{t}{\tau}
\end{equation*}
(supponendo che $\tau$ sia molto maggiore del periodo di
osservazione $t$); e da esso la probabilit\`a di osservare 0
eventi sempre nel tempo $t$, che \`e data dalla statistica
di Poisson e vale
\begin{equation*}
  P(0) \; = \; \frac{\alpha^0}{0!} \,
    e^{-\alpha} \; = \; e^{-\alpha} \peq .
\end{equation*}

Quello che si domanda \`e di calcolare, assumendo come certa
l'ipotesi che il protone sia instabile, il valore minimo che
deve avere la sua vita media perch\'e la probabilit\`a di
non osservare nulla sia almeno del 95\%: e questo avviene
quando
\begin{gather*}
  P(0) \; = \; e^{-\alpha} \; \approx \;
    e^{-N_0 \, \frac{t}{\tau}} \; \ge \;
    0.95 \\[1ex]
  - \, N_0 \, \frac{t}{\tau} \; \ge \;
    \ln 0.95 \\[1ex]
  \tau \; \ge \; - \, \frac{N_0 \, t}{\ln 0.95}
\end{gather*}
(abbiamo invertito il segno della disuguaglianza nell'ultimo
passaggio perch\'e $\ln 0.95 \approx -0.0513$ \`e un numero
negativo).

\subsection{Applicazione: ancora il rapporto di asimmetria}%
\index{asimmetria, rapporto di|(}%
\label{ch:8.rasim2}

Nel paragrafo \ref{ch:8.rasim1} abbiamo supposto che il
numero $N$ di osservazioni effettuate sia noto a priori e
costante: per\`o questo non \`e in generale corretto; e,
nella realt\`a, il numero di volte in cui un certo fenomeno
fisico si presenter\`a \`e di norma esso stesso una
variabile casuale.  Continuiamo la nostra analisi supponendo
che si tratti di un fenomeno nel quale $N$ segua la
distribuzione di Poisson con valore medio $\nu$.

Continuando ad usare gli stessi simboli del paragrafo
\ref{ch:8.rasim1}, la probabilit\`a congiunta di osservare
$N$ eventi dei quali $F$ in avanti \`e in realt\`a
\begin{align*}
  \Pr(F, N) &= \frac{\nu^N}{N!} \, e^{- \nu} \cdot
  \frac{N!}{F! \, (N - F)!} \, p^F q^{N-F} \\[1ex]
  &= \frac{ p^F q^{N-F} }{F! \, (N - F)!} \, \nu^N e^{- \nu}
  \peq ;
\intertext{o anche, cambiando coppia di variabili casuali
  passando da $\{ F, N \}$ a $\{ F, B \}$:}
  \Pr(F, B) &= \frac{p^F q^B}{F! \, B!} \, \nu^{F + B} e^{-
    \nu} \\[1ex]
  &= \frac{(\nu p)^F (\nu q)^B}{F! \, B!} \, e^{- \nu}
  \\[1ex]
  &= \frac{(\nu p)^F}{F!} \, e^{- \nu p} \cdot \frac{(\nu
    q)^B}{B!} \, e^{- \nu q} \peq .
\end{align*}
che \`e il \emph{prodotto di due funzioni di frequenza di
  Poisson}.

In definitiva abbiamo scoperto che la composizione di un
processo di Poisson e di un processo di Bernoulli equivale
al prodotto di due Poissoniane: il numero $N$ di eventi
osservato segue la statistica di Poisson; la scelta dello
stato finale $F$ o $B$ quella binomiale; ma tutto avviene
come se i decadimenti dei due tipi, in avanti ed
all'indietro, si verificassero \emph{separatamente} ed
\emph{indipendentemente} secondo la statistica di Poisson.

Accettato questo fatto appena dimostrato (ancorch\'e
inaspettato), e pensando sia ad $F$ che a $B$ come variabili
casuali statisticamente indipendenti tra loro e che seguono
singolarmente la statistica di Poisson, per il rapporto di
asimmetria \emph{asintoticamente} (ovvero per grandi $N$) si
ricava:
\begin{gather*}
  \var(F) \; = \; E(F) \; \simeq \; F \\[1ex]
  \var(B) \; = \; E(B) \; \simeq \; B
\end{gather*}
e, per il rapporto di asimmetria $R$:
\begin{align*}
  R &= \frac{F - B}{F + B} \\[1ex]
  &\simeq \frac{E(F) - E(B)}{E(F) + E(B)} + \frac{\partial
    R}{\partial F} \, \bigl[ F - E(F) \bigr] +
  \frac{\partial R}{\partial B} \, \bigl[ B - E(B) \bigr]
  \peq ; \\
  \intertext{visto che la speranza matematica degli ultimi
    due termini \`e nulla,}
  E(R) &\simeq \frac{E(F) - E(B)}{E(F) + E(B)} \\[1ex]
  &= \frac{2 F}{N} - 1 \peq ; \\[1ex]
  \var(R) &\simeq \left( \frac{\partial R}{\partial F}
  \right)^2 \var(F) + \left( \frac{\partial R}{\partial B}
  \right)^2 \var(B) \\[1ex]
  &= \frac{4}{(F + B)^4} \left[ B^2 \var(F) + F^2 \var(B)
  \right] \\[1ex]
  &= \frac{4 F B}{(F + B)^3} \\[1ex]
  &= 4 \, \frac{F B}{N^3} \peq ,
\end{align*}
e le cose, di fatto, non cambiano (almeno nel limite dei
grandi $N$) rispetto alla precedente analisi del paragrafo
\ref{ch:8.rasim1}.%
\index{asimmetria, rapporto di|)}

\subsection{La distribuzione esponenziale}%
\index{distribuzione!esponenziale|(}
Alla distribuzione di Poisson ne \`e strettamente legata
un'altra, quella \emph{esponenziale}: sia infatti un
fenomeno casuale qualsiasi che segua la distribuzione di
Poisson, ovvero tale che la probabilit\`a di osservare $x$
eventi nell'intervallo finito di tempo $t$ sia data dalla
\eqref{eq:8.poisson}; definiamo una nuova variabile casuale,
$\delta$, come l'intervallo di tempo che intercorre tra due
eventi successivi.

Visto che in un tempo $\delta$ nessun evento deve venire
osservato, la probabilit\`a che $\delta$ risulti maggiore di
un valore predeterminato $d$ coincide con la probabilit\`a
di osservare zero eventi nel tempo $d$:
\begin{gather*}
  \Pr (\delta > d) \; \equiv \; \Pr (0;d) \; = \; e^{-
    \lambda d} \\
  \intertext{e quindi la \emph{funzione di distribuzione} di
    $\delta$ \`e la}
  F(d) \; = \; \Pr (\delta \le d) \; = \; 1 - e^{- \lambda
    d}
\end{gather*}
Come conseguenza, la funzione di frequenza \`e
\emph{esponenziale}:
\begin{equation} \label{eq:8.expon}
  f(\delta) \; = \; \frac{\de \, F(\delta)}{\de \delta} \; =
  \; \lambda \, e^{- \lambda \delta} \peq ;
\end{equation}
e, volendo, da essa si pu\`o ricavare la funzione
caratteristica --- che vale
\begin{equation*}
  \phi_\delta (t) = \frac{\lambda}{\lambda - i t} \peq .
\end{equation*}

I momenti successivi della distribuzione esponenziale si
possono ottenere o integrando direttamente la funzione
densit\`a di probabilit\`a (moltiplicata per potenze
opportune di $\delta$) o derivando successivamente la
funzione caratteristica; troviamo i primi due momenti,
speranza matematica e varianza, usando questo secondo
metodo:
\begin{gather*}
  \frac{\de \, \phi_\delta (t)}{\de t} \; = \;
  \frac{\de}{\de t} \left( \frac{ \lambda }{ \lambda - i t
      } \right) \; = \; \frac{ i \lambda }{ (\lambda - i
    t)^2 } \\[2ex]
  \left. \frac{\de \, \phi_\delta (t)}{\de t} \right|_{t=0}
  \; = \; \frac{i}{\lambda} \; \equiv \; i \cdot E(\delta)
  \\
  \intertext{per cui la speranza matematica di $\delta$
    vale}
  E(\delta) = \frac{1}{\lambda} \peq ; \\
  \intertext{poi}
  \frac {\de^2 \, \phi_\delta (t)}{\de t^2} \; = \; \frac{-i
    \lambda \cdot 2 (\lambda - it)(- i)}{(\lambda - it)^4}
  \; = \; - \frac{2 \lambda}{(\lambda - it)^3} \\[2ex]
  \left. \frac {\de^2 \, \phi_\delta (t)}{\de t^2}
  \right|_{t=0} \; = \; - \frac{2}{\lambda^2} \; \equiv \;
  i^2 \, E \bigl( \delta^2 \bigr) \; = \; - E \bigl(
  \delta^2 \bigr) \peq , \\
  \intertext{ed infine la varianza \`e}
  \var (\delta) \; = \; E \bigl( \delta^2 \bigr) - \bigl[
  E(\delta) \bigr]^2 \; = \; \frac{1}{\lambda^2} \peq .
\end{gather*}

Se una variabile casuale $t$ rappresenta il tempo trascorso
tra due eventi casuali successivi che seguono una
distribuzione di Poisson, $t$ necessariamente ha una
distribuzione di probabilit\`a di tipo esponenziale data
dalla \eqref{eq:8.expon}; vogliamo ora calcolare la
probabilit\`a che $t$ sia maggiore di una quantit\`a $t_0 +
\Delta t$, \emph{condizionata} per\`o dal sapere in anticipo
che $t$ \`e sicuramente maggiore di $t_0$.  Sfruttando la
\eqref{eq:3.leprco}, abbiamo:
\begin{align*}
  \Pr( t > t_0 + \Delta t \, | \, t > t_0 ) &= \frac{ \Pr(
    t > t_0 + \Delta t ) }{ \Pr( t > t_0 ) } \\[2ex]
  &= \frac{ \int_{t_0 + \Delta t}^{+ \infty} e^{- \lambda t}
    \, \de t }{ \int_{t_0}^{+ \infty} e^{- \lambda t} \, \de
    t } \\[2ex]
  &= \frac{ \left[ -e^{-\lambda t} \right]_{t_0 + \Delta
      t}^{+ \infty} }{ \left[ -e^{-\lambda t}
    \right]_{t_0}^{+ \infty} } \\[2ex]
  &= \frac{ e^{-\lambda ( t_0 + \Delta t )} }{ e^{-\lambda
      t_0} } \\[2ex]
  &= e^{-\lambda \, \Delta t} \\[1ex]
  &\equiv \Pr \left( t > \Delta t \right) \peq .
\end{align*}

In conclusione, la distribuzione esponenziale (ovvero la
cadenza temporale di eventi casuali che seguono la
statistica di Poisson) \emph{non ricorda la storia
  precedente}: il presentarsi o meno di uno di tali eventi
in un tempo $\Delta t$ non dipende in alcun modo da quello
che \`e accaduto nell'arbitrario intervallo di tempo $t_0$
precedente; cos\`\i\ come ci dovevamo aspettare, vista
l'ipotesi numero \ref{it:pois2} formulata a pagina
\pageref{it:pois2}.%
\index{distribuzione!esponenziale|)}

\subsection{La distribuzione di Erlang}%
\index{distribuzione!di Erlang|(}%
\index{Erlang!distribuzione di|see{distribuzione di Erlang}}
La funzione di frequenza esponenziale \eqref{eq:8.expon} si
pu\`o considerare come un caso particolare di un'altra
funzione di frequenza, detta \emph{di
  Erlang}\thinspace\footnote{Agner Krarup Erlang fu un
  matematico danese vissuto dal 1878 al 1929; si occup\`o di
  analisi e di fisica oltre che di statistica.  Dette
  notevoli contributi alla tabulazione di varie funzioni, ed
  applic\`o in particolare la statistica a numerosi problemi
  relativi al traffico
  telefonico.}.%
\index{Erlang!Agner Krarup}
Supponiamo di voler trovare la densit\`a di probabilit\`a
$f_n(t; \lambda)$ dell'evento casuale consistente nel
presentarsi, dopo un tempo $t$, dell'$n$-esimo di una serie
di altri eventi che seguano la statistica di Poisson con
costante di tempo $\lambda$; la \eqref{eq:8.expon} \`e
ovviamente la prima di esse, $f_1(t; \lambda) = \lambda \,
e^{-\lambda t}$.

Il secondo evento si manifesta dopo un tempo $t$ con
densit\`a di probabilit\`a data da
\begin{align*}
  f_2(t; \lambda) &= \int_0^t f_1(x; \lambda) \, f_1(t-x;
  \lambda) \, \de x \\[1ex]
  &= \lambda^2 \int_0^t e^{ -\lambda x } \, e^{ -\lambda
    (t-x) } \, \de x \\[1ex]
  &= \lambda^2 \, e^{ -\lambda t} \int_0^t \! \de x \\[1ex]
  &= \lambda^2 \, t \, e^{ -\lambda t } \peq ;
\end{align*}
si \`e infatti supposto che il primo dei due eventi si sia
presentato dopo un tempo $x$ (con $0 < x < t$), si \`e
sfruttata l'indipendenza statistica degli eventi casuali tra
loro ed infine si \`e sommato su tutti i possibili valori di
$x$.  Allo stesso modo
\begin{align*}
  f_3(t; \lambda) &= \int_0^t f_2(x; \lambda) \, f_1(t-x;
  \lambda) \, \de x \\[1ex]
  &= \lambda^3 \int_0^t x \, e^{ -\lambda x } \, e^{
    -\lambda (t-x) } \de x \\[1ex]
  &= \lambda^3 \, e^{ -\lambda t} \int_0^t x \, \de x
  \\[1ex]
  &= \frac{ t^2 }{ 2 } \, \lambda^3 \, e^{ -\lambda t }
  \peq ;
\end{align*}
la formula generale (appunto la \emph{funzione di frequenza
  di Erlang}) \`e la
\begin{gather*}
  f_n(t; \lambda) = \frac{ t^{n-1} }{ (n-1)! } \, \lambda^n
  \, e^{ -\lambda t } \peq ,
  \intertext{con speranza matematica}
  E(t) = \frac{n}{\lambda}
  \intertext{e varianza}
  \var(t) = \frac{n}{\lambda^2} \peq .
\end{gather*}%
\index{distribuzione!di Erlang|)}

\subsection{La distribuzione composta di Poisson}%
\index{distribuzione!di Poisson!composta|(}
La \emph{distribuzione composta di Poisson} \`e quella
seguita da una variabile che sia somma \emph{di un numero
  casuale} $N$ di valori di un'altra variabile casuale $x$,
quando sia $N$ che $x$ seguono singolarmente delle
distribuzioni di Poisson.  Indichiamo con $\nu$ e $\xi$ i
valori medi delle popolazioni delle variabili $N$ e $x$
rispettivamente; le funzioni caratteristiche (di variabile
complessa) ad esse associate sono date, come sappiamo, dalla
\eqref{eq:8.fcapoi}:
\begin{gather*}
  \phi_N (z) = e^{ \nu (z -1)} \makebox[40mm]{e}
  \phi_x (z) = e^{ \xi (z -1)} \peq ; \\
  \intertext{ricordando la \eqref{eq:6.varn}, la variabile
    casuale}
  S = \sum_{i=1}^N x_i \\
  \intertext{ha funzione caratteristica di variabile
    complessa}
  \phi_S (z) \; = \; \phi_N \bigl[ \phi_x (z) \bigr] \; = \;
  e^{ \nu \left[ e^{\xi (z - 1)} - 1 \right] }
  \\
  \intertext{e funzione caratteristica di variabile reale}
  \phi_S (t) = e^{ \nu \left[ e^{\xi (e^{it} - 1)} - 1
    \right] } \peq ; \\
  \intertext{da quest'ultima poi si ricava}
  \frac{\de \, \phi_S (t)}{\de t} = \phi_S (t) \cdot \nu \,
  e^{\xi (e^{it} - 1)} \cdot \xi \, e^{it} \cdot i \\
  \intertext{ed infine}
  \left. \frac{\de \, \phi_S (t)}{\de t} \right|_{t=0} = i
  \nu \xi \peq .
  \intertext{La speranza matematica di una variabile che
    segua la distribuzione composta di Poisson vale quindi}
  E(S) = \nu \xi \peq , \\
  \intertext{e, similmente, si potrebbero ottenere}
  \var(S) = \nu \xi (1 + \xi) \\
  \intertext{per la varianza, e}
  \Pr (S) = \sum_{N=0}^\infty \left[ \frac{ (N \xi)^S }{ S!
      } \, e^{- N \xi} \cdot \frac{ \nu^N }{ N! } \, e^{-
      \nu} \right]
\end{gather*}
per la funzione di frequenza.

Quest'ultima formula non sorprende: \`e la somma (su tutti i
valori ammissibili) della probabilit\`a di ottenere un
determinato $N$, moltiplicata per la probabilit\`a di
ottenere il valore di $S$ condizionato da quello di $N$;
infatti la somma di $N$ variabili indipendenti distribuite
secondo Poisson con valore medio $\xi$ \`e ancora, in base a
quanto dedotto dall'equazione \eqref{eq:8.sumpoi}, una
variabile distribuita secondo Poisson e con valore medio $N
\xi$.%
\index{distribuzione!di Poisson!composta|)}

\subsection{Esempio: l'osservazione di un quark isolato}
Un esempio classico di applicazione delle formule precedenti
\`e la discussione di un articolo del
1969\/\footnote{McCusker e Cairns: Evidence of quarks in
  air-shower cores; Phys.~Rev.~Lett.~\textbf{23} (1969),
  pagg.~658--659.} in cui veniva annunciata l'osservazione
di un quark isolato; l'esperienza ivi descritta consisteva
nell'analizzare foto esposte in una camera a nebbia,
attraversata da un fascio di particelle (aventi carica
unitaria) che generavano tracce con un numero medio di gocce
per unit\`a di lunghezza $\alpha = 229$: su $55\updot 000$
tracce osservate ce ne era una con un numero di gocce per
unit\`a di lunghezza $n = 110$.

Questa traccia apparteneva indiscutibilmente al fascio di
particelle; la probabilit\`a che venisse osservato un numero
di gocce per unit\`a di lunghezza pari (o inferiore) a 110,
se il fenomeno \`e descritto da una distribuzione di Poisson
con media $229$, \`e data da
\begin{equation*}
  \Pr (n \le 110) \; = \; \sum_{k=0}^{110} \frac{ 229^k }{
    k! } \, e^{- 229} \; \approx \; 1.6 \times 10^{-18}
\end{equation*}
e risulta ben inferiore (per 13 ordini di grandezza!) alla
frequenza osservata $f = 1/55\updot 000 \approx 2 \times
10^{-5}$.  Per questo motivo gli autori sostenevano di avere
osservato una particella con carica frazionaria (un quark),
e che causava in conseguenza una ionizzazione assai
inferiore alle altre.

Una prima obiezione\/\footnote{Adair e Kasha: Analysis of
  some results of quark searches;
  Phys.~Rev.~Lett.~\textbf{23} (1969), pagg.~1355--1358.} fu
che, in ogni urto elementare tra le particelle del fascio e
le molecole di gas della camera, vengono generati in media
$\nu=4$ prodotti ionizzati indipendenti: e quindi 4 gocce.
Il numero medio effettivo di urti per unit\`a di lunghezza
era $229/4 = 57.25$, mentre la traccia osservata ne aveva
invece $\lambda = 110/4 = 27.5$; la probabilit\`a di
ottenere, per motivi puramente casuali, una fluttuazione
almeno pari a quella osservata doveva essere quindi
calcolata come probabilit\`a di avere meno di 28 eventi da
una distribuzione di Poisson con valore medio 57.25, che
vale
\begin{equation*}
  \Pr (n \le 110) \; = \; \sum_{k=0}^{27} \frac{ 57.25^k }{
    k! } \, e^{- 57.25} \; \approx \; 6.7 \times 10^{-6}
\end{equation*}
ed \`e quindi assai maggiore di quanto venisse ipotizzato
all'inizio dai due autori (pur mantenendosi pi\`u di 33
volte superiore alla frequenza osservata).

L'analisi del fenomeno \`e per\`o ancora pi\`u
complessa\/\footnote{Eadie, Drijard, James, Roos e Sadoulet:
  Statistical Methods in Experimental Physics; North-Holland
  Publishing Co.\ (1971), pag.~53.}: il numero $u$ di urti
elementari per unit\`a di lunghezza segue la distribuzione
di Poisson con valore medio $\lambda=57.25$, ed ogni urto
genera un numero di gocce che non \`e costante, ma segue
anch'esso la distribuzione di Poisson con valore medio
$\nu=4$; quindi il numero complessivo di gocce segue una
legge di distribuzione che \`e quella \emph{composta di
  Poisson}.

La probabilit\`a di osservare $k$ gocce per unit\`a di
lunghezza \`e quindi
\begin{gather*}
  \Pr(k) \; = \; \sum_{u=0}^\infty \left[ \frac{ (u \nu)^k
      }{ k! } \, e^{- u \nu} \cdot \frac{ \lambda^u }{ u! }
    e^{- \lambda} \right] \peq , \\
  \intertext{e la probabilit\`a cercata vale}
  \Pr (k \le 110) \; = \; \sum_{k=0}^{110} \Pr(k) \; \approx
  \; 4.7 \times 10^{-5}
\end{gather*}
(ben compatibile quindi con quanto osservato).

\subsection{Applicazione: segnale e fondo}
Supponiamo di osservare sperimentalmente un processo fisico,
per il quale il numero di eventi $s$ che potrebbero
presentarsi in un intervallo temporale prefissato (eventi di
\emph{segnale}) segua una distribuzione di Poisson con valore
medio $S$, e che indicheremo col simbolo $P(s; S)$;
\begin{equation*}
  \Pr(s) \; = \; P(s; S) \; = \; \frac{S^s}{s!} \, e^{-S}
  \peq :
\end{equation*}
in generale $S$ \`e ignoto, e ci si propone appunto di
determinarlo dall'esperimento.  Questo problema verr\`a poi
trattato anche nel paragrafo \ref{ch:11.stisuf} a pagina
\pageref{ch:11.stisuf}, usando altri metodi; qui vogliamo
solo vedere come, partendo dal dato di fatto consistente
nell'osservazione effettiva di $N$ eventi in un singolo
esperimento, si possa ricavare un \emph{limite superiore}
sui possibili valori di $S$.

Fissato arbitrariamente un valore della probabilit\`a
$\epsilon$, si tratta di trovare il valore $S_u$ per il
quale la probabilit\`a di osservare un numero di eventi non
superiore ad $N$ vale proprio $\epsilon$: vale a dire,
risolvere rispetto a $S_u$ l'equazione
\begin{equation*}
  \sum_{s=0}^N P(s; S_u) = \epsilon \peq ;
\end{equation*}
e diremo poi di poter affermare che $S \leq S_u$ con un
\emph{livello di confidenza} $\epsilon$.  Il significato
esatto della frase \`e che, se risultasse realmente $S \leq
S_u$, in una frazione pari almeno ad $\epsilon$ di
esperimenti analoghi a quello i cui risultati stiamo
esaminando ci aspetteremmo di ottenere al massimo $N$ eventi
come in esso.

Le cose si complicano in presenza di processi fisici che
possono produrre risultati che simulano il segnale: processi
fisici che indicheremo col nome complessivo di \emph{fondo}.
Se il fondo \`e presente, se \`e inoltre indipendente dal
segnale e se segue la distribuzione di Poisson con valore
medio \emph{noto} $F$, gi\`a sappiamo che la probabilit\`a
di osservare $N$ eventi in tutto \emph{tra fondo e segnale}
segue ancora la distribuzione di Poisson, con valore medio
$F+S$:
\begin{equation*}
  \Pr(N) \; \equiv \; P(N; F + S) \; = \; \frac{\left( F + S
    \right)^N}{N!} \, e^{- \left( F + S \right)} \peq .
\end{equation*}

Se sono stati realmente osservati $N$ eventi, si pu\`o
ancora determinare un \emph{limite superiore per $S$};
questo calcolando il valore $S_u$ per il quale la
probabilit\`a di osservare un numero di eventi (tra fondo e
segnale) non superiore a $N$ \emph{e condizionato all'avere
  ottenuto un numero di eventi di fondo che non pu\`o
  superare $N$} vale una quantit\`a predeterminata
$\epsilon$.  Insomma, si tratta di risolvere, rispetto a
$S_u$, l'equazione
\begin{equation} \label{eq:8.poisint}
  \frac{\sum\limits_{n=0}^N P(n; F +
    S_u)}{\sum\limits_{f=0}^N P(f; F)} = \epsilon \peq ;
\end{equation}
e, con lo steso significato della frase prima evidenziato,
potremo allora affermare che risulta $S \leq S_u$ con un
libello di confidenza $\epsilon$.
\begin{figure}[htbp]
  \vspace*{2ex}
  \begin{center} \input poisint.pstex_t \end{center}
  \caption[Limiti superiori sul segnale in presenza di fondo
    noto]{I limiti superiori sul segnale, ad un livello di
    confidenza fisso $\epsilon = 90\%$, in presenza di fondo noto.}
  \label{fig:8.poisint}
\end{figure}

Nella figura \ref{fig:8.poisint}, che \`e tratta dal
libretto pubblicato annualmente dal ``Particle Data Group''
(vedi la bibliografia), si possono trovare gi\`a
calcolate e rappresentate da curve continue le soluzioni
dell'equazione \eqref{eq:8.poisint} relative ad un livello
di confidenza fisso del 90\%.

\section{La distribuzione log-normale}%
\index{distribuzione!log-normale|(}
Sia una variabile casuale $y$ avente distribuzione normale
con media $\mu$ e varianza $\sigma^2$: la sua densit\`a di
probabilit\`a sia insomma data dalla funzione
\begin{gather*}
  g(y) \; = \; N(y; \mu, \sigma) \; = \; \frac{1}{\sigma
    \sqrt{2\pi}} \, e^{- \frac{1}{2 \sigma^2} \left( y - \mu
    \right)^2 } \peq ; \\
  \intertext{definiamo poi una nuova variabile casuale $x$
    attraverso la relazione}
  x = e^y \makebox[30mm]{$\Longleftrightarrow$} y = \ln x \\
  \intertext{(la corrispondenza tra $x$ ed $y$ \`e
    ovviamente biunivoca).  Il dominio di definizione della
    $x$ \`e il semiasse $x > 0$ e, in base alla
    \eqref{eq:6.cavaun}, la sua densit\`a di probabilit\`a
    $f(x)$ sar\`a data da}
  f(x) = \frac{1}{\sigma \sqrt{2\pi}} \, \frac{1}{x} \, e^{-
    \frac{1}{2 \sigma^2} \left( \ln x - \mu \right)^2 } \peq
  .
\end{gather*}

Questa funzione di frequenza si chiama \emph{log-normale};
sfruttando l'identit\`a
\begin{align*}
  \frac{\left( y - \mu - k \sigma^2 \right)^2}{2 \sigma^2} -
  k \mu - \frac{1}{2} k^2 \sigma^2 &= \frac{1}{2 \sigma^2}
  \Bigl( y^2 + \mu^2 + k^2 \sigma^4 - 2 \mu y \\[1ex]
  & \qquad  - 2 k \sigma^2 y + 2 k \mu \sigma^2 - 2 k \mu
  \sigma^2 - k^2 \sigma^4 \Bigr) \\[1ex]
  &= \frac{1}{2 \sigma^2} \left[ \left( y^2 - 2 \mu y +
      \mu^2 \right) - 2 k \sigma^2 y \right] \\[1ex]
  &= \frac{\left( y - \mu \right)^2}{2 \sigma^2} - k y
\end{align*}
se ne possono facilmente calcolare i momenti rispetto
all'origine, che valgono
\begin{align*}
  \lambda_k &= \int_0^{+\infty} \! x^k \, f(x) \, \de x
  \\[1ex]
  &= \int_{-\infty}^{+\infty} \! e^{k y} \, g(y) \, \de y
  \\[1ex]
  &= \int_{-\infty}^{+\infty} \! \frac{1}{\sigma \sqrt{2
      \pi}} \, e^{- \left[ \frac{\left( y - \mu \right)^2}{2
      \sigma^2} - k y \right]} \, \de y \\[1ex]
  &= e^{\left( k \mu + \frac{1}{2} k^2 \sigma^2 \right)}
  \int_{-\infty}^{+\infty} \! \frac{1}{\sigma \sqrt{2 \pi}}
  \, e^{- \frac{\left( y - \mu - k \sigma^2 \right)^2}{2
      \sigma^2}} \, \de y \\[1ex]
  &= e^{\left( k \mu + \frac{1}{2} k^2 \sigma^2 \right)}
\end{align*}
(infatti l'integrale \`e quello di una distribuzione normale
avente $\mu + k \sigma^2$ come valore medio e $\sigma$ come
varianza --- e vale dunque uno).

In particolare
\begin{align*}
  E(x) &\equiv \lambda_1 \; = \; e^{\left( \mu +
    \frac{\sigma^2}{2} \right)} \\[1ex]
  E(x^2) &\equiv \lambda_2 \; = \; e^{\left( 2 \mu + 2
      \sigma^2 \right) } \\[1ex]
  \var(x) &= \lambda_2 - {\lambda_1}^2 \; = \; e^{\left( 2
      \mu + \sigma^2 \right)} \left( e^{\sigma^2} - 1
  \right)
\end{align*}

\begin{figure}[hbtp]
  \vspace*{2ex}
  \begin{center} {
    \input{lognor.pstex_t}
  } \end{center}
  \caption[La distribuzione log-normale]
    {La distribuzione log-normale, per vari valori dei
    parametri ($\mu$ e $\sigma$) della funzione normale
    di partenza.}
  \label{fig:8.lognor}
\end{figure}

Nella figura \ref{fig:8.lognor} ci sono i grafici di alcune
distribuzioni log-normali corrispondenti a vari valori dei
parametri $\mu$ e $\sigma$ della funzione normale di
partenza (\emph{non} della funzione di frequenza esaminata);
per finire notiamo che, analogamente a quanto ricavato nel
teorema di pagina \pageref{th:8.colino} per quanto attiene
alle somme, si pu\`o dimostrare che il \emph{prodotto} di
variabili casuali \emph{log-normali ed indipendenti} debba
seguire \emph{una distribuzione log-normale}.%
\index{distribuzione!log-normale|)}

\section{La distribuzione normale in pi\`u dimensioni}%
\index{distribuzione!normale bidimensionale|(}
\begin{figure}[htbp]
  \vspace*{2ex}
  \begin{center} \input bigau2.pstex_t \end{center}
  \caption[Funzione normale bidimensionale]{La funzione
    normale in due dimensioni, nel piano $\{ u, v \}$ delle
    variabili standardizzate e per un coefficiente di
    correlazione $r = 0.8$.}
  \label{fig:8.bigau2}
\end{figure}
\begin{figure}[htbp]
  \vspace*{2ex}
  \begin{center} \input bigau1.pstex_t \end{center}
  \caption[Funzione normale bidimensionale (curve di
    livello)]{La sole curve di livello della stessa funzione
      rappresentata nella figura \ref{fig:8.bigau2}.}
  \label{fig:8.bigau1}
\end{figure}

Accenniamo solo brevemente alla cosiddetta
\emph{distribuzione normale bidimensionale}: per essa la
densit\`a di probabilit\`a congiunta delle due variabili $x$
ed $y$ \`e, nel caso generale, data dalla
\begin{equation*}
  f(x, y) = \frac{ e^{ \left\{ - \frac{1}{2 (1 - r^2)}
        \left[ \left( \frac{x - \mu_x}{ \sigma_x }
          \right)^2 - 2 r \frac{(x - \mu_x) (y - \mu_y
            )}{\sigma_x  \,\sigma_y} + \left( \frac{y -
              \mu_y}{ \sigma_y } \right)^2 \right]
      \right\} } }{2 \pi \, \sigma_x \, \sigma_y \sqrt{1 -
      r^2}}
\end{equation*}
o, espressa in funzione delle \emph{variabili
  standardizzate}
\begin{align*}
  u &= \frac{x - E(x)}{\sigma_x} &&\text{e} & v &= \frac{y
      - E(y)}{\sigma_y} \peq ,
\end{align*}
dalla
\begin{equation*}
  f(u,v) = \frac{ e^{ - \frac{1}{2} \frac{u^2 -2ruv +
        v^2}{1 - r^2} } }{ 2 \pi \sqrt{1 - r^2} } \peq ;
\end{equation*}
un esempio \`e dato dalle figure \ref{fig:8.bigau2} e
\ref{fig:8.bigau1}.

\begin{figure}[htbp]
  \vspace*{2ex}
  \begin{center} \input bigau3.pstex_t \end{center}
  \caption[Funzione normale bidimensionale (probabilit\`a
  condizionate)]{Le sezioni della figura \ref{fig:8.bigau2}
    con due piani di equazione $y=1$ e $y=2$
    rispettivamente: ovvero (a parte la normalizzazione) le
    densit\`a di probabilit\`a condizionate $\pi(x | y=1)$ e
    $\pi(x | y=2)$.}
  \label{fig:8.bigau3}
\end{figure}
$r$ \`e il coefficiente di correlazione lineare tra $x$ ed
$y$; $r = 0$ \`e condizione \emph{necessaria e sufficiente}
perch\'e le due variabili siano statisticamente indipendenti
tra loro.  Le due distribuzioni marginali $g(x)$ e $h(y)$
sono due funzioni normali, aventi speranza matematica e
varianza $\mu_x$ e ${\sigma_x}^2$ la prima, e $\mu_y$ e
${\sigma_y}^2$ la seconda:
\begin{align*}
  g(x) &= N(x; \mu_x, \sigma_x) &&\text{e} &
  h(y) &= N(y; \mu_y, \sigma_y) \peq .
\end{align*}
Le densit\`a di probabilit\`a condizionate sono anch'esse
sempre delle funzioni normali; come esempio, nella figura
\ref{fig:8.bigau3} si mostrano due di queste funzioni per
la stessa distribuzione di figura \ref{fig:8.bigau2}.%
\index{distribuzione!normale bidimensionale|)}

Nel caso pi\`u generale, la densit\`a di probabilit\`a di
una distribuzione di Gauss $N$-dimensionale \`e del tipo
\begin{equation} \label{eq:8.multigauss}
  f(x_1, x_2, \ldots, x_N) = K e^{- \frac{1}{2} H(x_1,
    x_2, \ldots, x_N)} \peq ,
\end{equation}
ove $H$ \`e una \emph{forma quadratica} nelle variabili
standardizzate
\begin{equation*}
  t_i = \frac{x_i - E(x_i)}{\sigma_i}
\end{equation*}
nella quale per\`o i coefficienti dei termini quadratici
sono \emph{tutti uguali}; le $t_i$ non sono generalmente
indipendenti tra loro, e quindi $H$ contiene anche i termini
rettangolari del tipo $t_i \cdot t_j$.

$K$, nella \eqref{eq:8.multigauss}, \`e un fattore di
normalizzazione che vale
\begin{equation*}
  K = \sqrt{ \frac{ \Delta }{ ( 2 \pi )^N } } \peq ;
\end{equation*}
a sua volta, $\Delta$ \`e il determinante della matrice
(\emph{simmetrica}) dei coefficienti della forma quadratica
$H$.  La condizione, poi, che la $f$ di equazione
\eqref{eq:8.multigauss} debba essere integrabile implica che
le ipersuperfici di equazione $H = \mathrm{cost.}$ siano
tutte al finito, e siano quindi \emph{iperellissi} nello
spazio $N$-dimensionale dei parametri; le funzioni di
distribuzione marginali e condizionate di \emph{qualsiasi}
sottoinsieme delle variabili sono ancora \emph{sempre
  normali}.

Si pu\`o inoltre dimostrare che \emph{esiste sempre} un
cambiamento di variabili $x_i \to y_k$ che muta $H$ nella
cosiddetta \emph{forma canonica} (senza termini
rettangolari); in tal caso
\begin{gather*}
  \Delta = \left[ \prod_{k=1}^N \var(y_k) \right]^{-1}
  \intertext{e}
  f(y_1,\ldots,y_k) = \frac{1}{\sqrt{\prod\limits_{k=1}^N
      \left[ 2 \pi \var(y_k) \right] }} \, e^{- \frac{1}{2}
    \sum\limits_{k=1}^N \frac{\left[ y_k - E(y_k)
      \right]^2}{\var(y_k)} } \peq :
\end{gather*}
e le $y_k$ sono quindi tutte \emph{statisticamente
  indipendenti} (oltre che normali).

\endinput

% $Id: chapter9.tex,v 1.1 2005/03/01 10:06:08 loreti Exp $

\chapter{La legge di Gauss}
Vogliamo ora investigare sulla distribuzione dei risultati
delle misure ripetute di una grandezza fisica, nell'ipotesi
che esse siano affette da errori esclusivamente
casuali\/\footnote{Il primo ad intuire la forma e
  l'equazione della distribuzione normale fu Abraham de
  Moivre%
  \index{de Moivre!Abraham}
  nel 1733, che la deriv\`o dalla distribuzione binomiale
  facendo uso della formula di Stirling per il fattoriale;
  fu poi studiata da Laplace,%
  \index{Laplace!Pierre Simon de}
  ma la teoria completa \`e dovuta a Gauss.}.

\section{La funzione di Gauss}%
\index{distribuzione!normale|(emidx}%
\label{ch:9.fungau}
Dall'esame di molte distribuzioni sperimentali di valori
ottenuti per misure ripetute in condizioni omogenee, si
possono astrarre due propriet\`a generali degli errori
casuali:
\begin{itemize}
\item La probabilit\`a di ottenere un certo scarto dal
  valore vero deve essere funzione del modulo di tale scarto
  e non del suo segno, se valori in difetto ed in eccesso
  rispetto a quello vero si presentano con uguale
  probabilit\`a; in definitiva la distribuzione degli scarti
  deve essere \emph{simmetrica rispetto allo zero}.
\item La probabilit\`a di ottenere un certo scarto dal
  valore vero (in modulo) deve essere \emph{decrescente} al
  crescere di tale scarto e \emph{tendere a zero} quando
  esso tende all'infinito; questo perch\'e deve essere pi\`u
  probabile commettere errori piccoli che errori grandi, ed
  infinitamente improbabile commettere errori infinitamente
  grandi.
\end{itemize}

A queste due ipotesi sulla distribuzione delle misure
affette da errori puramente casuali se ne pu\`o aggiungere
una terza, valida per \emph{tutte} le distribuzioni di
probabilit\`a; la condizione di normalizzazione,%
\index{normalizzazione!condizione di}
ossia l'equazione \eqref{eq:6.connor} di cui abbiamo gi\`a
parlato prima:
\begin{itemize}
\item L'area compresa tra la curva densit\`a di
  probabilit\`a dello scarto e l'asse delle ascisse, da
  $-\infty$ a $+\infty$, deve valere 1.
\end{itemize}

Da queste tre ipotesi e dal principio della media
aritmetica, Gauss\/\footnote{Karl Friedrich Gauss fu senza
  dubbio la maggiore personalit\`a del primo 800 nel campo
  della fisica e della matematica; si occup\`o di teoria dei
  numeri, analisi, geometria analitica e differenziale,
  statistica e teoria dei giochi, geodesia, elettricit\`a e
  magnetismo, astronomia e ottica.  Visse a G\"ottingen dal
  1777 al 1855 e, nel campo di cui ci stiamo occupando,
  teorizz\`o (tra le altre cose) la funzione normale ed il
  metodo dei minimi quadrati, quest'ultimo (studiato anche
  da Laplace) all'et\`a di soli 18 anni.}%
\index{Laplace!Pierre Simon de}%
\index{Gauss, Karl Friedrich|emidx}
dimostr\`o in modo euristico che la distribuzione degli
scarti $z$ delle misure affette da errori casuali \`e data
dalla funzione
\begin{equation} \label{eq:9.gauss}
  \boxed{ \rule[-6mm]{0mm}{14mm} \quad
    f(z) = \dfrac{h}{\sqrt{\pi}}
    \, e^{-h^2 z^2} \quad }
\end{equation}
che da lui prese il nome (\emph{funzione di Gauss} o
\emph{legge normale} di distribuzione degli errori).  Si
deve originalmente a Laplace%
\index{Laplace!Pierre Simon de}
una prova pi\`u rigorosa ed indipendente dall'assunto della
media aritmetica; una versione semplificata di questa
dimostrazione \`e data nell'appendice \ref{ch:d.applap}.

La funzione di Gauss ha una caratteristica forma a campana:
simmetrica rispetto all'asse delle ordinate (di equazione
$z=0$), decrescente man mano che ci si allontana da esso sia
nel senso delle $z$ positive che negative, e tendente a $0$
per $z$ che tende a $\pm \infty$; cos\`\i\ come richiesto
dalle ipotesi di partenza.
\begin{figure}[htbp]
  \vspace*{2ex}
  \begin{center} {
    \input{trigau.pstex_t}
  } \end{center}
  \caption[Dipendenza da $h$ della distribuzione di Gauss]
    {La funzione di Gauss per tre diversi
    valori di $h$.}
  \label{fig:9.gauss}
\end{figure}

Essa dipende da un parametro $h>0$ che prende il nome di
\emph{modulo di precisione}%
\index{modulo di precisione della misura|emidx}
della misura: infatti quando $h$ \`e piccolo la funzione \`e
sensibilmente diversa da zero in una zona estesa dell'asse
delle ascisse; mentre al crescere di $h$ l'ampiezza di tale
intervallo diminuisce, e la curva si stringe sull'asse delle
ordinate (come si pu\`o vedere nella figura
\ref{fig:9.gauss}).

\section{Propriet\`a della legge normale}
Possiamo ora, come gi\`a anticipato nei paragrafi
\ref{ch:4.medpes} e \ref{ch:6.mevaco}, determinare il valore
medio di una qualsiasi grandezza $W(z)$ legata alle misure,
nel limite di un numero infinito di misure effettuate;
questo valore medio sappiamo dall'equazione
\eqref{eq:6.mevaco} che si dovr\`a ricavare calcolando
l'integrale
\begin{equation*}
  \int_{- \infty}^{+ \infty} \! W(z) \cdot f(z) \, \de
    z
\end{equation*}
dove per $f(z)$ si intende la funzione densit\`a di
probabilit\`a dello scarto dal valore vero, che supporremo
qui essere la distribuzione normale \eqref{eq:9.gauss}.

Se vogliamo ad esempio calcolare il valore medio dello
scarto $z$, questo \`e dato dalla
\begin{equation*}
  E(z) \; = \; \frac{h}{\sqrt{\pi}}
  \int_{- \infty}^{+ \infty} \!
  z \, e^{-h^2 z^2} \, \de z \; = \; 0 \peq .
\end{equation*}

Il risultato \`e immediato considerando che $f(z)$ \`e una
funzione simmetrica mentre $z$ \`e antisimmetrica: in
definitiva, ad ogni intervallino centrato su un dato valore
$z>0$ possiamo associarne uno uguale centrato sul punto
$-z$, in cui il prodotto $z \, f(z) \, \de z $ assume valori
uguali in modulo ma di segno opposto; cos\`\i\ che la loro
somma sia zero.  Essendo poi
\begin{equation*}
  E(z) \; = \; E \left( x - x^* \right) \; = \;
  E(x) - x^*
\end{equation*}
abbiamo cos\`\i\ \emph{dimostrato} quanto assunto nel
paragrafo \ref{ch:5.medcl}, ossia che
\begin{quote}
  \index{media!aritmetica!come stima del valore vero}%
  \textit{Il valore medio della popolazione delle misure di
    una grandezza fisica affette solo da errori casuali
    esiste, e coincide con il valore vero della grandezza
    misurata.}
\end{quote}

\index{errore!medio!della distribuzione normale}%
Cerchiamo ora il valore medio del modulo dello scarto $ E
\bigl( | z | \bigr) $:
\begin{align*}
  E \bigl( | z | \bigr) &=
    \int_{- \infty}^{+ \infty} \! |z| \, f(z) \, \de z
    \\[1ex]
  &= 2 \int_0^{+ \infty} \! z \, f(z) \, \de z
    \\[1ex]
  &= \frac{2h}{\sqrt{\pi}} \int_0^{+\infty} \! z \,
    e^{-h^2 z^2} \, \de z \\[1ex]
  &= \frac{h}{\sqrt{\pi}} \int_0^{+\infty} \!
    e^{-h^2 t} \, \de t \\[1ex]
  &= - \frac{1}{h\sqrt{\pi}} \left[ e^{-h^2 t}
     \right]_0^{+\infty} \\[1ex]
  &= \frac{1}{h\sqrt{\pi}}
\end{align*}
dove si \`e eseguito il cambio di variabile $t=z^2$.

Il valore medio del modulo degli scarti \`e quella grandezza
che abbiamo definito come ``errore medio'': qui abbiamo
ricavato la relazione tra l'errore medio di misure affette
solo da errori casuali ed il modulo di precisione della
misura $h$.%
\index{errore!medio!della distribuzione normale}

Il rapporto invece tra l'errore quadratico medio ed $h$ si
trova calcolando il valore medio del quadrato degli scarti:%
\index{errore!quadratico medio!della distribuzione normale|(emidx}
\begin{align*}
  E \bigl( z^2 \bigr) &= \frac{h}{\sqrt{\pi}}
    \int_{-\infty}^{+\infty} \! z^2 \, e^{-h^2 z^2} \,
    \de z \\[1ex]
  &= \frac{1}{h^{2}\sqrt{\pi}}
    \int_{-\infty}^{+\infty} \! t^2 \, e^{-t^2} \, \de
    t \\[1ex]
  &= - \, \frac{1}{2h^2 \sqrt{\pi}}
    \int_{-\infty}^{+\infty} \! t \cdot \de \bigl(
    e^{-t^2} \bigr) \\[1ex]
  &= - \, \frac{1}{2h^2 \sqrt{\pi}} \, \left \{
    \left[ t e^{-t^2} \right]_{-\infty}^{+\infty} \:
    - \: \int_{-\infty}^{+\infty} \! e^{-t^2} \de t
    \right \} \\[1ex]
  &= \frac{1}{2h^2} \peq .
\end{align*}

Per giungere al risultato, per prima cosa si \`e effettuata
la sostituzione di variabile $t=hz$; poi si \`e integrato
per parti; ed infine si \`e tenuto conto del fatto che
\begin{equation*}
  \int_{-\infty}^{+\infty} \! e^{-t^2} \de t
  \: = \: \sqrt{\pi}
\end{equation*}
come si pu\`o ricavare dalla condizione di normalizzazione
della funzione di Gauss per il particolare valore $h=1$.

Concludendo:
\begin{quote}%
  \index{errore!medio!della distribuzione normale|(}%
  \index{modulo di precisione della misura|(}%
  \begin{itemize}
  \item \textit{Per misure affette da errori distribuiti
      secondo la legge normale, il rapporto tra l'errore
      quadratico medio $\sigma$ e l'errore medio $a$ vale}
    \begin{equation*}
      \frac{\sigma}{a} \: = \:
      \sqrt{\frac{\pi}{2}} \: = \:
      1.2533\ldots \peq .
    \end{equation*}
  \item \textit{Per misure affette da errori distribuiti
      secondo la legge normale, l'errore quadratico medio ed
      il modulo di precisione $h$ sono legati dalla}
    \begin{equation*}
      \sigma = \frac{1}{h\sqrt{2}} \peq .
    \end{equation*}
  \item \textit{L'errore medio ed il modulo di precisione
      sono invece legati dalla}
    \begin{equation*}
      a = \frac{1}{h\sqrt{\pi}} \peq .
    \end{equation*}
  \end{itemize}%
\end{quote}%
\index{modulo di precisione della misura|)}%
\index{errore!medio!della distribuzione normale|)}%
\index{errore!quadratico medio!della distribuzione normale|)}

Sostituendo nella \eqref{eq:9.gauss} il valore di $h$ in
funzione di $\sigma$, la legge di Gauss si pu\`o quindi
anche scrivere nella forma equivalente
\begin{equation} \label{eq:9.sgauss}
  \boxed{ \rule[-6mm]{0mm}{14mm} \quad
    f(z) = \dfrac{1}{\sigma \sqrt{2\pi}}
    \, e^{- \frac{z^2}{2 \sigma^2}} \quad }
\end{equation}

\section{Lo scarto normalizzato}%
\index{scarto normalizzato|(}%
\label{ch:9.scanor}
Introduciamo in luogo dello scarto $z$ il risultato della
misura $x$; questo \`e legato a $z$ dalla $ z=x-x^* $
(relazione che implica anche $ \de z = \de x $).  In luogo
del modulo di precisione $h$ usiamo poi l'errore quadratico
medio $\sigma$; la funzione di Gauss \eqref{eq:9.sgauss} si
pu\`o allora scrivere nella forma
\begin{equation*}
  \boxed{ \rule[-6mm]{0mm}{14mm} \quad
    f(x) =
    \dfrac{1}{\sigma \sqrt{2\pi}}
    \, e^{\textstyle -\frac{1}{2} \left(
        \frac{x - x^*}{\sigma} \right) ^2 } \quad }
\end{equation*}

Definiamo ora una nuova variabile $t$, legata alla $x$ dalla
relazione
\begin{align*}
  t &= \frac{x - x^*}{ \sigma }
  & &\Longrightarrow
  & \de t &= \frac{\de x}{\sigma} \peq .
\end{align*}

Essa prende il nome di \emph{scarto normalizzato} della $x$;
vogliamo trovare la funzione di frequenza $\varphi(t)$ della
$t$ nell'ipotesi che la $x$ abbia distribuzione normale.
Siano $x_1$ ed $x_2$ due qualunque valori della variabile
$x$ (con $ x_1 < x_2$); sappiamo che
\begin{equation} \label{eq:9.scnor1}
  \Pr \Bigl( x \in \left[ x_1 , x_2 \right] \Bigr)
  \; = \; \int_{x_1}^{x_2} \! f(x) \, \de x \; = \;
  \frac{1}{\sigma \sqrt{2\pi}} \int_{x_1}^{x_2} \!
  e^{\textstyle -\frac{1}{2} \left( \frac{x -
  x^*}{\sigma} \right) ^2 } \de x \peq .
\end{equation}
Siano poi $t_1$ e $t_2$ i valori per la $t$ che
corrispondono a $x_1$ e $x_2$; sar\`a
\begin{equation} \label{eq:9.scnor2}
  \Pr \Bigl( t \in \left[ t_1 , t_2 \right] \Bigr)
  \; = \; \int_{t_1}^{t_2} \! \varphi(t) \, \de t \peq .
\end{equation}

Quando la $x$ \`e compresa nell'intervallo $ \left[ x_1 ,
  x_2 \right] $, allora (e soltanto allora) la $t$ \`e
compresa nell'intervallo $ \left[ t_1 , t_2 \right]$;
pertanto la probabilit\`a che $x$ sia compresa in $ \left[
  x_1 , x_2 \right] $ deve essere identicamente uguale alla
probabilit\`a che $t$ sia compresa in $ \left[ t_1 , t_2
\right] $.

Eseguiamo sull'espressione \eqref{eq:9.scnor1} della
probabilit\`a per $x$ un cambiamento di variabile,
sostituendovi la $t$:
\begin{equation} \label{eq:9.scnor3}
  \Pr \Bigl( x \in \left[ x_1 , x_2 \right]
  \Bigr) \; = \; \frac{1}{\sqrt{2\pi}}
  \int_{t_1}^{t_2} \! e^{- \frac{1}{2}
    t^2} \de t \peq .
\end{equation}

Confrontando le due espressioni \eqref{eq:9.scnor2} e
\eqref{eq:9.scnor3} (che, ricordiamo, devono assumere lo
stesso valore per \emph{qualunque} coppia di valori $x_1$ e
$x_2$), si ricava immediatamente che deve essere
\begin{equation} \label{eq:9.pscnor}
  \boxed{ \rule[-6mm]{0mm}{14mm} \quad
    \varphi(t) = \frac{1}{\sqrt{2\pi}} \,
    e^{- \frac{1}{2} t^2} \quad }
\end{equation}

La cosa importante \`e che in questa espressione non
compaiono n\'e l'errore quadratico medio $\sigma$ n\'e
alcuna altra grandezza dipendente dal modo in cui la misura
\`e stata effettuata, ma \emph{solo costanti}: in altre
parole \emph{lo scarto normalizzato ha una distribuzione di
  probabilit\`a indipendente dalla precisione della misura}.

Di questa propriet\`a si fa uso, ad esempio, per comporre in
un unico grafico campioni di misure aventi precisione
diversa: se due osservatori misurano la stessa grandezza
commettendo solo errori casuali, le distribuzioni delle loro
misure saranno normali; ma se gli errori commessi sono
diversi, raggruppando i due insiemi di osservazioni in un
solo istogramma l'andamento di quest'ultimo non \`e
gaussiano.
\begin{figure}[htbp]
  \vspace*{2ex}
  \begin{center} {
    \input{tremis.pstex_t}
  } \end{center}
  \caption[Istogrammi di dati con differente precisione]
    {Gli istogrammi relativi a due campioni di
    misure aventi differente precisione, e quello
    relativo ai dati di entrambi i campioni.}
\end{figure}
Per\`o gli scarti normalizzati hanno la stessa legge di
distribuzione per entrambi i misuratori, indipendentemente
dall'entit\`a dei loro errori, e possono essere cumulati in
un unico istogramma.

Altra conseguenza dell'indipendenza da $\sigma$ della
funzione di frequenza \eqref{eq:9.pscnor} di $t$, \`e che la
probabilit\`a per una misura di avere scarto normalizzato
compreso tra due valori costanti prefissati risulta
indipendente dalla precisione della misura stessa; ad
esempio si ha

\begin{displaymath}
  \begin{array}{lcccl}
    \Pr \Bigl( t \in \left[ -1 , +1 \right] \Bigr) &
      = & \dfrac{1}{\sqrt{2\pi}} \displaystyle \int
      _{-1}^{+1} \! e^{
      -\frac{t^{2}}{2}} \, \de t & = & 0.6827\ldots
      \\[4ex]
    \Pr \Bigl( t \in \left[ -2 , +2 \right] \Bigr) &
      = & \dfrac{1}{\sqrt{2\pi}} \displaystyle \int
      _{-2}^{+2} \! e^{
      -\frac{t^{2}}{2}} \, \de t & = & 0.9545\ldots
      \\[4ex]
    \Pr \Bigl( t \in \left[ -3 , +3 \right] \Bigr) &
      = & \dfrac{1}{\sqrt{2\pi}} \displaystyle \int
      _{-3}^{+3} \! e^{
      -\frac{t^{2}}{2}} \, \de t & = & 0.9973\ldots
  \end{array}
\end{displaymath}
e, ricordando la relazione che intercorre tra $z$ e
$t$, questo implica che risulti anche
\begin{displaymath}
  \begin{array}{lcccl}
    \Pr \Bigl( z \in \left[ -\sigma , +\sigma \right]
      \Bigr) & \equiv & \Pr \Bigl( t \in \left[ -1 ,
      +1 \right] \Bigr) & \approx & 0.6827 \\[1ex]
    \Pr \Bigl( z \in \left[ -2\sigma , +2\sigma \right]
      \Bigr) & \equiv & \Pr \Bigl( t \in \left[ -2 ,
      +2 \right] \Bigr) & \approx & 0.9545 \\[1ex]
    \Pr \Bigl( z \in \left[ -3\sigma , +3\sigma \right]
      \Bigr) & \equiv & \Pr \Bigl( t \in \left[ -3 ,
      +3 \right] \Bigr) & \approx & 0.9973 \peq .
  \end{array}
\end{displaymath}%
\index{scarto normalizzato|)}

\index{errore!quadratico medio!della distribuzione normale|(}%
Possiamo quindi far uso di una qualsiasi di queste relazioni
per dare una \emph{interpretazione probabilistica}
dell'errore quadratico medio:
\begin{quote}
  \begin{itemize}
  \item \textit{Le misure affette da errori casuali (e
      quindi normali) hanno una probabilit\`a del 68\% di
      cadere all'interno di un intervallo di semiampiezza
      $\sigma$ centrato sul valore vero della grandezza
      misurata.}
  \item \textit{L'intervallo di semiampiezza $\sigma$
      centrato su di una misura qualsiasi di un campione ha
      pertanto una probabilit\`a del 68\% di contenere il
      valore vero, semprech\'e gli errori siano casuali e
      normali.}
  \end{itemize}
\end{quote}

\section[Il significato geometrico di $\sigma$]%
{Il significato geometrico di $\boldsymbol{\sigma}$}
Calcoliamo ora la derivata prima della funzione di
Gauss, nella sua forma \eqref{eq:9.sgauss}:
\begin{equation*}
  \frac{\de f}{\de z} = - \, \frac{z}{\sqrt{2 \pi}
    \, \sigma^3} \, e^{- \frac{z^2}{2 \sigma^2} } \peq .
\end{equation*}

La funzione $f(z)$ \`e crescente ($f'(z)>0$) quando $z$ \`e
negativa, e viceversa; ha quindi un massimo per $z=0$, come
d'altronde richiesto dalle ipotesi fatte nel paragrafo
\ref{ch:9.fungau} per ricavarne la forma analitica.  La
derivata seconda invece vale
\begin{align*}
  \frac{\de^2f}{\de z^2} &= - \, \frac{1}{
    \sqrt{2 \pi} \, \sigma^3} \, e^{
    - \frac{z^2}{2 \sigma^2} }
    \, - \, \frac{z}{\sqrt{2 \pi} \, \sigma^3}
    \, e^{- \frac{z^2}{2 \sigma^2} } \left( - \,
    \frac{z}{\sigma^2} \right) \\[0.7ex]
  &= \frac{1}{\sqrt{2 \pi} \, \sigma^3} \,
    e^{-\frac{z^2}{2 \sigma^2} }
    \left( \frac{z^2}{\sigma^2} -1 \right)
\end{align*}
e si annulla quando $z = \pm \sigma$.

Da qui si pu\`o allora ricavare il \emph{significato
  geometrico} dell'errore quadratico medio $\sigma$ in
relazione alla distribuzione normale:
\begin{quote}
  \textit{L'errore quadratico medio $\sigma$ pu\`o essere
    interpretato geometricamente come valore assoluto delle
    ascisse dei due punti di flesso della curva di Gauss.}
\end{quote}%
\index{errore!quadratico medio!della distribuzione normale|)}%
\index{distribuzione!normale|)}

\section{La curva di Gauss nella pratica}
Un campione di $N$ misure di una grandezza fisica con valore
vero $x^*$, affette da soli errori casuali normali con
errore quadratico medio $\sigma$, avr\`a media $\bar x$
prossima a $x^*$ (sappiamo infatti che la varianza della
media vale $ \sigma^2 / N $ e tende a zero al crescere di
$N$), e varianza $s^2$ prossima a $\sigma^2$ (anche la
varianza di $s^2$ tende a zero al crescere di $N$: vedi in
proposito l'appendice \ref{ch:b.errvar}).

Per $N$ abbastanza grande\/\footnote{Cosa si debba intendere
  esattamente per ``abbastanza grande'' risulter\`a chiaro
  dall'analisi dell'appendice \ref{ch:b.errvar}; normalmente
  si richiedono almeno 30 misure, dimensione del campione
  che corrisponde per $s$ ad un errore relativo di poco
  superiore al 10\%.}  si pu\`o dunque assumere $ s \approx
\sigma $ ed interpretare lo stesso scarto quadratico medio
del campione $s$, in luogo di $\sigma$ (peraltro ignoto),
come semiampiezza dell'intervallo di confidenza
corrispondente ad una probabilit\`a del 68\%.

\index{errori di misura!sistematici|(}%
Purtroppo non \`e generalmente possibile capire,
dall'andamento di un insieme di osservazioni, se fossero o
meno presenti nella misura errori sistematici; un campione
di misure ripetute, effettuate confrontando la lunghezza di
un oggetto con un regolo graduato mal tarato, avr\`a
distribuzione ancora normale: solo centrata attorno ad una
media che non corrisponde al valore vero.

Al contrario, se la distribuzione delle misure non \`e
normale sicuramente c'\`e qualcosa di sospetto nei dati che
stiamo esaminando; sorge quindi il problema di stimare se un
insieme di dati ha o non ha distribuzione conforme alla
funzione di Gauss (o meglio, di stimare con quale livello di
probabilit\`a possa provenire da una distribuzione normale).

Per far questo si pu\`o ricorrere ad alcune propriet\`a
matematiche della curva: ad esempio, si possono calcolare
l'errore medio e l'errore quadratico medio per verificare se
il loro rapporto ha un valore vicino a quello teorico;
oppure si pu\`o calcolare la frazione di dati che cadono tra
$\bar x - s$ e $\bar x + s$ e confrontare il numero ottenuto
con il valore teorico di 0.68.

Il modo migliore di eseguire il confronto \`e per\`o quello
che consiste nel disegnare assieme all'istogramma dei dati
anche la curva teorica relativa; a questo livello il
confronto pu\`o essere soltanto visuale, ma esistono metodi
matematici (\emph{metodo del chi quadro};%
\index{metodo!del $\chi^2$}
si veda in proposito il paragrafo \ref{ch:12.comdadis}) che
permettono di stimare con esattezza la probabilit\`a che i
dati di un istogramma provengano da una data distribuzione,
nel nostro caso quella normale.%
\index{errori di misura!sistematici|)}

\index{istogrammi!e curva normale|(}%
Per sovraimporre la curva di Gauss ad un istogramma, occorre
comunque moltiplicarne in ogni punto l'ordinata per un
fattore costante.  L'altezza dell'istogramma \`e infatti in
ogni intervallo data da
\begin{equation*}
  y_i = \frac{n_i \, A}{\Delta x_i}
\end{equation*}
dove $n_i$ \`e il numero di valori osservati nell'intervallo
di centro $x_i$ ed ampiezza $\Delta x_i$, mentre $A$ \`e
l'area del rettangolo corrispondente ad una osservazione.

\index{normalizzazione!della funzione normale agli istogrammi|(}%
Al tendere del numero $N$ di misure effettuate all'infinito,
risulta
\begin{align*}
  \lim_{N \rightarrow \infty} \frac{n_i}{N} &=
    \Pr \left( x_i - \frac{\Delta x_i}{2} \le x <
    x_i + \frac{\Delta x_i}{2} \right) \\[1ex]
  &= \frac{1}{\sigma \sqrt{2 \pi}}
    \int_{x_i - \frac{\Delta x_i}{2}}^{x_i
    + \frac{\Delta x_i}{2}} \! e^{- \frac{1}{2}
    \left( \frac{ x - x^* }{ \sigma } \right)^2 } \de x
\end{align*}
e dunque
\begin{equation*}
  y_i \;\;\; \xrightarrow[\quad N \to \infty \quad] \;\;\;
  \frac{N \, A}{\Delta x_i} \, \frac{1}{\sigma \sqrt{2 \pi}}
  \int_{x_i - \frac{\Delta x_i}{2}}^{x_i +
    \frac{\Delta x_i}{2}} \! e^{- \frac{1}{2}
    \left( \frac{x - x^*}{\sigma} \right) ^2} \de x \peq .
\end{equation*}

Cio\`e l'altezza dell'istogramma, in ognuna delle classi di
frequenza, tende al valore medio sull'intervallo
corrispondente della funzione di Gauss moltiplicato per un
fattore costante $NA$.  Allora la curva da sovrapporre
all'istogramma sperimentale deve essere quella che
corrisponde alla funzione
\begin{equation*}
  f(x) = \frac{NA}{s \sqrt{2\pi}} \,
  e^{- \frac{1}{2} \left( \frac{x -
    \bar x}{s} \right) ^2 }
\end{equation*}
(in luogo del valore vero $x^*$ e dell'errore quadratico
medio $\sigma$, generalmente ignoti, si pongono le loro
stime, $\bar x$ e $s$ rispettivamente, ottenute dal campione
stesso); osserviamo che $f(x)$ sottende la stessa area $NA$
dell'istogramma.

Se gli intervalli hanno tutti la medesima ampiezza $\Delta
x$, l'area del rettangolo elementare vale $A = \Delta x$,
assumendo l'arbitraria unit\`a di misura per le ordinate
pari all'altezza costante del rettangolo elementare, e la
funzione diviene
\begin{equation*}
  f(x) = \frac{N \, \Delta x}{s \sqrt{2\pi}} \,
  e^{- \frac{1}{2} \left( \frac{x -
    \bar x}{s} \right) ^2 } \peq .
\end{equation*}%
\index{normalizzazione!della funzione normale agli istogrammi|)}%
\index{istogrammi!e curva normale|)}

Sempre per quel che riguarda le implicazioni ``pratiche''
della legge normale di distribuzione degli errori, un altro
punto sul quale gli studenti hanno frequentemente dei dubbi
riguarda l'applicazione della funzione di Gauss a grandezze
misurate s\`\i\ commettendo errori casuali, ma che siano per
loro natura \emph{limitate}.  Ad esempio, una lunghezza \`e
una grandezza fisica implicitamente non negativa: quindi la
densit\`a di probabilit\`a associata ai particolari valori
ottenibili $x$ dovrebbe essere identicamente nulla quando $x
< 0$, mentre la funzione normale si annulla soltanto quando
$x = \pm \infty$.  Affermare che i risultati della misura
seguono la legge di Gauss sembra dunque una contraddizione.

La risposta a questa obiezione \`e che la funzione di
distribuzione della $x$ effettivamente \emph{non pu\`o
  essere normale}: ma che la reale differenza tra la vera
funzione di distribuzione e quella di Gauss \`e
assolutamente trascurabile.  Facciamo un esempio pratico:
supponiamo di voler misurare la dimensione del lato di un
quaderno (di valore vero $20\un{cm}$) con un regolo graduato,
e di commettere un errore di misura $\sigma = 1\un{mm}$; la
probabilit\`a di trovare un risultato in un intervallo ampio
$1\un{mm}$ appena alla sinistra dello zero secondo la legge
normale vale
\begin{equation*}
  p \; \simeq \; \frac{1}{\sqrt{2 \pi}} \, e^{- \frac{1}{2}
    \, 200^2} \; \approx \; 0.4 \, e^{-2 \times 10^4} \peq ;
\end{equation*}
quindi $\ln(p) \sim -2 \times 10^4$, e $\log_{10} (p) =
\ln(p) \cdot \log_{10} (e) \sim -10^4$, mentre dovrebbe
essere rigorosamente $p \equiv 0$.

Per valutare le reali implicazioni di un valore di $p$ come
quello che stiamo considerando, attualmente il numero di
atomi presenti nell'universo si stima essere dell'ordine di
$10^{79}$; mentre l'et\`a dell'universo stesso si stima in
circa $10^{10}$ anni, ovvero dell'ordine di $10^{18}$
secondi; se pensiamo ad un gruppo di misuratori in numero
pari al numero di atomi nell'universo, ed ognuno dei quali
esegua una misura al secondo, dovrebbe passare un tempo pari
circa a 7 volte l'et\`a dell'universo stesso per ottenere un
valore illegale qualora le misure seguissero veramente la
legge di Gauss: quindi la differenza tra la funzione di
distribuzione reale e quella ipotizzata \`e effettivamente
trascurabile.

\section{Esame dei dati}%
\index{esame dei dati|(emidx}
Talvolta nella misura si compiono errori non classificabili
n\'e come casuali n\'e come sistematici: ad esempio, dopo
aver misurato un angolo di un triangolo con un goniometro,
si pu\`o riportare come valore un numero diverso scambiando
tra di loro due cifre contigue.  La conseguenza sar\`a
quella di ottenere per il risultato finale della misura (la
somma degli angoli interni di un triangolo) un dato molto
differente dagli altri, che si impone quindi alla nostra
attenzione come \emph{sospetto}.

Nasce quindi il desiderio di avere un criterio preciso in
base al quale decidere se un dato possa o meno considerarsi
\emph{sospetto}, ed essere in conseguenza eliminato.

Normalmente la procedura consigliata \`e la seguente: dopo
aver calcolato media e scarto quadratico medio, si eliminano
dal campione i dati che differiscano da $\bar x$ per pi\`u
di tre volte $s$.  Sappiamo infatti che valori che si
trovino nella regione oltre $3 \sigma$ hanno probabilit\`a
molto bassa di presentarsi (del tre per mille circa);
bisogna comunque osservare che questo modo di procedere \`e
giustificato \emph{solo} in presenza di un \emph{numero
  piccolo} di dati.

Se le misure effettuate sono in numero ad esempio di 60, ci
si attende che (per fluttuazioni dovute esclusivamente al
caso) solo 0.18 misure (praticamente: nessuna) differiscano
dal valore medio, in modulo, per pi\`u di $3 \sigma$; se
troviamo una (o pi\`u) misure di questo tipo, possiamo
attribuire la loro presenza, piuttosto che ad una
fluttuazione casuale, a cause d'errore del tipo di quelle
considerate, quindi etichettarle come \emph{sospette} ed
infine scartarle.

Le cose cambiano se ci troviamo di fronte invece ad un
milione di misure, per le quali ci aspettiamo che ben 3000
cadano (per motivi perfettamente normali) al di fuori
dell'intervallo di $3 \sigma$, e non possiamo quindi
permetterci di scartare alcun dato particolare.%
\index{esame dei dati|)}

\section{Sommario delle misure dirette}
Per concludere, dovendo effettuare delle misure
dirette:
\begin{quote}
  \begin{itemize}
  \item \textit{Bisogna considerare criticamente le
      modalit\`a della misura e le formule usate, e
      controllare le caratteristiche di costruzione e d'uso
      degli strumenti per mettere in evidenza la
      possibilit\`a di errori sistematici; se questi sono
      presenti bisogna eliminarli: o cambiando gli
      strumenti, o modificando le modalit\`a delle
      operazioni da compiere, o correggendo opportunamente i
      risultati.}
  \item \textit{Potendo, bisogna effettuare misure ripetute:
      perch\'e in questo caso sappiamo stimare
      ragionevolmente l'errore commesso a partire dalle
      misure stesse (se non \`e possibile effettuare misure
      ripetute, si assumer\`a convenzionalmente come errore
      l'inverso della sensibilit\`a dello strumento,
      ovverosia la pi\`u piccola variazione della grandezza
      indicata sulla scala di lettura); e bisogna
      effettuarne quante pi\`u possibile per aumentare in
      corrispondenza la validit\`a statistica dei nostri
      risultati.}
  \item \textit{Se il numero di misure effettuate \`e
      basso\thinspace\footnote{``Basso'' si pu\`o ad esempio
        considerare un numero di misure tale che il numero
        atteso di eventi da scartare in base alla
        distribuzione normale sia inferiore all'unit\`a.}
      si scartano quei dati che differiscano dal valore
      medio per pi\`u di 3 volte lo scarto quadratico medio
      $s$.  Effettuata questa operazione si ricalcolano la
      media $\bar x$ e lo scarto quadratico medio $s$, e si
      ricava da quest'ultimo la stima dell'errore della
      media $\sigma_{\bar x}$ costituita da $s_{\bar x} = s
      / \sqrt{N} $.}
  \item \textit{Come valore pi\`u verosimile per la
      grandezza misurata si assume $\bar x$, e come errore
      di questo valore $s_{\bar x}$; se le misure sono in
      numero sufficiente e non si sono commessi errori
      sistematici, il significato dell'errore \`e quello di
      semiampiezza dell'intervallo di confidenza centrato
      sulla media e avente probabilit\`a di includere il
      valore vero pari al 68\%.}
  \end{itemize}
\end{quote}

\section{Il teorema del limite centrale}%
\index{limite centrale, teorema del|(}
Fino ad ora abbiamo pi\`u volte sottolineato il fatto che un
preciso significato (quello statistico) dell'errore
quadratico medio pu\`o essere enunciato solo se la
distribuzione delle misure effettuate \`e quella normale.

Con riguardo alla media aritmetica delle misure, se queste
seguono la legge normale e se, inoltre, sono statisticamente
indipendenti tra loro, il teorema di pagina
\pageref{th:8.colino} ci assicura che qualunque loro
combinazione lineare (ed in particolare la media aritmetica)
\`e ancora distribuita secondo la legge normale; ed
all'errore della media $\sigma_{\bar x}$ si pu\`o quindi
attribuire lo stesso significato statistico.

Vogliamo ora ampliare questo discorso dimostrando un
importantissimo teorema della statistica e discutendone le
implicazioni:
\begin{quote}
  \textsc{Teorema (del limite centrale):} \textit{siano $N$
    variabili casuali $x_i$, statisticamente indipendenti
    tra loro e provenienti da una distribuzione avente
    densit\`a di probabilit\`a ignota, della quale esistano
    finite sia la media $\mu$ che la varianza $\sigma^2$;
    sotto questa ipotesi, la distribuzione della media
    aritmetica del campione, $\bar x$, tende asintoticamente
    alla distribuzione normale con media $\mu$ e varianza
    $\sigma^2/N$ al crescere di $N$.}
\end{quote}

Dimostreremo questo teorema facendo l'ipotesi, pi\`u
restrittiva, che esistano i momenti della funzione di
frequenza delle $x_i$ di qualunque ordine $k$ (esso pu\`o
essere dimostrato, come si vede dall'enunciato, anche se
esistono solamente i primi due); e partiamo dal fatto che,
sotto le ipotesi su dette, la somma $S$ delle $N$ variabili
casuali
\begin{equation*}
  S = \sum_{i=1}^N x_i
\end{equation*}
ha valore medio e varianza date dalle
\begin{align*}
  E(S) &= N \mu &&\text{e} & {\sigma_S}^2 &= N \sigma^2 \peq
    .
\end{align*}

Inoltre, visto che i valori $x_i$ sono tra loro
statisticamente indipendenti, possiamo applicare l'equazione
\eqref{eq:6.fucacl} per trovare la funzione caratteristica
della $S$, che vale
\begin{align*}
  \phi_S(t) &= \prod_{i=1}^N \phi_{x_i}(t) \\[1ex]
  &= \bigl[ \phi_x(t) \bigr]^N
\end{align*}
visto che le $x_i$ hanno tutte la stessa distribuzione (e
quindi la stessa funzione caratteristica).  Se consideriamo
invece gli scarti $z_i = x_i - \mu$ delle $x_i$ dalla media,
dalla \eqref{eq:6.fuccav} possiamo ricavare la funzione
caratteristica della $z$:
\begin{gather}
  \phi_z(t) = e^{- i \mu t} \phi_x(t) \label{eq:9.phiz}
    \\
  \intertext{e, se esistono tutti i momenti fino a
    qualsiasi ordine della $x$ (e in conseguenza anche
    della $z$), la \eqref{eq:6.funcar1} implica}
  \phi_z(t) \; = \; \sum_{k=0}^{\infty}
    \frac{(it)^k}{k!} \, \lambda_k \; = \; 1 + 0 -
    \frac{1}{2} \, t^2 \sigma^2 + \mathcal{O} \bigl( t^3
    \bigr) \label{eq:9.phizsv}
\end{gather}
in cui i $\lambda_k$ sono i momenti della funzione di
frequenza della $z$, i primi due dei quali valgono 0 e
$\sigma^2$.

Introduciamo infine la nuova variabile
\begin{equation*}
  y \; = \; \frac{S - E(S)}{\sigma_S} \; = \;
    \frac{1}{\sigma \sqrt{N}} \, S -
    \frac{N \mu}{\sigma \sqrt{N}} \; = \;
    \frac{1}{\sigma \sqrt{N}} \sum_{i=1}^N \left( x_i -
    \mu \right)
\end{equation*}
e indichiamo poi con $\phi_y(t)$ la funzione caratteristica
della $y$; essendo quest'ultima lineare in $S$ abbiamo dalla
\eqref{eq:6.fuccav} che
\begin{align*}
  \phi_y(t) &= e^{- i \, t \, \frac{N \, \mu}{\sigma
    \sqrt{N}}} \cdot \phi_S \left( \frac{t}{\sigma
    \sqrt{N}} \right) \\[1ex]
  &= e^{- i \, t \, \frac{N \, \mu}{\sigma \sqrt{N}}}
    \left[ \phi_x \left( \frac{t}{\sigma \sqrt{N}}
    \right) \right]^N \\[1ex]
  &= \left[ e^{- i \, \mu \, \frac{t}{\sigma \sqrt{N}}}
    \cdot \phi_x \left( \frac{t}{\sigma \sqrt{N}}
    \right) \right]^N \\[1ex]
  &= \left[ \phi_z \left( \frac{t}{\sigma \sqrt{N}}
    \right) \right]^N
\end{align*}
ricordando la \eqref{eq:9.phiz}.  Da qui, introducendo
l'espressione \eqref{eq:9.phizsv} prima ottenuta per lo
sviluppo di $\phi_z(t)$,
\begin{align*}
  \phi_y(t) &= \left[ 1 - \frac{1}{2} \, \frac{t^2}{N
    \sigma^2} \, \sigma^2 + \mathcal{O} \left(
    \frac{t^3}{N^{ \frac{3}{2}}} \right) \right]^N
    \\[1ex]
  &= \left[ 1 - \frac{t^2}{2 N} + \mathcal{O}
    \left( N^{- \frac{3}{2} } \right) \right]^N
\end{align*}
e quando $N$ tende all'infinito
\begin{equation*}
  \lim_{N \to \infty} \phi_y(t) = e^{- \frac{t^2}{2}}
\end{equation*}
sfruttando il limite notevole
\begin{equation} \label{eq:9.linote}
  \lim_{x \to +\infty} \left( 1 + \frac{k}{x}
    \right)^x \; = \; e^k
\end{equation}
(qui, appunto, $k = -t^2/2$).  Insomma la funzione
caratteristica della $y$ tende a quella di una distribuzione
normale di media zero e varianza 1: quindi la $S$ tende
asintoticamente ad una distribuzione normale di media $N
\mu$ e varianza $N \sigma^2$; e $\bar x = S / N$ tende
asintoticamente ad una distribuzione normale di media $\mu$
e varianza $\sigma^2 / N$.

\index{media!aritmetica!come stima del valore vero|(}%
Il teorema \`e di fondamentale importanza perch\'e non fa
alcuna ipotesi sulla distribuzione delle variabili che
compongono il campione (all'infuori del requisito
dell'esistenza di media e varianza).  Con riguardo alle
misure ripetute di una stessa grandezza fisica esso ci dice
che, se anche la loro distribuzione non segue la legge di
Gauss, \emph{purch\'e se ne abbia un numero sufficiente} il
nostro risultato finale (la media aritmetica) tuttavia la
segue ugualmente in modo approssimato: cos\`\i\ che l'errore
della media conserva il consueto significato statistico (di
semiampiezza dell'intervallo, centrato su $\bar x$, che
contiene il valore vero con probabilit\`a costante
prefissata del 68\%) anche se questo non \`e verificato per
le singole misure.

Da notare che il teorema del limite centrale implica una
convergenza asintoticamente normale del valore medio del
campione al valore medio della popolazione delle misure; per
attribuire a quest'ultimo, come si \`e fatto nell'ultima
frase, il significato di valore vero della grandezza
misurata, si sottintende che le misure abbiano
distribuzione, ancorch\'e di forma non specificata,
simmetrica rispetto al valore vero $x^*$; insomma che errori
per difetto e per eccesso siano ugualmente probabili.%
\index{media!aritmetica!come stima del valore vero|)}%
\index{limite centrale, teorema del|)}

Incidentalmente, notiamo qui come il \emph{prodotto} di
molte variabili casuali indipendenti debba avere un
comportamento, indipendentemente dal tipo di distribuzione,
asintoticamente tendente a quello di una distribuzione
\emph{log-normale}.

\subsection{Applicazione: numeri casuali normali}%
\index{pseudo-casuali, numeri!con distribuzione normale|(}
Siano gli $u_i$ (con $i=1,\ldots,N$) dei numeri provenienti
da una popolazione $u$ distribuita uniformemente
nell'intervallo $[0,1]$; abbiamo visto nel paragrafo
\ref{ch:8.distun} che $E(u) = \frac{1}{2}$ e $\var(u) =
\frac{1}{12}$.  La loro media aritmetica $\bar u$, in
conseguenza del teorema del limite centrale, tende
asintoticamente (al crescere di $N$) alla distribuzione
normale con media $\frac{1}{2}$ e varianza $\frac{1}{12\cdot
  N}$; quindi la loro somma $N \bar u$ \`e asintoticamente
normale con media $\frac{N}{2}$ e varianza $\frac{N}{12}$;
e, infine, la variabile casuale
\begin{equation} \label{eq:9.nucaga}
  x = \frac{\sum\limits_{i=1}^N u_i -
    \dfrac{N}{2}}{\sqrt{\dfrac{N}{12}}}
\end{equation}
\`e asintoticamente normale con media 0 e varianza 1.

Di questa propriet\`a si pu\`o far uso per ottenere da un
computer dei numeri pseudo-casuali con distribuzione
(approssimativamente) normale, a partire da altri numeri
pseudo-casuali con distribuzione uniforme; in pratica
l'approssimazione \`e gi\`a buona quando $N \gtrsim 10$, e
scegliendo $N=12$ possiamo, ad esempio, porre semplicemente
\begin{equation*}
  x = \sum_{i=1}^{12} u_i - 6 \peq .
\end{equation*}

\`E da notare, comunque, che \textbf{non \`e} buona pratica
servirsi di questo metodo: anche se la parte centrale della
distribuzione normale \`e approssimata abbastanza bene, le
code mancano totalmente (essendo impossibile che risulti
$|x| > \sqrt{3N}$); l'effetto di questa mancanza, quando
(come nelle analisi fisiche basate su metodi di Montecarlo)
vengano richiesti numeri pseudo casuali per generare eventi
simulati in quantit\`a dell'ordine di milioni almeno, \`e
tale da invalidare completamente i risultati.

Soprattutto, poi, generare numeri pseudo-casuali normali
usando il teorema del limite centrale non \`e solo
sbagliato, ma inutile: esistono altri metodi (come ad
esempio quello di Box--Muller che discuteremo ora) che sono
in grado di generare numeri pseudo-casuali con una
\emph{vera} distribuzione normale usando, per il calcolo, un
tempo non molto superiore a quello richiesto dalla
\eqref{eq:9.nucaga}.

\index{Box--Muller, metodo di|(}%
Siano $x$ ed $y$ due variabili casuali \emph{statisticamente
  indipendenti}, ed aventi distribuzione uniforme
nell'intervallo $[0,1]$; consideriamo le altre due variabili
casuali $u$ e $v$ definite attraverso le
\begin{align} \label{eq:9.boxmul}
  u &= \sqrt{-2 \ln x} \cdot \cos( 2 \pi y ) &&\text{e}
  & v &= \sqrt{-2 \ln x} \cdot \sin( 2 \pi y ) \peq .
\end{align}

Queste funzioni si possono invertire, e risulta
\begin{align*}
  x &= e^{- \frac{1}{2} \left( u^2 + v^2
  \right)} &&\text{e} & y &= \frac{1}{2 \pi} \, \arctan
  \left( \frac{v}{u} \right)
\end{align*}
con derivate parziali prime date da
\begin{equation*}
  \begin{cases}
    \displaystyle \frac{\partial x}{\partial u} = - \,
      u \cdot e^{- \frac{1}{2} \left( u^2 + v^2
      \right)} \\[2.5ex]
    \displaystyle \frac{\partial x}{\partial v} = - \,
      v \cdot e^{- \frac{1}{2} \left( u^2 + v^2
      \right)}
  \end{cases}
\end{equation*}
e da
\begin{equation*}
  \begin{cases}
    \displaystyle \frac{\partial y}{\partial u} =
      \frac{1}{2 \pi} \: \frac{1}{1 + \frac{v^2}{u^2}}
      \left( - \, \frac{v}{u^2} \right) \\[3ex]
    \displaystyle \frac{\partial y}{\partial v} =
      \frac{1}{2 \pi} \: \frac{1}{1 + \frac{v^2}{u^2}}
      \; \frac{1}{u}
  \end{cases}
\end{equation*}

Il determinante Jacobiano%
\index{Jacobiano, determinante}
delle $(x,y)$ rispetto alle $(u,v)$ vale
\begin{align*}
  \frac{\partial (x, y)}{\partial (u, v)} & =
    \frac{\partial x}{\partial u} \, \frac{\partial
    y}{\partial v} - \frac{\partial x}{\partial v} \,
    \frac{\partial y}{\partial u} \\[1ex]
  &= - \, \frac{1}{2\pi} \, e^{- \frac{1}{2}
    \left( u^2 + v^2 \right)}
\end{align*}
per cui, essendo la densit\`a di probabilit\`a congiunta
delle due variabili casuali $x$ ed $y$ data dalla
\begin{gather*}
  f(x,y) = 1 \\
  \intertext{e applicando la \eqref{eq:6.cavamu}, la
    densit\`a di probabilit\`a congiunta della $u$ e
    della $v$ \`e data da}
  f(u,v) = \frac{1}{\sqrt{2\pi}}
    e^{- \frac{1}{2} u^2} \cdot
    \frac{1}{\sqrt{2\pi}} e^{- \frac{1}{2}
    v^2}
\end{gather*}
e quindi, in conseguenza della \eqref{eq:6.instmu}, la $u$ e
la $v$ sono due variabili casuali \emph{statisticamente
  indipendenti} tra loro ed entrambe aventi funzione di
frequenza data dalla \emph{distribuzione normale
  standardizzata}; questo \`e appunto il metodo cosiddetto
``di Box--Muller'' per la generazione di numeri
pseudo-casuali con distribuzione normale, a partire da
numeri pseudo-casuali con distribuzione uniforme.

Una variante che consente di sveltire questo metodo (lento,
perch\'e l'esecuzione delle funzioni logaritmo, seno e
coseno consuma molto tempo di \textsc{cpu}) consiste nel
generare dapprima due numeri pseudo-casuali $x'$ e $y'$
distribuiti uniformemente tra i limiti $-1$ e $+1$; e
nell'accettarli se $S = R^2 = {x'}^2 + {y'}^2 \leq 1$, in
modo che il punto $P$ le cui coordinate essi rappresentano
nel piano $\{ x', y' \}$ sia uniformemente distribuito entro
il cerchio avente centro nell'origine $O$ e raggio unitario
--- o nel rigettarli in caso contrario, ripetendo il passo
precedente.

Questa prima condizione in realt\`a \emph{rallenta} il
procedimento, perch\'e la coppia di numeri a caso viene
accettata con probabilit\`a $\frac{\pi}{4} \approx 78.5\%$;
ma se, a questo punto, si usa al posto della $x$ nella
\eqref{eq:9.boxmul} il valore di $S$ (che, come non \`e
difficile dimostrare, \`e anch'esso distribuito
uniformemente nell'intervallo $[0,1]$); e se si prende poi
in luogo dell'angolo $2 \pi y$ l'angolo polare $\theta$ tra
$\overline{OP}$ e l'asse delle $x'$, il calcolo risulta in
definitiva molto pi\`u rapido: perch\'e il seno ed il coseno
di $\theta$ si possono valutare come $y'/R$ ed $x'/R$
rispettivamente, eseguendo il calcolo di una radice quadrata
e due divisioni soltanto.%
\index{Box--Muller, metodo di|)}%
\index{pseudo-casuali, numeri!con distribuzione normale|)}

\endinput

% $Id: chapter10.tex,v 1.1 2005/03/01 10:06:08 loreti Exp $

\chapter{Le misure indirette}%
\index{misure!indirette|(}%
\label{ch:10.misind}
Misure indirette, come sappiamo, sono quelle eseguite non
sulla grandezza fisica che interessa determinare ma su altre
grandezze che siano a quest'ultima legate da una qualche
relazione funzionale; quest'ultima ci permetter\`a poi di
ricavarne il valore mediante il calcolo.

Supponiamo per semplicit\`a che queste altre grandezze
vengano misurate direttamente: gli inevitabili errori
commessi per ognuna di esse si ripercuoteranno poi
attraverso i calcoli effettuati, e si \emph{propagheranno}
fino al risultato finale; l'entit\`a dell'errore nelle
misure indirette dipender\`a dunque sia dai valori di quelli
commessi nelle misure dirette, sia dalla forma analitica
della funzione usata per il calcolo.

Consideriamo il caso del tutto generale di una qualsiasi
funzione $F$ di pi\`u variabili $ x, y, z, \ldots\, $:
ammettendo che i valori di queste variabili si possano
ottenere da misure di tipo diretto, vogliamo determinare un
algoritmo per ricavare da tali misure una stima del valore
vero di $F$; infine, nell'ipotesi che le variabili siano
anche tra loro statisticamente indipendenti, vedremo come si
pu\`o valutare l'errore collegato a tale stima.

\section{Risultato della misura}
Innanzi tutto, \`e chiaro che il valore vero $F^*$ della
grandezza $F$ \`e quello che corrisponde ai valori veri
delle variabili indipendenti da cui $F$ dipende:
\begin{equation} \label{eq:10.rimind}
  F^* = F( x^*, y^*, z^*,\ldots) \peq .
\end{equation}
Non avendo per\`o a disposizione tali valori veri, tutto
quello che possiamo fare \`e usare le migliori stime di cui
disponiamo: cio\`e, supponiamo, i valori medi di campioni di
determinazioni ripetute di tutte queste grandezze; insomma,
calcolare il valore \ob{F}\ assunto dalla funzione in
corrispondenza dei valori $\bar x, \bar y, \bar z,\ldots$
delle variabili.

Ricordiamo che il valore di una funzione di pi\`u variabili
$F$ in un qualsiasi punto si pu\`o ricavare dal valore
assunto dalla $F$ e dalle sue derivate successive in un
punto diverso, attraverso la formula dello sviluppo in serie
di Taylor:
\begin{align*}
  F(x,y,z,\ldots) &= F \left( x_0,y_0,z_0,
    \ldots\right) + && \text{(ordine zero)}
    \\[1ex]
  &+ \frac{\partial F}{\partial x} \left( x - x_0
    \right) + \frac{\partial F}{\partial y} \left(
    y - y_0 \right) +\cdots && \text{(primo
    ordine)} \\[1ex]
  &+ \frac{ \partial^2 F}{\partial x^2} \,
    \frac{\left( x - x_0 \right)^2}{2!} +\cdots
    && \text{(secondo ordine)} \\[1.5ex]
  &+ \mathcal{O}(3)
\end{align*}
(in cui per brevit\`a si \`e omesso di indicare che le
derivate parziali vanno calcolate per i valori delle
variabili $ x = x_0 $, $ y = y_0 $ e cos\`\i\ via).

Se \`e possibile trascurare i termini di ordine superiore al
primo, possiamo in particolare ricavare:
\begin{align*}
  \ob{F} &= F \left( \bar x, \bar y, \bar z,
   \ldots\right) \\[1ex]
  &\approx F \left( x^*, y^*, z^*,\ldots\right) +
    \frac{\partial F}{\partial x} \left( \bar x -
    x^* \right) + \frac{\partial F}{\partial y}
    \left( \bar y - y^* \right) + \cdots \peq .
\end{align*}
Prendendo poi il valore medio di entrambi i membri e tenendo
presente nei passaggi sia la \eqref{eq:10.rimind}, sia che
risulta $ E( \bar x - x^* ) \: = \: E(\bar x)-x^* \: \equiv
\: 0 $ in assenza di errori sistematici (e similmente per le
altre variabili), si ottiene
\begin{align*}
  E \Bigl \{ F \left( \bar x, \bar y, \bar z,
   \ldots\right) \Bigr \} &\approx F^* +
    \frac{\partial F}{\partial x} \, E \left( \bar x
    - x^* \right) + \frac{\partial F}{\partial y} \,
    E \left( \bar y - y^* \right) +\cdots \\[1ex]
  &= F^*
\end{align*}
cio\`e
\begin{equation*}
  E \left( \ob{F} \right) \approx F^*
\end{equation*}
e, in definitiva:
\begin{quote}
  \textit{In media, il valore di una funzione $F$ calcolato
    per le medie misurate delle variabili coincide col
    valore vero.}
\end{quote}
(ossia \ob{F}\ \`e una stima \emph{imparziale}%
\index{stima!imparziale}
di $F^*$).

Ricordiamo che questa conclusione \`e valida \emph{solo
  approssimativamente}, perch\'e nello sviluppo in serie di
Taylor abbiamo trascurato tutti i termini di ordine
superiore al primo; ma quali sono i limiti della validit\`a
della conclusione?  In quali casi si possono cio\`e
effettivamente considerare trascurabili i termini del
second'ordine e degli ordini superiori?

Ognuno dei termini di ordine $i$ nello sviluppo di $F$
contiene una delle derivate $i$-esime della funzione,
moltiplicata per un fattore del tipo $ ( \bar x - x^* ) $
elevato alla $i$-esima potenza e divisa per il fattoriale di
$i$; sar\`a senz'altro lecito trascurare questi termini se
le differenze tra i valori medi stimati per le variabili
indipendenti ed i loro valori veri sono piccole, in altre
parole \emph{se gli errori commessi nelle misure dirette
  sono piccoli}.

Un caso particolare \`e poi quello in cui la $F$ \`e una
funzione lineare in ognuna delle variabili da cui dipende;
in questo caso, ovviamente, tutte le derivate di ordine
successivo al primo sono identicamente nulle, e le
conclusioni precedenti sono valide \emph{esattamente}.

\section{Combinazioni lineari di misure dirette}
Supponiamo che le misure dirette delle variabili
indipendenti da cui dipende la $F$ siano esenti da errori
sistematici, e che siano pertanto distribuite secondo la
legge normale; consideriamo dapprima quelle particolari
funzioni che sono le combinazioni lineari di pi\`u
variabili:
\begin{equation*}
  F \; = \; k_1 \, x_1 + k_2 \, x_2 + k_3 \, x_3
    +\cdots \; = \; \sum \nolimits_i k_i \, x_i \peq .
\end{equation*}

Abbiamo gi\`a visto, nell'equazione \eqref{eq:5.medcol} a
pagina \pageref{eq:5.medcol}, che il valore medio di una
tale funzione \`e la combinazione lineare, con gli stessi
coefficienti, delle medie delle variabili; e, se supponiamo
inoltre che le variabili siano tutte \emph{statisticamente
  indipendenti} tra loro, sappiamo anche che la varianza di
$F$ \`e poi data (equazione \eqref{eq:5.varcol} a pagina
\pageref{eq:5.varcol}) dalla combinazione lineare delle loro
varianze con coefficienti pari ai quadrati dei rispettivi
coefficienti:
\begin{gather*}
  E(F) \; = \; \sum \nolimits_i k_i \, E ( x_i )
  \; = \; \sum \nolimits_i k_i \, x_i^*
  \; \equiv \; F^* \\
  \intertext{e}
  {\sigma_F}^2 = \sum \nolimits_i {k_i}^2 {\sigma_i}^2 \peq .
\end{gather*}

Abbiamo inoltre dimostrato, a pagina \pageref{th:8.colino},
un teorema secondo il quale una qualsiasi combinazione
lineare a coefficienti costanti di variabili casuali aventi
distribuzione normale, ed inoltre tra loro statisticamente
indipendenti, \`e anch'essa distribuita secondo la legge
normale.

Ora, sapendo che la distribuzione della $F$ \`e data dalla
funzione di Gauss, siamo anche in grado di attribuire un
significato pi\`u preciso al suo errore quadratico medio
$\sigma_F$: quello cio\`e di semiampiezza dell'intervallo,
avente centro sul valore medio $E(F) = F^*$, che contiene un
qualsiasi valore della $F$ (ed in particolare la nostra
miglior stima \ob{F}\,) con una probabilit\`a del 68\%; o,
di converso, la semiampiezza di un intervallo centrato sulla
nostra migliore stima \ob{F}\ e che contiene l'ignoto valore
vero $F^*$ con una probabilit\`a del 68\%.

In definitiva le formule precedenti risolvono il problema
delle misure indirette per quelle particolari funzioni che
sono le combinazioni lineari, permettendoci di calcolare per
esse sia il valore stimato pi\`u verosimile che l'errore, e
dandoci inoltre l'interpretazione probabilistica di questo
errore.

\section{La formula di propagazione degli errori}%
\index{propagazione degli errori, formula di|(emidx}
Ora, qualsiasi funzione di pi\`u variabili si pu\`o
considerare in prima approssimazione lineare; questo se ci
limitiamo a considerarla in un dominio di definizione
abbastanza ristretto da poter trascurare i termini di ordine
superiore al primo in uno sviluppo in serie di Taylor.  In
definitiva possiamo estendere le conclusioni del paragrafo
precedente ad una qualsiasi funzione di pi\`u variabili
\begin{multline*}
  F(x,y,z,\ldots) \; \approx \\
  F (\bar x, \bar y, \bar z,\ldots) +
  \frac{\partial F}{\partial x} \left( x - \bar x \right) +
  \frac{\partial F}{\partial y} \left( y - \bar y \right) +
  \frac{\partial F}{\partial z} \left( z - \bar z \right)
  +\cdots
\end{multline*}
per la cui varianza avremo
\begin{equation} \label{eq:10.proper}
  \boxed{ \rule[-6mm]{0mm}{14mm} \quad
    {\sigma_F}^2 \; \approx \;
    \left( \dfrac{\partial F}{\partial x}
      \right)^2 \! {\sigma_x}^2 +
    \left( \dfrac{\partial F}{\partial y}
      \right)^2 \! {\sigma_y}^2 +
    \left( \dfrac{\partial F}{\partial z}
      \right)^2 \! {\sigma_z}^2
      +\cdots \quad }
\end{equation}
(le derivate vanno calcolate per i valori $x = \bar x$, $y =
\bar y$, $z = \bar z$,\ldots\ delle variabili indipendenti).

Questa formula \`e nota sotto il nome di \emph{formula di
  propagazione degli errori}: ripetiamo che si tratta di una
formula \emph{approssimata}; che \`e valida solo se non si
commettono errori troppo grandi nelle misure dirette delle
variabili; e che presuppone che le variabili stesse siano
tra loro statisticamente indipendenti\/\footnote{Una formula
  di propagazione degli errori per variabili qualsiasi (che
  ossia non ne presupponga l'indipendenza statistica)
  verr\`a ricavata pi\`u avanti, nel paragrafo
  \ref{ch:c.proper}.}.  La formula di propagazione \`e
invece \emph{esatta} nel caso particolare (esaminato nel
paragrafo precedente) di una combinazione lineare di
variabili casuali indipendenti, caso questo nel quale tutte
le derivate parziali di ordine superiore al primo sono
identicamente nulle.%
\index{propagazione degli errori, formula di|)}%
\index{misure!indirette|)}

\section{Errore dei prodotti di potenze}%
\index{propagazione degli errori, formula di!per prodotti di potenze|(}
Applichiamo ora la formula \eqref{eq:10.proper} di
propagazione degli errori a quella particolare classe di
funzioni costituita dai prodotti di potenze delle variabili
indipendenti: cio\`e alle funzioni del tipo
\begin{equation*}
  F(x, y, z,\ldots) = K \cdot
  x^\alpha \cdot y^\beta \cdot z^\gamma\cdots \peq .
\end{equation*}

Calcoliamo innanzi tutto le derivate parziali di $F$;
risulta
\begin{displaymath}
  \left \{ \begin{array}{rcccl}
    \dfrac{\partial F}{\partial x}
      & = & K \cdot \alpha \, x^{\alpha - 1} \cdot
        y^\beta \cdot z^\gamma\cdots
      & = & \alpha \, \dfrac{F}{x} \\[3ex]
    \dfrac{\partial F}{\partial y}
      & = & K \cdot x^\alpha \cdot \beta \, y^{\beta -
        1} \cdot z^\gamma\cdots
      & = & \beta \, \dfrac{F}{y} \\[3ex]
    \cdots & & & &
  \end{array} \right.
\end{displaymath}
(ammettendo che nessuna delle variabili sia nulla; questo
implica che anche la $F$ abbia valore diverso da zero).
Introducendo questi valori delle derivate nella formula di
propagazione degli errori, avremo
\begin{align*}
  {\sigma_F}^2 &\approx
    \left( \frac{\partial F}{\partial x} \right) ^2
    \! {\sigma_x}^2 +
    \left( \frac{\partial F}{\partial y} \right) ^2
    \! {\sigma_y}^2 +\cdots \\[1ex]
  &= \alpha^2 \, \frac{F^2}{x^2} \, {\sigma_x}^2 +
    \beta^2 \, \frac{F^2}{y^2} \, {\sigma_y}^2
    +\cdots
\end{align*}
ed in definitiva
\begin{equation*}
  \Biggl( \frac{\sigma_F}{ F } \Biggr) ^2
  \approx \alpha^2 \Biggl( \frac{\sigma_x}{ x }
  \Biggr) ^2 + \beta^2 \Biggl( \frac{\sigma_y}{ y
      } \Biggr) ^2 +\cdots \peq ;
\end{equation*}
relazione che permette di ricavare con semplici calcoli
l'errore relativo di $F$ dagli errori relativi commessi
nella misura delle variabili indipendenti.  Per quanto detto
in precedenza, questa relazione \`e solo una prima
approssimazione; e possiamo ritenerla valida se le variabili
indipendenti sono misurate con errori piccoli.%
\index{propagazione degli errori, formula di!per prodotti di potenze|)}

\section{Errori massimi}%
\index{errore!massimo|(}
Quando si parla di errori di misura senza specificare
null'altro, si sottintende di norma che i numeri riportati
si riferiscono ad errori quadratici medi; talvolta per\`o si
\`e in grado di indicare un intervallo all'interno del quale
si sa con assoluta certezza che deve trovarsi il valore vero
della grandezza misurata: in questi casi si pu\`o ovviamente
attribuire un errore in termini \emph{assoluti} (sia in
difetto che in eccesso) al valore indicato.  Supponendo per
semplicit\`a che i valori limite siamo simmetrici rispetto
al risultato trovato, che si potr\`a quindi ancora esprimere
nella forma
\begin{equation*}
  x = x_0 \pm \Delta x \peq ,
\end{equation*}
vogliamo ora determinare la legge secondo la quale si
propagano questi \emph{errori massimi} nelle misure
indirette.

\`E immediato riconoscere che, se $F = Kx$ e $x \in [ x_0 -
\Delta x, x_0 + \Delta x ]$, necessariamente $F$ deve
appartenere ad un intervallo di semiampiezza $\Delta F$, con
$\Delta F = |K| \, \Delta x$.  Similmente, sommando (o
sottraendo) due grandezze indipendenti di cui si conoscano
gli errori massimi, il risultato $F = x \pm y$ dovr\`a
essere compreso in un intervallo di semiampiezza $\Delta F =
\Delta x + \Delta y$.  Usando entrambe queste conclusioni,
nel caso di una combinazione lineare%
\index{propagazione degli errori, formula di!per errori massimi|(}
\begin{gather*}
  F = \sum_{i=1}^N a_i \, x_i \\
  \intertext{l'errore massimo su $F$ vale}
  \Delta F = \sum_{i=1}^N \left| a_i \right| \Delta x_i \peq
  .
\end{gather*}

Per una relazione funzionale qualsiasi
\begin{equation*}
  F = F(x_1, x_2, \ldots, x_N) \peq ,
\end{equation*}
e nei limiti in cui si possano trascurare i termini di
ordine superiore al primo in uno sviluppo in serie di
Taylor, la formula di propagazione per gli errori massimi
\`e dunque
\begin{gather*}
  \Delta F \approx \sum_{i=1}^N \left| \frac{\partial
    F}{\partial x_i} \right| \Delta x_i \peq ; \\
  \intertext{e, per i prodotti di potenze del tipo $ F = K
    \cdot x^\alpha \cdot y^\beta \cdots$,}
  \frac{\Delta F}{|F|} \approx |\alpha| \, \frac{\Delta
    x}{|x|} + |\beta| \, \frac{\Delta y}{|y|}  + \cdots \peq
  .
\end{gather*}%
\index{propagazione degli errori, formula di!per errori massimi|)}%
\index{errore!massimo|)}

\endinput

% $Id: chapter11.tex,v 1.2 2006/11/06 10:30:15 loreti Exp $

\chapter{Stime di parametri}%
\label{ch:11.teldat}
In questo capitolo prenderemo in considerazione due speciali
tecniche di elaborazione dei dati che sono utilizzate per
stimare il valore di parametri ignoti dai quali le
distribuzioni teoriche dipendono: la media pesata di
determinazioni sperimentali aventi diversa precisione; e la
valutazione dei parametri da cui dipende l'equazione di una
curva che deve descrivere una relazione tra pi\`u variabili
interconnesse e misurate indipendentemente (curva
interpolante i dati sperimentali).

Il metodo usato per la soluzione \`e, in entrambi i casi,
quello della \emph{massima verosimiglianza} (introdotto
originariamente da Fisher\/\footnote{Sir Ronald Fisher
  nacque a Londra nel 1890 e mor\`\i\ ad Adelaide (in
  Australia) nel 1962.  \`E considerato, per l'importanza
  dei suoi lavori, uno dei fondatori della moderna
  statistica: oltre al concetto di verosimiglianza
  (\emph{likelihood} in inglese), introdusse per primo
  l'analisi delle varianze e scoperse la forma analitica
  delle funzioni di distribuzione di molte importanti
  variabili casuali; dette poi importanti contributi ai
  metodi per i piccoli campioni ed a quelli per la verifica
  delle ipotesi.}%
\index{Fisher, sir Ronald Aylmer|emidx}
nel 1921); la prima parte del capitolo riguarder\`a appunto
il problema della stima del valore dei parametri in
generale, e questo metodo in particolare.

\section{Stime e loro caratteristiche}%
\index{stime|(}%
\label{ch:11.sticar}
Supponiamo che la densit\`a di probabilit\`a $f(x;\theta)$
di una variabile casuale continua $x$ (che possa assumere
tutti i valori dell'asse reale) dipenda da un parametro
$\theta$, il cui valore vero $\theta^*$ ci sia ignoto; se si
hanno a disposizione $N$ determinazioni sperimentali
indipendenti $x_i$ della grandezza $x$, vogliamo trovare una
funzione $\bar \theta = \bar \theta(x_1, x_2,\ldots,x_N)$
che, a partire da esse, ci permetta di ricavare, nella
maniera migliore possibile, un valore numerico da attribuire
a $\theta^*$: le funzioni $\bar \theta$ di questo tipo si
chiamano appunto \emph{stime}.

Una stima \`e dunque una funzione di variabili casuali, e,
pertanto, una variabile casuale essa stessa; potremo in
conseguenza parlare del valore medio o della varianza di una
particolare stima, intendendo cos\`\i\ riferirci alle
caratteristiche della popolazione dei possibili valori
restituiti dalla stima stessa in corrispondenza di tutti i
possibili campioni che possono essere usati per calcolarla.

Nella statistica, alle stime si possono associare svariate
caratteristiche; la prima di esse (e la pi\`u importante)
\`e la \textbf{consistenza}.  Una stima si dice
\emph{consistente} quando converge (probabilisticamente) al
valore vero del parametro, ossia quando
\begin{equation*}
  \lim_{N\to\infty} \bar \theta(x_1,x_2,\ldots,x_N) \; = \;
    \theta^* \peq .
\end{equation*}
Ad esempio, il teorema di \v Ceby\v sef si pu\`o enunciare
sinteticamente affermando che ``il valore medio di un
campione \`e una stima consistente del valore medio della
popolazione''.

Una seconda caratteristica delle stime \`e la
\textbf{distorsione}: una stima si dice \emph{indistorta}, o
\emph{imparziale}, se mediamente coincide col valore vero
del parametro; insomma se
\begin{equation*}
  E ( \bar \theta ) = \theta^* \peq .
\end{equation*}
Gi\`a sappiamo, dai paragrafi \ref{ch:5.vmclc} e
\ref{ch:5.scedeqm} rispettivamente, che la media dei
campioni \`e una stima indistorta del valore medio della
popolazione; mentre la varianza del campione \`e una stima
distorta, ancorch\'e consistente, della varianza della
popolazione (a meno che non sia opportunamente corretta
moltiplicandola per un fattore $N / (N-1)$).

\begin{figure}[htbp]
  \vspace*{2ex}
  \begin{center}
    \input{stime.pstex_t}
  \end{center}
  \caption[Stime consistenti ed inconsistenti, imparziali e
  deviate]{Stime consistenti ed inconsistenti, imparziali
    e deviate.}
  \label{fig:11.stime}
\end{figure}

Nella figura \ref{fig:11.stime} sono riportati esempi di
stime consistenti ed inconsistenti, distorte ed indistorte,
per dare un'idea dell'andamento della densit\`a di
probabilit\`a delle stime stesse all'aumentare delle
dimensioni del campione.

Una terza caratteristica delle stime \`e
l'\textbf{efficienza}: diremo che una prima stima \`e pi\`u
\emph{efficiente} di una seconda se la sua varianza \`e
inferiore, e, quindi, se mediamente essa \`e pi\`u vicina al
valore centrale $E( \bar \theta)$; che coincide con
$\theta^*$ se la stima \`e anche imparziale.  Esiste un
teorema (teorema di Cram\'er--Rao)%
\index{Cram\'er--Rao, teorema di}
del quale ci occuperemo sia pi\`u avanti nel corso di questo
capitolo, sia in particolare nell'appendice
\ref{ch:e.maxlik}; questo teorema dimostra l'esistenza di un
limite inferiore per la varianza delle stime, e quindi di un
limite superiore per la loro efficienza.

Se abbiamo a disposizione, poi, $M$ stime differenti
$\theta_j$ dello stesso parametro $\theta$, ogni campione di
$N$ valori $x_i$ produrr\`a, attraverso l'applicazione di
ognuna di tali stime, $M$ diversi valori per $\theta$.  Se
ora indichiamo con $f$ la densit\`a di probabilit\`a
congiunta di questi $M$ valori, risulter\`a in generale
\begin{equation*}
  f ( \theta_1, \theta_2,\ldots,\theta_M; \theta^* ) = f^M (
    \theta_1; \theta^* ) \cdot \varphi( \theta_2,
    \theta_3,\ldots,\theta_M; \theta^* | \theta_1 )
\end{equation*}
dove con $f^M ( \theta_1; \theta^* )$ abbiamo, al solito,
indicato la funzione densit\`a di probabilit\`a marginale
della sola $\theta_1$ (ovvero la densit\`a di probabilit\`a
collegata al presentarsi di un certo valore per $\theta_1$
indipendentemente da quello ottenuto per le altre stime);
mentre $\varphi( \theta_2, \theta_3\ldots,\theta_M; \theta^*
| \theta_1 )$ \`e la densit\`a di probabilit\`a di queste
ulteriori $M-1$ stime condizionata dal valore della prima.

\index{stima!sufficiente|(emidx}%
Nel caso che $\varphi$ risulti \emph{indipendente} da
$\theta^*$, la conseguenza che da questo fatto si deduce \`e
che, una volta calcolata $\theta_1$, le altre stime
sarebbero distribuite comunque nello stesso modo per
\emph{qualunque} valore di $\theta^*$; esse non potrebbero
quindi aggiungere nulla alla conoscenza gi\`a ottenuta sul
valore del parametro $\theta$: ovverosia $\theta_1$
\emph{sfrutta tutta l'informazione} sul parametro ignoto che
\`e contenuta nei dati, ed in questo caso la stima
$\theta_1$ si dice \textbf{sufficiente}.  Non \`e detto che
una stima sufficiente per un certo parametro $\theta$
esista; ma se ne esiste una, $\bar \theta$, allora ne
esistono infinite: si pu\`o dimostrare infatti che ogni
funzione monotona in senso stretto di $\bar \theta$ gode
della stessa propriet\`a.%
\index{stima!sufficiente|)}%
\index{stime|)}

\section{La stima di massima verosimiglianza}%
\index{massima verosimiglianza, metodo della|(emidx}%
\label{ch:11.maxver}
Dato un campione di $N$ determinazioni \emph{indipendenti}
$x_i$, l'espressione
\begin{equation*}
  \prod_{i=1}^N f(x_i; \theta^*)
\end{equation*}
rappresenta la densit\`a di probabilit\`a da associare
all'evento casuale consistente nell'ottenere una determinata
$N$-pla di valori, essendo $\theta^*$ il valore del
parametro da cui la $f$ dipende.

\index{funzione!di verosimiglianza|(emidx}%
Se in questa espressione si sostituisce al valore vero (che
avevamo supposto noto) $\theta^*$ il generico valore
$\theta$; e se le $x_i$ non vengono considerate pi\`u
variabili casuali, ma costanti che sono state determinate
dalle nostre operazioni di misura, la funzione
\begin{equation} \label{eq:11.funver}
  \mathcal{L} ( x_1, x_2,\ldots, x_N; \theta ) =
    \prod_{i=1}^N f(x_i; \theta)
\end{equation}
(\emph{funzione di verosimiglianza}) rappresenta la
densit\`a di probabilit\`a da associare all'evento casuale
consistente nell'essere un certo $\theta$ il valore vero del
nostro parametro, nell'ipotesi di avere gi\`a ottenuto la
particolare $N$-pla di valori sperimentali $x_1, x_2,\ldots,
x_N$.%
\index{funzione!di verosimiglianza|)}

Il metodo della massima verosimiglianza consiste
nell'adottare, come stima del parametro $\theta$, quel
valore $\widehat \theta$ \emph{che rende massima} la
funzione di verosimiglianza \eqref{eq:11.funver}; ovvero la
soluzione delle
\begin{align} \label{eq:11.eqver}
  \frac{\de \mathcal{L}}{\de \theta} &= 0 &
  \frac{\de^2 \mathcal{L}}{\de \theta^2} &< 0
\end{align}
(nel caso che le \eqref{eq:11.eqver} abbiano pi\`u di una
soluzione, si sceglie quella che corrisponde al massimo
assoluto).

Visto che il logaritmo naturale \`e (essendo la base, $e$,
maggiore di uno) una funzione monotona strettamente
crescente dell'argomento, trovare il massimo di $\ln
\mathcal{L}$ condurrebbe ancora a tutti e soli i valori che
rendono massima $\mathcal{L}$; questo corrisponde al
sostituire (essendo $\mathcal{L} > 0$), alla prima delle
\eqref{eq:11.eqver}, l'equivalente
\begin{equation*}
  \frac{1}{\mathcal{L}} \, \frac{\de \mathcal{L}}{\de
    \theta} \; = \; \frac{\de \, ( \ln \mathcal{L} )}{\de
    \theta} \; = \; 0 \peq .
\end{equation*}

Enunciamo qui, senza dimostrarle, alcune propriet\`a
fondamentali della stima di massima verosimiglianza:
\begin{enumerate}
\item La stima di massima verosimiglianza \`e una stima
  \emph{asintoticamente consistente} al crescere della
  dimensione del campione.
\item La stima di massima verosimiglianza ha una densit\`a
  di probabilit\`a \emph{asintoticamente normale} al
  crescere della dimensione del campione.
\item La stima di massima verosimiglianza \`e
  asintoticamente, al crescere della dimensione del
  campione, anche \emph{la stima pi\`u efficiente possibile}
  (ossia quella di minima varianza).
\item Se esiste una stima sufficiente di $\theta$, essa
  pu\`o sempre essere espressa come funzione della sola
  stima di massima verosimiglianza $\widehat \theta$.
\end{enumerate}%
\index{massima verosimiglianza, metodo della|)}

Le ipotesi sotto le quali si riesce a dimostrare che la
stima di massima verosimiglianza gode asintoticamente delle
propriet\`a su dette sono estremamente generali: per la
normalit\`a basta che esistano i primi due momenti della
$f(x; \theta)$; per la consistenza e la massima efficienza
basta che $f(x; \theta)$ sia continua, dotata di derivata
prima e seconda rispetto al parametro, e che l'operazione di
integrazione rispetto a $x$ commuti con quella di
derivazione rispetto a $\theta$ (ovvero, in pratica, che il
dominio di definizione della $x$ non dipenda dal parametro).

Il teorema di Cram\'er--Rao%
\index{Cram\'er--Rao, teorema di|(}
(cui si \`e prima accennato) permette di dimostrare, sotto
ipotesi del tutto generali, che esiste \emph{un estremo
  inferiore} per le varianze delle stime \emph{imparziali}
di una qualsiasi grandezza dipendente dal parametro
$\theta$; non solo, ma che, se una stima di varianza minima
esiste, essa \emph{rende massima la funzione di
  verosimiglianza}.

Pi\`u in dettaglio: nell'ipotesi che la densit\`a di
probabilit\`a $f(x;\theta)$ sia una funzione definita in una
regione dell'asse $x$ avente estremi indipendenti dal
parametro $\theta$; che esista ovunque la derivata rispetto
a $\theta$ di $\ln f(x; \theta)$; e, infine, che esista
finito il valore medio del quadrato di questa derivata
\begin{equation*}
  E \left\{ \left[ \frac{\partial}{\partial
    \theta} \ln f(x; \theta) \right]^2 \right\} \; = \; \frac{1}{N} \cdot E
    \left\{ \left[ \frac{\partial (\ln \mathcal{L})}{\partial
    \theta} \right]^2 \right\}
\end{equation*}
il teorema di Cram\'er--Rao afferma che una \emph{qualsiasi}
stima \emph{imparziale} $\bar \theta$ di $\theta$ ha una
varianza che non pu\`o essere inferiore ad un valore
(\emph{limite di Cram\'er--Rao}) dato dalla
\begin{equation} \label{eq:11.crao1}
  \var( \bar \theta ) \;
    \ge \; \frac{1}{N \cdot E \left\{ \left[ \dfrac{\partial
    }{\partial \theta} \ln f(x; \theta) \right]^2 \right\} }
    \peq .
\end{equation}
Inoltre questo estremo inferiore viene raggiunto, e vale il
segno di uguaglianza nella \eqref{eq:11.crao1}, \emph{se e
  solo se} esiste una funzione $R(\theta)$ per la quale
risulti
\begin{equation} \label{eq:11.crao2}
  \frac{\partial (\ln \mathcal{L}) }{\partial \theta} \; = \;
    \sum_{i=1}^N \frac{\partial}{\partial
    \theta} \ln f(x_i; \theta) \; = \; \frac{\bar
    \theta(x_1, x_2, \ldots, x_N) - \theta}{R(\theta)}
\end{equation}
e, in tal caso, la stima di minima varianza \emph{rende
anche massima la funzione di verosimiglianza}.

La condizione \eqref{eq:11.crao2} \`e assai restrittiva,
potendosi tra l'altro dimostrare che essa implica che la
densit\`a di probabilit\`a $f(x; \theta)$ deve essere una
funzione di tipo esponenziale: nel caso generale non \`e
quindi affatto certo che una stima di varianza minima
esista, essendo questo subordinato alla validit\`a della
\eqref{eq:11.crao2}.

In ogni caso la stima di massima verosimiglianza deve, come
prima detto, tendere \emph{asintoticamente} a questo
comportamento al crescere di $N$; per\`o nulla si pu\`o dire
sulla rapidit\`a di tale convergenza.  Cos\`\i, per un
numero di misure finito, non c'\`e alcuna garanzia che la
funzione di verosimiglianza abbia un solo massimo; e, se
essa ne ammette pi\`u d'uno, non esiste modo di sapere quale
di essi corrisponde (asintoticamente) alla stima di minima
varianza, n\'e esiste modo di sapere quale di questi massimi
rappresenti la stima corretta del valore vero.%
\index{Cram\'er--Rao, teorema di|)}

Come abbiamo detto, la funzione di verosimiglianza
\eqref{eq:11.funver} pu\`o essere interpretata come
densit\`a di probabilit\`a del parametro una volta che si
sia ottenuto un certo insieme di valori misurati; sfruttando
la seconda delle propriet\`a su elencate, la densit\`a di
probabilit\`a di $\theta$ deve anche essere
(asintoticamente) data da
\begin{gather}
  \mathcal{L} (\theta) = \frac{1}{\sigma_\theta \, \sqrt{2
    \pi}} \, e^{- \frac{1}{2} \! \left(
    \frac{\theta - \widehat \theta}{\sigma_\theta} \right)^2}
    \notag \\
  \intertext{quindi, nell'intorno di $\widehat \theta$, deve
    essere}
  \ln \mathcal{L} = -\ln \left( \sigma_\theta \, \sqrt{2
    \pi} \right) - \frac{1}{2} \left( \frac{\theta - \widehat
    \theta}{\sigma_\theta} \right)^2 \notag \\
  \intertext{e, derivando due volte rispetto al parametro,}
  \frac{\de^2 ( \ln \mathcal{L} ) }{\de \theta^2} = -
    \frac{1}{{\sigma_\theta}^2} \notag \\
  \intertext{ed infine si giunge alla}
  \var( \widehat \theta ) \; \equiv \; {\sigma_\theta}^2
    \; = \; - \frac{1}{\displaystyle \left. \frac{\de^2 (
    \ln \mathcal{L} ) }{\de \theta^2} \right|_{\theta = \widehat
    \theta}} \label{eq:11.varlik}
\end{gather}
frequentemente usata per il calcolo dell'errore della stima
di massima verosimiglianza.

\subsection{Un esempio di stima sufficiente}%
\index{stima!sufficiente|(}%
\label{ch:11.stisuf}
Supponiamo di avere un campione di $N$ determinazioni
indipendenti $x_k$ di una variabile che segua la
distribuzione di Poisson; le probabilit\`a dei differenti
valori sono date dalla \eqref{eq:8.poiss2}, e dipendono da
un unico parametro: il valore medio della distribuzione,
$\alpha$.  La funzione di verosimiglianza \`e la
\begin{align}
  \Pr(x_1, \ldots, x_N; \alpha) &=
  \frac{ \alpha^{x_1} }{ x_1 ! } \, e^{- \alpha} \cdot
  \frac{ \alpha^{x_2} }{ x_2 ! } \, e^{- \alpha} \cdots
  \frac{ \alpha^{x_N} }{ x_N ! } \, e^{- \alpha} \notag
  \\[1ex]
  &= \frac{ \alpha^{\sum_k x_k} \cdot e^{- N \alpha} }{ x_1!
    \, x_2! \cdots x_N!} \cdot \frac{ (N \bar x)! }{ (N \bar
    x)! } \notag \\[1ex]
  &= \frac{ \alpha^{ N \bar x } }{ ( N \bar x )! } \, e^{- N
    \alpha} \cdot \frac{ ( N \bar x )! }{ x_1! \, x_2!
    \cdots x_N! } \cdot \frac{ N^{N \bar x} }{ N^{N \bar x}
    } \notag \\[1ex]
  &= \left\{ \frac{ (N \alpha)^{N \bar x} }{ (N \bar x)! }
    \, e^{-N \alpha} \right\} \Biggl\{ \frac{ (N \bar x)! }{
      x_1! \, x_2! \cdots x_N! } \, \frac{ 1 }{ N^{N \bar x}
      } \Biggr\} \label{eq:11.stisuf}
\end{align}
Nei passaggi, per due volte si \`e moltiplicato e diviso per
una stessa quantit\`a non nulla: prima per $(N \bar x)!$ e
poi per $N^{N \bar x}$.

La stima di massima verosimiglianza si trova annullando la
derivata della \eqref{eq:11.stisuf}; che, a meno di un
fattore costante, \`e della forma
\begin{gather*}
  f(\alpha) = \alpha^{N \bar x} e^{- N \alpha} \\
  \intertext{per cui}
  \frac{\de f}{\de \alpha} \; = \; N \, \bar x \,
  \alpha^{N \bar x - 1} e^{- N \alpha} - N \, \alpha^{N \bar
    x} e^{- N \alpha} \; = \; N \, \alpha^{N \bar x - 1}
  e^{- N \alpha} (\bar x - \alpha) \\
  \intertext{e quindi la stima cercata \`e}
  \hat \alpha = \bar x
\end{gather*}

Il primo termine dell'espressione finale
\eqref{eq:11.stisuf} per la funzione di verosimiglianza \`e
la probabilit\`a $\Pr(S)$ che la variabile casuale
\begin{equation*}
  S \; = \; \sum_{k=1}^N x_k \; = \; N \bar x
\end{equation*}
abbia un determinato valore: $\Pr(S)$ infatti, come gi\`a
sappiamo dal paragrafo \ref{ch:8.poisson}, segue la
distribuzione di Poisson con valore medio $N \alpha$.
Notiamo anche che, avendo $N$ un valore costante noto a
priori, $\Pr(S)$ coincide con $\Pr(\bar x)$: il secondo
termine \emph{deve} quindi essere la probabilit\`a che i
dati osservati valgano $x_1, x_2, \ldots, x_N$
\emph{condizionata} dal fatto che la loro somma vale $N \bar
x$; ma, non dipendendo questo termine da $\alpha$, tale
probabilit\`a \`e la stessa qualunque sia il parametro.

Qualunque sia $\bar x$, una volta noto il suo valore le
$x_k$ sono distribuite allo stesso modo: $\bar x$ riassume
insomma tutta l'informazione contenuta nei dati, ed \`e
quindi per definizione una stima \emph{sufficiente} del
parametro.  In effetti, se la probabilit\`a dei valori $x_k$
una volta nota $\bar x$ non dipende dal parametro, questo
implica che \emph{qualunque} funzione dei dati ha
probabilit\`a (condizionata) che gode della stessa
propriet\`a.  Citiamo senza dimostrarlo, in proposito, il
seguente
\begin{quote}
  \textsc{Teorema:} \textsl{$\bar \theta$ \`e una stima
    sufficiente di $\theta$ se e solo se la funzione di
    verosimiglianza \`e fattorizzabile nella forma}
  \begin{equation*}
    \mathcal{L} (x_1, x_2, \ldots, x_N; \theta) = f(\bar
    \theta, \theta) \cdot \phi(x_1, x_2, \ldots, x_N)
  \end{equation*}
\end{quote}%
\index{stima!sufficiente|)}

\section{Media pesata}%
\label{ch:11.mepeted}
Quando si abbiano a disposizione pi\`u determinazioni
ripetute di una stessa grandezza fisica, sappiamo che da
esse si pu\`o ricavare un valore unico da usare come
risultato finale attraverso il calcolo della media
aritmetica; questa (come gi\`a anticipato senza
dimostrazione nel paragrafo \ref{ch:4.giumed}) \`e la
funzione dei dati con la distribuzione pi\`u stretta attorno
al valore vero, e ci fornisce quindi la stima pi\`u
verosimile di esso.  Per\`o questo presuppone che i dati,
essendo considerati tutti allo stesso modo nella formula,
posseggano la stessa precisione sperimentale: ad esempio che
siano stati valutati dallo stesso sperimentatore, con lo
stesso strumento e nelle stesse condizioni; in altre parole,
che le misure provengano da un'unica popolazione.

Pu\`o capitare invece di disporre di pi\`u determinazioni
della stessa grandezza fisica fatte da sperimentatori
diversi, od in condizioni sperimentali differenti: e di
voler ugualmente estrarre da queste valutazioni, affette da
differenti errori, un valore unico da usare come risultato
complessivo.

Facendo le ipotesi che tutte le misure $x_i$ siano tra loro
statisticamente indipendenti, ed inoltre affette da errori
casuali distribuiti secondo la legge di Gauss, la densit\`a
di probabilit\`a corrispondente all'evento casuale
costituito dall'osservazione degli $N$ valori $x_1,
x_2,\ldots, x_N$ si pu\`o scrivere (applicando il teorema
della probabilit\`a composta)
\begin{equation*}
  \prod_{i=1}^N \frac{1}{\sigma_i \sqrt{2 \pi}}
    \, e^{- \frac{1}{2} \left( \frac{x^*
    - x_i}{\sigma_i} \right) ^2}
\end{equation*}
dove $x^*$ \`e il valore vero (ignoto) di $x$, e le
$\sigma_i$ sono gli errori quadratici medi (supposti noti)
delle diverse determinazioni.

La funzione di verosimiglianza \`e la
\begin{equation*}
  \mathcal{L} (x_1, x_2,\ldots, x_N ; x) \; = \;
    \prod_{i=1}^N \frac{1}{\sigma_i \sqrt{2 \pi}}
    \, e^{- \frac{1}{2} \left( \frac{x -
    x_i}{\sigma_i} \right) ^2}
\end{equation*}
(cio\`e la densit\`a di probabilit\`a di cui sopra, nella
quale il valore vero $x^*$ \`e sostituito dal parametro
variabile $x$); e ricordiamo che essa rappresenta la
densit\`a di probabilit\`a associata all'evento casuale
consistente nell'essere il numero $x$ il valore vero della
grandezza misurata, qualora di essa si siano ottenute le $N$
stime indipendenti $x_i$, di errori rispettivi $\sigma_i$,
supposte seguire la legge normale.

La stima pi\`u verosimile \`e quella che, rendendo massima
$\mathcal{L}$, individua quel numero che, sulla base delle
osservazioni disponibili, possiede la \emph{massima
  probabilit\`a} di coincidere con il valore vero: vedremo
tra poco che la soluzione \`e unica.  Prendendo il logaritmo
naturale di $\mathcal{L}$,
\begin{equation*}
  - 2 \, \ln\mathcal{L} \; = \;
  \sum_{i=1}^N \left( \frac{x - x_i}{\sigma_i}
  \right) ^2 \; + \; 2 \sum_{i=1}^N \ln \sigma_i
  \; + \; 2 N \ln \sqrt{2 \pi}
\end{equation*}
e ricordando, come prima detto, che il logaritmo naturale
\`e una funzione monotona strettamente crescente
dell'argomento, si vede che il massimo di $\mathcal{L}$
corrisponde al minimo di $-2 \, \ln \mathcal{L}$; la
determinazione del valore pi\`u verosimile di $x$ (nel caso
di errori normali) si riduce allora al problema analitico di
trovare il minimo della funzione
\begin{equation*}
  f(x) \; = \; \sum_{i=1}^{N} \left(
    \frac{ x - x_{i} }{ \sigma_{i} } \right) ^{2}
\end{equation*}
(infatti nessuno degli altri termini dipende dall'incognita
$x$).  Risolviamo il problema facendo uso del calcolo
infinitesimale:
\begin{gather*}
  \frac{\de f}{\de x} \; = \; 2 \sum_{i=1}^{N}
    \left( \frac{x - x_{i}}{ \sigma_{i} } \right)
    \frac{1}{\sigma_{i}} \; = \; 2 \left( x
    \sum_{i=1}^{N} \frac{1}{ {\sigma_i}^2 }
    \; - \; \sum_{i=1}^{N} \frac{x_i}{{\sigma_i}^2}
    \right) \peq ; \\[1ex]
  \frac{\de^2 f}{\de x^2} \; = \; 2 \sum_{i=1}^{N}
    \frac{1}{ {\sigma_i}^2 } \; > \; 0 \peq .
\end{gather*}

Se per brevit\`a poniamo
\begin{gather}
  K \; = \; \sum_{i=1}^{N} \frac{1}{{\sigma_i}^2}
    \notag \\
  \intertext{la condizione per l'estremante di
    $f(x)$ si scrive}
  \frac{\de f}{\de x} \; = \; 2 \left( K x -
    \sum_{i=1}^{N} \frac{x_i}{{\sigma_i}^2} \right)
    \; = \; 0 \notag \\
  \intertext{e la derivata prima di $f$ si annulla
    quando la variabile $x$ assume il valore}%
  \index{media!pesata|(emidx}
  \bar x \; = \; \frac{1}{K} \sum_{i=1}^{N}
   \frac{x_i}{{\sigma_i}^2} \peq . \label{eq:11.medpes}
\end{gather}

Il fatto che la derivata seconda sia positiva assicura poi
che si tratta effettivamente di un punto di minimo; si vede
come $\bar x$ sia una \emph{media pesata} dei valori
misurati $x_i$, ottenuta assegnando ad ognuno di essi peso
relativo inversamente proporzionale al \emph{quadrato}
dell'errore rispettivo.

Per determinare poi l'errore del risultato $ \bar x $, \`e
in questo caso possibile usare in tutta generalit\`a la
formula della propagazione degli errori: infatti $ \bar x $
\`e una particolare funzione delle variabili $x_i$, di
ognuna delle quali conosciamo per ipotesi l'errore
quadratico medio $\sigma_i$; ed inoltre dipende
\emph{linearmente} da ognuna di queste $N$ variabili, e
questo fa s\`\i\ che la formula di propagazione
\eqref{eq:10.proper} sia in questo caso \emph{esatta} e non
approssimata (dando insomma risultati sempre validi,
indipendentemente dall'entit\`a degli errori commessi).

Applichiamo direttamente l'equazione \eqref{eq:5.varcol} per
la varianza delle combinazioni lineari di variabili tra loro
indipendenti, invece della pi\`u complicata
\eqref{eq:10.proper}: $\bar x$ \`e calcolata come
combinazione lineare delle $x_i$ con coefficienti $1/\left(
  K \, {\sigma_i}^2 \right)$, e quindi avremo
\begin{gather}
  {\sigma_{\bar x}}^2 \; = \; \sum_{i=1}^N
    \left( \frac{1}{ K \, {\sigma_i}^2 }
    \right)^2 {\sigma_i}^2\; = \;
    \frac{1}{K^2} \sum_{i=1}^N \frac{1}
   {{\sigma_i}^2} \; = \; \frac{1}{K} \notag \\
  \intertext{cio\`e}
  {\sigma_{\bar x}}^2 = \frac{1} {\sum
    \limits_{i=1}^N \dfrac{1}{{\sigma_i}^2} }
    \peq . \label{eq:11.ermear}
\end{gather}

Per la osservata linearit\`a della formula, la media pesata
$\bar x$ (nelle ipotesi ammesse) \`e una variabile casuale
normale come le singole $x_i$; ed il suo errore quadratico
medio $\sigma_{\bar x}$ ha dunque l'analoga interpretazione
di semiampiezza dell'intervallo con centro in $\bar x$
avente probabilit\`a pari al 68\% di contenere il valore
vero $x^*$.

Per quanto concerne le propriet\`a della media pesata $\bar
x$ come stima del valore vero, la derivata del logaritmo
della funzione di verosimiglianza rispetto al parametro
incognito (che \`e $x$) vale
\begin{equation*}
  \frac{\de (\ln \mathcal{L}) }{\de x} \; = \; \sum_{i=1}^N
  \frac{x_i}{{\sigma_i}^2} - x \sum_{i=1}^N
  \frac{1}{{\sigma_i}^2} \; = \; K ( \bar x - x )
\end{equation*}
ed \`e soddisfatta la condizione \eqref{eq:11.crao2} sotto
la quale il teorema di Cram\'er--Rao%
\index{Cram\'er--Rao, teorema di}
(che esamineremo in dettaglio nell'appendice
\ref{ch:e.maxlik}) ci permette di affermare che la stima di
massima verosimiglianza \emph{\`e anche quella di varianza
  minima}: ovvero, tra tutte le possibili funzioni dei dati
che si potrebbero definire per stimare il valore vero $x^*$
dal campione, quella mediamente pi\`u vicina ad esso.%
\index{media!pesata|)}

\index{esame dei dati|(}%
\`E da notare come, prima di comporre tra loro
determinazioni indipendenti della stessa grandezza, sia
opportuno controllare che queste siano (entro i rispettivi
errori) tra loro \emph{compatibili}; analogamente a quanto
si fa per le misure ripetute, \`e preferibile non
considerare dati che non vadano d'accordo con gli altri
entro i limiti della pura casualit\`a.%
\index{esame dei dati|)}

\index{media!aritmetica!come stima del valore vero|(emidx}%
Il caso di $N$ misure ripetute effettuate nelle medesime
condizioni sperimentali non \`e altro che il caso
particolare in cui tutti gli errori quadratici medi
$\sigma_i$ sono uguali tra di loro: la media pesata
\eqref{eq:11.medpes} si riduce allora alla media aritmetica
\eqref{eq:4.mediar} (ed il suo errore \eqref{eq:11.ermear}
alla gi\`a nota espressione \eqref{eq:5.sbarx}).

Questo prova l'asserto del paragrafo \ref{ch:4.giumed}
(giustificazione della media); abbiamo finalmente
\emph{dimostrato} che la media aritmetica \`e il valore
\emph{pi\`u verosimile} della grandezza misurata: cio\`e
quello che ha la massima probabilit\`a di coincidere con il
valore vero sulla base del nostro campione di misure, e che
rappresenta la stima di minima varianza.%
\index{media!aritmetica!come stima del valore vero|)}

\section{Interpolazione dei dati con una curva}
Pu\`o in alcuni casi capitare di conoscere la forma
analitica della legge fisica che mette in relazione tra loro
due variabili, e di dover stimare dai dati misurati il
valore di uno o pi\`u parametri da cui tale funzione
dipende.

Ad esempio, nel moto dei corpi soggetti all'azione di una
forza costante le velocit\`a assunte in istanti successivi
dal corpo crescono linearmente rispetto ai tempi trascorsi,
secondo la nota formula $ v = v_0 + a t $; misurando in
istanti successivi del moto tempi e velocit\`a, i punti
aventi per coordinate cartesiane i valori determinati per
queste due grandezze devono disporsi approssimativamente
lungo una linea retta: e sarebbero tutti quanti esattamente
allineati se fosse possibile misurare senza commettere
errori.

In questo ultimo caso sarebbe possibile ricavare
immediatamente dal grafico il valore dell'accelerazione del
moto, che corrisponderebbe al coefficiente angolare (o
pendenza) della retta tracciata; vedremo ora come, pur
commettendo errori, sia comunque possibile ricavare una
stima sia dei valori dei parametri da cui l'equazione del
moto dipende, sia degli errori inerenti a tale valutazione.

C'\`e una qualche analogia fra questo problema e quello
delle misure indirette, nel senso che in entrambi i casi si
presuppone esistente una relazione funzionale tra pi\`u
grandezze fisiche; tuttavia, mentre in quel caso la funzione
era completamente nota e veniva usata per trovare il valore
di una di quelle grandezze una volta misurati quelli di
tutte le altre, qui si suppone di conoscere soltanto la
forma della funzione: ma sono ignoti uno o pi\`u parametri
da cui pure essa dipende, e si usano i valori osservati di
tutte le grandezze per stimare quelli dei parametri stessi.

\subsection{Interpolazione lineare per due variabili}%
\index{interpolazione lineare|(emidx}%
\label{ch:11.intlin}
Cominciamo col supporre che le variabili oggetto della
misura siano due sole, e che la legge che le mette in
relazione reciproca sia di tipo lineare:
\begin{equation*}
  y = a+bx \peq .
\end{equation*}

Supponiamo poi che siano state effettuate misure del valore
della $x$ e di quello corrispondente assunto dalla $y$ in
diverse condizioni, cos\`\i\ che si disponga in definitiva
di $N$ coppie di valori tra loro corrispondenti $(x_i,
y_i)$; abbiamo gi\`a detto che, una volta riportati sul
piano cartesiano $\{ x, y \}$ punti con queste coordinate,
essi si dovranno disporre approssimativamente lungo una
linea retta.

Ora, si pu\`o dimostrare che vale, sul piano, qualcosa di
analogo a quanto abbiamo gi\`a asserito riguardo alla media
aritmetica di misure ripetute di una stessa grandezza fisica
(cio\`e, geometricamente, su di una retta, visto che quelle
determinazioni potevano essere univocamente rappresentate da
punti su di una retta orientata); infatti
\begin{itemize}
\item Sulla base delle misure effettuate, non si pu\`o
  escludere con certezza che alcuna delle infinite rette del
  piano corrisponda a quella vera su cui le nostre
  osservazioni si disporrebbero in assenza di errori;
  tuttavia esse non appaiono tutte quante ugualmente
  verosimili, e la verosimiglianza sar\`a in qualche modo in
  relazione con la \emph{distanza complessiva} tra i nostri
  punti sperimentali e la retta stessa.
\item Nel caso particolare che siano verificate le seguenti
  ipotesi:
    \begin{enumerate}
    \item una sola delle variabili coinvolte (ad esempio la
      $y$) \`e affetta da errori;
    \item gli errori quadratici medi delle misure dei
      diversi valori di $y$ sono tutti uguali (o comunque
      non molto differenti);
    \item questi errori seguono la legge normale di
      distribuzione;
    \item le $N$ determinazioni effettuate sono tra loro
      statisticamente indipendenti;
    \end{enumerate}
    dimostreremo ora che per ``distanza complessiva'' si
    deve intendere \emph{la somma dei quadrati delle
      lunghezze dei segmenti di retta parallela all'asse $y$
      compresi tra i punti misurati e la retta esaminata}.
\end{itemize}

Infatti, detto $\sigma_y$ l'errore quadratico medio delle
$y_i$, la funzione di verosimiglianza \`e
\begin{equation*}%
\index{funzione!di verosimiglianza|(}
  \mathcal{L} (x_1, y_1, x_2, y_2,\ldots,x_N, y_N ;
    a, b ) \; = \; \prod_{i=1}^N \frac{1}{\sigma_y
    \sqrt{2 \pi}} \,  e^{- \frac{1}{2}
    \left( \frac{a + b x_i - y_i}{\sigma_y} \right)
    ^2 } \peq .
\end{equation*}

Per scrivere questa espressione si \`e fatto uso di tutte le
ipotesi postulate: in particolare, il fatto che le $x_i$
siano misurate senza errore ci permette di affermare che il
valore vero assunto in corrispondenza dalla $y$ \`e $ a + b
x_i$; visto che \`e $y = a + bx$ la legge fisica che lega le
due variabili tra loro.

Questa funzione di verosimiglianza rappresenta allora la
densit\`a di probabilit\`a collegata all'evento casuale
consistente nell'essere la legge fisica che lega $x$ ad $y$
rappresentata dall'equazione $y = a + bx$, qualora si siano
ottenuti gli $N$ valori misurati $(x_i, y_i)$, e sotto le
quattro ipotesi su elencate.%
\index{funzione!di verosimiglianza|)}

I valori pi\`u verosimili del parametro saranno quelli che
rendono massima $\mathcal{L}$: vedremo ora che la soluzione
\`e unica; e, ancora, il teorema di Cram\'er--Rao%
\index{Cram\'er--Rao, teorema di}
ci permetterebbe di dimostrare che la stima, appunto, pi\`u
verosimile (la retta che corrisponde al massimo della
probabilit\`a) \`e anche la stima di minima varianza (ovvero
la pi\`u precisa possibile).  Prendendo il logaritmo
naturale di entrambi i membri, risulta
\begin{equation*}
  - 2 \, \ln\mathcal{L} \; = \;
  \frac{1}{{\sigma_y}^2} \,
  \sum_{i=1}^N \left( a + b x_i - y_i \right) ^2
  \; + \; 2 N \ln \sigma_y
  \; + \; 2 N \ln \sqrt{2 \pi} \peq .
\end{equation*}

I valori pi\`u verosimili dei parametri $a$ e $b$ sono
quelli per cui \`e massima $\mathcal{L}$, ovvero \`e minima
$- 2 \ln\mathcal{L} $: il problema dell'interpolazione
lineare dunque si riduce (se sono soddisfatte le ipotesi
citate) a quello di trovare tra le infinite rette del piano
quella che rende minima la funzione
\begin{equation*}
  f(a,b) \; =  \;\sum_{i=1}^N \Bigl[ \left( a + b x_i
    \right) - y_i \Bigr] ^{2}
\end{equation*}
(essendo tutti gli altri termini indipendenti dalle due
incognite $a$ e $b$).

L'interpretazione geometrica \`e evidente: la retta
soluzione del nostro problema \`e (come gi\`a preannunciato)
quella che rende minima la somma dei quadrati delle
distanze, misurate per\`o \emph{parallelamente all'asse
  $y$}, dall'insieme dei punti misurati; queste ``distanze''
sono anche comunemente chiamate ``residui''.%
\index{residui|emidx}
Per trovare il valore dei coefficienti dell'equazione di
tale retta, calcoliamo ora le derivate prime della funzione
$f$:
\begin{gather*}
  \frac{\partial f}{\partial a} \; = \;
    2 \sum_{i=1}^N \left( a + b x_i - y_i \right)
    \; = \; 2 \left( N a \; + \;
    b \sum_{i=1}^N x_i \; - \;
    \sum_{i=1}^N y_i \right) \peq ; \\[1ex]
  \frac{\partial f}{\partial b} \; = \;
     2 \sum_{i=1}^N \left( a + b x_i - y_i \right) x_i
     \; = \; 2 \left( a \sum_{i=1}^N x_i \; + \;
     b \sum_{i=1}^N {x_i}^2 \; - \;
     \sum_{i=1}^N x_i y_i \right) \peq .
\end{gather*}

Imponendo che le due derivate prime siano contemporaneamente
nulle, dovranno essere verificate le
\begin{equation} \label{eq:11.normeq}
  \left \{ \begin{array}{ccccl}
    a \cdot N & + & b \cdot \sum_i x_i &
      = & \sum_i y_i \\*[5mm]
    a \cdot \sum_i x_i & + & b \cdot \sum_i {x_i}^2
      & = & \sum_i x_i y_i
    \end{array} \right.
\end{equation}
e questo sistema di due equazioni in due incognite ammette,
come si pu\`o verificare, sempre una ed una sola soluzione,
purch\'e vi siano almeno due punti sperimentali non
coincidenti; esaminando poi le derivate seconde si
troverebbe che essa corrisponde in effetti ad un minimo.  La
soluzione \`e
\begin{equation}%
  \label{eq:11.minqua}%
  \index{minimi quadrati, formule dei|emidx}
  \left \{
    \begin{array}{ccl}
      a & = & \dfrac{1}{\Delta}
        \Bigl[ \left( \sum_{i} {x_i}^2 \right) \cdot
        \left( \sum_i y_i \right) -
        \left( \sum_i x_i \right) \cdot
        \left( \sum_i x_i y_i \right) \Bigr] \\*[7mm]
      b & = & \dfrac{1}{\Delta}
        \Bigl[ N \cdot \left( \sum_i x_i y_i \right) -
        \left( \sum_i x_i \right) \cdot
        \left( \sum_i y_i \right)
        \Bigr]
    \end{array}
  \right.
\end{equation}
in cui si \`e posto per brevit\`a
\begin{equation*}
  \Delta \; = \; N \sum \nolimits_i {x_i}^2 \: - \:
  \left( \sum \nolimits_i x_i \right) ^2
\end{equation*}
(le formule \eqref{eq:11.minqua} sono note sotto il nome di
\emph{formule dei minimi quadrati}).

Per quanto attiene al calcolo degli errori commessi nella
valutazione di $a$ e $b$ in base ai dati, osserviamo che
entrambi si ricavano da relazioni lineari in ognuna delle
variabili affette da errore che, nelle nostre ipotesi, sono
le sole $ y_i$: possiamo dunque adoperare la formula della
propagazione degli errori \eqref{eq:10.proper}, che \`e in
questo caso esatta; oppure la pi\`u semplice
\eqref{eq:5.varcol}.  Possiamo esprimere $a$ e $b$ in
funzione delle $y_i$ come
\begin{align*}
  a &= \sum_{i=1}^N a_i \, y_i &&\text{e} &
  b &= \sum_{i=1}^N b_i \, y_i
\end{align*}
una volta posto
\begin{equation*}
  \left\{
    \begin{array}{rcl}
      a_i & = & \displaystyle \frac{1}{\Delta} \,
        \left( \sum\nolimits_j {x_j}^2 - x_i \,
        \sum\nolimits_j x_j \right) \\[3ex]
      b_i & = & \displaystyle \frac{1}{\Delta} \left(
        N \, x_i - \sum\nolimits_j x_j \right)
    \end{array}
  \right.
\end{equation*}
e, se indichiamo con ${\sigma_y}^2$ la varianza comune a
tutte le $y_i$, si ottiene, per l'errore di $a$:

\begin{align*}
  {\sigma_a}^2 &= \sum\nolimits_i {a_i}^2 \,
    {\sigma_y}^2 \\[2ex]
  &= {\sigma_y}^2 \; \sum\nolimits_i \left[
    \frac{1}{\Delta} \left( \sum\nolimits_j {x_j}^2 -
    x_i \sum\nolimits_j x_j \right) \right] ^2 \\[2ex]
  &= \frac{{\sigma_y}^2}{\Delta^2} \; \sum\nolimits_i
    \left[ \left( \sum\nolimits_j {x_j}^2 \right) ^2 +
    {x_i}^2 \left( \sum\nolimits_j x_j \right) ^2 - 2
    \, x_i \left( \sum\nolimits_j {x_j}^2 \right)
    \left( \sum\nolimits_j x_j \right) \right]
    \\[2ex]
  &= \frac{{\sigma_y}^2}{\Delta^2} \left[ N \left(
    \sum\nolimits_j {x_j}^2 \right) ^2 \! + \left(
    \sum\nolimits_i {x_i}^2 \right) \left( \sum\nolimits_j
    x_j \right) ^2 \! - 2 \left( \sum\nolimits_j
    {x_j}^2 \right) \left( \sum\nolimits_j x_j \right) ^2
    \right] \\[2ex]
  &= \frac{{\sigma_y}^2}{\Delta^2} \left[ N \left(
    \sum\nolimits_j {x_j}^2 \right) ^2 - \left(
    \sum\nolimits_j x_j \right) ^2 \left( \sum\nolimits_j
    {x_j}^2 \right) \right] \\[2ex]
  &= \frac{{\sigma_y}^2}{\Delta^2} \left(
    \sum\nolimits_j {x_j}^2 \right) \left[ N \left(
    \sum\nolimits_j {x_j}^2 \right) - \left(
    \sum\nolimits_j x_j \right) ^2 \right] \\[2ex]
  &= {\sigma_y}^2 \; \frac{\sum_j {x_j}^2 }{\Delta} \\
\end{align*}
e, similmente, per $b$:

\begin{align*}
  {\sigma_b}^2 &= \sum\nolimits_i {b_i}^2 \,
    {\sigma_y}^2 \\[1ex]
  &= {\sigma_y}^2 \; \sum\nolimits_i \left[
    \frac{1}{\Delta} \left( N \, x_i - \sum\nolimits_j
    x_j \right) \right] ^2 \\[1ex]
  &= \frac{{\sigma_y}^2}{\Delta^2} \; \sum\nolimits_i \left[
    N^2 {x_i}^2 + \left( \sum\nolimits_j x_j \right) ^2
    - 2 \, N \, x_i \sum\nolimits_j x_j \right] \\[1ex]
  &= \frac{{\sigma_y}^2}{\Delta^2} \left[ N^2 \left(
    \sum\nolimits_i {x_i}^2 \right) + N \left(
    \sum\nolimits_j x_j \right) ^2 - 2 \, N \left(
    \sum\nolimits_j x_j \right) ^2 \right] \\[1ex]
  &= \frac{N \, {\sigma_y}^2}{\Delta^2} \left[ N
    \left( \sum\nolimits_i {x_i}^2 \right) - \left(
    \sum\nolimits_j x_j \right) ^2 \right] \\[1ex]
  &= {\sigma_y}^2 \; \frac{N}{\Delta} \peq .
\end{align*}

In definitiva, $a$ e $b$ hanno errori quadratici medi dati
dalle
\begin{equation} \label{eq:11.errab}
  \left \{ \begin{array}{ccl}
    \sigma_{a} & = & \sigma_{y} \, \sqrt{
      \dfrac{\sum \nolimits_i x_i^2}
      {\Delta} } \\*
   & & \\*
   \sigma_{b} & = & \sigma_{y} \, \sqrt{
     \dfrac{N}{\Delta} }
   \end{array} \right.
\end{equation}
ed il fatto poi che $a$ e $b$ siano funzioni lineari di
variabili che seguono la legge di Gauss ci permette ancora
di affermare che anch'esse sono distribuite secondo la legge
normale; e di attribuire cos\`\i\ ai loro errori il consueto
significato statistico.%
\index{interpolazione lineare|)}

\subsection{Stima a posteriori degli errori di misura}%
\index{errore!a posteriori|(emidx}%
\label{ch:11.fisher}
\`E da osservare come nelle formule \eqref{eq:11.minqua} dei
minimi quadrati non compaia il valore di $\sigma_y$: la
soluzione del problema dell'interpolazione lineare \`e
indipendente dall'entit\`a degli errori di misura, nel senso
che i coefficienti della retta interpolante possono essere
calcolati anche se gli errori sulle $y$ non sono noti
(purch\'e naturalmente si assuma che siano tutti uguali tra
loro).

Se non \`e a priori nota la varianza delle $y$, essa pu\`o
per\`o essere stimata a partire dai dati stessi una volta
eseguita l'interpolazione lineare; infatti gli stessi
ragionamenti fatti per le variabili casuali unidimensionali
potrebbero essere ripetuti (con le opportune modifiche) sul
piano, per giungere a risultati analoghi.

In una dimensione abbiamo a suo tempo potuto collegare
l'errore commesso alla dispersione dei dati rispetto al
\emph{valore stimato} della grandezza misurata; sul piano
\`e in effetti ancora possibile calcolare l'errore commesso,
partendo dalla dispersione dei dati misurata rispetto alla
\emph{retta stimata} che passa attraverso di essi: dati
disposti mediamente lontano da questa retta indicheranno
errori maggiori rispetto a dati ben allineati (e quindi
vicini alla retta interpolante).

In una dimensione abbiamo visto che la dispersione dei dati,
misurata dal valore medio del quadrato degli scarti rispetto
alla loro media aritmetica (nostra migliore stima per la
grandezza misurata), era sistematicamente in difetto
rispetto alla corrispondente grandezza riferita all'intera
popolazione delle misure.  Sul piano si pu\`o, analogamente,
dimostrare che il valore medio del quadrato delle distanze
dei punti misurati dalla retta nostra migliore stima \`e
ancora sistematicamente in difetto rispetto alla varianza
riferita alla popolazione delle misure ed alla retta vera
che corrisponde alla legge fisica reale che collega le due
variabili.

Cos\`\i\ come abbiamo dimostrato che, al fine di correggere
questa sottostima (in media) per le misure ripetute, occorre
dividere la somma dei quadrati degli scarti per $N-1$ invece
che per $N$, si potrebbe analogamente dimostrare che una
corretta stima dell'errore dei punti misurati si ha, in
media, dividendo l'analoga somma per $N-2$; in definitiva,
che la corretta stima di $\sigma_y$ \`e data dalla formula
\begin{equation*}
  {\sigma_y}^2 = \frac{\sum\limits_{i=1}^N \Bigl[
    \left( a + b x_i \right) - y_i \Bigr]^2}{N - 2} \peq .
\end{equation*}

In essa a numeratore compare la somma dei quadrati dei
residui, cio\`e delle ``distanze'' dei punti misurati $(x_i,
y_i)$ dalla retta interpolante di equazione $a+bx$ calcolate
secondo la direzione parallela all'asse delle ordinate.
Questa formula\/\footnote{Una formula equivalente (ma pi\`u
  semplice) per il calcolo di $\sigma_y$ si pu\`o trovare
  nell'equazione \eqref{eq:c.fishalt} alla pagina
  \pageref{eq:c.fishalt}.} permette una corretta stima
dell'errore dei dati interpolati, qualora sia impossibile (o
scomodo) determinarli per altra via; l'errore \`e stimato
dai residui dei dati sperimentali, ed \`e quindi
scientificamente affidabile.

Il fatto che la corretta stima dell'errore si ottenga
dividendo per $N-2$ invece che per $N$ deve essere messo in
relazione con il fatto che gli scarti, invece che rispetto
al valore vero, sono calcolati rispetto ad un valore stimato
che dipende da \emph{due} parametri, che sono a loro volta
stati preventivamente determinati sulla base dei dati
sperimentali: cio\`e i due coefficienti $a$ e $b$
dell'equazione della retta.

Nell'analogo caso della stima dell'errore quadratico medio
di una variabile casuale unidimensionale, gli scarti erano
calcolati rispetto ad un valore che, unica grandezza
necessaria, veniva preventivamente determinato sulla base
delle misure: appunto la media aritmetica.

In generale, disponendo di $N$ dati sperimentali dai quali
possiamo determinare un valore dell'errore quadratico medio
che dipende da $M$ parametri che debbano essere
preventivamente derivati dai dati stessi, la modifica da
apportare alla formula per ottenere una corretta valutazione
dell'errore della popolazione consiste nel dividere la somma
dei quadrati degli scarti per un fattore $N-M$.%
\index{errore!a posteriori|)}

\subsection{Interpolazione con una retta per l'origine}%
\index{interpolazione lineare!con una retta per l'origine|(}
Se conosciamo altri vincoli cui debba soddisfare la legge
che mette in relazione i valori delle variabili misurate $x$
e $y$, possiamo imporre che la retta corrispondente
appartenga ad un particolare sottoinsieme delle rette del
piano; ad esempio, un caso che si pu\`o presentare \`e che
la retta sia vincolata a passare per una posizione
particolare, che supporremo qui essere l'origine degli assi
coordinati.

Una generica retta per l'origine ha equazione $y=mx$;
ammesso ancora che gli errori commessi riguardino soltanto
la misura della $y$ e non quella della $x$, che tutti i vari
$y_i$ abbiano errori distribuiti secondo la legge normale e
tra loro uguali, e che le misure siano tra loro
indipendenti, il problema dell'interpolazione lineare si
riduce a trovare tra le infinite rette passanti per
l'origine quella che rende massima la funzione di
verosimiglianza
\begin{equation*}
  \mathcal{L} (x_1, y_1, x_2, y_2,\ldots,x_N, y_N ;
    m) \; = \; \prod_{i=1}^N \frac{1}{\sigma_y \,
    \sqrt{2 \pi}} \,  e^{- \frac{1}{2}
    \left( \frac{m x_i - y_i}{\sigma_y} \right)
    ^2 } \peq .
\end{equation*}

Passando al logaritmo naturale di $\mathcal{L}$, \`e facile
vedere che la soluzione ricercata \`e sempre quella che
rende minima la somma dei quadrati dei residui dai punti
misurati: che cio\`e minimizza la funzione
\begin{equation*}
  f(m) = \sum_{i=1}^N \left( m x_i - y_i \right) ^2 \peq .
\end{equation*}

La derivata prima di $f$ vale
\begin{equation*}
  \frac{\de f}{\de m} \; = \;
  2 \sum_{i=1}^N \left( m x_i - y_i \right) x_i
  \; = \; 2 \left( m \sum_{i=1}^N {x_i}^2 \; - \;
    \sum_{i=1}^N x_i y_i \right)
\end{equation*}
e, imponendo che essa sia nulla, l'estremante si ha per
\begin{equation*}
  m = \frac{\sum_i x_i y_i}{\sum_i {x_i}^2}
\end{equation*}
e corrisponde in effetti ad un minimo.  La legge di
propagazione degli errori \`e esatta anche in questo caso,
perch\'e $m$ \`e una combinazione lineare delle variabili
affette da errore (le $y_i$); il coefficiente di $y_i$ nella
combinazione vale
\begin{gather*}
  \frac{x_i}{\sum_k {x_k}^2} \\
  \intertext{e quindi}
  {\sigma_m}^2 \; = \; \sum_{i=1}^N \left(
    \frac{x_i}{\sum_k {x_k}^2} \right)^2
    {\sigma_y}^2 \; = \; \frac{{\sigma_y}^2}{
    \left( \sum_k {x_k}^2 \right)^2 } \,
    \sum_{i=1}^N {x_i}^2 \; = \; \frac{{\sigma_y}^2}
    {\sum_k {x_k}^2} \\
  \intertext{e la formula per il calcolo degli
    errori a posteriori diventa}%
  \index{errore!a posteriori}
  {\sigma_y}^2 = \frac{\sum_i \left( m x_i - y_i \right)
    ^2}{N-1}
\end{gather*}
visto che il parametro da cui l'errore quadratico medio
dipende e che deve essere stimato sulla base dei dati \`e
uno soltanto: $m$.%
\index{interpolazione lineare!con una retta per l'origine|)}

\subsection{Interpolazione lineare nel caso generale}%
\index{interpolazione lineare|(}
Le condizioni 1) e 2) sugli errori delle grandezze misurate
$x$ e $y$ date nel paragrafo \ref{ch:11.intlin} non potranno
ovviamente mai essere verificate esattamente; come ci si
deve comportare quando nemmeno in prima approssimazione le
possiamo considerare vere?

Se gli errori quadratici medi delle $y_i$ sono tra loro
diversi, non \`e pi\`u possibile raccogliere a fattore
comune $1 / \sigma^2$ nell'espressione del logaritmo della
verosimiglianza; e ciascun addendo sar\`a diviso per il
corrispondente errore $\sigma_i$.  In definitiva la retta
pi\`u verosimile si trova cercando il minimo della funzione
\begin{equation*}
  f(a,b) = \sum_{i=1}^N \left[
    \frac{ ( a + b x_i ) - y_i }{ \sigma_i } \right] ^2 \peq .
\end{equation*}

Questo avviene quando
\begin{equation*}
  \left \{ \begin{array}{ccl}
    a & = & \dfrac{1}{\Delta}
      \left[ \left( \sum \nolimits_i
      \dfrac{{x_i}^2}{{\sigma_i}^2} \right)
      \cdot \left( \sum \nolimits_i
      \dfrac{y_i}{{\sigma_i}^2} \right) -
      \left( \sum \nolimits_i
      \dfrac{x_i}{{\sigma_i}^2} \right)
      \cdot \left( \sum \nolimits_i
      \dfrac{x_i y_i}{{\sigma_i}^2}
      \right) \right] \\*[7mm]
    b & = & \dfrac{1}{\Delta}
      \left[ \left( \sum \nolimits_i
      \dfrac{1}{{\sigma_i}^2} \right) \cdot
      \left( \sum \nolimits_i
      \dfrac{x_i y_i}{{\sigma_i}^2} \right)
      - \left( \sum \nolimits_i
      \dfrac{x_i}{{\sigma_i}^2} \right) \cdot
      \left( \sum \nolimits_i
      \dfrac{y_i}{{\sigma_i}^2} \right)
      \right]
    \end{array} \right.
\end{equation*}
in cui si \`e posto
\begin{equation*}
  \Delta \; = \; \left( \sum \nolimits_i
    \frac{1}{{\sigma_i}^2} \right) \cdot
    \left( \sum \nolimits_i \frac{{x_i}^2}
    {{\sigma_i}^2}  \right) - \left(
    \sum\nolimits_i \frac{x_i}{{\sigma_i}^2}
    \right) ^2 \peq .
\end{equation*}

Le varianze di $a$ e di $b$ saranno poi date dalle
\begin{equation*}
  \left\{
  \begin{array}{ccl}
    {\sigma_a}^2 & = & \dfrac{1}{\Delta} \,
      \sum \nolimits_i \dfrac{{x_i}^2}{{\sigma_i}^2}
      \\*[7mm]
    {\sigma_b}^2 & = & \dfrac{1}{\Delta} \,
      \sum \nolimits_i \dfrac{1}{{\sigma_i}^2}
  \end{array}
  \right.
\end{equation*}

Si deve tuttavia osservare che per applicare questo metodo
\`e necessario conoscere, per altra via e preventivamente,
tutte le $N$ varianze ${\sigma_i}^2$.  Ci\`o pu\`o essere
molto laborioso o addirittura impossibile, e non risulta
conveniente rinunciare ad una stima unica e ragionevole
${\sigma_y}^2$ di queste varianze per tener conto di una
variazione, generalmente debole, delle $\sigma_i$ in un
intervallo limitato di valori della $x$.

Volendo tener conto dell'errore su entrambe le variabili $x$
ed $y$, non \`e generalmente possibile usare un metodo,
descritto in alcuni testi, consistente nel cercare la retta
che rende minima la somma dei quadrati delle distanze dai
punti, misurate per\`o \emph{ortogonalmente} alla retta
stessa: a prescindere dalla complicazione della soluzione di
un sistema di equazioni non lineari, resta il fatto che se
$x$ ed $y$ sono due grandezze fisiche diverse, o anche
soltanto misurate con strumenti e metodi diversi, i loro
errori quadratici medi sono generalmente differenti; mentre
la distanza sul piano attribuisce lo stesso peso agli scarti
in $x$ ed a quelli in $y$.

Per applicare questo metodo si dovrebbe conoscere, per via
indipendente, almeno il rapporto tra $\sigma_x$ e
$\sigma_y$; e rappresentare i valori misurati $( x_i , y_i
)$ non gi\`a sul piano $\{ x,y \}$, bens\`\i\ su quello
delle variabili ridotte $\{ x/\sigma_x \, , \, y/\sigma_y
\}$.

Per solito nella pratica si preferisce considerare affetta
da errore una soltanto delle variabili, ad esempio la $y$,
la scelta cadendo generalmente su quella determinata in
maniera \emph{pi\`u indiretta}, e che risente perci\`o degli
errori di tutte le altre grandezze misurate direttamente;
cos\`\i, in un diagramma velocit\`a-tempo trascorso o
velocit\`a-spazio percorso, si assumer\`a affetta da errore
la sola velocit\`a.

Un eventuale errore sulla $x$ si \emph{propagher\`a}
attraverso la relazione funzionale anche alla $y$, e, se
l'errore quadratico medio $\sigma_y$ \`e stimato dai dati
sperimentali, esso conglober\`a anche l'indeterminazione
dovuta alla $x$.

Per meglio chiarire il concetto, consideriamo la legge $v =
v(t) = v_0 + g t$ che descrive la caduta di un grave, e
pensiamo di misurare la sua velocit\`a in un certo istante:
nell'ipotesi originale $t$ sarebbe determinabile
esattamente, ma l'imprecisione nella misura delle velocit\`a
ci darebbe valori di $v$ compresi in un intervallo di
ampiezza non nulla (dipendente dall'errore quadratico medio
$\sigma_v$).

Se delle due grandezze, al contrario, fosse la velocit\`a ad
essere conoscibile esattamente, l'impossibilit\`a di
determinare con precisione l'istante $t$ in cui essa deve
essere misurata ci darebbe ugualmente valori di $v$
distribuiti in un intervallo di ampiezza non nulla (legata
stavolta a $g \cdot \sigma_t$).

Indicando, insomma, con $\sigma_x$ e $\sigma_y$ gli errori
(sempre supposti costanti) di ognuna delle determinazioni
(sempre supposte indipendenti) $x_i$ e $y_i$, la formula
dell'errore a posteriori ci permette di ricavare dai dati
una ragionevole stima non tanto del solo $\sigma_y$ quanto,
piuttosto, della combinazione (quadratica) dell'errore
intrinseco delle ordinate e di quello intrinseco delle
ascisse propagato sulle ordinate:
\begin{equation*}
  \sigma^2 \approx {\sigma_y}^2 + \left( b^* \: \sigma_x
    \right)^2
\end{equation*}
(ove $b^*$ \`e il valore vero della pendenza della retta).%
\index{interpolazione lineare|)}

\subsection{Interpolazione non lineare}
Formule analoghe a quelle trovate si possono ricavare per
risolvere il problema dell'interpolazione di curve di ordine
superiore al primo (parabole, cubiche, polinomiali in
genere) ad un insieme di dati sperimentali, sempre usando il
metodo della massima verosimiglianza.

Nel caso poi ci si trovasse di fronte ad una curva di
equazione diversa da un polinomio, in parecchi casi \`e
possibile \emph{linearizzare} la relazione cambiando
variabile: cos\`\i, ad esempio, se due grandezze hanno tra
loro una relazione di tipo esponenziale, il logaritmo
naturale ne avr\`a una di tipo lineare:
\begin{align*}
  y &= K e^{-bx} & &\Longleftrightarrow &
   \ln y \; = \; \ln K - bx \; &= \; a-bx \peq .
\end{align*}

\section{Altre applicazioni della stima di massima
   verosimiglianza}%
\index{massima verosimiglianza, metodo della|(}%
\label{ch:11.exampl}
Per concludere il capitolo, presentiamo altre tre
applicazioni del metodo della massima verosimiglianza:
la stima delle probabilit\`a ignote di un insieme di
modalit\`a esclusive ed esaurienti cui pu\`o dar luogo un
fenomeno casuale; la stima sia della media che della
varianza di una popolazione normale; e la stima del range di
una popolazione uniforme.

\subsection{Stima di probabilit\`a}
Supponiamo che un fenomeno casuale possa dare origine ad un
numero finito $M$ di eventualit\`a, ognuna delle quali sia
associata ad un valore $p_i$ ignoto della probabilit\`a; se,
eseguite $N$ prove indipendenti, indichiamo con $n_i$ la
frequenza assoluta con cui ognuna delle $M$ eventualit\`a si
\`e presentata nel corso di esse, quale \`e la stima di
massima verosimiglianza, $\widehat p_i$, per le incognite
probabilit\`a $p_i$?

La funzione di verosimiglianza \`e, visto che la generica
delle $M$ eventualit\`a, di probabilit\`a $p_i$, si \`e
presentata $n_i$ volte, data\/\footnote{A meno di un fattore
  moltiplicativo costante, corrispondente al numero di modi
  in cui $N$ oggetti si possono ripartire tra $M$ gruppi in
  modo che ogni gruppo sia composto da $n_i$ oggetti; numero
  delle \emph{partizioni ordinate}%
  \index{partizioni ordinate}
  (vedi in proposito il paragrafo \ref{ch:a.parord}).
} da
\begin{gather}
  \mathcal{L} ( \boldsymbol{n}; \boldsymbol{p} ) =
    \prod_{i=1}^M {p_i}^{n_i} \notag \\
  \intertext{(in cui abbiamo indicato sinteticamente con
    due vettori, $\boldsymbol{n}$ e $\boldsymbol{p}$,
    entrambi di dimensione $M$, l'insieme degli
    $M$ valori $n_i$ e quello degli $M$ valori $p_i$
    rispettivamente); ed il suo logaritmo da}
  \ln \mathcal{L} ( \boldsymbol{n}; \boldsymbol{p} ) =
    \sum_{i=1}^M n_i \, \ln p_i \peq . \label{eq:11.lnlipi}
\end{gather}

Il problema della ricerca del massimo della
\eqref{eq:11.lnlipi} \`e complicato dal fatto che i valori
delle $p_i$ non sono liberi, ma \emph{vincolati} dalla
condizione
\begin{equation} \label{eq:11.sumpi}
  \sum_{i=1}^M p_i = 1 \peq .
\end{equation}
Usiamo quindi il metodo dei moltiplicatori di Lagrange,
costruendo la funzione
\begin{equation} \label{eq:11.likpro}
  f( \boldsymbol{n}; \boldsymbol{p} ) \; = \; \sum_{i=1}^M
    n_i \, \ln p_i - \lambda \left( \sum_{i=1}^M p_i - 1
    \right)
\end{equation}
e risolvendo il sistema delle $M+1$ equazioni, nelle $M+1$
incognite $p_i$ e $\lambda$, composto dalla
\eqref{eq:11.sumpi} e dalle altre $M$ ottenute derivando la
\eqref{eq:11.likpro} rispetto ad ognuna delle $p_k$:
\begin{gather*}
  \frac{\partial f}{\partial p_k} \; = \; n_k \,
    \frac{1}{p_k} - \lambda \; = \; 0 \hspace{2cm} (k = 1,
    2,\ldots, M) \peq . \\
  \intertext{Da quest'ultima si ricava}
  \widehat p_k = \frac{n_k}{\lambda} \\
  \intertext{e, sostituendo nella \eqref{eq:11.sumpi},}
  \sum_{i=1}^M \widehat p_i \; = \; \frac{1}{\lambda}
    \sum_{i=1}^M n_i \; = \; \frac{N}{\lambda} \; = \; 1
    \\
  \intertext{si ottiene}
  \lambda = N \\
  \intertext{per cui in definitiva la soluzione di massima
    verosimiglianza \`e (cosa non sorprendente) data dalle}
  \widehat p_i = \frac{n_i}{N} \peq .
\end{gather*}

\subsection{Media e varianza di una popolazione
  normale}
Abbiamo gi\`a visto nel paragrafo \ref{ch:11.mepeted} che,
\emph{ammessa nota la varianza} $\sigma^2$ di una
popolazione normale, il suo valore medio $\mu$ ha come stima
di massima verosimiglianza la media aritmetica $\bar x$ di
un campione di stime indipendenti; vogliamo ora stimare
\emph{contemporaneamente sia $\mu$ che $\sigma$} dai dati,
usando sempre il metodo della massima verosimiglianza.

La densit\`a di probabilit\`a vale
\begin{gather*}
  f( x; \mu, \sigma ) = \frac{1}{\sigma \sqrt{2 \pi}} \,
    e^{- \frac{1}{2} \left( \frac{x - \mu}{
          \sigma } \right)^2 } \\
  \intertext{ed il suo logaritmo}
  \ln f ( x; \mu, \sigma ) \; = \; - \ln \sigma - \ln \sqrt{
    2 \pi } - \frac{1}{2} \left( \frac{x - \mu}{ \sigma }
    \right)^2 \peq .
\end{gather*}

Il logaritmo della funzione di verosimiglianza \`e
\begin{gather*}
  \ln \mathcal{L} ( \boldsymbol{x}; \mu, \sigma) \; =
    \; \sum_{i=1}^N \ln f ( x_i ; \mu, \sigma ) \\
  \intertext{e dunque}
  \ln \mathcal{L} ( \boldsymbol{x}; \mu, \sigma) \; = \; - N
    \ln \sigma - N \, \ln \sqrt{2 \pi} - \frac{1}{2 \sigma^2}
    \sum_{i=1}^N \left( x_i - \mu \right)^2 \peq ;
\end{gather*}
e le sue derivate parziali prime sono
\begin{gather*}
  \frac{ \partial }{ \partial \mu } \ln \mathcal{L} \; = \;
    \frac{1}{\sigma^2} \, \sum_{i=1}^N \left( x_i - \mu
    \right) \; = \; \frac{1}{ \sigma^2 } \left(
  \sum\limits_{i=1}^N x_i - N \mu \right) \\
  \intertext{e}
  \frac{ \partial }{ \partial \sigma } \ln \mathcal{L} \; =
    \; - \frac{N}{\sigma} + \frac{1}{\sigma^3} \sum_{i=1}^N
    \left( x_i - \mu \right)^2 \; = \; \frac{1}{\sigma^3} \,
    \left[ \sum_{i=1}^N \left( x_i - \mu \right)^2 - N
    \sigma^2  \right] \peq .
\end{gather*}

Il sistema ottenuto annullando le due derivate parziali
prime ha l'unica soluzione (in effetti un massimo) data da
\begin{align*}
  \widehat \mu &= \bar x = \frac{1}{N} \sum_{i=1}^N x_i
   &&\text{e} &
   \widehat \sigma^2 &= \frac{1}{N} \sum_{i=1}^N \left( x_i
     - \widehat \mu \right)^2 \peq .
\end{align*}
Questo era gi\`a noto: entrambe le stime, come sappiamo,
sono consistenti; per\`o la seconda non \`e imparziale (ma
pu\`o essere resa tale moltiplicandola per un opportuno
fattore di correzione).

In sostanza il fatto che la varianza della popolazione abbia
un determinato valore (come assunto nel paragrafo
\ref{ch:11.mepeted}) non cambia il fatto che la nostra
migliore stima del valore medio della popolazione sia
comunque data dalla media aritmetica del campione: vedremo
poi nel paragrafo \ref{th:12.inmest} che il valore medio del
campione e la sua varianza sono variabili casuali
\emph{statisticamente indipendenti tra loro}.%
\index{media!aritmetica!e varianza}%
\index{varianza!e media aritmetica}

\subsection{Range di una popolazione uniforme}%
\index{distribuzione!uniforme!range|(}
Sia una variabile casuale $x$ distribuita uniformemente tra
un estremo inferiore noto, che senza perdere in generalit\`a
possiamo supporre sia lo zero, ed un estremo superiore
ignoto $A$; in questo caso dobbiamo innanzi tutto osservare
sia che il dominio di definizione della funzione di
frequenza $f(x)$ della $x$ \emph{dipende dal parametro} che
dobbiamo stimare, sia che $f(x)$ e la sua derivata prima
hanno dei punti di discontinuit\`a: e non possiamo in
conseguenza garantire \emph{a priori} n\'e la consistenza,
n\'e la massima efficienza asintotica del metodo usato.

Comunque, introducendo la cosiddetta \emph{funzione gradino}
(o \emph{step function}) $S(x)$, definita attraverso la
\begin{equation*}
  \begin{cases}
    S(x) = 0 &\qquad(x < 0) \\[1ex]
    S(x) = 1 &\qquad(x \ge 0)
  \end{cases}
\end{equation*}
la funzione di frequenza $f(x)$ si pu\`o anche scrivere
\begin{gather*}
  f(x) \; = \; \frac{1}{A} \; S(x) \; S(A - x) \\
  \intertext{e la funzione di verosimiglianza}
  \mathcal{L}(x_1, x_2, \ldots, x_N; A) = \frac{1}{A^N}
  \; S(x_{\min}) \; S(A - x_{\max}) \peq .
\end{gather*}

Come sappiamo, ammesso noto il valore del parametro $A$ essa
rappresenta la densit\`a di probabilit\`a di ottenere gli
$N$ valori $x_i \in [-\infty , +\infty]$; se invece si
considera $A$ come l'unica variabile e si ammettono noti gli
$N$ valori $x_i$, rappresenta la densit\`a di probabilit\`a
che un dato $A$ abbia prodotto i dati osservati.  Ma in
quest'ultimo caso $S(x_{\min}) \equiv 1$, e la funzione di
verosimiglianza si riduce alla
\begin{equation} \label{eq:11.range1}
  \mathcal{L}(A) = \frac{1}{A^N} \; S(A - x_{\max})
\end{equation}
che \`e nulla per $A < x_{\max}$ e monotona strettamente
decrescente per $A \ge x_{\max}$; ed ammette quindi un unico
massimo all'estremo del dominio, che vale
\begin{equation} \label{eq:11.range2}
  \hat A = x_{\max} \peq .
\end{equation}

Valore medio e varianza della stima valgono, come gi\`a
sappiamo dal paragrafo \ref{ch:8.estremi},
\begin{gather*}
  E( \hat A ) \; = \; A^* - \frac{A^*}{N + 1} \; = \;
  \frac{N}{N + 1} \, A^* \\
  \intertext{e}
  \var( \hat A ) \; = \; \frac{N}{(N + 1)^2 \, (N + 2)} \,
  \left( A^* \right)^2
  \intertext{e quindi la stima \`e consistente, ma non
    imparziale; una stima imparziale \`e invece}
  \bar A \; = \; \frac{N + 1}{N} \, x_{\max} \; = \;
  x_{\max} \, \left( 1 + \frac{1}{N} \right) \peq ,
\end{gather*}
di varianza ovviamente superiore per un fattore $(1 +
1/N)^2$.  \`E anche ovvio, dalla forma sia della
\eqref{eq:11.range1} che della \eqref{eq:11.range2}, che
$\hat A$ \`e una stima sufficiente di $A^*$.%
\index{distribuzione!uniforme!range|)}%
\index{massima verosimiglianza, metodo della|)}

\subsection{Stima della vita media di una particella}%
\index{vita media|(}
Nel processo di decadimento di una particella instabile,
indichiamo con $\tau$ l'incognita vita media e con $t$ i
tempi (propri) di decadimento osservati; tempi che (come
sappiamo) seguono la distribuzione esponenziale:
\begin{equation*}
  f(t; \tau) = \frac{1}{\tau} \, e^{- \frac{t}{\tau} }
  \makebox[40mm]{$\Longrightarrow$}
  \begin{cases}
    E(t) = \tau \\[2ex]
    \var(t) = \tau^2
  \end{cases}
\end{equation*}
Ammettendo per semplicit\`a che l'osservazione avvenga con
una efficienza unitaria, o, in altre parole, che tutti i
decadimenti vengano osservati, la funzione di
verosimiglianza si scrive
\begin{gather*}
  \mathcal{L} \; = \; \prod_{k=1}^N f(t_k; \tau) \; = \;
  \frac{1}{\tau^N} \, e^{- \frac{1}{\tau} \sum_k t_k } \peq
  , \\
  \intertext{ed il suo logaritmo vale}
  \ln( \mathcal{L} ) \; = \; - N \ln \tau -
  \frac{1}{\tau} \sum_{k=1}^N t_k \; = \; - N \left(
    \frac{\bar t}{\tau} + \ln \tau \right) \peq . \\
  \intertext{Derivando rispetto al parametro e cercando gli
    estremanti,}
  \frac{\de \, \ln( \mathcal{L} )}{\de \tau} \; = \;
  \frac{N}{\tau^2} ( \bar t - \tau ) \; = \; 0 \peq ; \\
  \intertext{e quindi l'unico estremante della funzione di
    verosimiglianza si ha per}
  \hat \tau = \bar t \peq . \\
  \intertext{Se calcoliamo la derivata seconda,}
  \frac{\de^2 \, \ln( \mathcal{L} )}{\de \tau^2} = -
  \frac{N}{\tau^3} \left( 2 \bar t - \tau \right) \\
\end{gather*}
essa, calcolata per $t = \bar t$ \`e negativa; quindi
l'unico estremante \`e effettivamente un punto di massimo.

La soluzione di massima verosimiglianza $\hat \tau = \bar t$
\`e \emph{consistente ed imparziale} (essendo il valore medio
del campione); \emph{di varianza minima} (per il teorema di
Cram\'er--Rao); inoltre la stima \`e \emph{sufficiente}
(riassume insomma tutta l'informazione del campione).

Normalmente l'efficienza non \`e per\`o unitaria; ad esempio
il nostro rivelatore pu\`o avere dimensioni confrontabili
col cammino medio delle particelle, che possono quindi
uscirne prima di decadere.  In questo caso, visto che i
decadimenti possono essere stati osservati solo essendo
avvenuti all'interno di un intervallo compreso tra un valore
temporale minimo (eventualmente nullo) ed uno massimo (ad
esempio dipendente dalla posizione del decadimento, dalla
direzione di emissione dei suoi prodotti, dalle dimensioni
del rivelatore, \ldots) --- intervallo differente per ognuno
dei decadimenti --- dovremo costruire la funzione di
verosimiglianza considerando le probabilit\`a di
osservazione \emph{condizionate} dall'essere il decadimento
$i$-esimo avvenuto tra un certo $(t_{\min})_i$ ed un certo
$(t_{\max})_i$:
\begin{equation} \label{eq:11.taulik}
  \mathcal{L} = \prod_{i=1}^N \left[ \frac{ \frac{1}{\tau}
      e^{- \frac{t_i}{\tau}} }{ e^{-
        \frac{(t_{\min})_i}{\tau}} - e^{-
        \frac{(t_{\max})_i}{\tau}} } \right]
\end{equation}

Il denominatore della \eqref{eq:11.taulik} rappresenta
infatti la probabilit\`a di decadere tra il tempo $t_{\min}$
e quello $t_{\max}$, come \`e immediato ricavare dalla
funzione di distribuzione della densit\`a di probabilit\`a
esponenziale, che vale
\begin{gather*}
  F(t) \; = \; \int_0^t \frac{1}{\tau} \, e^{-
    \frac{x}{\tau} } \, \de x \; = \; 1 - e^{-
    \frac{t}{\tau} } \peq ; \\
  \intertext{dalla \eqref{eq:11.taulik} si ricava poi}
  \ln( \mathcal{L} ) = -N \ln \tau + \sum_{i=1}^N \left\{
    - \frac{t_i}{\tau} - \ln \left[ e^{-
        \frac{(t_{\min})_i}{\tau}} -  e^{-
        \frac{(t_{\max})_i}{\tau}} \right] \right\} \\
  \intertext{e, posto per brevit\`a}
  \varphi_i(\tau) = \frac{ (t_{\min})_i^2 \cdot e^{-
      \frac{(t_{\min})_i}{\tau}} - (t_{\max})_i^2 \cdot
    e^{- \frac{(t_{\max})_i}{\tau}}}{e^{-
      \frac{(t_{\min})_i}{\tau}} - e^{-
      \frac{(t_{\max})_i}{\tau}}} \\
  \intertext{si arriva alla}
  \frac{\de \, \ln( \mathcal{L} ) }{\de \tau} \; = \;
  \frac{1}{\tau^2} \left\{ \sum_{i=1}^N \bigl[ t_i -
    \varphi_i(\tau) \bigr] - N \tau \right\} \; = \; 0
\end{gather*}
che bisogna risolvere in modo numerico.  Non si pu\`o
inoltre in questo caso garantire che le propriet\`a
precedentemente delineate per $\hat \tau$ (consistenza,
normalit\`a, efficienza, \ldots) siano ancora valide, almeno
per $N$ finito.  Pu\`o darsi che la funzione di
verosimiglianza ammetta pi\`u di un massimo, e non si sa a
priori quale di essi converger\`a verso $\tau^*$; e, per
finire, l'errore della stima deve essere ricavato dalla
concavit\`a della funzione di verosimiglianza, supposta
approssimativamente normale.%
\index{vita media|)}

\endinput

% $Id: chapter12.tex,v 1.3 2006/05/16 11:10:55 loreti Exp $

\chapter{La verifica delle ipotesi (I)}%
\label{ch:12.veripo}
Una volta eseguita una misura, si pu\`o voler controllare se
i nostri risultati possono confermare o rigettare una
determinata ipotesi riguardante il fenomeno fisico che li ha
prodotti; naturalmente, visto che risultati di una misura
comunque lontani dal valore vero sono sempre possibili
(anche se con probabilit\`a sempre pi\`u piccole al crescere
dello scarto), una qualunque ipotesi sulla grandezza fisica
misurata potr\`a essere confermata o rigettata dai dati solo
ad un certo livello di probabilit\`a.

Qui ci occuperemo inoltre di alcune funzioni di frequenza
collegate a quella di Gauss, ossia della distribuzione del
$\chi^2$, di quella di Student\/\footnote{``Student'' \`e lo
  pseudonimo con cui vennero pubblicati i lavori statistici
  di William Gosset, scienziato inglese vissuto dal 1876 al
  1937.  Uno dei pionieri di questo ramo della matematica,
  svolse le sue ricerche essendo dipendente (prima come
  chimico, poi come dirigente) della Guinness Brewery di
  Dublino.}%
\index{Gosset, William (``Student'')}%
\index{Student|see{Gosset, William}}
e di quella di Fisher; e dell'uso che di esse si pu\`o fare
per la verifica di ipotesi statistiche: quali ad esempio
quella che un campione di dati sperimentali provenga da una
popolazione descritta da una densit\`a di probabilit\`a nota
a priori; o quella che il valore vero della grandezza
misurata coincida con un valore determinato, noto anch'esso
a priori.

\section[La distribuzione del $\chi^2$]
{La distribuzione del
  $\boldsymbol{\chi}^{\boldsymbol{2}}$}%
\index{distribuzione!del $\chi^2$|(}
Se le $N$ variabili casuali $x_i$, tra loro statisticamente
indipendenti, sono variabili normali standardizzate
(ovverosia distribuite secondo la legge normale con media 0
e varianza 1), si pu\`o dimostrare che la nuova variabile
casuale
\begin{gather}
  X = \sum_{i=1}^N {x_i}^2 \notag \\
  \intertext{(ovviamente non negativa) \`e
    distribuita con una densit\`a di probabilit\`a
    data dalla}
  \frac{\de p}{\de X} \; = \; f(X ; N) \; = \;
    K_N \, X^{\left(\frac{N}{2}-1 \right)}
    e^{-\frac{X}{2}} \label{eq:12.denchi}
\end{gather}
(\emph{distribuzione del chi quadro}); la costante $K_N$
viene fissata dalla condizione di normalizzazione, ed il
parametro $N$ prende il nome di \emph{numero di gradi di
  libert\`a} della distribuzione.

\begin{figure}[hbtp]
  \vspace*{2ex}
  \begin{center} {
    \input{chi.pstex_t}
  } \end{center}
  \caption[La distribuzione del $\chi^2$]
    {La distribuzione del $\chi^2$ per alcuni
    valori del parametro $N$.}
\end{figure}

La funzione caratteristica della $X$ si pu\`o trovare
facilmente considerando che, se la $x$ \`e una variabile
normale standardizzata, il suo quadrato $y = x^2$ ha una
funzione caratteristica
\begin{align*}
  \phi_y(t) &= E \bigl( e^{ity} \bigr) \\[1ex]
  &= E \left( e^{itx^2} \right) \\[1ex]
  &= \int_{-\infty}^{+\infty} \! e^{itx^2}
    \frac{1}{\sqrt{2\pi}} \, e^{- \frac{x^2}{2}} \, \de
    x \\[1ex]
  &= \int_{-\infty}^{+\infty} \frac{1}{\sqrt{2\pi}}
    \, e^{- \frac{x^2}{2} (1-2it)} \, \de x \\[1ex]
  &= \frac{1}{\sqrt{1-2it}} \int_{-\infty}^{+\infty}
    \frac{1}{\sqrt{2\pi}} \, e^{- \frac{u^2}{2}} \, \de
    u \\[1ex]
  &= (1-2it)^{- \frac{1}{2}}
\end{align*}
(si \`e eseguita la sostituzione di variabile $u = x
\sqrt{1-2it}$; l'integrale definito \`e quello di una
distribuzione normale $N(u; 0,1)$ e vale dunque 1).  Di
conseguenza, applicando l'equazione \eqref{eq:6.fucacl}, la
funzione caratteristica della $X$ vale
\begin{equation} \label{eq:12.fucachi}
  \phi_X(t) = (1 - 2it)^{- \frac{N}{2}} \peq .
\end{equation}

Per dimostrare che la funzione di frequenza della $X$ \`e
effettivamente la \eqref{eq:12.denchi}, si parte poi
dall'espressione \eqref{eq:12.fucachi} della funzione
caratteristica e le si applica la trasformazione inversa di
Fourier%
\index{Fourier, trasformata di}
gi\`a definita nella \eqref{eq:6.trinfo}.

Con simili passaggi si potrebbe ricavare la funzione
generatrice dei momenti, che vale
\begin{equation*}
  M_X(t) = (1 - 2t)^{- \frac{N}{2}}
\end{equation*}
e, da queste, si ottiene infine che il valore medio e la
varianza di una variabile casuale distribuita come il
$\chi^2$ a $N$ gradi di libert\`a sono
\begin{gather*}
  E(X) \; = \; N \makebox[40mm]{\mbox{e}} \var(X) \; =
  \; 2N \\
  \intertext{mentre i coefficienti di asimmetria e di
    curtosi valgono}
  \gamma_1 \; = \; 2 \sqrt{\frac{2}{N}}
  \makebox[40mm]{\mbox{e}} \gamma_2 \; = \; \frac{12}{N} \peq
  .
\end{gather*}

\index{distribuzione!del $\chi^2$!e distribuzione normale|(}%
La distribuzione del $\chi^2$ tende asintoticamente ad una
distribuzione normale con la stessa media $N$ e la stessa
varianza $2N$; infatti la funzione caratteristica della
variabile standardizzata
\begin{gather*}
  y \; = \; \frac{X - N}{\sqrt{2N}} \; = \;
    \frac{X}{\sqrt{2N}} - \frac{N}{\sqrt{2N}} \\
  \intertext{vale, ricordando la \eqref{eq:6.fuccav},}
  \phi_y(t) = e^{- \frac{i N t}{\sqrt{2N}}} \left[ 1 -
    \frac{2 i t}{\sqrt{2N}} \right]^{- \frac{N}{2}} \peq . \\
  \intertext{Passando ai logaritmi naturali,}
  \ln \phi_y(t) = - \, \frac{i N t}{\sqrt{2N}} -
    \frac{N}{2} \ln \left( 1 - \frac{2 i t}{\sqrt{2N}}
    \right)
\end{gather*}
e, sviluppando in serie di McLaurin il logaritmo,
\begin{align*}
  \ln \phi_y(t) &= - \, \frac{i N t}{\sqrt{2N}} -
    \frac{N}{2} \left[ - \, \frac{2 i t}{\sqrt{2N}} -
    \frac{1}{2} \left( \frac{2 i t}{\sqrt{2N}}
    \right)^2 + \mathcal{O} \left( N^{-\frac{3}{2}} \right)
  \right] \\[1ex]
  &= - \, \frac{t^2}{2} + \mathcal{O} \left(
    N^{-\frac{1}{2}} \right)
\end{align*}
da cui
\begin{equation*}
  \lim_{N \to \infty} \phi_y(t) = e^{- \frac{t^2}{2}}
\end{equation*}
che \`e appunto la funzione caratteristica di una
distribuzione normale standardizzata.

In definitiva:
\begin{itemize}
\item Quando $N$ assume valori sufficientemente grandi, la
  distribuzione del $\chi^2$ \`e ben approssimata da una
  distribuzione normale avente la stessa media $N$ e la
  stessa varianza $2N$; tale approssimazione si pu\`o
  ritenere in pratica gi\`a buona quando $N$ \`e superiore a
  30.
\item Inoltre si potrebbe analogamente dimostrare che la
  variabile casuale $\sqrt{2 X}$, anche per valori
  relativamente piccoli di $N$, ha una distribuzione che \`e
  assai bene approssimata da una funzione normale con media
  $\sqrt{2N-1}$ e varianza 1; l'approssimazione \`e gi\`a
  buona per $N\gtrsim 8$.
\end{itemize}%
\index{distribuzione!del $\chi^2$!e distribuzione normale|)}

Dalla definizione (o dalla funzione caratteristica
\eqref{eq:12.fucachi}) discende immediatamente la cosiddetta
\emph{regola di somma del $\chi^2$}\label{th:12.resochi}:%
\index{distribuzione!del $\chi^2$!regola di somma} ossia
che, se $X$ ed $Y$ sono due variabili casuali
statisticamente indipendenti entrambe distribuite come il
$\chi^2$, con $N$ ed $M$ gradi di libert\`a rispettivamente,
la loro somma $Z=X+Y$ \`e una variabile casuale ancora
distribuita come il $\chi^2$; per\`o con $N+M$ gradi di
libert\`a.

Ovviamente, se le $x_i$ (con $i=1,\ldots,N$) sono $N$
variabili casuali statisticamente indipendenti tra loro e
provenienti da una stessa distribuzione normale con media
$\mu$ e varianza $\sigma^2$, discende da quanto detto che la
nuova variabile casuale
\begin{equation*}
  X' = \sum_{i=1}^N \left( \frac{x_i - \mu}{\sigma}
   \right)^2
\end{equation*}
\`e distribuita come il $\chi^2$ a $N$ gradi di libert\`a.
Indichiamo ora, al solito, con $\bar x$ la media aritmetica
delle $x_i$: vogliamo dimostrare che la variabile casuale
\begin{equation*}
  X'' = \sum_{i=1}^N \left( \frac{x_i - \bar x}{\sigma}
  \right)^2
\end{equation*}
\`e distribuita \emph{ancora come il $\chi^2$, ma con $N -
  1$ gradi di libert\`a}.

A questo scopo facciamo dapprima alcune considerazioni,
indipendenti dalle ipotesi prima fatte sulle $x_i$ e che
risultano quindi valide per variabili casuali qualunque:
supponiamo di definire $N$ nuove variabili $y_i$ come
generiche combinazioni lineari delle $x_j$, con coefficienti
che indicheremo col simbolo $A_{ij}$; in modo insomma che
risulti
\begin{equation*}
  y_i = \sum_{j=1}^N A_{ij} \, x_j \peq .
\end{equation*}

La somma dei quadrati delle $y_i$ \`e data da
\begin{equation*}
  \sum_{i=1}^N {y_i}^2 = \sum_{i=1}^N \left(
  \sum_{j=1}^N A_{ij} \, x_j \right) \left( \sum_{k=1}^N
  A_{ik} \, x_k \right) = \sum\nolimits_{jk} x_j \, x_k
  \sum\nolimits_i A_{ij} \, A_{ik} \peq ;
\end{equation*}
\`e possibile che questa somma risulti uguale alla somma dei
quadrati delle $x_i$ \emph{qualunque} sia il valore di
queste ultime?  Ovviamente questo avviene se e solo se vale
la
\begin{equation} \label{eq:12.conort}
  \sum\nolimits_i A_{ij} \, A_{ik} \; = \; \delta_{jk} \;
  = \;
  \begin{cases}
    0 & \text{per $j \ne k$} \\[2ex]
    1 & \text{per $j = k$}
  \end{cases}
\end{equation}
(il simbolo $\delta_{jk}$, che assume il valore 1 quando gli
indici sono uguali e 0 quando sono invece diversi, si chiama
\emph{simbolo di Kronecker} o \emph{delta di Kronecker}).%
\index{Kronecker, delta di}

Consideriamo gli $A_{ij}$ come gli elementi di una matrice
quadrata $\boldsymbol{A}$ di ordine $N$; gli $x_j$ e le
$y_i$ si possono invece considerare come le componenti di
due \emph{vettori} $\boldsymbol{X}$ ed $\boldsymbol{Y}$
definiti in uno spazio $N$-dimensionale --- ossia come gli
elementi di due matrici rettangolari con $N$ righe ed 1
colonna.

La trasformazione che muta $\boldsymbol{X}$ in
$\boldsymbol{Y}$ si pu\`o scrivere, in forma matriciale,
come $\boldsymbol{Y} = \boldsymbol{A} \boldsymbol{X}$; la
somma dei quadrati delle $x_j$ o delle $y_i$ altro non \`e
se non il prodotto scalare, di $\boldsymbol{X}$ ed
$\boldsymbol{Y}$ rispettivamente, per loro stessi: ovverosia
la loro \emph{norma}, il quadrato della loro lunghezza nello
spazio a $N$ dimensioni.  Quella che abbiamo ricavato adesso
\`e la condizione perch\'e una \emph{trasformazione lineare}
applicata ad un vettore ne conservi la lunghezza: occorre e
basta che la matrice $\boldsymbol{A}$ sia \emph{ortogonale}.
Infatti la \eqref{eq:12.conort} si pu\`o scrivere
\begin{align*}
  \boldsymbol{\widetilde A} \boldsymbol{A} &=
  \boldsymbol{1} && \text{ossia}
  & \boldsymbol{\widetilde A} &= \boldsymbol{A}^{-1}
\end{align*}
($\boldsymbol{\widetilde A}$ \`e la matrice trasposta di
$\boldsymbol{A}$, di elementi $\boldsymbol{\widetilde
  A}_{ij} = A_{ji}$; $\boldsymbol{1}$ \`e la matrice
unit\`a, di elementi $\boldsymbol{1}_{ij} = \delta_{ij}$;
$\boldsymbol{A}^{-1}$ \`e la matrice inversa di
$\boldsymbol{A}$; ed una matrice per cui
$\boldsymbol{\widetilde A} = \boldsymbol{A}^{-1}$ si dice,
appunto, ortogonale).

Consideriamo adesso una trasformazione lineare definita
dalle seguenti relazioni:
\begin{equation} \label{eq:12.hack}
  \begin{cases}
    y_1 = \displaystyle \frac{1}{\sqrt{N}} \, (x_1 +
      x_2 +\cdots+ x_N) \\[2ex]
    y_2 = \displaystyle \frac{1}{\sqrt{2}} \, (x_1 -
      x_2) \\[2ex]
    y_3 = \displaystyle \frac{1}{\sqrt{6}} \, (x_1 +
      x_2 - 2 x_3) \\[2ex]
    \cdots \\[1ex]
    y_N = \displaystyle \frac{1}{\sqrt{N(N-1)}} \,
      \bigl[ x_1 + x_2 +\cdots+ x_{N-1} - (N-1) x_N
      \bigr]
  \end{cases}
\end{equation}
e per la quale la matrice di trasformazione abbia, insomma,
elementi $A_{ij}$ definiti come
\begin{equation*}
  A_{ij} \; \equiv \;
  \begin{cases}
    \text{$i=1$:} &  \displaystyle \frac{1}{\sqrt{N}}
      \\[4ex]
    \text{$i>1$:} &
    \begin{cases}
      \text{$j<i$:} &  \displaystyle
        \frac{1}{\sqrt{i(i-1)}} \\[2ex]
      \text{$j=i$:} &  \displaystyle - \,
        \frac{i-1}{\sqrt{i(i-1)}} \\[2ex]
      \text{$j>i$:} &  \displaystyle 0
    \end{cases}
  \end{cases}
\end{equation*}
Non \`e difficile controllare che la matrice
$\boldsymbol{A}$ \`e ortogonale; inoltre la prima riga \`e
stata scelta in modo tale che
\begin{gather}
  y_1 \; = \; \sum_{i=1}^N \frac{1}{\sqrt{N}} \, x_i \;
    = \; \frac{1}{\sqrt{N}} \cdot N \bar x \; = \;
    \sqrt{N} \, \bar x \notag \\
  \intertext{e quindi}
  \sum_{i=1}^N {x_i}^2 \; = \; \sum_{i=1}^N {y_i}^2 \;
    = \; N {\bar x}^2 + \sum_{i=2}^N {y_i}^2 \peq . \notag \\
  \intertext{Inoltre risulta (per $i > 1$)}
  \sum_{j=1}^N A_{ij} \; = \; \sum_{j=1}^{i-1}
    \frac{1}{\sqrt{i(i-1)}} \, - \,
    \frac{i-1}{\sqrt{i(i-1)}} \; = \; 0
    \label{eq:12.hackmean} \\
  \intertext{e, per ogni $i$,}
  \sum_{j=1}^N {A_{ij}}^2 \; = \; \left( \boldsymbol{A}
    \boldsymbol{\widetilde A} \right)_{ii} \; = \;
    \delta_{ii} \; = \; 1 \peq . \label{eq:12.hackstd}
\end{gather}

Tornando al nostro problema, supponiamo ora che tutte le
$x_j$ siano variabili aventi distribuzione normale; che
abbiano tutte valore medio $\mu$ e varianza $\sigma^2$; ed
inoltre che siano tra loro tutte statisticamente
indipendenti.  Una qualsiasi loro combinazione lineare,
quindi anche ognuna delle $y_i$ legate alle $x_j$ da quella
particolare matrice di trasformazione \eqref{eq:12.hack} che
abbiamo prima definita, \`e anch'essa distribuita secondo la
legge normale; inoltre risulta

\begin{align*}
  \frac{1}{\sigma^2} \sum_{i=1}^N \left( x_i - \bar x
    \right)^2 &= \frac{1}{\sigma^2} \left(
    \sum_{i=1}^N {x_i}^2 - N {\bar x}^2 \right) \\[1ex]
  &= \frac{1}{\sigma^2} \left( N {\bar x}^2 +
    \sum_{i=2}^N {y_i}^2 - N {\bar x}^2  \right)
    \\[1ex]
  &= \sum_{i=2}^N \frac{{y_i}^2}{\sigma^2} \peq .
\end{align*}

Applicando alle $y_i = \sum_j A_{ij} x_j$ le formule per la
media e la varianza delle combinazioni lineari di variabili
casuali statisticamente indipendenti gi\`a ricavate nel
capitolo \ref{ch:5.varcun}, si trova facilmente (tenendo
presenti la \eqref{eq:12.hackmean} e la
\eqref{eq:12.hackstd}) che la varianza di ognuna di esse \`e
ancora $\sigma^2$; e che, per $i \ne 1$, il loro valore
medio \`e 0.  Di conseguenza, per $i \geq 2$ le $y_i /
\sigma$ sono variabili casuali normali aventi media 0 e
varianza 1: e questo implica che
\begin{equation} \label{eq:12.xii}
  X'' = \sum_{i=1}^N \left( \frac{x_i - \bar x}{\sigma}
    \right)^2
\end{equation}
sia effettivamente distribuita come il $\chi^2$ a
$N - 1$ gradi di libert\`a.

\`E interessante confrontare questo risultato con quello
precedentemente ricavato, e riguardante la stessa
espressione --- in cui per\`o gli scarti erano calcolati
rispetto alla media della popolazione $\mu$.  Nel primo caso
la distribuzione era ancora quella del $\chi^2$, ma con $N$
gradi di libert\`a: riferendoci invece alla media aritmetica
del campione, i gradi di libert\`a diminuiscono di una
unit\`a.  Questo \`e conseguenza di una legge generale,
secondo la quale il numero di gradi di libert\`a da
associare a variabili che seguono la distribuzione del
$\chi^2$ \`e dato dal numero di contributi
\emph{indipendenti}: ovvero il numero di termini con
distribuzione normale sommati in quadratura (qui $N$, uno
per ogni determinazione $x_i$) diminuito del numero di
parametri che compaiono nella formula e che sono stati
stimati dai dati stessi (qui uno: appunto la media della
popolazione, stimata usando la media aritmetica delle
misure).

Un'ultima notevole conseguenza del fatto che la variabile
casuale $X''$ definita dalla \eqref{eq:12.xii} sia
distribuita come il $\chi^2$ a $N - 1$ gradi di libert\`a
\`e la seguente: la stima della varianza della popolazione
ottenuta dal campione, $s^2$, vale
\begin{equation} \label{eq:12.xiis2}
  s^2 = X'' \, \frac{\sigma^2}{N-1}
\end{equation}
e, essendo proporzionale a $X''$, \`e anch'essa distribuita
come il $\chi^2$ a $N - 1$ gradi di libert\`a; quindi la sua
densit\`a di probabilit\`a \`e data dalla
\eqref{eq:12.denchi} e dipende solamente da $N$; non
dipende, in particolare, dalla media del campione $\bar x$.
Quindi:
\begin{quote}
  \index{media!aritmetica!e varianza|emidx}%
  \index{varianza!e media aritmetica|emidx}%
  \label{th:12.inmest}
  \textit{Il valore medio $\bar x$ e la varianza campionaria
    $s^2$, calcolati su valori estratti a caso da una stessa
    popolazione normale, sono due variabili casuali
    \textbf{statisticamente indipendenti} tra loro.}
\end{quote}%
\index{distribuzione!del $\chi^2$|)}

Questo risulta anche intuitivamente comprensibile; se
infatti ci \`e noto che un certo campione di dati ha una
dispersione pi\`u o meno grande, questo non deve alterare la
probabilit\`a che il suo valore medio abbia un valore
piuttosto che un altro; n\'e, viceversa, il fatto che il
campione sia centrato attorno ad un certo valore deve
permetterci di prevedere in qualche modo la sua dispersione.

\section[Verifiche basate sulla distribuzione del $\chi^2$]
{Verifiche basate sulla distribuzione del
  $\boldsymbol{\chi}^{\boldsymbol{2}}$}
\subsection{Compatibilit\`a dei dati con una distribuzione}%
\index{compatibilit\`a!con una distribuzione|(}%
\label{ch:12.comdadis}
Supponiamo di avere dei dati raccolti in un istogramma, e di
voler verificare l'ipotesi che i dati provengano da una
certa distribuzione; ad esempio, dalla distribuzione
normale.  Ora, per una misura, la probabilit\`a $p_i$ di
cadere nell'intervallo $i$-esimo (di ampiezza prefissata
$\Delta x$ e corrispondente alla generica classe di
frequenza usata per la realizzazione dell'istogramma) \`e
data dal valore medio della funzione densit\`a di
probabilit\`a nell'intervallo stesso moltiplicato per
$\Delta x$.

Il numero di misure effettivamente ottenute in una classe di
frequenza su $N$ prove deve obbedire poi alla distribuzione
binomiale: il loro valore medio \`e quindi $N p_i$, e la
loro varianza $N \, p_i \, (1 - p_i)$; quest'ultimo termine
si pu\`o approssimare ancora con $N p_i$ se si ammette che
le classi di frequenza%
\index{classi di frequenza|(}
siano sufficientemente ristrette da poter trascurare i
termini in ${p_i}^2$ rispetto a quelli in $p_i$ (cio\`e se
$p_i \ll 1$).

In questo caso il numero di misure in ciascuna classe segue
approssimativamente la distribuzione di Poisson; questa \`e
infatti la funzione di frequenza che governa il presentarsi,
su un grande numero di osservazioni, di eventi aventi
probabilit\`a trascurabile di verificarsi singolarmente in
ognuna: distribuzione nella quale l'errore quadratico medio
\`e effettivamente dato dalla radice quadrata del valore
medio, $\sigma = \sqrt{N \, p_i \, (1 - p_i)} \simeq \sqrt{N
  p_i}$.

Nei limiti in cui il numero di misure attese in una classe
\`e sufficientemente elevato da poter confondere la relativa
funzione di distribuzione con la funzione normale, la
quantit\`a
\begin{equation} \label{eq:12.chi2fit}
  X \; = \;\sum_{i=1}^M
    \frac{(n_i - N p_i)^2}{N p_i} \; = \; \sum_{i=1}^M
    \frac{(O_i - A_i)^2}{A_i}
\end{equation}
cio\`e la somma, su tutte le classi di frequenza (il cui
numero abbiamo supposto sia $M$), del quadrato della
differenza tra il numero di misure ivi \emph{attese} ($A_i =
N p_i$) ed ivi \emph{effettivamente osservate} ($O_i =
n_i$), diviso per la varianza del numero di misure attese
(approssimata da $N p_i = A_i$), ha
\emph{approssimativamente} la distribuzione del $\chi^2$,
con $M - 1$ gradi di libert\`a; il motivo di quest'ultima
affermazione \`e che esiste un vincolo sulle $O_i$, quello
di avere per somma il numero totale di misure effettuate $N$
(che viene usato nella formula \eqref{eq:12.chi2fit},
mediante la quale abbiamo definito $X$, per calcolare il
numero $A_i$ di misure attese in ogni intervallo).

La condizione enunciata si pu\`o in pratica supporre
verificata se le $A_i$ in ogni intervallo sono almeno pari a
5; o, meglio, se il numero di classi di frequenza%
\index{classi di frequenza|)}
in cui ci si aspetta un numero di misure minore di 5 \`e
trascurabile rispetto al totale (meno del 10\%).  In
realt\`a, se le classi di frequenza si possono scegliere
arbitrariamente, la cosa migliore consiste nel definirle di
ampiezze differenti: in modo tale che quegli intervalli dove
cadono poche misure vengano riuniti assieme in un'unica
classe pi\`u ampia, ove $n_i$ valga almeno 5 (ma nemmeno
troppo ampia, per soddisfare al vincolo di avere ${p_i}^2
\ll p_i$; in genere si cerca di riunire assieme pi\`u classi
in modo da avere degli $n_i \sim 5\div 10$).

Tornando al problema iniziale, per la verifica dell'ipotesi
statistica che i dati vengano dalla distribuzione usata per
il calcolo delle $A_i$ basta:
\begin{itemize}
\item fissare arbitrariamente un livello di probabilit\`a
  che rappresenti il confine tra eventi ammissibili
  nell'ipotesi della pura casualit\`a ed eventi invece tanto
  improbabili da far supporre che il loro verificarsi sia
  dovuto non a fluttuazioni statistiche, ma al non essere
  verificate le ipotesi fatte in partenza (il provenire i
  dati dalla distribuzione nota a priori): ad esempio il
  95\% o il 99\%.
\item Cercare nelle apposite tabelle\/\footnote{Alcuni
    valori numerici di questo tipo sono tabulati
    nell'appendice \ref{ch:f.tabelle}.  \`E bene anche
    ricordare che quando il numero di gradi di libert\`a $N$
    \`e superiore a 30 si pu\`o far riferimento alla
    distribuzione normale con media $N$ ed errore quadratico
    medio $\sqrt{2N}$; e che, gi\`a per piccoli $N$,
    $\sqrt{2\chi^2}$ \`e approssimativamente normale con
    media $\sqrt{2N-1}$ e varianza 1.} il valore di taglio
  corrispondente alla coda superiore della distribuzione del
  $\chi^2$ ad $M - 1$ gradi di libert\`a avente area pari al
  livello di confidenza desiderato; ossia quell'ascissa
  $\xi$ che lascia alla propria sinistra, sotto la curva
  della distribuzione del $\chi^2$ ad $M - 1$ gradi di
  libert\`a, un'area pari a tale valore.
\item Calcolare $X$; ed infine rigettare l'ipotesi (al
  livello di confidenza prescelto) perch\'e incompatibile
  con i dati raccolti, se $X$ risultasse superiore a $\xi$
  (o, altrimenti, considerare l'ipotesi compatibile con i
  dati al livello di confidenza prescelto e quindi
  accettarla).
\end{itemize}

Quanto detto a proposito della particolare distribuzione del
$\chi^2$ da usare per il la verifica della nostra ipotesi,
per\`o, \`e valido solo se le caratteristiche della
distribuzione teorica con cui confrontare i nostri dati sono
note a priori; se, invece, $R$ parametri da cui essa dipende
fossero stati stimati a partire dai dati, il numero di gradi
di libert\`a sarebbe inferiore e pari ad $M - R - 1$.

Cos\`\i\ se le $p_i$ sono state ricavate integrando sulle
classi di frequenza una distribuzione normale la cui media e
la cui varianza siano state a loro volta ottenute dal
campione istogrammato, il numero di gradi di libert\`a,
essendo $R=2$, sarebbe pari a $M - 3$.

\begin{figure}[hbtp]
  \vspace*{2ex}
  \begin{center} {
    \input{chicdf.pstex_t}
  } \end{center}
  \caption[La funzione di distribuzione del $\chi^2$]
  {L'integrale da $x$ a $+\infty$ della funzione di
    frequenza del $\chi^2$, per alcuni valori del parametro
    $N$.}
  \label{fig:chicdf}
\end{figure}

\begin{figure}[hbtp]
  \vspace*{2ex}
  \begin{center} {
    \input{chirid.pstex_t}
  } \end{center}
  \caption[La funzione di distribuzione del $\chi^2$
  ridotto]
  {I valori del $\chi^2$ ridotto ($\chi^2/N$) che
    corrispondono, per differenti gradi di libert\`a $N$, ad
    un certo livello di confidenza.}
  \label{fig:chirid}
\end{figure}

Per dare un'idea dei valori del $\chi^2$ che corrispondono
al rigetto di una ipotesi (ad un certo livello di
confidenza), e senza ricorrere alle tabelle numeriche, nella
figura \ref{fig:chicdf} sono riportati in grafico i valori
$P$ dell'integrale da $x$ a $+\infty$ della funzione di
frequenza del $\chi^2$ (ovvero il complemento ad uno della
funzione di distribuzione), per alcuni valori del parametro
$N$.

Le curve della figura \ref{fig:chirid} permettono invece di
identificare (per differenti scelte del livello di
confidenza $\varepsilon$) i corrispondenti valori di taglio
del $\chi^2$ \emph{ridotto} --- ovvero del rapporto
$\chi^2/N$ tra esso ed il numero di gradi di libert\`a $N$.
Insomma, ogni punto di queste curve al di sopra di
un'ascissa (intera) $N$ ha come ordinata un numero $X/N$
tale che l'integrale da $X$ a $+\infty$ della funzione di
frequenza del $\chi^2$ ad $N$ gradi di libert\`a sia uguale
ad $\varepsilon$.%
\index{compatibilit\`a!con una distribuzione|)}

\subsection[Il metodo del minimo $\chi^2$]{Il metodo
  del minimo $\boldsymbol{\chi}^{\boldsymbol{2}}$}%
\index{metodo!del minimo $\chi^2$|(}
Supponiamo di sapere a priori che i nostri dati istogrammati
debbano seguire una data distribuzione, ma che essa dipenda
da $R$ parametri incogniti che dobbiamo stimare a partire
dai dati stessi; visto che l'accordo tra i dati e la
distribuzione \`e dato dalla $X$ definita nella
\eqref{eq:12.chi2fit}, ed \`e tanto migliore quanto pi\`u il
valore ottenuto per essa \`e basso, un metodo plausibile di
stima potrebbe essere quello di trovare per quali valori dei
parametri stessi la $X$ \`e minima (\emph{metodo del minimo}
$\chi^2$).

Indicando con $\alpha_k$ ($k=1,\ldots,R$) i parametri da
stimare, ognuna delle $p_i$ sar\`a esprimibile in funzione
delle $\alpha_k$; ed imponendo che le derivate prime della
$X$ rispetto ad ognuna delle $\alpha_k$ siano tutte nulle
contemporaneamente, otteniamo
\begin{gather}
  \frac{\partial X}{\partial \alpha_k} \; = \;
    \sum_{i=1}^M \frac{-2 \left(n_i - N p_i \right) N^2
    p_i - N \left( n_i - N p_i \right)^2}{N^2 {p_i}^2}
    \, \frac{\partial p_i}{\partial \alpha_k}
    \; = \; 0 \peq , \notag \\
  \intertext{ossia}
  - \frac{1}{2} \, \frac{\partial X}{\partial \alpha_k}
    \; = \; \sum_{i=1}^M \left[ \frac{n_i - N p_i}{p_i}
    + \frac{\left( n_i - N p_i \right)^2}{2 N {p_i}^2}
    \right] \frac{\partial p_i}{\partial \alpha_k} \; =
    \; 0 \peq . \label{eq:12.michi1}
\end{gather}

L'insieme delle \eqref{eq:12.michi1} costituisce un sistema
di $R$ equazioni, nelle $R$ incognite $\alpha_k$, che ci
permetter\`a di stimarne i valori (salvo poi, nel caso il
sistema delle \eqref{eq:12.michi1} abbia pi\`u di una
soluzione, controllare quali di esse corrispondono in
effetti ad un minimo e quale tra queste ultime corrisponde
al minimo assoluto); le condizioni sotto le quali il metodo
\`e applicabile sono quelle gi\`a enunciate in
precedenza\/\footnote{Se la prima di esse non si pu\`o
  ritenere accettabile, delle equazioni ancora valide ma
  pi\`u complesse si possono ottenere dalla
  \eqref{eq:12.chi2fit} sostituendo $N p_i (1 - p_i)$ al
  posto di $N p_i$ nel denominatore.}, ossia ${p_i}^2 \ll
p_i$ e $n_i \gtrsim 5$.

In genere per\`o si preferisce servirsi, in luogo delle
equazioni \eqref{eq:12.michi1}, di una forma semplificata,
ottenuta trascurando il secondo termine nella parentesi
quadra: che, si pu\`o dimostrare, \`e molto inferiore al
primo per grandi $N$ (infatti il rapporto tra i due termini
vale
\begin{equation*}
  \frac{ \left( n_i - N p_i \right)^2 }{ 2 N {p_i}^2 } \,
    \frac{ p_i }{ n_i - N p_i } \; = \; \frac{ n_i - N p_i
    }{ 2 N p_i } \; = \; \frac{1}{2 p_i} \left(
    \frac{n_i}{N} - p_i \right)
\end{equation*}
e converge ovviamente a zero all'aumentare di $N$); e
risolvere, insomma, il sistema delle
\begin{equation} \label{eq:12.michi2}
  \sum_{i=1}^M \left( \frac{n_i - N p_i}{p_i} \right)
    \frac{\partial p_i}{\partial \alpha_k} = 0
\end{equation}
(metodo \emph{semplificato} del minimo $\chi^2$).

Si pu\`o dimostrare che le soluzioni $\bar \alpha_k$ del
sistema delle \eqref{eq:12.michi2} tendono stocasticamente
ai valori veri $\alpha_k^*$ (in assenza di errori
sistematici) al crescere di $N$; inoltre il valore di $X$
calcolato in corrispondenza dei valori ricavati
$\bar \alpha_k$ d\`a, se rapportato alla distribuzione del
$\chi^2$ con $M - R - 1$ gradi di libert\`a, una misura
della bont\`a della soluzione stessa.

Ora, le equazioni \eqref{eq:12.michi2} si possono scrivere
anche
\begin{gather*}
  \sum_{i=1}^M \left( \frac{n_i - N p_i}{p_i} \right)
    \frac{\partial p_i}{\partial \alpha_k} =
    \sum_{i=1}^M \frac{n_i}{p_i} \, \frac{\partial
    p_i}{\partial \alpha_k} - N \sum_{i=1}^M
    \frac{\partial p_i}{\partial \alpha_k} \\
  \intertext{e si possono ulteriormente semplificare,
    visto che l'ultimo termine si annulla, essendo}
  \sum_{i=1}^M \frac{\partial p_i}{\partial \alpha_k}
    \; = \; \frac{\partial}{\partial \alpha_k}
    \sum_{i=1}^M p_i \; = \; \frac{\partial}{\partial
    \alpha_k} \, 1 \; \equiv \; 0
\end{gather*}
se si fa l'ulteriore ipotesi che l'intervallo dei valori
indagati copra, anche approssimativamente, tutti quelli in
pratica permessi; per cui il sistema di equazioni da
risolvere \`e in questo caso quello delle
\begin{equation} \label{eq:12.michi4}
  \sum_{i=1}^M \frac{n_i}{p_i} \, \frac{\partial
    p_i}{\partial \alpha_k} = 0 \peq .
\end{equation}

\index{massima verosimiglianza, metodo della|(}%
Per la stima di parametri incogniti a partire da dati
misurati abbiamo gi\`a affermato che teoricamente \`e da
preferire il metodo della massima verosimiglianza, le cui
soluzioni sono quelle affette, come sappiamo, dal minimo
errore casuale (almeno asintoticamente); in questo caso
particolare (dati in istogramma), come lo si dovrebbe
applicare?  Se le misure sono indipendenti, la probabilit\`a
di avere $n_i$ eventi nella generica classe di frequenza \`e
data da $p_i^{n_i}$; la funzione di
verosimiglianza\/\footnote{Per essere precisi, la
  probabilit\`a che $n_1$ misure si trovino nella prima
  classe di frequenza, $n_2$ nella seconda e cos\`\i\ via,
  \`e dato dalla espressione \eqref{eq:12.michi3}
  moltiplicata per il numero di modi differenti in cui $N$
  oggetti possono essere suddivisi in $M$ gruppi composti da
  $n_1, n_2,\ldots,n_M$ oggetti rispettivamente (numero
  delle \emph{partizioni ordinate});%
  \index{partizioni ordinate}
  questo vale, come mostrato nel paragrafo
  \ref{ch:a.parord}, $N! / (n_1!\, n_2!\cdots n_M!)$, e
  rappresenta un fattore costante che non incide nella
  ricerca del massimo della \eqref{eq:12.michi3}.}  da
\begin{gather}
  \mathcal{L}(\alpha_1,\ldots,\alpha_R) = \prod_{i=1}^M
    p_i^{n_i} \label{eq:12.michi3} \\
  \intertext{ed il suo logaritmo da}
  \ln \mathcal{L} = \sum_{i=1}^M \left( n_i \cdot \ln
    p_i \right) \peq . \notag
\end{gather}

La soluzione di massima verosimiglianza (e quindi di minima
varianza) si trova cercando il massimo di $\ln \mathcal{L}$:
e risolvendo quindi il sistema delle
\begin{equation*}
  \frac{\partial}{\partial \alpha_k} \, \ln \mathcal{L}
    \; = \; \sum_{i=1}^M n_i \, \frac{1}{p_i} \,
    \frac{\partial p_i}{\partial \alpha_k} \; = \; 0 \peq ;
\end{equation*}
in questo caso, vista l'equazione \eqref{eq:12.michi4} in
precedenza ricavata, i due metodi (della massima
verosimiglianza e del minimo $\chi^2$ semplificato)
conducono dunque \emph{alla stessa soluzione}.%
\index{massima verosimiglianza, metodo della|)}%
\index{metodo!del minimo $\chi^2$|)}

\subsection{Test di omogeneit\`a per dati raggruppati}%
\index{compatibilit\`a!tra dati sperimentali|(}%
\index{omogeneit\`a, test di|see{compatibilit\`a tra dati sperimentali}}
Supponiamo di avere a disposizione $Q$ campioni di dati,
indipendenti l'uno dall'altro e composti da $n_1,
n_2,\ldots, n_Q$ elementi rispettivamente; e, all'interno di
ognuno di tali campioni, i dati siano suddivisi nei medesimi
$P$ gruppi: indichiamo infine col simbolo $\nu_{ij}$ il
numero di dati appartenenti al gruppo $i$-esimo all'interno
del campione $j$-esimo.

Per fare un esempio, i campioni si potrebbero riferire alle
regioni italiane e i gruppi al livello di istruzione
(licenza elementare, media, superiore, laurea): cos\`\i\ che
i $\nu_{ij}$ rappresentino il numero di persone, per ogni
livello di istruzione, residenti in ogni data regione;
oppure (e questo \`e un caso che si presenta frequentemente
nelle analisi fisiche) si abbiano vari istogrammi
all'interno di ognuno dei quali i dati siano stati
raggruppati secondo le medesime classi di frequenza:%
\index{classi di frequenza} allora i $\nu_{ij}$ saranno il
numero di osservazioni che cadono in una determinata classe
in ogni istogramma.

Il problema che ci poniamo \`e quello di verificare
l'ipotesi che tutti i campioni provengano dalla stessa
popolazione e siano perci\`o compatibili tra loro
(\emph{test di omogeneit\`a}).  Indichiamo con il simbolo
$N$ il numero totale di dati a disposizione; e con $m_i$
(con $i=1,\ldots,P$) il numero totale di dati che cadono
nell'$i$-esimo gruppo in tutti i campioni a disposizione.

\begin{table}[htbp]
  \vspace*{2ex}
  \begin{center}
    \begin{tabular}{|r|ccccc|c|}
      \cline{2-6}
      \multicolumn{1}{c}{\tabtop\tabbot} &
        \multicolumn{5}{|c|}{Campioni} \\
      \hline
      & $\nu_{11}$ & $\nu_{12}$ & $\nu_{13}$ & $\cdots$
        & $\nu_{1Q}$ & $m_1$\tabtop \\
      & $\nu_{21}$ & $\nu_{22}$ & $\cdots$ & $\cdots$ &
        $\nu_{2Q}$ & $m_2$ \\
      Gruppi & $\nu_{31}$ & $\cdots$ & $\cdots$ &
        $\cdots$ & $\cdots$ & $m_3$ \\
      & $\cdots$ & $\cdots$ & $\cdots$ & $\cdots$ &
        $\cdots$ & $\cdots$ \\
      & $\nu_{P1}$ & $\nu_{P2}$ & $\cdots$ & $\cdots$ &
        $\nu_{PQ}$ & $m_P$\tabbot \\
      \hline
      \multicolumn{1}{c|}{} & $n_1$ & $n_2$ & $n_3$ &
        $\cdots$ & $n_Q$ & $N$\tabtop\tabbot \\
      \cline{2-7}
    \end{tabular}
  \end{center}
  \caption{Un esempio delle cosiddette \emph{tabelle
    delle contingenze}.}
  \label{tab:12.contin}
\end{table}

\`E consuetudine che dati di questo genere siano
rappresentati in una tabella del tipo della
\ref{tab:12.contin}, che si chiama \emph{tabella delle
  contingenze};%
\index{contingenze, tabella delle}
e risulta ovviamente
\begin{align*}
  n_j &= \sum_{i=1}^P \nu_{ij} && \qquad \qquad
    (j=1,2,\ldots,Q) \peq ; \\[1ex]
  m_i &= \sum_{j=1}^Q \nu_{ij} && \qquad \qquad
    (i=1,2,\ldots,P) \peq ; \\[1ex]
  N &= \sum_{j=1}^Q n_j = \sum_{i=1}^P m_i = \sum_{i,j}
    \nu_{ij} \peq .
\end{align*}

Vogliamo ora dimostrare che la variabile casuale
\begin{equation} \label{eq:12.chiomo}
  X = N \left[ \sum_{i,j} \frac{\left( \nu_{ij}
  \right)^2}{m_i \, n_j} - 1 \right]
\end{equation}
\`e distribuita come il $\chi^2$ a $(P-1)(Q-1)$ gradi di
libert\`a: a questo scopo supponiamo innanzi tutto sia
valida l'ipotesi che i dati provengano tutti dalla medesima
popolazione, ed indichiamo con i simboli $p_i$ e $q_j$ le
probabilit\`a che un componente di tale popolazione scelto a
caso cada rispettivamente nel gruppo $i$-esimo o nel
campione $j$-esimo; e sappiamo inoltre che (ammessa per\`o
vera l'ipotesi che \emph{tutti} i campioni provengano dalla
stessa distribuzione) questi due eventi sono statisticamente
indipendenti: per cui ognuno dei dati ha probabilit\`a
complessiva $p_i q_j$ di cadere in una delle caselle della
tabella delle contingenze.

Possiamo stimare i $P$ valori $p_i$ a partire dai dati
sperimentali: si tratta in realt\`a solo di $P-1$ stime
\emph{indipendenti}, perch\'e, una volta ricavate le prime
$P-1$ probabilit\`a, l'ultima di esse risulter\`a
univocamente determinata dalla condizione che la somma
complessiva valga 1.  Analogamente possiamo anche stimare i
$Q$ valori $q_j$ dai dati sperimentali, e si tratter\`a in
questo caso di effettuare $Q-1$ stime indipendenti.

Le stime di cui abbiamo parlato sono ovviamente
\begin{align} \label{eq:12.piqj}
  p_i &= \frac{m_i}{N} &&\text{e} & q_j &=
    \frac{n_j}{N}
\end{align}
e, applicando le conclusioni del paragrafo precedente
(l'equazione \eqref{eq:12.chi2fit}), la variabile
\begin{align*}
  X &= \sum_{i,j} \frac{\left( \nu_{ij} - N p_i q_j
    \right)^2}{N p_i q_j} \\[1ex]
  &= \sum_{i,j} \left[ \frac{\left( \nu_{ij}
    \right)^2}{N p_i q_j} - 2 \nu_{ij} + N p_i q_j
    \right] \\[1ex]
  &= \sum_{i,j} \frac{\left( \nu_{ij} \right)^2}{N p_i
    q_j} -2N + N \\[1ex]
  &= \sum_{i,j} \frac{\left( \nu_{ij} \right)^2}{N p_i
    q_j} - N
\end{align*}
deve essere distribuita come il $\chi^2$.

Sostituendo in quest'ultima espressione i valori
\eqref{eq:12.piqj} per $p_i$ e $q_j$, essa si riduce alla
\eqref{eq:12.chiomo}; il numero di gradi di libert\`a \`e
pari al numero di contributi sperimentali indipendenti, $PQ
- 1$ (c'\`e il vincolo che la somma totale sia $N$),
diminuito del numero $(P-1) + (Q-1)$ di parametri stimato
sulla base dei dati: ovverosia proprio $(P-1) (Q-1)$ come
anticipato.%
\index{compatibilit\`a!tra dati sperimentali|)}

\subsection{Un esempio: diffusione elastica protone-protone}
\begin{figure}[hbtp]
  \vspace*{2ex}
  \begin{center} {
    \input{scat.pstex_t}
  } \end{center}
  \caption{Urto elastico protone-protone.}
  \label{fig:12.scat}
\end{figure}
Nella figura \ref{fig:12.scat} \`e schematicamente
rappresentato un processo di urto elastico tra due
particelle, una delle quali sia inizialmente ferma; dopo
l'urto esse si muoveranno lungo traiettorie rettilinee ad
angoli $\vartheta_1$ e $\vartheta_2$ rispetto alla direzione
originale della particella urtante.

Gli angoli $\vartheta_i$ vengono misurati; supponendo che il
processo di misura introduca errori che seguono la
distribuzione normale ed abbiano una entit\`a che (per
semplificare le cose) assumiamo sia costante, nota ed
indipendente dall'ampiezza dell'angolo, vogliamo verificare
l'ipotesi che le due particelle coinvolte nel processo
d'urto siano di massa uguale (ad esempio che siano entrambe
dei protoni).

La prima cosa da fare \`e quella di ricavare dai dati
misurati $\vartheta_i$, che per ipotesi hanno una funzione
di frequenza
\begin{equation*}
  f(\vartheta; \vartheta^*, \sigma) \; = \; \frac{1}{\sigma
    \sqrt{2 \pi}} \, e^{- \frac{1}{2} \left( \frac{\vartheta
        - \vartheta^*}{\sigma} \right)^2}
\end{equation*}
una stima dei valori veri $\vartheta^*$.  Il logaritmo della
funzione di verosimiglianza \`e dato da
\begin{equation*}
  \ln \mathcal{L} \; = \; - 2 \ln \left( \sigma \sqrt{2 \pi}
    \right) - \frac{1}{2} \left( \frac{\vartheta_1 -
        \vartheta_1^*}{\sigma} \right)^2  - \frac{1}{2}
    \left( \frac{\vartheta_2 - \vartheta_2^*}{\sigma}
    \right)^2 \peq ;
\end{equation*}
ma le variabili $\vartheta_1$ e $\vartheta_2$ \emph{non sono
  indipendenti}, visto che il processo deve conservare sia
energia che quantit\`a di moto.  Ammessa vera l'ipotesi che
le due particelle abbiano uguale massa (e restando nel
limite non-relativistico), le leggi di conservazione
impongono il vincolo che l'angolo tra le due particelle dopo
l'urto sia di 90\gra (o, in radianti, $\pi / 2$); usando il
metodo dei moltiplicatori di Lagrange, la funzione da
massimizzare \`e
\begin{equation*}
  \varphi( \vartheta_1^*, \vartheta_2^*, \lambda) \; =
  \; - \frac{1}{2} \left( \frac{\vartheta_1 -
      \vartheta_1^*}{\sigma} \right)^2 - \frac{1}{2}
  \left( \frac{\vartheta_2 - \vartheta_2^*}{\sigma}
  \right)^2 + \lambda \left( \vartheta_1^* +
    \vartheta_2^* - \frac{\pi}{2} \right)
\end{equation*}
e, annullando contemporaneamente le sue derivate rispetto
alle tre variabili, si giunge al sistema
\begin{equation*}
  \left\{
    \begin{array}{cclcc}
      \dfrac{\partial \varphi}{\partial \lambda} & = &
      \vartheta_1^* + \vartheta_2^* - \dfrac{\pi}{2} & = &
      0 \\[2.5ex]
      \dfrac{\partial \varphi}{\partial \vartheta_1^*} & = &
      \dfrac{1}{\sigma^2} \left( \vartheta_1 - \vartheta_1^*
      \right) + \lambda & = & 0 \\[2.5ex]
      \dfrac{\partial \varphi}{\partial \vartheta_2^*} & = &
      \dfrac{1}{\sigma^2} \left( \vartheta_2 - \vartheta_2^*
      \right) + \lambda & = & 0
    \end{array}
  \right.
\end{equation*}
Eliminando $\lambda$ dalle ultime due equazioni otteniamo
\begin{gather*}
  \vartheta_1 - \vartheta_1^* \; = \; \vartheta_2 -
  \vartheta_2^* \\
  \intertext{e, sostituendo l'espressione per
    $\vartheta_2^*$ ricavata dalla prima equazione,}
  \vartheta_1 - \vartheta_1^* \; = \; \vartheta_2 - \left(
    \frac{\pi}{2} - \vartheta_1^* \right)
\end{gather*}
per cui le due stime di massima verosimiglianza sono
\begin{equation*}
  \left\{
    \begin{array}{ccl}
      \hat \vartheta_1^* & = & \vartheta_1 + \dfrac{1}{2}
      \left( \dfrac{\pi}{2} - \vartheta_1 - \vartheta_2
      \right) \\[2.5ex]
      \hat \vartheta_2^* & = & \vartheta_2 + \dfrac{1}{2}
      \left( \dfrac{\pi}{2} - \vartheta_1 - \vartheta_2
      \right)
    \end{array}
  \right.
\end{equation*}

Ammesso che queste soluzioni siano buone stime dei valori
veri, la variabile casuale
\begin{equation*}
  X \; = \; \left( \frac{ \vartheta_1 -
      \vartheta_1^*}{\sigma} \right)^2 + \left( \frac{
      \vartheta_2 - \vartheta_2^*}{\sigma} \right)^2 \;
  = \; \frac{1}{2 \sigma^2} \left( \frac{\pi}{2} -
    \vartheta_1 - \vartheta_2 \right)^2
\end{equation*}
\`e distribuita come il $\chi^2$ ad un grado di libert\`a
(due contributi, un vincolo); ed il valore di $X$
confrontato con le tabelle del $\chi^2$ pu\`o essere usato
per la verifica dell'ipotesi.

\section{Compatibilit\`a con un valore prefissato}%
\index{compatibilit\`a!con un valore|(}
Un altro caso che frequentemente si presenta \`e il
seguente: si vuole controllare se un determinato valore
numerico, a priori attribuibile alla grandezza fisica in
esame, \`e o non \`e confermato dai risultati della misura;
cio\`e se quel valore \`e o non \`e \emph{compatibile} con i
nostri risultati --- pi\`u precisamente, a che livello di
probabilit\`a (o, per usare la terminologia statistica, a
che \emph{livello di confidenza}) \`e con essi compatibile.

Ammettiamo che gli errori di misura seguano la legge
normale; sappiamo che la probabilit\`a per il risultato di
cadere in un qualunque intervallo prefissato dell'asse reale
si pu\`o calcolare integrando la funzione di Gauss fra gli
estremi dell'intervallo stesso.  Riferiamoci per comodit\`a
alla variabile \emph{scarto normalizzato}
\begin{equation*}
  t = \frac{x - E(x)}{\sigma}
\end{equation*}
che sappiamo gi\`a dal paragrafo \ref{ch:9.scanor} essere
distribuita secondo una legge che \`e indipendente
dall'entit\`a degli errori di misura.

Se fissiamo arbitrariamente un numero positivo $\tau$,
possiamo calcolare la probabilit\`a che si verifichi
l'evento casuale consistente nell'ottenere, in una
particolare misura, un valore di $t$ che in modulo
superi $\tau$; come esempio particolare, le condizioni
$|t| > 1$ o $|t|> 2$ gi\`a sappiamo che si verificano
con probabilit\`a rispettivamente del 31.73\% e del
4.55\%, visto che l'intervallo $-1 \leq t \leq 1$
corrisponde al 68.27\% dell'area della curva normale, e
quello $-2 \leq t \leq 2$ al 95.45\% .

Se consideriamo poi un campione di $N$ misure indipendenti,
avente valore medio $\bar x$ e proveniente da questa stessa
popolazione di varianza $\sigma^2$, \`e immediato capire
come la variabile
\begin{equation*}
  t = \frac{\bar x - E(x)}{\dfrac{\sigma}{\sqrt{N}}}
\end{equation*}
soddisfer\`a a queste stesse condizioni: accadr\`a cio\`e
nel 31.73\% dei casi che $|t|$ sia maggiore di $\tau=1$, e
nel 4.55\% dei casi che $|t|$ sia superiore a $\tau=2$ .

Per converso, se fissiamo arbitrariamente un qualunque
valore ammissibile $P$ per la probabilit\`a, possiamo
calcolare in conseguenza un numero $\tau$, tale che la
probabilit\`a di ottenere effettivamente da un particolare
campione un valore dello scarto normalizzato $t$ superiore
ad esso (in modulo) sia data dal numero $P$.  Ad esempio,
fissato un valore del 5\% per $P$, il limite per $t$ che se
ne ricava \`e $\tau = 1.96$: insomma
\begin{equation*}
  \int_{-1.96}^{+1.96} \frac{1}{\sqrt{2 \pi}} \,
    e^{- \frac{1}{2} t^2} \de t
    \; = \; 0.95
\end{equation*}
e solo nel cinque per cento dei casi si ottiene un valore di
$t$ che supera (in modulo) 1.96.

Se si fissa per convenzione un valore della probabilit\`a
che indichi il confine tra un avvenimento accettabile ed uno
inaccettabile nei limiti della pura casualit\`a, possiamo
dire che l'ipotesi consistente nell'essere un certo numero
$\xi$ il valore vero della grandezza misurata sar\`a
compatibile o incompatibile con i nostri dati a seconda che
lo scarto normalizzato
\begin{equation*}
  t = \frac{\bar x - \xi}{\dfrac{\sigma}{\sqrt{N}}}
\end{equation*}
relativo a tale numero sia, in valore assoluto, inferiore o
superiore al valore di $\tau$ che a quella probabilit\`a
corrisponde; e diremo che la compatibilit\`a (o
incompatibilit\`a) \`e riferita a quel certo livello di
confidenza prescelto.

La difficolt\`a \`e che tutti questi ragionamenti
coinvolgono una quantit\`a numerica (lo scarto quadratico
medio) relativa \emph{alla popolazione} e per ci\`o stesso
in generale ignota; in tal caso, per calcolare lo scarto
normalizzato relativo ad un certo valore numerico $\xi$ non
possiamo che servirci, in luogo di $\sigma$, della
corrispondente stima ricavata dal campione, $s$:
\begin{equation*}
  t = \frac{\bar x - \xi}{\dfrac{s}{\sqrt{N}}}
\end{equation*}
e quindi si deve presupporre di avere un campione di
dimensioni tali che questa stima si possa ritenere
ragionevole, ossia sufficientemente vicina ai corrispondenti
valori relativi alla popolazione a meno di fluttuazioni
casuali abbastanza poco probabili.

In generale si ammette che \emph{almeno 30 dati} siano
necessari perch\'e questo avvenga: in corrispondenza a tale
dimensione del campione, l'errore della media \`e circa 5.5
volte inferiore a quello dei dati; e l'errore relativo di
$s$ \`e approssimativamente del 13\%.

Bisogna anche porre attenzione alla esatta natura
dell'ipotesi che si intende verificare.  Per un valore
limite di $\tau = 1.96$ abbiamo visto che il 95\% dell'area
della curva normale \`e compreso tra $-\tau$ e $+\tau$:
superiormente a $+\tau$ si trova il 2.5\% di tale area; ed
anche inferiormente a $-\tau$ se ne trova un'altra porzione
pari al 2.5\%.

\begin{table}[htbp]
  \vspace*{2ex}
  \begin{center}
    \begin{tabular}{rcc}
      \toprule
      \multicolumn{1}{c}{$P$ (\%)} & $\tau_2$ & $\tau_1$ \\
      \midrule
      10.0 & 1.64485 & 1.28155 \\
      5.0  & 1.95996 & 1.64485 \\
      2.0  & 2.32635 & 2.05375 \\
      1.0  & 2.57583 & 2.32635 \\
      0.5  & 2.81297 & 2.57583 \\
      0.2  & 3.09023 & 2.87816 \\
      0.1  & 3.29053 & 3.09023 \\
      \bottomrule
    \end{tabular}
  \end{center}
  \caption{Alcuni valori della
    probabilit\`a $P$ e dei corrispondenti limiti
    $\tau$ sullo scarto normalizzato, per verifiche
    two-tailed ($\tau_2$) o one-tailed ($\tau_1$).}
  \label{tab:12.conlev1}
\end{table}

\begin{table}[hbtp]
  \vspace*{2ex}
  \begin{center}
    \begin{tabular}{crr}
      \toprule
      $\tau$ & \multicolumn{1}{c}{$P_2$ (\%)} &
      \multicolumn{1}{c}{$P_1$ (\%)} \\
      \midrule
      0.5 & 61.708 & 30.854 \\
      1.0 & 31.731 & 15.866 \\
      1.5 & 13.361 &  6.681 \\
      2.0 &  4.550 &  2.275 \\
      2.5 &  1.242 &  0.621 \\
      3.0 &  0.270 &  0.135 \\
      \bottomrule
    \end{tabular}
  \end{center}
  \caption{I valori della
    probabilit\`a per verifiche two-tailed ($P_2$) ed
    one-tailed ($P_1$) che corrispondono a valori
    prefissati dello scarto normalizzato $\tau$.}
  \label{tab:12.conlev2}
\end{table}

Se l'ipotesi da verificare riguarda l'\emph{essere
  differenti tra loro} due entit\`a (il presupposto valore
vero della grandezza misurata e la media aritmetica dei
nostri dati, nell'esempio precedente) quel valore di $\tau$
corrisponde in effetti ad una verifica relativa ad un
livello di confidenza del 5\% (usando il termine inglese,
stiamo effettuando un \emph{two-tailed test});%
\index{two-tailed test}
ma se l'ipotesi riguarda l'essere un valore numerico
\emph{superiore} (od \emph{inferiore}) alla nostra media
aritmetica (ad esempio, i dati misurati potrebbero essere
relativi al rendimento di una macchina, e si vuole
verificare l'ipotesi che tale rendimento misurato sia
superiore ad un valore prefissato), allora un limite $\tau =
1.96$ corrisponde in effetti ad un livello di confidenza del
2.5\% (\emph{one-tailed test}):%
\index{one-tailed test}
nell'esempio fatto, soltanto l'intervallo $[-\infty, -\tau]$
deve essere preso in considerazione per il calcolo della
probabilit\`a.  Alcuni limiti relativi a diversi livelli di
confidenza si possono trovare nelle tabelle
\ref{tab:12.conlev1} e \ref{tab:12.conlev2}; altri si
possono facilmente ricavare dalle tabelle dell'appendice
\ref{ch:f.tabelle}.%
\index{compatibilit\`a!con un valore|)}

\section{I piccoli campioni e la distribuzione di
  Student}%
\index{piccoli campioni|(}
Cosa si pu\`o fare riguardo alla verifica di ipotesi
statistiche come quella (considerata nel paragrafo
precedente) della compatibilit\`a del risultato delle misure
con un valore noto a priori, quando si abbiano a
disposizione solamente piccoli campioni?  Ci riferiamo,
pi\`u esattamente, a campioni costituiti da un numero di
dati cos\`\i\ esiguo da farci ritenere che non si possa
ottenere da essi con ragionevole probabilit\`a una buona
stima delle varianze delle rispettive popolazioni (sempre
per\`o supposte normali).

\index{distribuzione!di Student|(emidx}%
Sia $X$ una variabile casuale distribuita come il $\chi^2$
ad $N$ gradi di libert\`a, ed $u$ una seconda variabile
casuale, indipendente dalla prima, e avente distribuzione
normale standardizzata $N(u;0,1)$; consideriamo la nuova
variabile casuale $t$ definita attraverso la
\begin{equation} \label{eq:12.stuvar}
  t = \frac{u}{\sqrt{\dfrac{X}{N}}} \peq .
\end{equation}

Si pu\`o dimostrare che la funzione densit\`a di
probabilit\`a relativa alla variabile casuale $t$ \`e data
dalla
\begin{equation*}
  f(t;N) = \frac{T_N}{\left( 1 + \frac{t^2}{N}
    \right)^{\frac{N + 1}{2}}}
\end{equation*}
che si chiama \emph{distribuzione di Student ad $N$ gradi di
  libert\`a}.

\begin{figure}[hbtp]
  \vspace*{2ex}
  \begin{center} {
    \input{student.pstex_t}
  } \end{center}
  \caption[La distribuzione di Student]
    {La distribuzione di Student per $N=2$ ed $N=4$,
    confrontata con la funzione normale.}
  \label{fig:12.student}
\end{figure}

Il coefficiente $T_N$ \`e una costante che viene fissata
dalla condizione di normalizzazione; se $N$ viene poi fatto
tendere all'infinito il denominatore della funzione (come si
potrebbe facilmente provare partendo dal limite notevole
\eqref{eq:9.linote}) tende a $e^{t^2/2}$, e dunque la
distribuzione di Student tende alla distribuzione normale
(con media 0 e varianza 1).%
\index{distribuzione!di Student!e distribuzione normale}
Anche la forma della funzione di Student ricorda molto
quella della funzione di Gauss, come appare evidente dalla
figura \ref{fig:12.student}; soltanto, rispetto a dati che
seguano la distribuzione normale, valori elevati dello
scarto sono relativamente pi\`u probabili\/\footnote{Per
  valori di $N \gtrsim 35$ la distribuzione di Student si
  pu\`o approssimare con la distribuzione normale a media 0
  e varianza 1.}.

La distribuzione di Student \`e simmetrica, quindi tutti i
momenti di ordine dispari (compreso il valore medio
$\lambda_1$) sono nulli; mentre la varianza della
distribuzione \`e
\begin{gather*}
  \var(t) = \frac{N}{N-2} \\
  \intertext{(se $N > 2$); ed il coefficiente di
    curtosi vale}
  \gamma_2 = \frac{6}{N-4}
\end{gather*}
(se $N > 4$).

Indicando con $\bar x$ la media aritmetica di un campione di
dimensione $N$, estratto a caso da una popolazione normale
avente valore medio $E(x)$ e varianza $\sigma^2$; e con $s$
la stima della deviazione standard della popolazione
ottenuta dal campione stesso, cio\`e
\begin{gather}
  s^2 = \frac{\sum_i (x_i - \bar x)^2}{N-1} \notag \\
  \intertext{sappiamo, ricordando l'equazione
    \eqref{eq:12.xiis2}, che la variabile casuale}
  X'' = (N - 1) \, \frac{s^2}{\sigma^2} \notag \\
  \intertext{\`e distribuita come il $\chi^2$ ad
    $N - 1$ gradi di libert\`a; inoltre, ovviamente,}
  u = \frac{\bar x - E(x)}{\dfrac{\sigma}{\sqrt{N}}}
    \notag \\
  \intertext{segue la legge normale, con media 0 e
    varianza 1. Di conseguenza la variabile casuale}
  t \; = \; \frac{u}{\sqrt{\dfrac{X''}{N - 1}}} \; = \;
    \frac{\bar x - E(x)}{\dfrac{s}{\sqrt{N}}}
    \label{eq:12.xsecond}
\end{gather}
\emph{segue la distribuzione di Student ad $N - 1$
  gradi di libert\`a}.

Insomma: se i campioni a disposizione non hanno dimensioni
accettabili, una volta calcolato lo scarto normalizzato
relativo alla differenza tra la media di un campione ed un
valore prefissato occorrer\`a confrontare il suo valore con
i limiti degli intervalli di confidenza relativi alla
distribuzione di Student e non alla distribuzione
normale\/\footnote{Per taluni pi\`u usati valori del livello
  di confidenza, i limiti rilevanti si possono trovare
  tabulati anche nell'appendice \ref{ch:f.tabelle}.}.%
\index{distribuzione!di Student|)}%
\index{piccoli campioni|)}

\section{La compatibilit\`a di due valori misurati}%
\index{compatibilit\`a!tra valori misurati|(}
Un altro caso frequente \`e quello in cui si hanno a
disposizione due campioni di misure, e si vuole verificare
l'ipotesi statistica che essi provengano da popolazioni
aventi lo stesso valore medio: un caso particolare \`e
quello dell'ipotesi consistente nell'essere i due campioni
composti da misure della stessa grandezza fisica, che hanno
prodotto differenti stime come effetto della presenza in
entrambi degli errori; errori che assumiamo ancora seguire
la legge normale.

Siano ad esempio un primo campione di $N$ misure $x_i$, ed
un secondo campione di $M$ misure $y_j$; indichiamo con
$\bar x$ e $\bar y$ le medie dei due campioni, con
${\sigma_x}^2$ e ${\sigma_y}^2$ le varianze delle
popolazioni da cui tali campioni provengono, e con $\delta =
\bar x - \bar y$ la differenza tra le due medie.

Sappiamo gi\`a che i valori medi e le varianze delle medie
dei campioni sono legati ai corrispondenti valori relativi
alle popolazioni dalle
\begin{equation*}
  E(\bar x) \; = \; E(x)
    \makebox[4cm]{\mbox{,}}
    E(\bar y) \; = \; E(y)
\end{equation*}
e
\begin{equation*}
  \var (\bar x) \; = \; \frac{{\sigma_x}^2}{N}
    \makebox[4cm]{\mbox{,}}
    \var (\bar y) \; = \; \frac{{\sigma_y}^2}{M}
\end{equation*}
per cui risulter\`a, se i campioni sono tra loro
statisticamente indipendenti e se si ammette valida
l'ipotesi (da verificare) che abbiano la stessa media,
\begin{gather*}
  E(\delta) \; = \; E(\bar x - \bar y)
    \; =  \;E(x) - E(y) = 0 \\
  \intertext{e}
  \var (\delta) \; = \;
    \var (\bar x - \bar y) \; = \;
    \frac{{\sigma_x}^2}{N} + \frac{{\sigma_y}^2}{M} \peq .
\end{gather*}

Inoltre, essendo $\bar x$, $\bar y$ (e quindi $\delta$)
combinazioni lineari di variabili normali, seguiranno
anch'esse la legge normale; e la verifica dell'ipotesi che i
campioni provengano da popolazioni aventi la stessa media si
traduce nella verifica dell'ipotesi che $\delta$ abbia
valore vero nullo.

Tale verifica, essendo $\delta$ distribuita secondo la legge
normale, si esegue come abbiamo visto nel paragrafo
precedente: si fissa arbitrariamente un valore del livello
di confidenza, si determina il corrispondente valore limite
degli scarti normalizzati, e lo si confronta con il valore
di
\begin{equation*}
  \frac{\delta - E(\delta)}{\sigma_\delta} \; = \;
    \frac{\bar x - \bar y}{\sqrt{
    \dfrac{{\sigma_x}^2}{N} +
    \dfrac{{\sigma_y}^2}{M}}} \peq .
\end{equation*}

Ovviamente vale anche qui l'osservazione fatta nel paragrafo
precedente: non conoscendo le deviazioni standard delle
popolazioni, $\sigma_x$ e $\sigma_y$, siamo costretti ad
usare in loro vece le stime ottenute dai campioni, $s_x$ ed
$s_y$; e questo si ammette generalmente lecito quando la
dimensione di entrambi i campioni \`e almeno pari a 30.

\index{piccoli campioni|(}%
In caso contrario, presupponendo cio\`e di avere a
disposizione piccoli campioni per almeno una delle due
variabili, limitiamo la nostra analisi al caso in cui si
sappia con sicurezza che le due popolazioni $x$ ed $y$
\emph{abbiano la stessa varianza},
\begin{equation*}
  {\sigma_x}^2 \; = \; {\sigma_y}^2 \; \equiv \;
    \sigma^2
\end{equation*}
e definiamo la grandezza $S^2$ (\emph{varianza globale} dei
campioni) come
\begin{align*}
  S^2 &= \frac{1}{N + M - 2} \cdot \left\{ \sum_{i=1}^N
    \left[ x_i - \bar x \right]^2 + \sum_{j=1}^M \left[
    y_j - \bar y \right]^2 \right\} \\[2ex]
  &= \frac{(N-1) \, {s_x}^2 + (M-1) \, {s_y}^2}{N + M -
    2} \peq .
\end{align*}

Sapendo, dall'equazione \eqref{eq:12.xiis2}, che le due
variabili
\begin{align*}
  &(N-1) \, \frac{{s_x}^2}{\sigma^2} &&\text{e}
    &&(M-1)\, \frac{{s_y}^2}{\sigma^2}
\end{align*}
sono entrambe distribuite come il $\chi^2$, con $N - 1$ ed
$M - 1$ gradi di libert\`a rispettivamente, sfruttando la
regola di somma enunciata a pagina \pageref{th:12.resochi}
si ricava che la variabile casuale
\begin{equation*}
  X \; = \; \frac{(N-1) \, {s_x}^2 + (M-1) \,
    {s_y}^2}{\sigma^2} \; = \; (N + M - 2) \,
    \frac{S^2}{\sigma^2}
\end{equation*}
\`e distribuita come il $\chi^2$ ad $N + M - 2$ gradi di
libert\`a; essendo inoltre $\delta = \bar x - \bar y$ una
variabile normale con media e varianza date da
\begin{align*}
  E(\delta) &= E(x) - E(y) &&\text{e}
    &{\sigma_\delta}^2 &= \frac{\sigma^2}{N} +
    \frac{\sigma^2}{M}
\end{align*}
la variabile casuale
\begin{equation*}
  u \; = \; \frac{\delta - E(\delta)}{\sigma_\delta} \;
    = \; \frac{(\bar x - \bar y) - \left[ E(x) - E(y)
    \right]}{\sqrt{{\sigma}^2 \left( \dfrac{1}{N} +
    \dfrac{1}{M} \right)}}
\end{equation*}
\`e normale con media 0 e varianza 1.  Per concludere,
\begin{equation} \label{eq:12.tdif2c}
  t \; = \; \frac{u}{\sqrt{\dfrac{X}{N+M-2}}} \; = \;
    \frac{(\bar x - \bar y) - \left[ E(x) - E(y)
    \right]}{\sqrt{S^2 \left( \dfrac{1}{N} +
    \dfrac{1}{M} \right)}}
\end{equation}
deve seguire la distribuzione di Student%
\index{distribuzione!di Student}
con $N + M - 2$ gradi di libert\`a; di conseguenza, per
verificare l'ipotesi che le due popolazioni \emph{normali}
da cui i campioni provengono abbiano la stessa media
\emph{ammesso gi\`a che posseggano la stessa varianza},
basta confrontare con le apposite tabelle della
distribuzione di Student il valore della $t$ ottenuta dalla
\eqref{eq:12.tdif2c} ponendovi $E(x) - E(y) = 0$:
\begin{equation*}
  t = \frac{\bar x - \bar y}{\sqrt{S^2 \left(
    \dfrac{1}{N} + \dfrac{1}{M} \right)} } \peq .
\end{equation*}%
\index{piccoli campioni|)}%
\index{compatibilit\`a!tra valori misurati|)}

\section{La distribuzione di Fisher}%
\index{distribuzione!di Fisher|(}
Sia $X$ una variabile casuale distribuita come il $\chi^2$
ad $M$ gradi di libert\`a; ed $Y$ una seconda variabile
casuale, indipendente dalla prima, distribuita ancora come
il $\chi^2$, ma con $N$ gradi di libert\`a.

La variabile casuale $w$ (sempre positiva) definita in
funzione di esse attraverso la relazione
\begin{equation*}
  w \; = \; \frac{\phantom{t} \dfrac{X}{M}
    \phantom{t}}{\dfrac{Y}{N}}
\end{equation*}
ha una densit\`a di probabilit\`a che segue la cosiddetta
\emph{funzione di frequenza di Fisher} con $M$ ed $N$ gradi
di libert\`a.  La forma analitica della funzione di Fisher
\`e data dalla
\begin{equation} \label{eq:12.fisher}
  F(w;M,N) = K_{MN} \, \frac{w^{\frac{M}{2} - 1}}{\left( M
    w + N \right)^{\frac{M + N}{2}}}
\end{equation}
(nella quale $K_{MN}$ \`e un fattore costante determinato
dalla condizione di normalizzazione).

Il valore medio e la varianza della funzione di frequenza di
Fisher sono dati poi rispettivamente da
\begin{align*}
  E(F) &= \frac{N}{N-2} &&\text{(se $N>2$)} \\
  \intertext{e da}
  \var(F) &= \frac{2 \, N^2 (M + N - 2)}{M (N - 2)^2
    (N - 4)} &&\text{(se $N>4$)} \peq .
\end{align*}

Si pu\`o dimostrare che, se $Y$ \`e una variabile casuale
distribuita come il $\chi^2$ ad $N$ gradi di libert\`a,
\begin{equation*}
  \lim_{N \to +\infty} \frac{Y}{N} = 1
\end{equation*}
in senso statistico (ovverosia la probabilit\`a che il
rapporto $Y/N$ sia differente da 1 tende a zero quando $N$
viene reso arbitrariamente grande); per cui, indicando con
$f(x;M)$ la funzione di frequenza del $\chi^2$ ad $M$ gradi
di libert\`a,
\begin{gather*}
  F(w; M, \infty) \; \equiv \; \lim_{N \to +\infty}
    F(w; M, N) \; = \; \frac{f(w; M)}{M} \peq . \\
  \intertext{Allo stesso modo}
  F(w; \infty, N) \; \equiv \; \lim_{M \to +\infty}
    F(w; M, N) \; = \; \frac{N}{f(w;N)}
\end{gather*}
e quindi esiste una stretta relazione tra le distribuzioni
di Fisher e del chi quadro.

Inoltre, ricordando che, se $u$ \`e una variabile casuale
distribuita secondo la legge normale standardizzata
$N(u;0,1)$, l'altra variabile casuale $u^2$ \`e distribuita
come il $\chi^2$ ad un grado di libert\`a, il rapporto
\begin{gather*}
  w \; = \; \frac{\phantom{t} u^2
    \phantom{t}}{\dfrac{Y}{N}} \\
  \intertext{deve essere distribuito secondo
    $F(w;1,N)$; ma, se definiamo}
  t \; = \; \frac{\phantom{t} u
    \phantom{t}}{\sqrt{\dfrac{Y}{N}}}
\end{gather*}
sappiamo anche dalla \eqref{eq:12.stuvar} che la $t$ segue
la distribuzione di Student ad $N$ gradi di libert\`a.  La
conclusione \`e che il quadrato di una variabile $t$ che
segua la distribuzione di Student ad $N$ gradi di libert\`a
\`e a sua volta distribuito con una densit\`a di
probabilit\`a data da $F(t^2; 1, N)$.

Per terminare, quando i due parametri $M$ ed $N$ (da cui la
funzione di frequenza di Fisher \eqref{eq:12.fisher}
dipende) vengono resi arbitrariamente grandi, essa tende ad
una distribuzione normale; ma la convergenza \`e lenta, e
l'approssimazione normale alla distribuzione di Fisher si
pu\`o pensare in pratica usabile quando sia $M$ che $N$ sono
superiori a 50.%
\index{distribuzione!di Fisher|)}

\subsection{Confronto tra varianze}%
\index{compatibilit\`a!tra varianze|(}
Supponiamo di avere a disposizione due campioni di misure,
che ipotizziamo provenire da due differenti popolazioni che
seguano delle distribuzioni normali.

Siano $M$ ed $N$ le dimensioni di tali campioni, e siano
${\sigma_1}^2$ e ${\sigma_2}^2$ le varianze delle rispettive
popolazioni di provenienza; indichiamo poi con ${s_1}^2$ ed
${s_2}^2$ le due stime delle varianze delle popolazioni
ricavate dai campioni.  Vogliamo ora capire come si pu\`o
verificare l'ipotesi statistica che le due popolazioni
abbiano \emph{la stessa varianza}, ossia che $\sigma_1 =
\sigma_2$.

Ora sappiamo gi\`a dalla equazione \eqref{eq:12.xiis2}
che le due variabili casuali
\begin{align*}
  X &= (M - 1) \, \frac{{s_1}^2}{{\sigma_1}^2}
    &&\text{e} & Y &= (N - 1) \,
    \frac{{s_2}^2}{{\sigma_2}^2}
\end{align*}
sono entrambe distribuite come il $\chi^2$, con $M - 1$
ed $N - 1$ gradi di libert\`a rispettivamente; quindi
la quantit\`a
\begin{equation*}
  w \; = \; \frac{X}{M - 1} \, \frac{N - 1}{Y} \; = \;
    \frac{{s_1}^2}{{\sigma_1}^2} \,
    \frac{{\sigma_2}^2}{{s_2}^2}
\end{equation*}
ha densit\`a di probabilit\`a data dalla funzione di
Fisher con $M - 1$ ed $N - 1$ gradi di libert\`a.

Assunta a priori vera l'ipotesi statistica $\sigma_1 =
\sigma_2$, la variabile casuale
\begin{equation*}
  w = \frac{{s_1}^2}{{s_2}^2}
\end{equation*}
ha densit\`a di probabilit\`a data dalla funzione di Fisher
prima menzionata, $F(w; M-1, N-1)$; per cui, fissato un
livello di confidenza al di l\`a del quale rigettare
l'ipotesi, e ricavato dalle apposite tabelle\/\footnote{Per
  un livello di confidenza pari a 0.95 o 0.99, e per alcuni
  valori dei due parametri $M$ ed $N$, ci si pu\`o riferire
  ancora alle tabelle dell'appendice \ref{ch:f.tabelle}; in
  esse si assume che sia $s_1 > s_2$, e quindi $w > 1$.}  il
valore $W$ che lascia alla propria sinistra, al di sotto
della funzione $F(w; M-1, N-1)$, un'area pari al livello di
confidenza prescelto, si pu\`o escludere che i due campioni
provengano da popolazioni con la stessa varianza se $w >
W$.%
\index{compatibilit\`a!tra varianze|)}

\section{Il metodo di Kolmogorov e Smirnov}%
\index{Kolmogorov e Smirnov, test di|(}%
\index{compatibilit\`a!con una distribuzione|(}%
\index{compatibilit\`a!tra dati sperimentali|(}%
\index{Kolmogorov, Andrei Nikolaevich}
Il \emph{test di Kolmogorov e Smirnov} \`e un metodo di
analisi statistica che permette di confrontare tra loro un
campione di dati ed una distribuzione teorica (oppure due
campioni di dati) allo scopo di verificare l'ipotesi
statistica che la popolazione da cui i dati provengono sia
quella in esame (oppure l'ipotesi che entrambi i campioni
provengano dalla stessa popolazione).

Una caratteristica interessante di questo metodo \`e che
esso non richiede la preventiva, e pi\`u o meno arbitraria,
suddivisione dei dati in classi di frequenza; definendo
queste ultime in modo diverso si ottengono ovviamente, dal
metodo del $\chi^2$, differenti risultati per gli stessi
campioni.

Il test di Kolmogorov e Smirnov si basa infatti sulla
\emph{frequenza cumulativa relativa} dei dati, introdotta
nel paragrafo \ref{def:4.frcure} a pagina
\pageref{def:4.frcure}; e sull'analogo concetto di
\emph{funzione di distribuzione} di una variabile continua
definito nel paragrafo \ref{def:6.fundis} a pagina
\pageref{def:6.fundis}.  Per la compatibilit\`a tra un
campione ed una ipotetica legge che si ritiene possa
descriverne la popolazione di provenienza, e collegata ad
una funzione di distribuzione $\Phi(x)$, bisogna confrontare
la frequenza cumulativa relativa $F(x)$ del campione con
$\Phi(x)$ per ricavare \emph{il valore assoluto del massimo
  scarto tra esse},
\begin{equation*}
  \delta = \max \Bigl\{ \bigl| F(x) - \Phi(x) \bigr|
    \Bigr\} \peq .
\end{equation*}

Si pu\`o dimostrare che, se l'ipotesi da verificare fosse
vera, la probabilit\`a di ottenere casualmente un valore di
$\delta$ non inferiore ad una prefissata quantit\`a
(positiva) $\delta_0$ sarebbe data da
\begin{gather}
  \Pr \left( \delta \ge \delta_0 \right) = F_{\mathrm{KS}}
    \left( \delta'_0 \right) \notag \\
  \intertext{ove $F_{\mathrm{KS}}$ \`e la serie}
  F_{\mathrm{KS}} (x) = 2 \sum_{k=1}^\infty ( -1 )^{k-1}
    e^{-2 \, k^2 x^2} \label{eq:12.kosmfu} \\
  \intertext{e $\delta'_0$ vale}
  \delta'_0 = \left( \sqrt{N} + 0.12 +
    \frac{0.11}{\sqrt{N}} \right) \delta_0 \peq .
    \label{eq:12.kosmva}
\end{gather}

La legge ora enunciata \`e approssimata, ma il test di
Kolmogorov e Smirnov pu\`o essere usato gi\`a per dimensioni
del campione $N$ uguali a 5.  Attenzione per\`o che, se
qualche parametro da cui la distribuzione teorica dipende
\`e stato stimato sulla base dei dati, l'integrale della
densit\`a di probabilit\`a per la variabile $\delta$ di
Kolmogorov e Smirnov \emph{non segue pi\`u} la legge
\eqref{eq:12.kosmfu}: non solo, ma non \`e pi\`u possibile
ricavare teoricamente una funzione che ne descriva il
comportamento in generale (in questi casi, nella pratica, la
distribuzione di $\delta$ viene studiata usando metodi di
Montecarlo).

Se si vogliono invece confrontare tra loro due campioni
indipendenti per verificarne la compatibilit\`a, bisogna
ricavare dai dati il massimo scarto (in valore assoluto),
$\delta$, tra le due frequenze cumulative relative; e
ricavare ancora dalla \eqref{eq:12.kosmfu} la probabilit\`a
che questo possa essere avvenuto (ammessa vera l'ipotesi)
per motivi puramente casuali.  L'unica differenza \`e che la
funzione \eqref{eq:12.kosmfu} va calcolata in un'ascissa
$\delta'_0$ data dalla \eqref{eq:12.kosmva}, nella quale $N$
vale
\begin{equation*}
  N \; = \; \frac{1}{\frac{1}{N_1} + \frac{1}{N_2}} \; = \;
  \frac{N_1 \, N_2}{N_1 + N_2}
\end{equation*}
($N_1$ ed $N_2$ sono le dimensioni dei due campioni).

Oltre al gi\`a citato vantaggio di non richiedere la
creazione di pi\`u o meno arbitrarie classi di frequenza per
raggrupparvi i dati, un'altra caratteristica utile del test
di Kolmogorov e Smirnov \`e quella di essere, entro certi
limiti, \emph{indipendente dalla variabile usata} nella
misura: se al posto di $x$ si usasse, per caratterizzare il
campione, $\ln(x)$ o $\sqrt{x}$, il massimo scarto tra
frequenza cumulativa e funzione di distribuzione rimarrebbe
invariato.

Un altrettanto ovvio svantaggio \`e collegato al fatto che
per valori molto piccoli (o molto grandi) della variabile
casuale usata, qualsiasi essa sia, \emph{tutte} le funzioni
di distribuzione e tutte le frequenze cumulative \emph{hanno
  lo stesso valore} (0, o 1 rispettivamente).  Per questo
motivo il test di Kolmogorov e Smirnov \`e assai sensibile a
differenze nella zona centrale dei dati (attorno al valore
medio), mentre non \`e affatto efficace per discriminare tra
due distribuzioni che differiscano significativamente tra
loro solo nelle code; ad esempio che abbiano lo stesso
valore medio e differente ampiezza.%
\index{compatibilit\`a!tra dati sperimentali|)}%
\index{compatibilit\`a!con una distribuzione|)}%
\index{Kolmogorov e Smirnov, test di|)}

\endinput

% $Id: chapter13.tex,v 1.1 2005/03/01 10:06:08 loreti Exp $

\chapter{La verifica delle ipotesi (II)}
Nel precedente capitolo \ref{ch:12.veripo} abbiamo esaminato
varie tecniche che ci permettono di decidere se una
caratteristica del processo fisico che ha prodotto un
campione di dati \`e o non \`e confermata dai dati stessi;
tutte queste tecniche non sono che casi particolari di una
teoria generale, di cui ora ci occuperemo, senza per\`o
scendere in profondit\`a nei dettagli.

In sostanza, nei vari casi del capitolo \ref{ch:12.veripo},
abbiamo formulato una certa ipotesi $H_0$ sulla natura di un
fenomeno casuale; e, ammesso per assurdo che questa ipotesi
fosse vera, abbiamo associato un ben definito valore della
densit\`a di probabilit\`a ad ogni punto $E$ dello spazio
$\mathcal{S}$ degli eventi.

Se indichiamo con $K$ un valore (arbitrariamente scelto)
della probabilit\`a, \emph{livello di confidenza} nel
linguaggio statistico, abbiamo in sostanza diviso
$\mathcal{S}$ in due sottoinsiemi esclusivi ed esaurienti:
uno $\mathcal{R}$ di eventi con probabilit\`a complessiva $1
- K$, ed uno $\mathcal{A} = \mathcal{S} - \mathcal{R}$ di
eventi con probabilit\`a complessiva $K$.

Per verificare l'ipotesi $H_0$ occorre scegliere a priori un
valore di $K$ da assumere come il confine che separi, da una
parte, eventi che riteniamo ragionevole si possano
presentare nell'ambito di pure fluttuazioni casuali se \`e
vera $H_0$; e, dall'altra, eventi cos\`\i\ improbabili
(sempre ammesso che $H_0$ sia vera) da far s\`\i\ che la
loro effettiva realizzazione debba implicare la falsit\`a
dell'ipotesi.

Normalmente si sceglie $K = 0.95$ o $K = 0.997$, i
valori della probabilit\`a che corrispondono a scarti di due
o tre errori quadratici medi per la distribuzione di Gauss,
anche se altri valori (come ad esempio $K = 0.999$ o
$K=0.99$) sono abbastanza comuni; e, una volta fatto
questo, si \emph{rigetta l'ipotesi} $H_0$ se il dato a
disposizione (un evento $E$ ottenuto dall'effettivo studio
del fenomeno in esame) appartiene ad $\mathcal{R}$; e la si
accetta se appartiene ad $\mathcal{A}$.

In realt\`a nella pratica si presenta in generale la
necessit\`a di discriminare tra \emph{due} ipotesi, sempre
mutuamente esclusive, che indicheremo con i simboli $H_0$ ed
$H_a$ e che, usando la terminologia della statistica, si
chiamano rispettivamente \emph{ipotesi nulla}%
\index{ipotesi!nulla}
ed \emph{ipotesi alternativa};%
\index{ipotesi!alternativa}
i casi precedenti corrispondono al caso particolare in cui
l'ipotesi alternativa coincida con il \emph{non realizzarsi}
di $H_0$.

Ipotesi nulla ed ipotesi alternativa possono essere entrambe
eventi semplici, oppure composti (ossia somma logica di
pi\`u eventualit\`a semplici); e lo scopo di questo capitolo
\`e quello di mostrare dei criteri sulla base dei quali si
possa opportunamente definire nello spazio degli eventi una
\emph{regione di rigetto} $\mathcal{R}$ per l'ipotesi nulla
(e, in corrispondenza, ovviamente, una regione $\mathcal{A}
= \mathcal{S} - \mathcal{R}$ nella quale tale ipotesi viene
accettata).

\`E chiaro che si corre sempre il rischio di sbagliare: o
rigettando erroneamente ipotesi in realt\`a vere
(\emph{errori di prima specie})%
\index{errore!di prima specie}
o accettando invece ipotesi in realt\`a false (\emph{errori
di seconda specie});%
\index{errore!di seconda specie}
e che, allargando o restringendo la regione di rigetto, si
pu\`o diminuire la probabilit\`a di uno di questi due tipi
di errori solo per aumentare la probabilit\`a di quelli
dell'altra categoria.  Se indichiamo con $P_I$ e $P_{II}$ le
probabilit\`a degli errori di prima e seconda specie
rispettivamente, sulla base della definizione risulta
\begin{align*}
  P_I \; &= \; \Pr( E \in \mathcal{R} | H_0 )
  &&\text{e} &
  P_{II} \; &= \; \Pr( E \in \mathcal{A} | H_a ) \peq .
\end{align*}

Quello che abbiamo finora chiamato ``livello di confidenza''
non \`e altro che $1 - P_I$; $P_I$ viene anche indicato col
simbolo $\alpha$ e chiamato \emph{significanza}%
\index{significanza}
del criterio adottato.  Infine, la probabilit\`a di
\emph{non} commettere un errore di seconda specie, ovvero la
probabilit\`a di rigettare $H_0$ quando l'ipotesi nulla \`e
falsa (e quindi quella alternativa \`e vera) si indica col
simbolo $\beta$ e si chiama \emph{potenza}%
\index{potenza}
del criterio adottato; essa vale quindi
\begin{equation*}
  \beta \; = \; \Pr( E \in \mathcal{R} | H_a ) \; = \; 1 -
  P_{II} \peq .
\end{equation*}

Per fare un esempio concreto, il fisico si trova spesso ad
esaminare ``eventi'' sperimentali e deve decidere se essi
sono del tipo desiderato (segnale) o no (fondo): in questo
caso l'ipotesi nulla $H_0$ consiste nell'appartenenza di un
evento al segnale, mentre l'ipotesi alternativa $H_a$
corrisponde invece all'appartenenza dello stesso evento al
fondo; che in genere non \`e l'intero insieme di eventi
complementare all'ipotesi nulla, $\ob{H}_0$, ma si sa
restringere ad una classe ben definita di fenomeni.

Gli errori di prima specie consistono in questo caso nello
scartare eventi buoni (errori di \emph{impoverimento} del
segnale), e quelli di seconda specie nell'introduzione nel
segnale di eventi di fondo (errori di
\emph{contaminazione}).

I criteri da seguire per definire una regione $\mathcal{R}$
nella quale rigettare $H_0$ sono dettati dalle
caratteristiche del processo di generazione: se gli eventi
di fondo sono preponderanti rispetto al segnale, ad esempio,
bisogner\`a evitare gli errori di seconda specie per quanto
possibile; anche al prezzo di scartare in questo modo una
parte consistente del segnale.

Estendendo al caso generale il metodo seguito nei vari casi
del capitolo \ref{ch:12.veripo} e prima delineato, se si \`e
in grado di associare ad ogni punto dello spazio degli
eventi \emph{due} valori della probabilit\`a (o della
densit\`a di probabilit\`a nel caso di variabili continue),
sia ammessa vera l'ipotesi nulla che ammessa invece vera
l'ipotesi alternativa, si pu\`o pensare di usare \emph{il
  loro rapporto} per definire la regione di rigetto.

Limitandoci al caso delle variabili continue, insomma, dopo
aver definito una nuova variabile casuale $\lambda$
attraverso la
\begin{equation*}
  \lambda = \frac{\mathcal{L} (\boldsymbol{x} |
    H_0)}{\mathcal{L} (\boldsymbol{x} | H_a)} \peq ,
\end{equation*}
possiamo scegliere arbitrariamente un numero reale $k$ e
decidere di accettare l'ipotesi $H_0$ se $\lambda \ge k$ o
di rifiutarla se $\lambda < k$; in definitiva ad ogni $k$
ammissibile \`e associata una differente regione di rigetto
$\mathcal{R}_k$ definita da
\begin{equation*}
  \mathcal{R}_k \; \equiv \; \left\{ \lambda =
    \frac{\mathcal{L} (\boldsymbol{x} | H_0)}{\mathcal{L}
    (\boldsymbol{x} | H_a)} < k \right\} \peq .
\end{equation*}

$\mathcal{L}$, nelle espressioni precedenti, \`e la funzione
di verosimiglianza;%
\index{funzione!di verosimiglianza|(}
che rappresenta appunto la densit\`a di probabilit\`a
corrispondente all'ottenere (sotto una certa ipotesi) un
campione di $N$ valori $x_1, x_2,\ldots,x_N$ (qui indicato
sinteticamente come un vettore $\boldsymbol{x}$ a $N$
componenti).  Ma in base a quale criterio dobbiamo scegliere
$k$?

\section{Un primo esempio}%
\label{ch:13.es1}
Cominciamo con un esempio didattico: supponiamo che i valori
$x_i$ si sappiano provenienti da una popolazione normale
$N(x; \mu, \sigma)$ di varianza $\sigma^2$ nota: e che il
nostro scopo consista nel discriminare tra due possibili
valori $\mu_1$ e $\mu_2$ per $\mu$; valori che, senza
perdere in generalit\`a, supponiamo siano 0 e 1 (potendosi
sempre effettuare un opportuno cambiamento di variabile
casuale che ci porti in questa situazione).  Riassumendo:
siano
\begin{equation*}
  \begin{cases}
    \setbox0=\hbox{$H_0$}
    \makebox[\wd0]{$x$} \sim N(x; \mu, \sigma)
    &\text{\hspace{2cm}(con $\sigma > 0$ noto)} \\[1ex]
    H_0 \equiv \{ \mu = 0 \} \\[1ex]
    H_a \equiv \{ \mu = 1 \}
  \end{cases}
\end{equation*}
le nostre ipotesi.

La densit\`a di probabilit\`a della $x$ vale
\begin{gather*}
  N(x; \mu, \sigma) = \frac{1}{\sigma \sqrt{2 \pi}} \, e^{
    - \frac{1}{2} \left( \frac{x - \mu}{\sigma}
    \right)^2 } \\
  \intertext{e, quindi, la funzione di verosimiglianza ed il
    suo logaritmo valgono}
  \mathcal{L} (\boldsymbol{x}; \mu, \sigma) =
    \frac{1}{\bigl( \sigma \sqrt{2 \pi} \bigr)^N}
    \prod_{i=1}^N e^{- \frac{1}{2 \sigma^2}
    \left( x_i - \mu \right)^2 }
\end{gather*}
e, rispettivamente,
\begin{align*}
  \ln \mathcal{L} (\boldsymbol{x}; \mu, \sigma) &= - N \,
    \ln \bigl( \sigma \sqrt{2 \pi} \bigr) - \frac{1}{2
    \sigma^2} \sum_{i=1}^N \left( x_i - \mu \right)^2
    \\[1ex]
  &= - N \, \ln \bigl( \sigma \sqrt{2 \pi} \bigr) -
    \frac{1}{2 \sigma^2} \sum_{i=1}^N \left( {x_i}^2 - 2 \mu
    x_i + \mu^2 \right) \peq ; \\
\end{align*}
per cui
\begin{equation} \label{eq:13.lnlino}
  \ln \mathcal{L} (\boldsymbol{x}; \mu, \sigma) = - N \, \ln
    \bigl( \sigma \sqrt{2 \pi} \bigr) - \frac{1}{2
    \sigma^2} \left( \sum_{i=1}^N {x_i}^2 - 2 N \bar x \mu +
    N \mu^2 \right) \peq .
\end{equation}
Dalla \eqref{eq:13.lnlino} si ricava immediatamente
\begin{equation*}
  \ln \lambda \; = \; \ln \mathcal{L} (\boldsymbol{x}; 0,
    \sigma) - \ln \mathcal{L} (\boldsymbol{x}; 1, \sigma) \;
    = \; \frac{N}{2 \sigma^2} \left( 1 - 2 \bar x \right)
\end{equation*}
e la regione di rigetto $\mathcal{R}_k$ \`e definita dalla
\begin{gather*}
  \mathcal{R}_k \; \equiv \; \left\{ \; \ln \lambda =
    \frac{N}{2 \sigma^2} \left( 1 - 2 \bar x \right) <
    \ln k \; \right\} \\
  \intertext{da cui consegue, con facili passaggi,}
  \mathcal{R}_k \; \equiv \; \left\{ \; \bar x > \frac{N -
    2 \sigma^2 \ln k}{2 N} = c \; \right\} \peq ;
\end{gather*}
ed insomma $H_0$ va rigettata se la media aritmetica del
campione $\bar x$ risulta superiore a $c$; ed accettata
altrimenti.

\begin{figure}[htbp]
  \vspace*{2ex}
  \begin{center} {
    \input{duegau.pstex_t}
  } \end{center}
  \caption[Un esempio: errori di prima e seconda specie]
    {L'esempio del paragrafo \ref{ch:13.es1}, con delineate
    (in corrispondenza ad un particolare valore di $c$) le
    probabilit\`a degli errori di prima e seconda specie; le
    due curve sono $N(0, \sigma / \sqrt{N})$ e $N(1, \sigma
    / \sqrt{N})$.}
  \label{fig:13.duegau}
\end{figure}

Come si pu\`o scegliere un valore opportuno di $k$ (e quindi
di $c$)?  Gli errori di prima specie (si faccia riferimento
anche alla figura \ref{fig:13.duegau}) hanno probabilit\`a
\begin{gather}
  P_I \; = \; 1 - \alpha \; = \; \Pr \bigl( \bar x > c | H_0
    \bigr) \; = \; \int_c^{+\infty} \! N \left( x; 0,
    \frac{\sigma}{\sqrt{N}} \right) \, \de x
    \label{eq:13.es1_1} \\
  \intertext{e quelli di seconda specie}
  P_{II} \; = \; 1 - \beta \; = \; \Pr \bigl( \bar x < c | H_a
    \bigr) \; = \; \int_{-\infty}^c \! N \left( x; 1,
    \frac{\sigma}{\sqrt{N}} \right) \, \de x
    \label{eq:13.es1_2}
\end{gather}
per cui si hanno svariate possibilit\`a: ad esempio, se
interessa contenere gli errori di prima specie e la
dimensione del campione \`e nota, si fissa un valore
opportunamente grande per $\alpha$ e dalla
\eqref{eq:13.es1_1} si ricava $c$; o, se interessa contenere
gli errori di seconda specie e la dimensione del campione
\`e nota, si fissa $\beta$ e si ricava $c$ dalla
\eqref{eq:13.es1_2}; o, infine, se si vogliono contenere gli
errori di entrambi i tipi, si fissano sia $\alpha$ che
$\beta$ e si ricava la dimensione minima del campione
necessaria per raggiungere lo scopo utilizzando entrambe le
equazioni \eqref{eq:13.es1_1} e \eqref{eq:13.es1_2}.

\section{Il lemma di Neyman--Pearson}%
\index{Neyman--Pearson, lemma di|(}
L'essere ricorsi per la definizione della regione di rigetto
$\mathcal{R}$ al calcolo del \emph{rapporto} delle funzioni
di verosimiglianza non \`e stato casuale; esiste infatti un
teorema (il cosiddetto \emph{lemma di Neyman--Pearson}) il
quale afferma che
\begin{quote}
  \textit{Se si ha a disposizione un campione di $N$ valori
    indipendenti $x_i$ da utilizzare per discriminare tra
    un'ipotesi nulla ed un'ipotesi alternativa entrambe
    semplici, e se \`e richiesto un livello fisso $\alpha$
    di significanza, la massima potenza del test (ovvero la
    minima probabilit\`a di errori di seconda specie) si
    raggiunge definendo la regione di rigetto
    $\mathcal{R}_k$ attraverso una relazione del tipo}
\end{quote}
\begin{equation} \label{eq:13.lambdak}
  \mathcal{R}_k \; \equiv \; \left\{ \; \lambda =
    \frac{\mathcal{L} (\boldsymbol{x} | H_0)}{\mathcal{L}
    (\boldsymbol{x} | H_a)} < k \; \right\} \peq .
\end{equation}

Infatti, indicando con $f = f(x; \theta)$ la densit\`a di
probabilit\`a della variabile $x$ (che supponiamo dipenda da
un solo parametro $\theta$), siano $H_0 \equiv \{ \theta =
\theta_0 \}$ e $H_a \equiv \{ \theta = \theta_a \}$ le due
ipotesi (semplici) tra cui decidere; la funzione di
verosimiglianza vale, come sappiamo,
\begin{equation*}
  \mathcal{L} (\boldsymbol{x}; \theta) = \prod_{i=1}^N
    f(x_i; \theta) \peq .
\end{equation*}
Indichiamo con $\Re$ l'insieme di tutte le regioni
$\mathcal{R}$ per le quali risulti
\begin{equation} \label{eq:13.explainx}
  P_I \; = \; \int_\mathcal{R} \! \mathcal{L}
    (\boldsymbol{x}; \theta_0) \, \de \boldsymbol{x} \; = \;
    1 - \alpha
\end{equation}
con $\alpha$ costante prefissata (nella
\eqref{eq:13.explainx} abbiamo sinteticamente indicato con
$\de \boldsymbol{x}$ il prodotto $\de x_1 \, \de x_2 \cdots
\de x_N$).
Vogliamo trovare quale di queste regioni rende massima
\begin{equation*}
  \beta \; = \; 1 - P_{II} \; = \; \int_\mathcal{R} \!
    \mathcal{L} (\boldsymbol{x}; \theta_a) \, \de
    \boldsymbol{x} \peq .
\end{equation*}

\noindent Ora, per una qualsiasi regione $\mathcal{R} \ne
\mathcal{R}_k$, valgono sia la
\begin{gather*}
  \mathcal{R}_k \; = \; (\mathcal{R}_k \cap \mathcal{R})
    \cup (\mathcal{R}_k \cap \overbar{\mathcal{R}}\,) \\
  \intertext{che la}
  \mathcal{R} \; = \; (\mathcal{R} \cap \mathcal{R}_k) \cup
    (\mathcal{R} \cap \overbar{\mathcal{R}}_k) \peq ; \\
  \intertext{e quindi, per una qualsiasi funzione
    $\phi(\boldsymbol{x})$, risulta sia}
  \int_{\mathcal{R}_k} \! \phi(\boldsymbol{x}) \, \de
    \boldsymbol{x} \; = \; \int_{\mathcal{R}_k \cap
    \mathcal{R}} \phi(\boldsymbol{x}) \, \de \boldsymbol{x}
    + \int_{\mathcal{R}_k \cap \overbar{\mathcal{R}}}
    \phi(\boldsymbol{x}) \, \de \boldsymbol{x} \\
  \intertext{che}
  \int_{\mathcal{R}} \! \phi(\boldsymbol{x}) \, \de
    \boldsymbol{x} \; = \; \int_{\mathcal{R} \cap
    \mathcal{R}_k} \phi(\boldsymbol{x}) \, \de
    \boldsymbol{x} + \int_{\mathcal{R} \cap
    \overbar{\mathcal{R}}_k} \phi(\boldsymbol{x}) \, \de
    \boldsymbol{x}
\end{gather*}
e, sottraendo membro a membro,
\begin{equation} \label{eq:13.neypea2}
  \int_{\mathcal{R}_k} \! \phi(\boldsymbol{x}) \, \de
    \boldsymbol{x} - \int_{\mathcal{R}} \!
    \phi(\boldsymbol{x}) \, \de \boldsymbol{x} \; = \;
    \int_{\mathcal{R}_k \cap \overbar{\mathcal{R}}}
    \phi(\boldsymbol{x}) \, \de \boldsymbol{x} -
    \int_{\mathcal{R} \cap \overbar{\mathcal{R}}_k}
    \phi(\boldsymbol{x}) \, \de \boldsymbol{x} \peq .
\end{equation}

\noindent Applicando la \eqref{eq:13.neypea2} alla funzione
$\mathcal{L}(\boldsymbol{x} | \theta_a)$ otteniamo:
\begin{multline} \label{eq:13.neypea3}
  \int_{\mathcal{R}_k} \! \mathcal{L}(\boldsymbol{x} |
    \theta_a) \, \de \boldsymbol{x} -\int_{\mathcal{R}} \!
    \mathcal{L}(\boldsymbol{x} | \theta_a) \, \de
    \boldsymbol{x} \; = \\
  = \; \int_{\mathcal{R}_k \cap \overbar{\mathcal{R}}}
    \mathcal{L}(\boldsymbol{x} | \theta_a) \, \de
    \boldsymbol{x} - \int_{\mathcal{R} \cap
    \overbar{\mathcal{R}}_k} \mathcal{L}(\boldsymbol{x}
    | \theta_a) \, \de \boldsymbol{x} \peq ; \qquad
\end{multline}
ma, nel primo integrale del secondo membro, essendo la
regione di integrazione contenuta in $\mathcal{R}_k$, deve
valere la \eqref{eq:13.lambdak}; e quindi risultare ovunque
\begin{gather*}
  \mathcal{L}(\boldsymbol{x} | \theta_a) \; > \; \frac{1}{k}
    \cdot \mathcal{L}(\boldsymbol{x} | \theta_0) \\
  \intertext{mentre, per lo stesso motivo, nel secondo
    integrale}
  \mathcal{L}(\boldsymbol{x} | \theta_a) \; \le \;
    \frac{1}{k} \cdot \mathcal{L}(\boldsymbol{x} | \theta_0)
\end{gather*}
e quindi la \eqref{eq:13.neypea3} implica che
\begin{multline*}
  \int_{\mathcal{R}_k} \! \mathcal{L}(\boldsymbol{x} |
    \theta_a) \, \de \boldsymbol{x} -\int_{\mathcal{R}} \!
    \mathcal{L}(\boldsymbol{x} | \theta_a) \, \de
    \boldsymbol{x} \; > \\
  > \; \frac{1}{k} \cdot \left[ \int_{\mathcal{R}_k \cap
    \overbar{\mathcal{R}}} \mathcal{L}(\boldsymbol{x} |
    \theta_0) \, \de \boldsymbol{x} - \int_{\mathcal{R}
    \cap \overbar{\mathcal{R}}_k}
    \mathcal{L}(\boldsymbol{x} | \theta_0) \, \de
    \boldsymbol{x} \right] \peq .
\end{multline*}
Ricordando la \eqref{eq:13.neypea2},
\begin{equation*}
  \int_{\mathcal{R}_k} \! \mathcal{L}(\boldsymbol{x} |
    \theta_a) \, \de \boldsymbol{x} -\int_{\mathcal{R}} \!
    \mathcal{L}(\boldsymbol{x} | \theta_a) \, \de
    \boldsymbol{x} \; > \; \frac{1}{k} \cdot \left[
    \int_{\mathcal{R}_k} \! \mathcal{L}(\boldsymbol{x} |
    \theta_0) \, \de \boldsymbol{x} -\int_{\mathcal{R}} \!
    \mathcal{L}(\boldsymbol{x} | \theta_0) \, \de
    \boldsymbol{x} \right]
\end{equation*}
e, se $\mathcal{R} \in \Re$ e quindi soddisfa anch'essa alla
\eqref{eq:13.explainx},
\begin{equation*}
  \int_{\mathcal{R}_k} \! \mathcal{L}(\boldsymbol{x} |
    \theta_a) \, \de \boldsymbol{x} -\int_{\mathcal{R}} \!
    \mathcal{L}(\boldsymbol{x} | \theta_a) \, \de
    \boldsymbol{x} \; > \; \frac{1}{k} \cdot \left[ P_I -
    P_I \right] \; = \; 0
\end{equation*}
che era la nostra tesi.%
\index{Neyman--Pearson, lemma di|)}

\section{Tests di massima potenza uniforme}
Consideriamo ora un esempio del tipo di quello del paragrafo
\ref{ch:13.es1}; e sia sempre disponibile un campione di $N$
misure indipendenti derivante da una popolazione normale di
varianza nota.  Assumiamo ancora come ipotesi nulla quella
che la popolazione abbia un certo valore medio, che
supponiamo essere 0, ma sostituiamo alla vecchia ipotesi
alternativa $H_a$ una nuova ipotesi \emph{composta}; ovvero
quella che il valore medio della popolazione sia positivo:
\begin{equation*}
  \begin{cases}
    \setbox0=\hbox{$H_0$}
    \makebox[\wd0]{$x$} \sim N(x; \mu, \sigma)
      &\text{\hspace{2cm}(con $\sigma > 0$ noto)} \\[1ex]
    H_0 \equiv \{ \mu = 0 \} \\[1ex]
    H_a \equiv \{ \mu > 0 \}
  \end{cases}
\end{equation*}
(l'ipotesi alternativa \`e dunque somma logica di infinite
ipotesi semplici del tipo $\mu = \mu_a$ con $\mu_a > 0$).

Dalla \eqref{eq:13.lnlino} ricaviamo immediatamente le
\begin{gather*}
  \mathcal{L} ( \boldsymbol{x}; 0, \sigma ) \; = \; - N \,
    \ln \bigl( \sigma \sqrt{2 \pi} \bigr) - \frac{1}{2
    \sigma^2} \sum_{i=1}^N {x_i}^2 \\
  \intertext{e}
  \mathcal{L} ( \boldsymbol{x}; \mu_a, \sigma ) \; = \; - N
    \, \ln \bigl( \sigma \sqrt{2 \pi} \bigr) - \frac{1}{2
    \sigma^2} \left( \sum_{i=1}^N {x_i}^2 - 2 N \bar x \mu_a
    + N {\mu_a}^2 \right)
  \intertext{(sempre con $\mu_a > 0$); e, sostituendole
    nella \eqref{eq:13.lambdak}, che definisce la generica
    regione di rigetto $\mathcal{R}_k$, otteniamo}
  \ln \lambda \; = \; \ln \mathcal{L} ( \boldsymbol{x}; 0,
    \sigma ) - \mathcal{L} ( \boldsymbol{x}; \mu_a, \sigma )
    \; = \; \frac{N \mu_a}{2 \sigma^2} \left( \mu_a - 2 \bar
    x \right) \; < \; \ln k \\
  \intertext{equivalente alla}
  \mathcal{R}_k \; \equiv \; \left\{ \; \bar x > \frac{N
      {\mu_a}^2 - 2 \sigma^2 \ln k}{2 N \mu_a} = c \;
  \right\} \peq .
\end{gather*}

Si rigetta quindi $H_0$ se la media aritmetica del campione
\`e superiore a $c$ e la si accetta altrimenti: la
probabilit\`a di commettere errori di prima specie vale
\begin{equation*}
  P_I \; = \; 1 - \alpha \; = \; \int_c^{+\infty} \! N
    \left( x; 0, \frac{\sigma}{\sqrt{N}} \right) \, \de x
\end{equation*}
ed \`e ben definita; ma, al contrario, la probabilit\`a di
commettere errori di seconda specie dipende dal particolare
valore di $\mu_a$, e non pu\`o quindi essere calcolata.

Se interessa solo contenere gli errori di prima specie e la
dimensione del campione \`e nota, si fissa $\alpha$ e si
ricava il corrispondente valore di $c$ dall'equazione
precedente; altrimenti occorre fare delle ulteriori ipotesi
sulla funzione di frequenza dei differenti valori di
$\mu_a$, e, ad esempio, calcolare la probabilit\`a degli
errori di seconda specie con tecniche di Montecarlo.

In ogni caso, per\`o, osserviamo che la regione di rigetto
\`e sempre dello stesso tipo \eqref{eq:13.lambdak} per
\emph{qualsiasi} $\mu_a > 0$; e quindi un confronto separato
tra $H_0$ ed ognuna delle differenti ipotesi semplici che
costituiscono $H_a$ \`e comunque del tipo per cui il lemma
di Neyman--Pearson garantisce la massima potenza.

Tests di questo tipo, per i quali \emph{la significanza \`e
  costante e la potenza \`e massima per ognuno dei casi
  semplici che costituiscono l'ipotesi alternativa}, si
dicono ``tests di massima potenza uniforme''.

\section{Il rapporto delle massime verosimiglianze}%
\index{metodo!del rapporto delle massime verosimiglianze|(}
Nel caso generale in cui sia l'ipotesi nulla che quella
alternativa siano composte, la situazione \`e pi\`u
complicata: non esiste normalmente un test di massima
potenza uniforme, e, tra i vari criteri possibili per
decidere tra le due ipotesi, bisogna capire quali abbiano
caratteristiche (significanza e potenza) adeguate; un metodo
adatto a costruire una regione di rigetto dotata
asintoticamente (per grandi campioni) di caratteristiche,
appunto, desiderabili, \`e quello seguente (\emph{metodo del
  rapporto delle massime verosimiglianze}).

Sia una variabile casuale $x$, la cui densit\`a di
probabilit\`a supponiamo sia una funzione $f(x; \theta_1,
\theta_2,\ldots, \theta_M )$ dipendente da $M$ parametri:
indicando sinteticamente la $M$-pla dei valori dei parametri
come un vettore $\boldsymbol{\theta}$ in uno spazio a $M$
dimensioni (\emph{spazio dei parametri}), consista $H_0$
nell'essere $\boldsymbol{\theta}$ compreso all'interno di
una certa regione $\Omega_0$ di tale spazio; mentre $H_a$
consista nell'appartenere $\boldsymbol{\theta}$ alla
regione $\Omega_a$ complementare a $\Omega_0$: $\Omega_a
\equiv \ob{H}_0$, cos\`\i\ che $(\Omega_0 \cup
\Omega_a)$ coincida con l'intero spazio dei parametri
$\mathcal{S}$.

In particolare, $\Omega_0$ pu\`o estendersi, in alcune delle
dimensioni dello spazio dei parametri, da $-\infty$ a
$+\infty$; e, in tal caso, il vincolo sulle $\theta_i$ cui
corrisponde l'ipotesi nulla riguarder\`a un numero di
parametri minore di $M$.

Scritta la funzione di verosimiglianza,
\begin{equation} \label{eq:13.genlik}
  \mathcal{L} ( \boldsymbol{x}; \boldsymbol{\theta} ) =
    \prod_{i=1}^N f(x_i; \boldsymbol{\theta} )
\end{equation}
indichiamo con $\mathcal{L} (\widehat S)$ il suo massimo
valore nell'intero spazio dei parametri; e con $\mathcal{L}
(\widehat R)$ il massimo valore assunto sempre della
\eqref{eq:13.genlik}, ma con i parametri vincolati a
trovarsi nella regione $\Omega_0$ (quindi limitatamente a
quei casi nei quali $H_0$ \`e vera).  Il rapporto
\begin{equation} \label{eq:13.gelira}
  \lambda = \frac{\mathcal{L} (\widehat R)}{\mathcal{L} (\widehat
    S)}
\end{equation}
deve essere un numero appartenente all'intervallo $[0,1]$;
se si fissa un arbitrario valore $k$ ($0<k<1$), esso
definisce una generica regione di rigetto, $\mathcal{R}_k$,
attraverso la
\begin{equation*}
  \mathcal{R}_k \; \equiv \; \left\{ \; \lambda =
    \frac{\mathcal{L} (\widehat R)}{\mathcal{L} (\widehat
      S)} < k \; \right\}
\end{equation*}
(ovvero si accetta $H_0$ quando $\lambda \ge k$ e la si
rigetta quando $\lambda < k$).  Nel caso si sappia
determinare la densit\`a di probabilit\`a di $\lambda$
condizionata all'assunzione che $H_0$ sia vera, $g(\lambda |
H_0)$, la probabilit\`a di un errore di prima specie \`e
data ovviamente da
\begin{equation*}
  P_I \; = \; \alpha \; = \; \Pr \bigl( \lambda \in [0,k] |
    H_0 \bigr) \; = \; \int_0^k \! g(\lambda | H_0) \, \de
    \lambda \peq .
\end{equation*}

L'importanza del metodo sta nel fatto che si pu\`o
dimostrare il seguente
\begin{quote}
  \textsc{Teorema:} \textit{se l'ipotesi nulla $H_0$
    consiste nell'appartenenza di un insieme di $P \le M$
    dei parametri $\theta_i$ ad una determinata regione
    $\Omega_0$, e se l'ipotesi alternativa $H_a$ consiste
    nel fatto che essi non vi appartengano ($H_a \equiv
    \ob{H}_0$), allora $- 2 \ln \lambda$, ove $\lambda$ \`e
    definito dalla \eqref{eq:13.gelira}, ha densit\`a di
    probabilit\`a che, ammessa vera l'ipotesi nulla,
    converge in probabilit\`a (all'aumentare di $N$) alla
    distribuzione del $\chi^2$ a $P$ gradi di libert\`a.}
\end{quote}%
\index{metodo!del rapporto delle massime verosimiglianze|)}
che, dicendoci quale \`e (almeno nel limite di grandi
campioni) la forma di $g(\lambda | H_0)$, ci mette comunque
in grado di calcolare la significanza del test.

Illustriamo il metodo con un esempio: disponendo ancora di
un campione di $N$ determinazioni indipendenti, provenienti
da una popolazione normale di varianza nota, vogliamo
applicarlo per discriminare tra l'ipotesi nulla che il valore
medio abbia valore 0 ($H_0 \equiv \{ \mu = 0 \}$) e quella
che esso abbia valore differente ($H_a \equiv \{ \mu \ne 0
\}$).

Il logaritmo della funzione di verosimiglianza \`e ancora
dato dalla \eqref{eq:13.lnlino}; e gi\`a sappiamo, dal
paragrafo \ref{ch:11.mepeted}, che $\mathcal{L}$ assume il
suo massimo valore quando $\mu = \bar x$, per cui
\begin{gather*}
  \ln \mathcal{L} (\widehat S) = - N \, \ln \bigl( \sigma
    \sqrt{2 \pi} \bigr) - \frac{1}{2 \sigma^2} \left(
    \sum_{i=1}^N {x_i}^2 - N \bar x^2 \right) \peq . \\
  \intertext{Inoltre $\Omega_0$ si riduce ad un unico punto,
    $\mu = 0$; per cui}
  \ln \mathcal{L} (\widehat R) = - N \, \ln \bigl( \sigma
    \sqrt{2 \pi} \bigr) - \frac{1}{2 \sigma^2} \sum_{i=1}^N
    {x_i}^2 \peq . \\
  \intertext{Dalla \eqref{eq:13.gelira} si ricava}
  \ln \lambda \; = \; \ln \mathcal{L} (\widehat R) - \ln
    \mathcal{L} (\widehat S) \; = \; - \frac{1}{2 \sigma^2} \, N
    \bar x^2 \\
  \intertext{e la regione di rigetto \`e definita dalla $\ln
    \lambda < \ln k$; ovvero (ricordando che $\ln k < 0$) da}
  \mathcal{R}_k \; \equiv \; \left\{ \; \bar x^2 > - \frac{2
    \sigma^2 \ln k}{N} \; \right\} \\
  \intertext{e, posto}
  c = \sigma \sqrt{- \frac{2 \ln k}{N}}
\end{gather*}
si accetter\`a $H_0$ se $| \bar x | \le c$ (e la si
rigetter\`a se $| \bar x | > c$).

In questo caso il teorema precedentemente citato afferma che
\begin{equation*}
  - 2 \ln \lambda = \frac{\phantom{M} \bar x^2
    \phantom{M}}{\dfrac{\sigma^2}{N}}
\end{equation*}
\`e distribuito asintoticamente come il $\chi^2$ ad un grado
di libert\`a (cosa che del resto gi\`a sapevamo, vista
l'espressione di $- 2 \ln \lambda$); per cui, indicando con
$F(t; N)$ la densit\`a di probabilit\`a della distribuzione
del $\chi^2$ a $N$ gradi di libert\`a, avremo
\begin{equation*}
  P_I \; = \; \alpha \; = \; \int_0^k \! g( \lambda | H_0 )
    \, \de \lambda \; = \; \int_{- 2 \ln k}^{+\infty} \!
    F(t; 1) \, \de t
\end{equation*}
della quale ci possiamo servire per ricavare $k$ se vogliamo
che la significanza del test abbia un certo valore: ad
esempio un livello di confidenza del 95\% corrisponde ad
$\alpha = 0.05$ e, dalle tabelle della distribuzione del
$\chi^2$, ricaviamo
\begin{align*}
  - 2 \ln k &= 3.84 &&\text{e quindi} & c &= 1.96 \,
    \frac{\sigma}{\sqrt{N}} \peq .
\end{align*}

Anche senza dover ricorrere al teorema sul comportamento
asintotico di $- 2 \ln \lambda$, allo stesso risultato si
pu\`o pervenire per altra via: in questo caso si conosce
infatti esattamente $\alpha$, che vale
\begin{equation*}
  P_I \; = \; \alpha \; = \; \Pr \Bigl( | \bar x | > c
    \bigl| H_0 \bigr. \Bigr) \; = \; 2 \int_c^{+\infty} \!
    N \left( t; 0, \frac{\sigma}{\sqrt{N}} \right) \, \de t
\end{equation*}
e, dalle tabelle della distribuzione normale standardizzata,
si ricava che un'area two-tailed del 5\% corrisponde ad un
valore assoluto dello scarto normalizzato $t_0 = 1.96$; per
cui, ancora, si ricaverebbe $| \bar x | > 1.96 ( \sigma /
\sqrt{N} )$ come test per un livello di confidenza del
95\%.

\section{Applicazione: ipotesi sulle probabilit\`a}
Nel paragrafo \ref{ch:11.exampl} abbiamo preso in
considerazione il caso di un evento casuale che si pu\`o
manifestare in un numero finito $M$ di modalit\`a, aventi
ognuna probabilit\`a incognita $p_i$; la stima di massima
verosimiglianza delle $p_i$ \`e data dal rapporto tra la
frequenza assoluta di ogni modalit\`a, $n_i$, ed il numero
totale di prove, $N$.

Vogliamo ora applicare il metodo del rapporto delle massime
verosimiglianze per discriminare, sulla base di un campione
di determinazioni indipendenti, l'ipotesi nulla che le
probabilit\`a abbiano valori noti a priori e l'ipotesi
alternativa complementare, $H_a \equiv \ob{H}_0$:
\begin{equation*}
  \begin{cases}
    H_0 \; \equiv \; \left\{ p_i = \pi_i \right\} &
    \hspace{2cm} (\forall  i \in \{ 1, 2,\ldots, M \})
    \\[1ex]
    H_a \; \equiv \; \left\{ p_i \ne \pi_i \right\} &
    \hspace{2cm} (\exists i \in \{ 1, 2,\ldots, M \})
  \end{cases}
\end{equation*}

Ricordiamo che la funzione di verosimiglianza, a meno di un
fattore moltiplicativo costante, \`e data da
\begin{gather*}
  \mathcal{L} ( \boldsymbol{n}; \boldsymbol{p} ) =
    \prod_{i=1}^M {p_i}^{n_i} \\
  \intertext{e che, essendo la stima di massima
    verosimiglianza data da}
  \widehat p_i = \frac{n_i}{N} \\
  \intertext{il massimo assoluto di $\mathcal{L}$ \`e}
  \mathcal{L} (\widehat S) \; = \; \prod_{i=1}^M \left(
    \frac{n_i}{N} \right)^{n_i} \; = \; \frac{1}{N^N}
    \prod_{i=1}^M {n_i}^{n_i} \peq . \\
  \intertext{Inoltre, nell'unico punto dello spazio dei
    parametri che corrisponde ad $H_0$,}
  \mathcal{L} (\widehat R) = \prod_{i=1}^M {\pi_i}^{n_i} \\
  \intertext{per cui}
  \lambda \; = \; \frac{\mathcal{L} (\widehat R)}{\mathcal{L}
    (\widehat S)} \; = \; N^N \prod_{i=1}^M \left(
    \frac{\pi_i}{n_i} \right)^{n_i}
\end{gather*}
dalla quale si pu\`o, come sappiamo, derivare una generica
regione di rigetto attraverso la consueta $\mathcal{R}_k
\equiv \{ \lambda < k \}$.

\begin{equation*}
  -2 \, \ln \lambda \; = \; -2 \left[ N \, \ln N +
    \sum_{i=1}^M n_i \left( \ln \pi_i - \ln n_i \right)
    \right]
\end{equation*}
\`e inoltre asintoticamente distribuita come il $\chi^2$ a
$M-1$ gradi di libert\`a (c'\`e un vincolo: che le $n_i$
abbiano somma $N$), e questo pu\`o servire a scegliere un
$k$ opportuno (nota la dimensione del campione) una volta
fissata $\alpha$.

Il criterio di verifica dell'ipotesi dato in precedenza
consisteva nel calcolo del valore della variabile casuale
\begin{equation*}
  X = \sum_{i=1}^M \frac{ \left( n_i - N \pi_i \right)^2 }{
    N \pi_i }
\end{equation*}
e nel suo successivo confronto con la distribuzione del
$\chi^2$ a $M-1$ gradi di libert\`a; lo studio del rapporto
delle massime verosimiglianze porta dunque ad un criterio
\emph{differente} e, senza sapere nulla della probabilit\`a
di commettere errori di seconda specie, non \`e possibile
dire quale dei due risulti migliore (a parit\`a di
significanza).

\section{Applicazione: valore medio di una
  popolazione normale}
Ancora un esempio: sia una popolazione normale $N(x; \mu,
\sigma)$ dalla quale vengano ottenuti $N$ valori
indipendenti $x_i$, ma questa volta \emph{la varianza}
$\sigma$ \emph{sia ignota}; vogliamo discriminare, sulla
base del campione, tra l'ipotesi nulla che il valore medio
della popolazione abbia un valore prefissato e l'ipotesi
alternativa complementare,
\begin{equation*}
  \begin{cases}
    H_0 \; \equiv \; \left\{ \mu = \mu_0 \right\} \\[1ex]
    H_a \; \equiv \; \left\{ \mu \ne \mu_0 \right\}
  \end{cases}
\end{equation*}

Il logaritmo della funzione di verosimiglianza \`e
\begin{equation} \label{eq:13.likgau}
  \ln \mathcal{L} ( \boldsymbol{x}; \mu, \sigma ) \; = \;
    - N \, \ln \sigma - \frac{N}{2} \, \ln ( 2 \pi )
    - \frac{1}{2 \sigma^2} \sum_{i=1}^N ( x_i - \mu)^2
\end{equation}
ed essendo le stime di massima verosimiglianza date, come
avevamo trovato nel paragrafo \ref{ch:11.exampl}, da
\begin{align*}
  \widehat \mu &= \bar x = \frac{1}{N} \sum_{i=1}^N x_i
   &&\text{e} &
   \widehat \sigma^2 &= \frac{1}{N} \sum_{i=1}^N \left( x_i
     - \widehat \mu \right)^2
\end{align*}
ne deriva, sostituendo nella \eqref{eq:13.likgau}, che
\begin{equation*}
  \ln \mathcal{L} (\widehat S) \; = \; - \frac{N}{2} \ln \left[
    \sum_{i=1}^N \left( x_i - \bar x \right)^2 \right] +
    \frac{N}{2} \, \ln N - \frac{N}{2} \, \ln ( 2 \pi ) -
    \frac{N}{2} \peq .
\end{equation*}

D'altra parte, ammessa vera $H_0$, abbiamo che
\begin{gather*}
  \ln \mathcal{L} ( \boldsymbol{x} | H_0 ) = - N \, \ln
    \sigma - \frac{N}{2} \, \ln ( 2 \pi ) - \frac{1}{2
    \sigma^2} \, \sum_{i=1}^N \left( x_i - \mu_0 \right)^2
    \\
  \intertext{e, derivando rispetto a $\sigma$,}
  \frac{\de}{\de \sigma} \, \ln \mathcal{L} ( \boldsymbol{x}
    | H_0 ) =  - \frac{N}{\sigma} + \frac{1}{\sigma^3}
    \sum_{i=1}^N \left( x_i - \mu_0 \right)^2 \peq . \\
  \intertext{Annullando la derivata prima, si trova che
    l'unico estremante di $\mathcal{L} ( \boldsymbol{x} |
    H_0 )$ si ha per}
  \sigma_0 = \frac{1}{N} \sum_{i=1}^N \left( x_i - \mu_0
    \right)^2 \\
  \intertext{mentre la derivata seconda vale}
  \frac{\de^2}{\de \sigma^2} \, \ln \mathcal{L} (
    \boldsymbol{x} | H_0 ) = \frac{N}{\sigma^2} -
    \frac{3}{\sigma^4} \sum_{i=1}^N \left( x_i - \mu_0
    \right)^2 \\
  \intertext{e, calcolata per $\sigma = \sigma_0$,}
  \left. \frac{\de^2 (\ln \mathcal{L}) }{\de \sigma^2}
    \right|_{\sigma = \sigma_0} \; = \; - \frac{2 N^2}{
    \sum_i \left( x_i - \mu_0 \right)^2 } \; < \; 0 \\
\end{gather*}
per cui l'estremante \`e effettivamente un massimo.
Sostituendo,
\begin{gather*}
  \ln \mathcal{L} (\widehat R) \; = \; - \frac{N}{2} \ln \left[
    \sum_{i=1}^N \left( x_i - \mu_0 \right)^2 \right] +
    \frac{N}{2} \, \ln N - \frac{N}{2} \, \ln ( 2 \pi ) -
    \frac{N}{2} \\[1ex]
  \ln \lambda \; = \; \ln \mathcal{L} (\widehat R) - \ln
    \mathcal{L} (\widehat S) \; = \; - \frac{N}{2} \left\{ \ln
    \left[ \sum_{i=1}^N \left( x_i - \mu_0 \right)^2 \right]
    - \ln \left[ \sum_{i=1}^N \left( x_i - \bar x \right)^2
    \right] \right\}
\end{gather*}
ed infine
\begin{align*}
  \ln \lambda &= - \frac{N}{2} \, \ln \left[ \frac{ \sum_i
    \left( x_i - \mu_0 \right)^2 }{ \sum_i \left( x_i - \bar
    x \right)^2 } \right] \\[1.5ex]
  &= - \frac{N}{2} \, \ln \left[ 1 + \frac{N \left( \bar x -
    \mu_0 \right)^2 }{ \sum_i \left( x_i - \bar x \right)^2
    } \right] \\[1.5ex]
  &= - \frac{N}{2} \, \ln \left( 1 + \frac{t^2}{N - 1}
    \right)
\end{align*}
tenendo conto dapprima del fatto che $\sum_i ( x_i - \mu_0
)^2 = \sum_i ( x_i - \bar x )^2 + N (\bar x - \mu_0 )^2$, e
definendo poi una nuova variabile casuale
\begin{equation*}
  t \; = \; \left( \bar x - \mu_0 \right) \sqrt{ \frac{N (N
    - 1)}{\sum_i \left( x_i - \bar x \right)^2} } \; = \;
    \frac{\phantom{t} \bar x - \mu_0 \phantom{t}}{
    \dfrac{s}{\sqrt{N}} } \peq .
\end{equation*}

Un qualunque metodo per il rigetto di $H_0$ definito
confrontando $\lambda$ con un prefissato valore $k$ si
traduce, in sostanza, in un corrispondente confronto da
eseguire per $t$:
\begin{gather*}
  \mathcal{R}_k \; \equiv \; \bigl\{ \ln \lambda < \ln k
    \bigr\} \\
  \intertext{che porta alla}
  - \frac{N}{2} \, \ln \left( 1 + \frac{t^2}{N - 1}
    \right) \; < \; \ln k \\
  \intertext{ed alla condizione}
  t^2 \; > \; (N - 1) \, \left( k^{- \tfrac{2}{N}} - 1
    \right) \peq ;
\end{gather*}
ovvero si rigetta l'ipotesi nulla se $|t|$ \`e maggiore di
un certo $t_0$ (derivabile dall'equazione precedente), e la
si accetta altrimenti.

Ma $t$ (vedi anche l'equazione \eqref{eq:12.xsecond}) segue
la distribuzione di Student a $N-1$ gradi di libert\`a, e
quindi accettare o rigettare $H_0$ sotto queste ipotesi si
riduce ad un test relativo a quella distribuzione: come
gi\`a si era concluso nel capitolo \ref{ch:12.veripo}.
Il livello di significanza $\alpha$ \`e legato a $t_0$ dalla
\begin{equation*}
  \frac{\alpha}{2} = \int_{t_0}^{+\infty} \! F(t; N-1)
    \, \de t
\end{equation*}
(indicando con $F(t;N)$ la funzione di frequenza di Student
a $N$ gradi di libert\`a), tenendo conto che abbiamo a che
fare con un two-tailed test ($\mathcal{R}_k \equiv \bigl\{
|t| > t_0 \bigr\}$).

Insomma non c'\`e differenza, in questo caso, tra quanto
esposto nel capitolo precedente e la teoria generale
discussa in quello presente: nel senso che i due criteri di
verifica dell'ipotesi portano per questo problema allo
stesso metodo di decisione (ma, come abbiamo visto nel
paragrafo precedente, non \`e sempre cos\`\i).%
\index{funzione!di verosimiglianza|)}

\endinput

\appendix
\renewcommand{\thefigure}{{\normalfont\Alph{chapter}\arabic{figure}}}
\renewcommand{\thetable}{{\normalfont\Alph{chapter}.\arabic{table}}}
% $Id: chaptera.tex,v 1.1 2005/03/01 10:06:08 loreti Exp $

\chapter{Cenni di calcolo combinatorio}%
\index{calcolo combinatorio|(}
Il \emph{calcolo combinatorio} \`e una branca della
matematica orientata alla discussione ed allo sviluppo di
formule che permettano di ottenere il numero di casi
distinti che si possono presentare in un esperimento, od il
numero di elementi che compongono un insieme, senza
ricorrere alla loro enumerazione esplicita.

Il calcolo combinatorio trova importanti applicazioni nella
teoria della probabilit\`a e nella statistica: alcune
formule, specificatamente quelle per le permutazioni e le
combinazioni, vengono usate nel corso del testo; qui se ne
d\`a una breve giustificazione.

\section{Il lemma fondamentale del calcolo combinatorio}%
\index{calcolo combinatorio!lemma fondamentale|(}
\begin{quote}
  \textsc{Lemma fondamentale del calcolo combinatorio}:
  \textit{dati due insiemi $I_1$ ed $I_2$, composti da $N_1$
    ed $N_2$ elementi distinti rispettivamente, l'insieme $I
    = I_1 \otimes I_2$ di tutte le coppie ordinate che si
    possono costruire associando un elemento di $I_1$ con un
    elemento di $I_2$ \`e composto da $N_1 \cdot N_2$
    elementi.}
\end{quote}

Questo lemma si pu\`o immediatamente generalizzare (per
induzione completa) a $K$ insiemi $I_1,\ldots, I_K$ composti
da $N_1,\ldots, N_K$ elementi distinti rispettivamente:
l'insieme $I = I_1 \otimes I_2 \otimes \cdots \otimes I_K$,
costituito da tutte le possibili associazioni ordinate di
$K$ elementi ognuno dei quali provenga da un differente
insieme $I_j$, con $j=1,\ldots, K$, \`e composto da $N_1
\cdot N_2 \cdots N_K$ elementi.%
\index{calcolo combinatorio!lemma fondamentale|)}

\section{Fattoriale di un numero intero}
Si definisce come \emph{fattoriale} di un numero intero
positivo $N$, e si indica con il simbolo $N!$, il prodotto
dei primi $N$ numeri interi:
\begin{equation*}
  N! = 1 \cdot 2 \cdot 3 \cdots N \peq ;
\end{equation*}
per motivi che appariranno chiari pi\`u avanti\footnote{La
  ``definizione'' $0!  = 1$ non \`e cos\`\i\ arbitraria come
  pu\`o sembrare: in realt\`a si comincia definendo una
  certa funzione di variabile complessa $\Gamma(z)$ che,
  quando l'argomento $z$ \`e un numero intero positivo,
  coincide con il suo fattoriale; e per la quale si vede che
  $\Gamma(0) = 1$.}, si definisce poi il fattoriale di zero
come $0!  = 1$.

\section{Disposizioni}%
\index{disposizioni|(}
Se $N$ e $K$ sono due numeri interi positivi tali che sia $K
\le N$, si definisce come numero delle disposizioni di $N$
oggetti di classe $K$ (che si indica con il simbolo $D^N_K$)
il numero dei gruppi \emph{distinti} di $K$ oggetti che \`e
possibile formare a partire dagli $N$ originali; definendo
come \emph{distinti} due gruppi se essi differiscono o per
qualche elemento o per l'ordine.

Come esempio, le disposizioni di classe 2 che si possono
formare con le 21 lettere dell'alfabeto italiano sono le
seguenti:
\begin{equation*}
  \left\{ \begin{array}{lllllll}
       & AB & AC & AD & \cdots & AV & AZ \\
    BA &    & BC & BD & \cdots & BV & BZ \\
       &    &    &    & \cdots &    &    \\
    ZA & ZB & ZC & ZD & \cdots & ZV &    \\
  \end{array} \right.
\end{equation*}

Il valore di $D^N_K$ si pu\`o facilmente trovare sfruttando
il lemma fondamentale del calcolo combinatorio: il primo
elemento di una disposizione si pu\`o infatti scegliere in
$N$ modi distinti, il secondo in $N-1$, e cos\`\i\ via.  Di
conseguenza $D^N_K$ \`e il prodotto di $K$ numeri interi
decrescenti a partire da $N$:
\begin{equation} \label{eq:a.dnk}
  D^N_K \; = \; N \cdot (N-1) \cdot (N-2) \cdots
    (N-K+1) \; = \; \frac{N!}{(N-K)!}
\end{equation}
(nel caso dell'esempio fatto, le disposizioni sono $D^{21}_2
= 21 \cdot 20 = 420$; nella tabella in cui sono state
elencate vi sono 21 righe di 20 elementi ciascuna).

L'espressione \eqref{eq:a.dnk} \`e verificata anche se
$K=N$, per\`o purch\'e (come prima detto) si ponga $0!=1$.%
\index{disposizioni|)}

\section{Permutazioni}%
\index{permutazioni|(}
Se $N$ \`e un numero intero positivo, si definisce come
numero delle permutazioni di $N$ oggetti, e si indica con
$P_N$, il numero di maniere distinte in cui si possono
ordinare gli $N$ oggetti stessi.  Evidentemente risulta
\begin{equation*}
  P_N \; \equiv \; D^N_N \; = \; N! \peq .
\end{equation*}%
\index{permutazioni|)}

\section{Permutazioni con ripetizione}%
\index{permutazioni!con ripetizione|(}
Se gli $N$ oggetti che si hanno a disposizione sono tali da
poter essere divisi in $M$ gruppi (composti da $N_1,
N_2,\ldots,N_M$ oggetti rispettivamente; ovviamente $N_1 +
N_2 + \cdots+N_M = N$), tali che gli oggetti in ognuno di
questi gruppi siano \emph{indistinguibili} tra loro, il
numero di permutazioni che con essi si possono realizzare
\`e inferiore a $P_N$; pi\`u precisamente, visto che gli
oggetti di ogni gruppo si possono scambiare tra loro in
qualsiasi modo senza per questo dare luogo a una sequenza
distinta, il numero di \emph{permutazioni con ripetizione}
\`e dato da
\begin{equation} \label{eq:a.perip}
  \frac{N!}{N_1! \cdot N_2! \cdots N_M!} \peq .
\end{equation}%
\index{permutazioni!con ripetizione|)}

\section{Combinazioni}%
\index{combinazioni|(}%
\label{ch:a.combina}
Se $N$ e $K$ sono due numeri interi positivi tali che sia $K
\le N$, si definisce come numero delle combinazioni di
classe $K$ di $N$ oggetti il numero dei sottoinsiemi
\emph{distinti} composti da $K$ oggetti che \`e possibile
formare a partire dagli $N$ originali; definendo come
\emph{distinti} due sottoinsiemi se essi differiscono per
qualche elemento.  Il numero delle combinazioni di classe
$K$ di $N$ oggetti si indica con uno dei due simboli
\begin{equation*}
  C^N_K \makebox[20mm]{o} \binom{N}{K}
\end{equation*}%
\index{coefficienti binomiali}
(l'ultimo dei quali si chiama \emph{coefficiente
  binomiale}).

Consideriamo l'insieme composto da tutte le disposizioni di
classe $K$ di $N$ oggetti, e pensiamo di raggruppare i suoi
elementi in sottoinsiemi in modo che ciascuno di essi
contenga tutte e sole quelle disposizioni che differiscano
esclusivamente per l'ordine ma siano composte dagli stessi
oggetti; ovviamente il numero di questi sottoinsiemi \`e
$C^N_K$: ed ognuno di essi contiene un numero di elementi
che \`e $P_K$.

Da qui ricaviamo
\begin{equation} \label{eq:a.binom}
  C^N_K \; \equiv \; \binom{N}{K} \; = \;
    \frac{D^N_K}{P_K} \; = \;
    \frac{N \cdot (N-1) \cdots (N-K+1)}{K \cdot (K-1)
    \cdots 1} \; = \; \frac{N!}{K! \, (N-K)!}
\end{equation}
O, in altre parole, il numero di combinazioni di classe $K$
di $N$ oggetti \`e uguale al rapporto tra il prodotto di $K$
numeri interi decrescenti a partire da $N$ ed il prodotto di
$K$ numeri interi crescenti a partire dall'unit\`a.

\index{coefficienti binomiali|(}%
Si dimostrano poi facilmente, a partire dalla definizione,
due importanti propriet\`a dei coefficienti binomiali:
\begin{gather*}
  \binom{N}{K} = \binom{N}{N-K} \\
  \intertext{e}
  \binom{N+1}{K} = \binom{N}{K-1} + \binom{N}{K} \peq .
\end{gather*}

\`E da osservare che, cos\`\i\ come sono stati ricavati
(dalla definizione delle possibili combinazioni di $N$
oggetti), i coefficienti binomiali hanno senso solo se $N$ e
$K$ sono numeri interi; ed inoltre se risulta sia $N > 0$
che $0 \le K \le N$.  La definizione \eqref{eq:a.binom}
pu\`o comunque essere estesa a valori interi qualunque, ed
anche a valori reali di $N$ --- ma questo esula dal nostro
interesse.%
\index{coefficienti binomiali|)}%
\index{combinazioni|)}

\section{Partizioni ordinate}%
\index{partizioni ordinate|(emidx}%
\label{ch:a.parord}
Consideriamo un insieme di $N$ oggetti; vogliamo calcolare
il numero di maniere in cui essi possono essere divisi in
$M$ gruppi che siano composti da $N_1, N_2,\ldots,N_M$
oggetti rispettivamente (essendo $N_1 + N_2 + \cdots + N_M =
N$).

Gli $N_1$ oggetti che compongono il primo gruppo possono
essere scelti in $C^N_{N_1}$ modi differenti; quelli del
secondo gruppo in $C^{N-N_1}_{N_2}$ modi; e cos\`\i\ via.
Per il lemma fondamentale del calcolo combinatorio, il
numero delle \emph{partizioni ordinate} deve essere uguale a
\begin{multline*}
  \binom{N}{N_1} \binom{N-N_1}{N_2}
    \binom{N-N_1-N_2}{N_3} \cdots
    \binom{N-N_1-\cdots-N_{M-1}}{N_M} = \\[1.5ex]
  = \frac{N!}{N_1! \, (N-N_1)!} \cdot
    \frac{(N-N_1)!}{N_2! \, (N-N_1-N_2)!} \cdots
    \frac{(N-N_1-\cdots-N_{M-1})!}{N_M! \,
    (N-N_1-\cdots-N_M)!} = \\[1.5ex]
  = \frac{N!}{N_1! \, N_2! \cdots N_M!}
\end{multline*}
(sfruttando il fatto che tutti i numeratori dei termini dal
secondo in poi si semplificano con uno dei fattori del
denominatore del termine precedente; inoltre, nell'ultimo
termine, $N-N_1-\cdots-N_M \equiv 0$).  Si pu\`o notare che
l'ultimo termine della prima espressione, essendo
$N-N_1-\cdots-N_{M-1} = N_M$, vale sempre uno; cosa non
sorprendente visto che, quando i primi $M-1$ gruppi sono
stati scelti, anche l'ultimo risulta univocamente
determinato.

Insomma il numero delle partizioni ordinate \`e uguale al
numero delle permutazioni con ripetizione di $N$ oggetti
raggruppabili in $M$ insiemi, composti rispettivamente da
$N_1, N_2,\ldots,N_M$ oggetti indistinguibili tra loro, dato
dalla formula \eqref{eq:a.perip}%
\index{partizioni ordinate|)}%
\index{calcolo combinatorio|)}

\endinput

% $Id: chapterb.tex,v 1.1 2005/03/01 10:06:08 loreti Exp $

\chapter{L'errore della varianza}%
\label{ch:b.errvar}
Pu\`o a volte essere utile valutare l'errore della stima
della varianza ricavata da un campione di dati sperimentali.
Facendo un esempio concreto, supponiamo di disporre di un
ampio insieme di valutazioni della stessa grandezza fisica:
$N \cdot M$ misure ripetute $x_1, x_2,\ldots, x_{N \cdot M}
$.  Dividiamo questi valori in $M$ sottoinsiemi costituiti
da $N$ dati ciascuno, e per ognuno di questi $M$
sottocampioni calcoliamo la media aritmetica dei dati;
otterremo cos\`\i\ $M$ medie parziali, che indicheremo con i
simboli $ \bar x_1, \ldots, \bar x_M $.

Lo scopo di queste operazioni pu\`o essere quello di
verificare che le medie di questi sottocampioni sono
distribuite su un intervallo di valori pi\`u ristretto di
quello su cui si distribuisce l'insieme dei dati originali:
in sostanza, per verificare che le medie di $N$ dati hanno
errore quadratico medio inferiore a quello dei dati di
partenza.

L'errore delle medie dei sottocampioni pu\`o essere stimato
sperimentalmente calcolandone la varianza:
\begin{align*}
  {\sigma_{\bar x}}^2 &= \frac{1}{M-1} \,
    \sum_{i=1}^M \Bigl( \bar x_i - \left< \bar x
    \right> \Bigr) ^2 && \text{\itshape (sperimentale)}
\end{align*}
intendendo con $ \left< \bar x \right> $ la media delle $M$
medie parziali, che coincider\`a necessariamente con la
media complessiva dell'intero campione di $N \cdot M$ dati.

Questo valore pu\`o essere poi confrontato con quello
previsto dalla teoria per la varianza della media di un
gruppo di dati, allo scopo di verificare in pratica
l'adeguatezza della teoria stessa; tale previsione teorica
\`e come sappiamo data dal rapporto tra la varianza di
ognuno dei dati che contribuiscono alla media ed il numero
dei dati stessi:
\begin{align*}
  {\sigma_{\bar x}}^2 &= \frac{\sigma^2}{N}
    && \text{\itshape (teorico)} \peq .
\end{align*}

Come stima di $\sigma$ si pu\`o usare l'errore quadratico
medio dell'insieme di tutti gli $N \cdot M$ dati; ma,
naturalmente, perch\'e il confronto tra questi due numeri
abbia un significato, \emph{occorre conoscere gli errori} da
cui sia la valutazione sperimentale che la previsione
teorica di $\sigma_{\bar x}$ sono affette.

Consideriamo (come gi\`a fatto precedentemente) una
popolazione a media zero per semplificare i calcoli:
\begin{equation*}
  E(x) \; \equiv \; x^* \; = \; 0 \peq ;
\end{equation*}
i risultati si potranno in seguito facilmente estendere ad
una popolazione qualsiasi, tenendo presente il teorema di
pagina \pageref{def:5.varind} ed i ragionamenti conseguenti.
La varianza di una qualsiasi variabile casuale $x$, indicata
di seguito come $\var (x)$, si pu\`o scrivere come
\begin{gather*}
  \var( x ) = E \bigl( x^2 \bigr) -
    \bigl[ E( x ) \bigr]^2 \\
  \intertext{e, usando questa formula per calcolare la
    varianza della varianza di un campione di $N$
    misure $s^2$, avremo}
  \var \bigl( s^2 \bigr) = E \bigl( s^4 \bigr)
    - \left[ E \bigl( s^2 \bigr) \right]^2 \peq .
\end{gather*}

Ora
\begin{align*}
  s^4 &= \left[ \frac{\sum_i {x_i}^2}{N} -
    \left( \frac{\sum_i x_i}{N} \right) ^2 \right]^2
    \\[1ex]
  &= \frac{1}{N^2} \left( \sum \nolimits_i {x_i}^2
    \right) ^2 - \frac{2}{N^3} \left( \sum\nolimits_i
    {x_i}^2 \right) \left( \sum \nolimits_i x_i
    \right)^2 + \frac{1}{N^4} \left( \sum\nolimits_i
    x_i \right) ^4 \peq .
\end{align*}
Sviluppiamo uno per volta i tre termini a secondo membro;
per il primo risulta
\begin{align*}
  \left( \sum \nolimits_i {x_i}^2 \right) ^2 &=
    \left( \sum \nolimits_i {x_i}^2 \right)
    \Bigl( \sum \nolimits_j {x_j}^2 \Bigr)
    \\[1ex]
  &= \sum \nolimits_i \left( {x_i}^2
    \sum_{j=i} {x_j}^2 \right) +
    \sum \nolimits_i \left( {x_i}^2
    \sum_{j \neq i} {x_j}^2 \right) \\[1ex]
  &= \sum_i {x_i}^4 +
    \sum_{\substack{i,j\\j \neq i}} {x_i}^2 {x_j}^2
    \\[1ex]
  &= \sum_i {x_i}^4 +
    2 \sum_{\substack{i,j\\ j < i}} {x_i}^2 {x_j}^2 \peq .
\end{align*}

La prima sommatoria comprende $N$ addendi distinti; la
seconda \`e estesa a tutte le possibili \emph{combinazioni}
dei valori distinti di $i$ e $j$ presi a due a due: \`e
costituita quindi da
\begin{equation*}
  C^N_2 = \frac{N \, (N-1)}{2}
\end{equation*}
addendi distinti.

Il fattore 2 che compare davanti ad essa \`e dovuto al fatto
che una coppia di valori degli indici si presentava nella
sommatoria su $i \neq j$ una volta come ${x_i}^2 {x_j}^2$ e
un'altra come ${x_j}^2 {x_i}^2$, termini diversi per
l'ordine ma con lo stesso valore.  In definitiva, passando
ai valori medi e tenendo conto dell'indipendenza statistica
di $x_i$ e $x_j$ quando \`e $i \neq j$, risulta
\begin{equation*}
  E \left \{ \left( \sum \nolimits_i {x_i}^2 \right) ^2
    \right \} = N \: E \bigl( x^4 \bigr) + N \, (N-1) \,
    \left[ E \bigl( x^2 \bigr) \right] ^2 \peq .
\end{equation*}

Con simili passaggi, si ricava per il secondo termine
\begin{align*}
   \left( \sum \nolimits_i {x_i}^2 \right)
     \Bigl( \sum \nolimits_j x_j \Bigr) ^2 &=
     \left( \sum \nolimits_i {x_i}^2 \right)
     \left( \sum_j {x_j}^2 +
     \sum_{\substack{j,k\\ j \neq k}} x_j x_k
     \right) \\[1ex]
   &= \sum_i {x_i}^4 +
     \sum_{\substack{i,j\\ i \neq j}} {x_i}^2
     {x_j}^2 + \sum_{\substack{i,j\\ i \neq j}}
     {x_i}^3 x_j + \sum_{\substack{i,j,k\\ i \neq j
     \neq k}} {x_i}^2 x_j \, x_k
\end{align*}
dove gli indici aventi simboli diversi si intendono avere
anche valori sempre diversi tra loro nelle sommatorie.

Il valore medio del terzo e del quarto termine si annulla
essendo $E(x)=0$; inoltre gli addendi nella prima sommatoria
sono in numero di $N$ e quelli nella seconda in numero di $N
\, (N-1) /2$ e vanno moltiplicati per un fattore 2.
Pertanto anche
\begin{equation*}
  E \left \{ \left( \sum \nolimits_i x_i^2 \right)
    \left( \sum \nolimits_i x_i \right) ^2 \right \} =
    N \: E \bigl( x^4 \bigr) +
    N \, (N-1) \, \left[ E \bigl( x^2 \bigr) \right] ^2 \peq
    .
\end{equation*}

Infine avremo, con la medesima convenzione sugli indici,
\begin{multline*}
   \left( \sum \nolimits_i x_i \right) ^4 =
     \left( \sum \nolimits_i x_i \right)
     \Bigl( \sum \nolimits_j x_j \Bigr)
     \left( \sum \nolimits_k x_k \right)
     \left( \sum \nolimits_l x_l \right) \\[1ex]
   = \sum_i {x_i}^4 +
     \sum_{\substack{i,j \\ i \neq j}} {x_i}^3 x_j +
     \sum_{\substack{i,j \\ i \neq j}} {x_i}^2 {x_j}^2 +
     \sum_{\substack{i,j,k \\i \neq j \neq k}} {x_i}^2
       x_j \, x_k +
     \sum_{\substack{i,j,k,l \\ i \neq j \neq k \neq l}}
       x_i \, x_j \, x_k \, x_l \peq .
\end{multline*}

I valori medi del secondo, quarto e quinto termine (che
contengono potenze dispari delle $x$) sono nulli.  Gli
addendi nella prima sommatoria sono in numero di $N$; nella
terza vi sono $N \, (N-1) / 2$ termini distinti: ma ciascuno
appare in 6 modi diversi solo per l'ordine, corrispondenti
al numero $ C^4_2 $ di combinazioni dei quattro indici
originari $i$, $j$, $k$ ed $l$ presi a due a due.  Allora
\begin{equation*}
  E \left( \sum \nolimits_i x_i \right) ^4 =
    N \: E \bigl( x^4 \bigr) +
    3 \,N \,(N-1) \, \left[ E \bigl( x^2 \bigr)
    \right] ^2 \peq ;
\end{equation*}
e, riprendendo la formule di partenza,
\begin{equation*}
  E \bigl( s^4 \bigr) =
    \frac{(N-1)^2}{N^3} \, E \bigl( x^4 \bigr) +
    \frac{(N-1)(N^2-2N+3)}{N^3} \,
    \left[ E \bigl( x^2 \bigr) \right] ^2 \peq .
\end{equation*}

Per il valore medio di $s^2$, gi\`a sappiamo come risulti
per la varianza del campione
\begin{gather*}
  E \bigl( s^2 \bigr) = \sigma^2 -
    {\sigma_{\bar x}}^2 \\
  \intertext{inoltre}
  \sigma^2 \; = \; E \left \{ \left( x - x^*
    \right)^2 \right \} \; = \; E \bigl( x^2 \bigr) \\
  \intertext{(essendo $x^*=0 $) e}
  {\sigma_{\bar x}} ^2 \; = \; E \left \{ \left( \bar x
    - x^* \right) ^2 \right \} \; = \;
    \frac{\sigma^2}{N} \\
  \intertext{da cui abbiamo ottenuto a suo tempo la}
  E \bigl( s^2 \bigr) \; = \;
    \frac{N-1}{N} \, \sigma^2 \; = \;
    \frac{N-1}{N} \, E \bigl( x^2 \bigr) \peq .
\end{gather*}

Per la varianza di $s^2$, che vogliamo determinare:
\begin{align*}
  \var \bigl( s^2 \bigr) &= E \bigl( s^4 \bigr) -
    \left[ E \bigl( s^2 \bigr) \right] ^2 \\[1ex]
  &= \frac{(N-1)^2}{N^3} \, E \bigl( x^4 \bigr) +
    \\[1ex]
  &\qquad + \left[ \, \frac{(N-1)(N^2-2N+3)}{N^3}
    \, - \, \frac{(N-1)^2}{N^2} \, \right]
    \left[ E \bigl( x^2 \bigr) \right] ^2 \\[1ex]
  &= \frac{(N-1)^2}{N^3} \, E \bigl( x^4 \bigr)
    \, - \, \frac{(N-1)(N-3)}{N^3} \, \left[ E
    \bigl( x^2 \bigr) \right] ^2 \\[1ex]
  &= \frac{N-1}{N^3} \, \left \{ (N-1) \, E
    \bigl( x^4 \bigr) \, - \, (N-3) \, \left[ E
    \bigl( x^2 \bigr) \right] ^2 \right \} \peq .
\end{align*}

Questa relazione ha validit\`a generale.  \emph{Nel caso poi
  che la popolazione ubbidisca alla legge normale}, potremo
calcolare il valore medio di $ x^4 $ usando la forma
analitica della funzione di Gauss: per distribuzioni normali
qualsiasi, i momenti di ordine pari rispetto alla media sono
dati dalla formula \eqref{eq:8.mopaga}, che qui ricordiamo:
\begin{equation*}
  \mu_{2k} \; = \; E \left\{ \bigl[ x - E(x)
    \bigr]^{2k} \right\} \; = \; \frac{(2k)!}{2^k \,
    k!} \, {\mu_2} ^k \; = \; \frac{(2k)!}{2^k \, k!}
    \, \sigma^{2k} \peq .
\end{equation*}

Per la varianza di $s^2$ se ne ricava
\begin{gather*}
  E \bigl( x^4 \bigr) = 3 \sigma^4 \\
  \intertext{e, sostituendo,}
  \var \bigl( s^2 \bigr) \; = \;
    \frac{2(N-1)}{N^2} \, \left[ E \bigl(
    x^2 \bigr) \right] ^2 \; = \;
    \frac{2(N-1)}{N^2} \, \sigma^4 \peq ;
\end{gather*}
insomma \emph{l'errore quadratico medio della varianza $s^2$
  del campione} vale
\begin{equation*}
  \sigma_{s^2} = \frac{\sqrt{2(N-1)}}{N}
    \, \sigma^2 \peq .
\end{equation*}

La varianza, invece, della stima della varianza della
popolazione
\begin{gather*}
  \sigma^2 = \frac{N}{N-1} \, s^2 \\
  \intertext{vale}
  \var \bigl( \sigma^2 \bigr) \; = \;
    \left( \frac{N}{N-1} \right) ^2 \,
    \var \bigl( s^2 \bigr) \; = \;
    \frac{2}{N-1} \, \sigma^4 \peq ;
\end{gather*}
ed infine \emph{l'errore quadratico medio della stima della
  varianza della popolazione} ricavata dal campione \`e
\begin{equation*}%
\index{errore!della varianza stimata}
  \boxed{ \rule[-7mm]{0mm}{17mm} \quad
    \sigma_{\sigma^2} = \sqrt{
    \dfrac{2}{N-1} }
    \, \sigma^2 \quad}
\end{equation*}

Sottolineiamo ancora come queste formule che permettono di
calcolare, per una popolazione \emph{avente distribuzione
  normale}, gli errori quadratici medi sia della varianza di
un campione di $N$ misure che della stima della varianza
della popolazione ricavata da un campione di $N$ misure,
siano \emph{esatte}.

Se si vuole invece calcolare l'errore da attribuire agli
\emph{errori quadratici medi}, cio\`e alle quantit\`a $s$ e
$\sigma$ radici quadrate delle varianze di cui sopra, non
\`e possibile dare delle formule esatte: la ragione ultima
\`e che il valore medio di $s$ non pu\`o essere espresso in
forma semplice in termini di grandezze caratteristiche della
popolazione.

Per questo motivo \emph{\`e sempre meglio riferirsi ad
  errori di varianze} piuttosto che ad errori di scarti
quadratici medi; comunque, in prima approssimazione,
l'errore di $\sigma$ si pu\`o ricavare da quello su
$\sigma^2$ usando la formula di propagazione:
\begin{gather}
  \var ( \sigma ) \; \approx \; \left(
    \frac{1}{\tfrac{\de \left( \sigma^2 \right)}
    {\de \sigma}} \right)^2 \var \bigl( \sigma^2
    \bigr) \; = \; \frac{1}{4 \, \sigma^2} \, \var
    \bigl( \sigma^2 \bigr) \; = \; \frac{\sigma^2}
   {2 \, (N - 1)} \peq ; \notag \\
  \intertext{cio\`e}%
  \index{errore!dell'errore stimato}
  \boxed{ \rule[-8mm]{0mm}{16mm} \quad
    \sigma_\sigma \approx
    \dfrac{\sigma}{\sqrt{2 \, (N - 1)}} \quad}
    \label{eq:b.errstd}
\end{gather}
(il fatto che questa formula sia approssimata risulta
chiaramente se si considera che la relazione tra $\sigma^2$
e $\sigma$ \`e non lineare).

Una conseguenza dell'equazione \eqref{eq:b.errstd} \`e che
l'errore relativo di $\sigma$ dipende \emph{solo dal numero
  di misure}; diminuisce poi all'aumentare di esso, ma
questa diminuzione \`e inversamente proporzionale alla
radice quadrata di $N$ e risulta perci\`o lenta.

\index{cifre significative|(}%
In altre parole, per diminuire l'errore relativo di $\sigma$
di un ordine di grandezza occorre aumentare il numero delle
misure di \emph{due} ordini di grandezza; $\sigma_\sigma /
\sigma$ \`e (circa) il 25\% per 10 misure, il 7\% per 100
misure ed il 2\% per 1000 misure effettuate: e questo \`e
sostanzialmente il motivo per cui, di norma, si scrive
l'errore quadratico medio \emph{dando per esso una sola
  cifra significativa}.

Due cifre significative \emph{reali} per $\sigma$
corrisponderebbero infatti ad un suo errore relativo
compreso tra il 5\% (se la prima cifra significativa di
$\sigma$ \`e 1, ad esempio $\sigma = 10 \pm 0.5$) e lo 0.5\%
($\sigma = 99 \pm 0.5$); e presupporrebbero quindi che
siano state effettuate almeno 200 misure nel caso pi\`u
favorevole e quasi $20\updot 000$ in quello pi\`u
sfavorevole.%
\index{cifre significative|)}

\endinput

% $Id: chapterc.tex,v 1.1 2005/03/01 10:06:08 loreti Exp $

\chapter{Covarianza e correlazione}%
\label{ch:c.covcor}
\section{La covarianza}%
\index{covarianza|(emidx}%
\label{ch:c.covar}
Per due variabili casuali $x$ ed $y$ si definisce la
\emph{covarianza}, che si indica con uno dei due simboli $
\, \cov (x, y) $ o $ K_{xy} $, nel seguente modo:
\begin{align*}
  \cov (x, y) &= E \Bigl \{ \left[
    \vphantom{y} x - E(x) \right] \left[ y - E(y)
    \right] \Bigr \} \\[1ex]
   &= E(x y) - E(x) \cdot E(y) \peq .
\end{align*}
Per provare l'equivalenza delle due forme, basta osservare
che\/\footnote{Nel seguito useremo per la varianza, per le
  probabilit\`a di ottenere i vari valori $x_i $ o $y_j $ e
  cos\`\i\ via, le stesse notazioni gi\`a introdotte nel
  capitolo \ref{ch:5.varcun}.}
\begin{align*}
   \cov (x, y) &= E \Bigl \{ \left[ \vphantom{y} x
     - E(x) \right] \left[ y - E(y) \right] \Bigr \}
     \\[1ex]
   &= \sum \nolimits_{ij} P_{ij} \left[ \vphantom{y}
     x_i - E(x) \right] \left[ y_j - E(y) \right]
     \\[1ex]
   &= \sum \nolimits_{ij} P_{ij} \, x_i \, y_j \; - \;
     E(x) \sum \nolimits_{ij} P_{ij} \, y_j \; - \;
     E(y) \sum \nolimits_{ij} P_{ij} \, x_i \; +
     \\[1ex]
   &\qquad + E(x) \, E(y) \sum \nolimits_{ij} P_{ij}
     \\[1ex]
   &= E(x y) \; - \; E(x) \sum \nolimits_j q_j \, y_j
     \; - \; E(y) \sum \nolimits_i p_i \, x_i \; + \;
     E(x) \, E(y) \\[1ex]
   &= E(x y) \; - \; E(x) \cdot E(y)
\end{align*}
ricordando alcune relazioni gi\`a ricavate nel capitolo
\ref{ch:5.varcun}, e valide per variabili casuali
\emph{qualunque}: in particolare, anche \emph{non}
statisticamente indipendenti.

\`E chiaro come per variabili \emph{statisticamente
  indipendenti} la covarianza sia nulla:%
\index{statistica!indipendenza|(}
infatti per esse vale la
\begin{equation*}
  E(xy) \; = \; \sum \nolimits_{ij} P_{ij} \, x_i
    \, y_j \; = \; \sum \nolimits_{ij} \, p_i \, q_j
    \, x_i \,y_j \; = \; E(x) \cdot E(y) \peq .
\end{equation*}
Non \`e per\`o vero l'inverso: consideriamo ad esempio le
due variabili casuali $x$ ed $y = x^2$, ovviamente
dipendenti l'una dall'altra: la loro covarianza vale
\begin{equation*}
  \cov(x,y) \; = \; E(xy) - E(x) \cdot E(y) \; = \; E(x^3)
    - E(x) \cdot E(x^2)
\end{equation*}
ed \`e chiaramente nulla per qualunque variabile casuale $x$
con distribuzione simmetrica rispetto allo zero; quindi
l'annullarsi della covarianza \`e condizione
\emph{necessaria ma non sufficiente} per l'indipendenza
statistica di due variabili casuali.%
\index{statistica!indipendenza|)}

\index{combinazioni lineari!varianza!di variabili correlate|(}%
\index{varianza!di combinazioni lineari!di variabili correlate|(}%
Possiamo ora calcolare la varianza delle combinazioni
lineari di due variabili casuali qualunque, estendendo la
formula gi\`a trovata nel capitolo \ref{ch:5.varcun} nel
caso particolare di variabili statisticamente indipendenti;
partendo ancora da due variabili $x$ e $y$ con media zero
per semplificare i calcoli, per la loro combinazione lineare
$z = ax + by $ valgono le:
\begin{align*}
  E(z) \; &= \; a \, E(x) + b \, E(y) \; = \; 0 \\[1ex]
  \cov (x, y) \; &= \; E(xy) - E(x) \cdot E(y) \; = \;
    E(xy) \\[1ex]
  \var (z) \; &= \; E \left\{ \bigl[ z - E(z) \bigr]^2
    \right\} \\[1ex]
  &= \; E \bigl( z^2 \bigr) \\[1ex]
  &= \; E \left[ (ax + by) ^2 \right] \\[1ex]
  &= \; \sum \nolimits_{ij} P_{ij} (a \, x_i +
    b \, y_j) ^2 \\[1ex]
  &= \; a^2 \sum \nolimits_{ij} P_{ij} \,
    {x_i}^2 \; + \; b^2 \sum \nolimits_{ij}
    P_{ij} \, {y_j}^2 \; + \; 2 a b
    \sum \nolimits_{ij} P_{ij} \, x_i \, y_j
    \\[1ex]
  &= \; a^2 \sum \nolimits_i p_i {x_i}^2 \; + \;
    b^2 \sum \nolimits_j q_j {y_j}^2 \; + \;
    2 a b \, E(xy)
\end{align*}
ed infine
\begin{equation} \label{eq:c.varcol}
  \var (z) \; = \; a^2 \, \var (x) + b^2 \, \var (y)
    + 2 a b \, \cov(x, y) \peq .
\end{equation}

Questa si estende immediatamente a variabili casuali con
media qualsiasi: introducendo ancora le variabili ausiliarie
\begin{align*}
  \xi &= x - E(x) && \text{ed} &
  \eta &= y - E(y)
\end{align*}
per le quali gi\`a sappiamo che vale la
\begin{equation*}
  E(\xi) \; = \; E(\eta) \; = \; 0
\end{equation*}
con le
\begin{align*}
  \var (\xi) &= \var (x) && \text{e} &
  \var (\eta) &= \var (y) \peq ;
\end{align*}
basta osservare infatti che vale anche la
\begin{equation*}
  \cov (x, y) \; = \;
    E \Bigl \{ \left[ \vphantom{y} x - E(x)
    \right] \left[ y - E(y) \right] \Bigr \}
    \; = \; \cov (\xi, \eta) \peq .
\end{equation*}

La \eqref{eq:c.varcol} si pu\`o poi generalizzare, per
induzione completa, ad una variabile $z$ definita come
combinazione lineare di un numero qualsiasi $N$ di variabili
casuali: si trova che, se
\begin{gather}
  z = \sum_{i=1}^N a_i \, x_i \notag \\
  \intertext{risulta}
  \var(z) = \sum_i {a_i}^2 \, \var(x_i) +
    \sum_{\substack{i,j \\ j > i}} 2 \, a_i \, a_j \,
    \cov(x_i , x_j) \peq . \label{eq:c.varcon}
\end{gather}

\index{covarianza!matrice di|(emidx}%
Per esprimere in modo compatto la \eqref{eq:c.varcon}, si
ricorre in genere ad una notazione che usa la cosiddetta
\emph{matrice delle covarianze} delle variabili $x$;
ovverosia una matrice quadrata $\boldsymbol{V}$ di ordine
$N$, in cui il generico elemento $V_{ij}$ \`e uguale alla
covarianza delle variabili casuali $x_i$ e $x_j$:
\begin{gather}
  V_{ij} \; = \; \cov(x_i, x_j) \; = \; E (x_i \cdot
  x_j) - E(x_i) \cdot E(x_j) \peq . \label{eq:c.matcov} \\
  \intertext{La matrice \`e ovviamente
    \emph{simmetrica} ($V_{ij} = V_{ji}$); e, in
    particolare, gli elementi diagonali $V_{ii}$
    valgono}
  V_{ii} \; = \; E( {x_i}^2 ) - \bigl[ E (x_i)
    \bigr]^2 \; \equiv \; \var(x_i) \peq . \notag
\end{gather}%
\index{covarianza!matrice di|)}

Consideriamo poi le $a_i$ come le $N$ componenti di un
\emph{vettore} $\boldsymbol{A}$ di dimensione $N$ (che
possiamo concepire come una matrice rettangolare di $N$
righe ed una colonna); ed introduciamo la matrice
\emph{trasposta} di $\boldsymbol{A}$,
$\boldsymbol{\widetilde A}$, che \`e una matrice
rettangolare di una riga ed $N$ colonne i cui elementi
valgono
\begin{gather*}
  \widetilde A_i = A_i \peq . \\
  \intertext{Possiamo allora scrivere la
    \eqref{eq:c.varcon} nella forma}
  \var(z) = \sum_{i,j} \widetilde A_i \: V_{ij}
    \: A_j
\end{gather*}
(la simmetria di $\boldsymbol{V}$ e quella tra
$\boldsymbol{\widetilde A}$ ed $\boldsymbol{A}$ produce,
nello sviluppo delle sommatorie, il fattore 2 che moltiplica
le covarianze); o anche, ricordando le regole del prodotto
tra matrici,
\begin{equation*}
  \boxed{ \rule[-5mm]{0mm}{12mm} \quad
    \var(z) \; = \; \boldsymbol{\widetilde A} \,
      \boldsymbol{V} \, \boldsymbol{A}
    \quad}
\end{equation*}%
\index{varianza!di combinazioni lineari!di variabili correlate|)}%
\index{combinazioni lineari!varianza!di variabili correlate|)}

Si pu\`o poi facilmente dimostrare il seguente teorema, che
ci sar\`a utile pi\`u avanti:
\begin{quote}
  \index{combinazioni lineari!e loro correlazione}%
  \textsc{Teorema:} \textit{due differenti combinazioni
    lineari delle stesse variabili casuali sono sempre
    correlate.  }
\end{quote}
Infatti, dette $A$ e $B$ le due combinazioni lineari:
\begin{align*}
  A &= \sum_{i=1}^N a_i \, x_i &&\text{$\Rightarrow$}
    & E(A) &= \sum_{i=1}^N a_i \, E(x_i) \\[1ex]
  B &= \sum_{j=1}^N b_j \, x_j &&\text{$\Rightarrow$}
    & E(B) &= \sum_{j=1}^N b_j \, E(x_j)
\end{align*}
abbiamo che la covarianza di $A$ e $B$ vale
\begin{align*}
  \cov(A, B) &= E \Biggl\{ \Bigl[ A - E(A) \Bigr] \,
    \Bigl[ B - E(B) \Bigr] \Biggr\} \\[1ex]
  &= E \left\{ \sum_{i,j} a_i \, b_j \, \Bigl[ x_i - E
    \bigl( x_i \bigr) \Bigr] \, \Bigl[ x_j - E \bigl(
    x_j \bigr) \Bigr] \right\} \\[1ex]
  &= \sum_{i,j} a_i \, b_j \: E \Biggl\{ \Bigl[ x_i - E
    \bigl( x_i \bigr) \Bigr] \, \Bigl[ x_j - E \bigl(
    x_j \bigr) \Bigr] \Biggr\} \\[1ex]
  &= \sum_i a_i \, b_i \: \var(x_i) +
    \sum_{\substack{i,j\\i\ne j}} a_i \, b_j \:
    \cov(x_i, x_j)
\end{align*}
e non \`e in genere nulla.  In forma matriciale e con ovvio
significato dei simboli,
\begin{equation*}
  \cov(A, B) = \boldsymbol{\widetilde A} \,
    \boldsymbol{V} \, \boldsymbol{B} \peq .
\end{equation*}

\`E da notare come $A$ e $B$ siano di norma sempre correlate
\emph{anche se le variabili di partenza $x_i$ sono tutte tra
  loro statisticamente indipendenti}: in questo caso infatti
tutti i termini non diagonali della matrice delle covarianze
si annullano, e risulta
\begin{equation} \label{eq:c.covcol}
  \cov( A,B ) = \sum_{i=1}^N a_i \, b_i \, \var( x_i ) \peq .
\end{equation}%
\index{covarianza|)}

\section{La correlazione lineare}%
\index{correlazione lineare, coefficiente di|(emidx}
Per due variabili casuali qualunque si definisce poi il
\emph{coefficiente di correlazione lineare} $ \cor(x,y) $
(anche indicato col simbolo $ r_{xy} $, o semplicemente come
$r$) nel modo seguente:
\begin{equation*}
  r_{xy} \; \equiv \;
    \cor (x, y) \; = \;
    \frac{\cov (x, y)} {\sqrt{ \var(x) \, \var(y) }}
    \; = \; \frac{\cov (x, y)} {\sigma_x \, \sigma_y} \peq .
\end{equation*}

Il coefficiente di correlazione di due variabili \`e
ovviamente adimensionale; \`e nullo quando le variabili
stesse sono statisticamente indipendenti%
\index{statistica!indipendenza}
(visto che \`e zero la loro covarianza); ed \`e comunque
compreso tra i due limiti $-1$ e $+1$.  Che valga
quest'ultima propriet\`a si pu\`o dimostrare calcolando
dapprima la varianza di una variabile casuale ausiliaria $z$
definita attraverso la relazione $z = \sigma_y \, x -
\sigma_x \, y$, ed osservando che essa deve essere una
quantit\`a non negativa:
\begin{align*}
  \var (z) &= {\sigma_y}^2 \, \var(x) + {\sigma_x}^2
    \, \var(y) - 2 \, \sigma_x \, \sigma_y \,
    \cov (x, y) \\[1ex]
  &= 2 \, \var(x) \, \var(y) -
    2 \, \sigma_x \, \sigma_y \, \cov (x, y)
    \\[1ex]
  &\ge 0 \peq ;
\end{align*}
da cui
\begin{equation*}
  \cor (x, y) \le 1 \peq .
\end{equation*}
Poi, compiendo analoghi passaggi su un'altra variabile
definita stavolta come $z = \sigma_y \, x + \sigma_x \, y$,
si troverebbe che deve essere anche $ \cor (x, y) \ge -1 $.

Se il coefficiente di correlazione lineare raggiunge uno dei
due valori estremi $\pm 1$, risulta $\var(z) = 0$; e dunque
deve essere
\begin{equation*}
  z \; = \; \sigma_y \, x \mp \sigma_x \, y
  \; = \; \text{costante}
\end{equation*}
cio\`e $x$ ed $y$ devono essere legati da una relazione
funzionale \emph{di tipo lineare}.

Vale anche l'inverso: partendo infatti dall'ipotesi che le
due variabili siano legate da una relazione lineare data da
$ y =a + b x $, con $b$ finito e non nullo, ne consegue che:

\begin{align*}
  E(y) &= a + b \, E(x) \\[1ex]
  \var (y) &= b^2 \, \var(x) \\[1ex]
  E(xy) &= E(a \, x + b\, x^2) \\[1ex]
  &= a \, E(x) + b \, E(x^2) \\[1ex]
  \cov (x, y) &= E(xy) - E(x) \cdot E(y) \\[1ex]
  &= a \, E(x) + b \, E(x^2) - E(x)
    \bigl[ a + b \, E(x) \bigr] \\[1ex]
  &= b \left\{ E(x^2) - \bigl[ E(x)
    \bigr] ^2 \right\} \\[1ex]
  &= b \cdot \var(x) \\[1ex]
  \cor (x, y) &= \frac{\cov (x, y)}
    {\sqrt{ \var(x) \, \var(y) }} \\[1ex]
  &= \frac{ b \, \var(x) } {\sqrt{ b^2
    \bigl[ \var(x) \bigr] ^2 }} \\[1ex]
  &= \frac{b}{|b|} \\[1ex]
  &= \pm 1 \peq .
\end{align*}

Il segno del coefficiente di correlazione \`e quello del
coefficiente angolare della retta.  Sono da notare due cose:
innanzi tutto il rapporto $b / |b|$ perde significato quando
$b = 0$ o quando $b = \infty$, cio\`e quando la retta \`e
parallela ad uno degli assi coordinati: in questi casi ($x$
= costante o $y$ = costante) una delle due grandezze non \`e
in realt\`a una variabile casuale, e l'altra \`e dunque
indipendente da essa; \`e facile vedere che tanto il
coefficiente di correlazione tra $x$ e $y$ quanto la
covarianza valgono zero, essendo $E(xy) \equiv E(x) \cdot
E(y)$ in questo caso.

Anche quando esiste una relazione funzionale esatta tra $x$
e $y$, se questa non \`e rappresentata da una funzione
lineare il coefficiente di correlazione non raggiunge i
valori estremi $\pm 1 $; per questa ragione appunto esso si
chiama pi\`u propriamente ``coefficiente di correlazione
\emph{lineare}''.%
\index{correlazione lineare, coefficiente di|)}

\section{Propagazione degli errori per variabili
  correlate}%
\index{propagazione degli errori, formula di|(}%
\label{ch:c.proper}
Vediamo ora come si pu\`o ricavare una formula di
propagazione per gli errori (da usare in luogo
dell'equazione \eqref{eq:10.proper} che abbiamo incontrato a
pagina \pageref{eq:10.proper}) se le grandezze fisiche
misurate direttamente non sono tra loro statisticamente
indipendenti; nel corso di questo paragrafo continueremo ad
usare la notazione gi\`a introdotta nel capitolo
\ref{ch:10.misind}.

Consideriamo una funzione $F$ di $N$ variabili, $F = F(x_1,
x_2,\ldots, x_N)$; ed ammettiamo che sia lecito svilupparla
in serie di Taylor nell'intorno del punto $(\bar x_1, \bar
x_2,\ldots, \bar x_N)$ trascurando i termini di ordine
superiore al primo (questo avviene, come sappiamo, o se gli
errori di misura sono piccoli o se $F$ \`e lineare rispetto
a tutte le variabili).  Tenendo presente il teorema di
pagina \pageref{def:5.varind}, ed applicando alla formula
dello sviluppo
\begin{gather}
  F(x_1, x_2,\ldots, x_N) \; \approx \;
  F(\bar x_1, \bar x_2,\ldots, \bar x_N) \; + \;
    \sum_{i=1}^N \frac{\partial F}{\partial x_i}
    \left( x_i - \bar x_i \right) \notag \\
  \intertext{l'equazione \eqref{eq:c.varcon}, otteniamo}
  \var(F) \; \approx \; \sum_i \left( \frac{\partial F}
    {\partial x_i} \right)^2 \! \var(x_i) + 2
    \sum_{\substack{i, j\\ j > i}} \frac{\partial
    F}{\partial x_i} \: \frac{\partial F}{\partial x_j}
    \: \cov(x_i, x_j) \peq . \label{eq:c.proper}
\end{gather}

Per esprimere in modo compatto la \eqref{eq:c.proper}, si
pu\`o ricorrere ancora alla matrice delle covarianze%
\index{covarianza!matrice di}
$\boldsymbol{V}$ delle variabili $x_i$; ricordandone la
definizione (data dall'equazione \eqref{eq:c.matcov} a
pagina \pageref{eq:c.matcov}) ed introducendo poi un vettore
$\boldsymbol{F}$ di dimensione $N$ di componenti
\begin{gather*}
  F_i \; = \; \frac{\partial F}{\partial x_i} \\
  \intertext{ed il suo trasposto $\boldsymbol{\widetilde
    F}$, la \eqref{eq:c.proper} si pu\`o riscrivere nella
    forma}
  \var(F) \; = \; \sum_{i,j} \widetilde F_i \: V_{ij}
    \: F_j \\
  \intertext{ossia}
  \boxed{ \rule[-5mm]{0mm}{12mm} \quad
    \var(F) \; = \; \boldsymbol{\widetilde F}
      \, \boldsymbol{V} \, \boldsymbol{F}
  \quad}
\end{gather*}%
\index{propagazione degli errori, formula di|)}

\section{Applicazioni all'interpolazione lineare}%
\index{interpolazione lineare|(}
Riprendiamo adesso il problema dell'interpolazione lineare,
gi\`a discusso nel capitolo \ref{ch:11.teldat}: si sia
cio\`e compiuto un numero $N$ di misure indipendenti di
coppie di valori di due grandezze fisiche $x$ e $y$, tra le
quali si ipotizza che esista una relazione funzionale di
tipo lineare data da $y=a+bx $.  Supponiamo inoltre che
siano valide le ipotesi esposte nel paragrafo
\ref{ch:11.intlin}; in particolare che le $x_i$ siano prive
di errore, e che le $y_i$ siano affette da errori normali e
tutti uguali tra loro.

\subsection{Riscrittura delle equazioni dei minimi
  quadrati}%
\label{ch:c.intmed}
Sebbene i valori della $x$ siano scelti dallo sperimentatore
e privi di errore, e non siano pertanto variabili casuali in
senso stretto; e sebbene la variabilit\`a delle $y$ sia
dovuta non solo agli errori casuali di misura ma anche alla
variazione della $x$, introduciamo ugualmente (in maniera
\emph{puramente formale}) le medie e le varianze degli $N$
valori $x_i$ e $y_i$, date dalle espressioni
\begin{gather*}
  \bar x \; = \; \frac{ \sum_i x_i }{N}
  \makebox[30mm]{e}
  \var (x) \; = \; \frac{ \sum_i \left( x_i
    - \bar x \right) ^2 }{N}
    \; = \; \frac{ \sum_i {x_i}^2 }{N} -
    {\bar x} ^2 \\
  \intertext{(e simili per la $y $); e la covarianza
    di $x$ e $y$, data dalla}
  \cov (x, y) \; = \; \frac{ \sum_i x_i \,
    y_i }{N} \, - \, \bar x \bar y \peq .
\end{gather*}

Queste grandezze permettono di riscrivere le equazioni
\eqref{eq:11.normeq} risolutive del problema
dell'interpolazione lineare per un insieme di dati, che
abbiamo gi\`a incontrato a pagina \pageref{eq:11.normeq},
nella forma
\begin{equation*}%
  \index{minimi quadrati, formule dei}
  \left\{ \begin{array}{lclcl}
    a & + & b \, \bar x & = & \bar y \\[1ex]
    a \, \bar x & + & b \, \bigl[ \var(x) +
      {\bar x}^2 \bigr] & = &
    \cov(x,y) \; + \; \bar x \bar y
  \end{array} \right.
\end{equation*}

La prima equazione intanto implica che la retta interpolante
deve passare per il punto $(\bar x, \bar y)$ le cui
coordinate sono le medie dei valori misurati delle due
variabili in gioco; poi, ricavando da essa $a = \bar y - b
\bar x$ e sostituendo nella seconda equazione, dopo aver
semplificato alcuni termini si ottiene la soluzione per
l'altra incognita:
\begin{equation} \label{eq:c.valueb}
  b \; = \; \frac{\cov (x, y)}{ \var (x) } \;
    \equiv \; \cor(x,y) \, \sqrt{ \frac{\var(y)}
    {\var(x)} }
\end{equation}
e la retta interpolante ha quindi equazione
\begin{gather*}
  y \; = \; a + bx \; = \; \left( \bar y - b \bar x
    \right) + b x \\
  \intertext{o anche}
  \left( y - \bar y \right) \; = \; b \left(
    x - \bar x \right) \\
  \intertext{(in cui $b$ ha il valore
    \eqref{eq:c.valueb}).
    Introduciamo ora le due variabili casuali
    ausiliarie $\xi = x - \bar x $ e $\eta = y
    - \bar y$, per le quali valgono le}
  \bar \xi \; = \; 0
    \makebox[30mm]{e}
    \var (\xi) \; = \; \var (x) \\
  \intertext{(con le analoghe per $\eta$ ed $y $),
    ed inoltre la}
  \cov (\xi, \eta) = \cov (x, y)
\end{gather*}
ed indichiamo poi con $\widehat y_i$ il valore della $y$
sulla retta interpolante in corrispondenza dell'ascissa $x_i
$:
\begin{equation} \label{eq:c.resid}
  \widehat y_i \; = \; a + b x_i \; = \;
    \bar y + b \left( x_i - \bar x \right)
\end{equation}
e con $\delta_i$ la differenza $\widehat y_i - y_i $.  Le
differenze $\delta_i$ prendono il nome di \emph{residui},%
\index{residui}
e di essi ci occuperemo ancora pi\`u avanti; risulta che
\begin{align*}
  \sum\nolimits_i {\delta_i}^2 &= \sum\nolimits_i \Bigl \{ \left[
    \bar y + b \left( x_i - \bar x \right) \right]
    - y_i \Bigr \} ^2 \\[1ex]
  &= \sum\nolimits_i \left( b \xi_i - \eta_i \right) ^2
    \\[1ex]
  &= b^2 \sum\nolimits_i {\xi_i}^2 + \sum\nolimits_i {\eta_i}^2
    - 2 b \sum\nolimits_i \xi_i \eta_i \\[1ex]
  &= N \, b^2 \, \var(\xi) + N \, \var(\eta)
    - 2 N b \, \cov(\xi, \eta) \\[1ex]
  &= N \, b^2 \, \var(x) + N \, \var(y)
    - 2 N b \, \cov(x, y) \\[1ex]
  &= N \, \var(x) \left[ \frac{ \cov(x, y) }
    { \var(x) } \right] ^2
    + N \, \var(y) - 2 N \, \frac{ \cov(x, y) }
    { \var(x) } \, \cov(x, y) \\[1ex]
  &= N \left\{ \var(y)
    - \frac{ \left[ \cov(x, y) \right] ^2 }
    { \var(x) } \right\} \\[1ex]
  &= N \; \var(y) \: ( 1 - r^2 )
\end{align*}%
\index{correlazione lineare, coefficiente di|(}
in cui $r$ \`e il coefficiente di correlazione lineare
calcolato usando, sempre solo formalmente, i campioni dei
valori misurati delle $x$ e delle $y$.

Visto che quest'ultimo, nel calcolo dell'interpolazione
lineare fatto con le calcolatrici da tasca, viene in genere
dato come sottoprodotto dell'algoritmo, la sua conoscenza
permette (usando questa formula) di ottenere facilmente
l'errore a posteriori sulle ordinate interpolate (studiato
nel paragrafo \ref{ch:11.fisher}):
\begin{equation}%
  \label{eq:c.fishalt}%
  \index{errore!a posteriori}
  \boxed{ \rule[-6mm]{0mm}{16mm} \quad
    \mu_y \; = \; \sqrt{
    \dfrac{\sum_i {\delta_i}^2}{N-2} } \; = \;
    \sqrt{ \dfrac{N}{N-2} \, \var(y)
    \left( 1 - r^2 \right)}
  \quad }
\end{equation}
oltre a fornire una grossolana stima dell'allineamento dei
punti; quanto pi\`u infatti esso \`e rigoroso, tanto pi\`u
$r$ si avvicina a $\pm 1$.

Il valore $\mu_y$ dell'errore rappresenta sempre anche una
stima dell'allineamento dei punti, a differenza del
coefficiente di correlazione lineare, per cui $r = \pm 1$
implica allineamento perfetto, ma \emph{non} inversamente:
potendo essere ad esempio (punti su di una retta parallela
all'asse $x$) $r = 0$ e $\var(y) = 0$ ancora con
allineamento perfetto.

\`E opportuno qui osservare che $r$ \emph{non} \`e il
coefficiente di correlazione fra gli \emph{errori} delle
grandezze $x$ ed $y $; infatti, per ipotesi, la $x$ non \`e
affetta da errore e, se pur lo fosse, la correlazione fra
gli errori sarebbe nulla qualora ciascuna $x_i$ fosse
misurata indipendentemente dalla corrispondente $y_i$ o,
altrimenti, sarebbe tanto pi\`u prossima ai valori estremi
$\pm 1$ quanto maggiore fosse il numero di cause d'errore
comuni alle misure di $x$ e di $y$.

Invece $r$ \`e il coefficiente di correlazione per l'insieme
dei punti aventi coordinate date dalle coppie di valori
misurati, e nell'ipotesi di effettiva dipendenza lineare
delle grandezze $x$ ed $y$ sarebbe sempre rigorosamente
uguale a $\pm 1$ se non intervenissero gli errori
sperimentali.%
\index{correlazione lineare, coefficiente di|)}

\subsection{Verifica di ipotesi sulla correlazione lineare}%
\index{correlazione lineare, coefficiente di|(}
Non \`e pensabile di poter svolgere una teoria completa
della correlazione in queste pagine; tuttavia talvolta un
fisico si trova nella necessit\`a di verificare delle
ipotesi statistiche sul coefficiente di correlazione lineare
ricavato sperimentalmente da un insieme di $N$ coppie di
valori misurati $\{ x_i, y_i \}$.  Nel seguito riassumiamo,
senza darne alcuna dimostrazione, alcune propriet\`a di
questi coefficienti:
\begin{itemize}
\item Se si vuole verificare che la correlazione tra $x$ ed
  $y$ sia significativamente differente da zero, si pu\`o
  calcolare il valore della variabile casuale
  \begin{equation*}
    t = \frac{r \sqrt{N-2}}{\sqrt{1 - r^2}}
  \end{equation*}
  che \`e distribuita secondo Student%
  \index{distribuzione!di Student}
  con $(N-2)$ gradi di libert\`a, e controllare la sua
  compatibilit\`a con lo zero.
\item Rispetto al valore vero $\rho$ della correlazione
  lineare tra le due variabili, il valore $r$ ricavato da un
  insieme di $N$ coppie $\{ x_i, y_i \}$ estratte a caso
  dalle rispettive popolazioni \`e tale che la variabile
  casuale
  \begin{equation} \label{eq:c.varfish}
    Z(r) \; = \; \ln \sqrt{ \frac{1 + r}{1 - r} } \; = \;
      \frac{1}{2} \, \bigl[ \ln ( 1 + r ) - \ln ( 1 - r)
      \bigr]
  \end{equation}
  (detta \emph{variabile di Fisher}%
  \index{Fisher, sir Ronald Aylmer}%
  ) segue una distribuzione approssimativamente normale con
  valore medio e varianza date da
  \begin{align*}
    E(Z) &= Z(\rho) &&\text{e} & \var(Z) &= {\sigma_Z}^2 =
      \frac{1}{N-3}
  \end{align*}
  rispettivamente; la trasformazione inversa della
  \eqref{eq:c.varfish} \`e la
  \begin{equation*}
    r \; = \; \frac{e^{2Z} - 1}{e^{2Z} + 1} \peq .
  \end{equation*}
  Quindi:
  \begin{itemize}
  \item per verificare se il valore vero della correlazione
    pu\`o essere una quantit\`a prefissata $\rho$, si
    controlla la compatibilit\`a con la distribuzione
    normale $N(Z(\rho), \sigma_Z)$ del valore ottenuto
    $Z(r)$;
  \item per calcolare un intervallo di valori corrispondente
    ad un certo livello di confidenza, si usano i
    corrispondenti intervalli per la distribuzione normale
    con deviazione standard $\sigma_Z$;
  \item per verificare se due coefficienti di correlazione
    lineare $r_1$ ed $r_2$, ricavati da $N_1$ ed $N_2$
    coppie di valori $\{ x_i, y_i \}$ rispettivamente, siano
    o meno significativamente differenti, si calcola la
    variabile casuale
    \begin{equation*}
      \delta \; =  \; \frac{Z_1 - Z_2}{\sqrt{\var( Z_1 ) +
      \var( Z_2 ) }} \; = \; \frac{Z_1 - Z_2}{\sqrt{
      \dfrac{1}{N_1 - 3} + \dfrac{1}{N_2 - 3} } }
    \end{equation*}
    (ove $Z_1$ e $Z_2$ sono le variabili di Fisher ricavate
    dalla \eqref{eq:c.varfish} per i due campioni; $\delta$
    segue asintoticamente la distribuzione normale con media
    $E(Z_1) - E(Z_2)$ e varianza 1) e si verifica se il
    risultato ottenuto \`e compatibile con lo zero.
  \end{itemize}
\end{itemize}%
\index{correlazione lineare, coefficiente di|)}

\subsection{La correlazione tra i coefficienti della retta}
Notiamo anche che l'intercetta e la pendenza $a$ e $b$ della
retta interpolante un insieme di punti sperimentali, essendo
ottenute come combinazioni lineari delle stesse variabili
casuali, sono tra loro correlate (come osservato nel
paragrafo \ref{ch:c.covar}); le equazioni
\eqref{eq:11.minqua} possono essere riscritte (l'abbiamo
gi\`a visto nel paragrafo \ref{ch:11.intlin}) come
\begin{align*}
  a &= \sum\nolimits_i a_i \, y_i &&\text{e} &
  b &= \sum\nolimits_i b_i \, y_i
\end{align*}
una volta che si sia posto
\begin{equation*}
  \begin{cases}
    a_i = \displaystyle \frac{1}{\Delta} \,
      \left[ \sum\nolimits_j {x_j}^2 - \left(
      \sum\nolimits_j x_j \right) x_i \right]
      \\[3ex]
    b_i = \displaystyle \frac{1}{\Delta} \left[
      N \, x_i - \sum\nolimits_j x_j \right]
  \end{cases}
\end{equation*}
Ricordando la definizione di $\Delta$, possiamo esprimerlo
come
\begin{align*}
  \Delta &= N \left( \sum\nolimits_j {x_j}^2
    \right) - \left( \sum\nolimits_j x_j \right)^2 \\[1ex]
  &= N^2 \left[ \frac{\sum\nolimits_j {x_j}^2}{N} -
    \left( \frac{\sum\nolimits_j x_j}{N} \right)^2
    \right]
\end{align*}
ed infine come
\begin{equation} \label{eq:c.delta}
  \Delta = N^2 \: \var(x) \peq .
\end{equation}

Visto che le $y_i$ sono per ipotesi statisticamente
indipendenti tra loro, possiamo applicare la
\eqref{eq:c.covcol}; ne ricaviamo, sostituendovi la
\eqref{eq:c.delta} e ricordando che gli errori sono tutti
uguali, che
\begin{align*}
  \cov( a, b ) &= \sum\nolimits_i a_i \, b_i \,
    \var(y_i) \\[1ex]
  &= \frac{{\sigma_y}^2}{\Delta^2} \, \sum\nolimits_i
    \left[ \sum\nolimits_j {x_j}^2 - \left(
    \sum\nolimits_j x_j \right) x_i \right] \left[ N
    x_i - \sum\nolimits_j x_j \right] \\[1ex]
  &= \frac{{\sigma_y}^2}{\Delta^2} \, \sum\nolimits_i
    \Biggl[ \left( \sum\nolimits_j {x_j}^2 \right) N
    x_i - \left( \sum\nolimits_j {x_j}^2  \right)
    \left( \sum\nolimits_j x_j \right) -
    \\[1ex]
  &\qquad - \left( \sum\nolimits_j x_j \right) N
    {x_i}^2 + \left( \sum\nolimits_j x_j \right)^2 \!
    x_i \Biggr] \\[1ex]
  &= \frac{{\sigma_y}^2}{\Delta^2} \, \Biggl[ N \left(
    \sum\nolimits_j {x_j}^2 \right) \left(
    \sum\nolimits_i x_i \right) - N \left(
    \sum\nolimits_j {x_j}^2 \right) \left(
    \sum\nolimits_j x_j \right) - \\[1ex]
  &\qquad - N \left( \sum\nolimits_j x_j \right) \left(
    \sum\nolimits_i {x_i}^2 \right) + \left(
    \sum\nolimits_j x_j \right)^3 \Biggr]
    \\[1ex]
  &= \frac{{\sigma_y}^2}{\Delta^2} \, \left(
    \sum\nolimits_j x_j \right) \, \left[ \left(
    \sum\nolimits_j x_j \right)^2 - N \left(
    \sum\nolimits_j {x_j}^2 \right) \right]
\end{align*}
ed infine, vista la definizione di $\Delta$,
\begin{gather}%
\index{covarianza!dei coefficienti della retta interpolante}
  \cov( a, b ) = - \, \frac{\sum\nolimits_j
    x_j}{\Delta} \, {\sigma_y}^2 \peq ; \label{eq:c.covab} \\
  \intertext{o anche}
  \cov( a, b ) = - \, \frac{\bar x}{N} \,
    \frac{{\sigma_y}^2}{\var(x)} \peq , \notag
\end{gather}
diversa da zero se $\bar x \ne 0$; inoltre il segno della
correlazione tra $a$ e $b$ \`e opposto a quello di $\bar x$.
Questo non sorprende: la retta interpolante deve
necessariamente passare per il punto $(\bar x, \bar y)$,
come abbiamo notato nel paragrafo \ref{ch:c.intmed}: se
$\bar x$ \`e positivo, aumentando la pendenza della retta
deve diminuire il valore della sua intercetta; e viceversa
per $\bar x < 0$.

\subsection{Stima puntuale mediante l'interpolazione
  lineare}
Bisogna tener presente che le formule dei minimi quadrati ci
danno l'equazione della retta che meglio approssima la
relazione tra le due variabili nella parte di piano in cui
esistono punti misurati; ma non ci dicono nulla sulla
dipendenza tra le variabili stesse in zone in cui non siano
state effettuate delle osservazioni.

In altre parole, non si pu\`o mai escludere sulla base delle
misure che la $y = f(x)$ sia una funzione comunque complessa
ma approssimativamente lineare \emph{solamente}
nell'intervallo in cui abbiamo investigato; per questo
motivo bisogna evitare per quanto possibile di usare
l'equazione della retta interpolante per ricavare valori
stimati $\widehat y$ della variabile indipendente (dalla
$\widehat y = a + b x$) in corrispondenza di valori della
$x$ non compresi nell'intorno delle misure, e questo tanto
pi\`u rigorosamente quanto pi\`u $x$ \`e distante da tale
intorno: \emph{non \`e lecito usare l'interpolazione per
  estrapolare} su regioni lontane da quelle investigate.

A questo proposito, se lo scopo primario dell'interpolazione
non \`e tanto quello di ottenere una stima di $a$ o $b$
quanto quello di ricavare il valore $\widehat y$ assunto
dalla $y$ in corrispondenza di un particolare valore
$\widehat x$ della variabile indipendente $x$, applicando la
formula di propagazione degli errori \eqref{eq:c.proper}
alla $\widehat y = a + b \widehat x$ ricaviamo:
\begin{equation*}
  \var(\widehat y) \; = \; \var(a) + {\widehat x}^2 \,
  \var(b) + 2 \, \widehat x \, \cov(a,b) \peq .
\end{equation*}

Sostituendo nell'equazione precedente le espressioni
\eqref{eq:11.errab} per le varianze di $a$ e $b$ e quella
\eqref{eq:c.covab} della loro covarianza, si ha poi
\begin{gather*}
  \var(\widehat y) \; = \; \frac{\sum\nolimits_i {x_i}^2}
  {\Delta} \, {\sigma_y}^2 + {\widehat x}^2 \,
    \frac{N}{\Delta} \, {\sigma_y}^2 - 2 \, \widehat x \,
    \frac{\sum\nolimits_i x_i}{\Delta} \, {\sigma_y}^2 \\
  \intertext{e, introducendo nell'equazione precedente sia
    l'espressione \eqref{eq:c.delta} per $\Delta$ che la}
  \sum\nolimits_i {x_i}^2 = N \, \left[ \var(x) + \bar x^2
    \right] \\
  \intertext{e la}
  \sum\nolimits_i x_i = N \bar x
\end{gather*}
otteniamo
\begin{align*}
  \var(\widehat y) &= \frac{{\sigma_y}^2}{\Delta} \, \left[
    \sum\nolimits_i {x_i}^2 - 2 \widehat x \sum\nolimits_i
    x_i + N {\widehat x}^2 \right] \\[1ex]
  &= \frac{{\sigma_y}^2}{N \, \var(x)} \, \left[ \var(x) +
    \bar x^2 - 2 \widehat x \bar x + {\widehat x}^2 \right]
\end{align*}
ed infine la
\begin{equation} \label{eq:c.errhaty}
  \var(\widehat y) = \frac{{\sigma_y}^2}{N} \, \left[ 1 +
    \frac{\left( \widehat x - \bar x \right)^2}{\var(x)}
  \right] \peq .
\end{equation}

Si vede immediatamente dalla \eqref{eq:c.errhaty} che
l'errore sul valore stimato $\widehat y$ (che \`e funzione
di $\widehat x$) \`e minimo quando $\widehat x = \bar x$:
quindi, per ricavare una stima della $y$ con il pi\`u
piccolo errore casuale possibile, bisogna che il
corrispondente valore della $x$ sia nel centro
dell'intervallo in cui si sono effettuate le misure (questo
in accordo con le considerazioni qualitative precedenti a
riguardo di interpolazione ed estrapolazione).

\subsection{Verifica di ipotesi nell'interpolazione lineare}
Nel paragrafo \ref{ch:11.intlin} abbiamo ricavato le formule
\eqref{eq:11.minqua} dei minimi quadrati e le formule
\eqref{eq:11.errab} per l'errore dei coefficienti della
retta interpolata: queste ultime richiedono la conoscenza
dell'errore comune sulle ordinate $\sigma_y$ che, di
consueto, viene ricavato a posteriori dai dati attraverso
l'equazione \eqref{eq:c.fishalt}.

Assai di frequente \`e necessario verificare delle ipotesi
statistiche sui risultati dell'interpolazione lineare; una
volta ricavata, ad esempio, la pendenza della retta
interpolante, si pu\`o o voler confrontare la stima ottenuta
con un valore noto a priori, o voler costruire attorno ad
essa un intervallo corrispondente ad un certo livello di
confidenza; o, ancora, si pu\`o voler effettuare il
confronto (o calcolare l'ampiezza dell'intervallo di
confidenza) per un valore $\widehat y = a + b \widehat x$
della $y$ stimato sulla base dell'interpolazione, e la cui
varianza \`e data dalla \eqref{eq:c.errhaty}.

\`E naturale pensare di sfruttare per questo scopo le
tabelle della distribuzione normale; ma questo
implicitamente richiede che il numero di coppie di dati a
disposizione \emph{sia sufficientemente elevato} perch\'e la
stima di $\sigma_y$ ottenuta a posteriori dai dati si possa
considerare esatta.  In realt\`a quando l'errore \`e
ricavato a posteriori \emph{tutte} le grandezze
precedentemente citate \emph{non} seguono la distribuzione
normale \emph{ma la distribuzione di Student}%
\index{distribuzione!di Student}
con $(N-2)$ gradi di libert\`a.

\subsection{Adeguatezza dell'interpolazione lineare o
  polinomiale in genere}
Talvolta non si sa a priori che la relazione tra due
variabili $x$ ed $y$ \`e di tipo lineare; ma, una volta
eseguite le misure, si nota un loro approssimativo disporsi
lungo una linea retta.  In questo caso si pu\`o cercare di
inferire una legge fisica dai dati sperimentali; ma come si
pu\`o essere ragionevolmente sicuri che la relazione tra le
due grandezze sia effettivamente lineare, anche limitandoci
all'intervallo in cui sono distribuite le osservazioni?

Una possibilit\`a \`e quella di eseguire interpolazioni
successive con pi\`u curve polinomiali di grado $M$
crescente, del tipo
\begin{equation} \label{eq:c.polin}
  y \; = \; P_M(x) \; = \; \sum_{k=0}^M a_k \, x^k
\end{equation}
e di osservare l'andamento del \emph{valore} dei residui in
funzione di $M$: al crescere del grado del polinomio questi
diminuiranno, dapprima in modo rapido per poi assestarsi su
valori, sempre decrescenti, ma pi\`u o meno dello stesso
ordine di grandezza; ed infine si annulleranno quando $M =
N-1$.  Il valore di $M$ che segna la transizione tra questi
due comportamenti ci d\`a il grado della curva polinomiale
che descrive in modo soddisfacente la relazione tra le
variabili senza per questo seguire le fluttuazioni casuali
di ogni singola misura.

Nel caso in cui ognuno degli $N$ valori $y_i$ ha errore noto
$\sigma_i$ (e le $y_i$ non sono correlate), la somma dei
quadrati dei residui%
\index{residui}
pesati in maniera inversamente proporzionale ai quadrati
degli errori,
\begin{equation*}
  S_M \; = \; \sum_{i=1}^N \frac{ {\delta_i}^2 }{
    {\sigma_i}^2 } \; = \; \sum_{i=1}^N \frac{\left( y_i -
    \widehat y_i \right)^2}{ {\sigma_i}^2 }
\end{equation*}
(le $\widehat y_i$ si intendono calcolate usando il
polinomio interpolante $y = P_M(x)$ \eqref{eq:c.polin}, i
cui $M+1$ coefficienti siano stati stimati dai dati) \`e
distribuita come il $\chi^2$ a $N - M - 1$ gradi di
libert\`a; la probabilit\`a di ottenere un determinato
valore di $S_M$ sotto l'ipotesi $y = P_M(x)$ pu\`o dunque
essere stimata dalle tabelle della distribuzione, e ci d\`a
una misura statisticamente corretta dell'``accordo
complessivo'' tra i punti misurati ed un polinomio di grado
$M$.

Per valutare numericamente la significativit\`a della
variazione di questo ``accordo complessivo'' dovuta ad un
aumento di grado del polinomio interpolante, la regola di
somma del $\chi^2$ ci dice che la variazione della somma
pesata dei quadrati dei residui, $S_{M-1} - S_M$, deve
essere distribuita come il $\chi^2$ ad un grado di
libert\`a; normalmente, in luogo di usare direttamente
$S_{M-1} - S_M$, si considera l'altra variabile casuale
\begin{equation*}
  F = \frac{\phantom{m} S_{M-1} - S_M \phantom{m}}{
    \dfrac{S_M}{N - M - 1} }
\end{equation*}
(che rappresenta una sorta di ``variazione relativa
dell'accordo complessivo''), e la si confronta con la
funzione di frequenza di Fisher a 1 e $(N - M - 1)$ gradi di
libert\`a; un valore di $F$ elevato, e che con piccola
probabilit\`a potrebbe essere ottenuto da quella
distribuzione, implicher\`a che l'aumento di grado \`e
significativo.

\subsection{Il \emph{run test} per i residui}%
\index{run test|(}%
\index{residui|(}
Un'altra tecnica che ci permette di capire se una funzione
di primo grado \`e o meno adeguata a rappresentare un
insieme di dati \`e quella che consiste nell'osservare
l'andamento, in funzione della $x$, del solo segno dei
residui $\delta_i$ differenza tra i valori misurati e quelli
stimati della $y$:
\begin{equation*}
  \delta_i \; = \; y_i - \widehat{y}_i \; = \; y_i - (a + b
  \, x_i) \peq .
\end{equation*}

\begin{figure}[htbp]
  \vspace*{2ex}
  \begin{center}
    \input{runfig.pstex_t}
  \end{center}
  \caption{Esempio di interpolazione lineare per un insieme
    di 12 punti.}
  \label{fig:c.intex}
\end{figure}

Per meglio chiarire questo concetto, osserviamo la figura
\ref{fig:c.intex} tratta dal paragrafo 8.3.2 del testo di
Barlow citato nella bibliografia (appendice
\ref{ch:g.biblio}, a pagina \pageref{ch:g.biblio}).  \`E
evidente come l'andamento dei dati sperimentali non
suggerisca affatto l'ipotesi di una dipendenza lineare del
tipo $y=A+Bx$; questo anche se l'entit\`a degli errori
assunti sulle $y_i$ fa s\`\i\ che l'accordo tra i dati
stessi e la retta interpolante, se valutato con il calcolo
del $\chi^2$, risulti comunque accettabile: infatti il
metodo citato usa come stima la somma dei quadrati dei
\emph{rapporti tra i residui e gli errori stimati},
ovviamente piccola se questi ultimi sono stati
sopravvalutati.

Quello che \`e in grado di suggerire il sospetto di un
andamento non lineare della legge $y=y(x)$, in casi come
questo, \`e un altro tipo di controllo basato appunto
\emph{sul solo segno dei residui e non sul loro valore}
(come il calcolo del $\chi^2$, o dell'errore a posteriori, o
del coefficiente di correlazione lineare, o dalla somma
pesata dei quadrati dei residui).  Segni che siano (come
nell'esempio di figura \ref{fig:c.intex}) per piccole $x$
tutti positivi, poi tutti negativi, ed infine tutti positivi
per i valori pi\`u grandi delle $x$ suggeriranno che si sta
tentando di approssimare con una retta una funzione che in
realt\`a \`e una curva pi\`u complessa (ad esempio una
parabola) avente concavit\`a rivolta verso l'alto.

Cominciamo con l'osservare che il valore medio $\bar \delta$
dei residui \`e identicamente nullo: infatti dalla
\eqref{eq:c.resid} ricaviamo immediatamente
\begin{equation*}
  \bar \delta = \frac{1}{N} \sum_{i=1}^N \Bigl[ \left( y_i -
      \bar y \right) - b \left( x_i - \bar x \right) \Bigr]
  \equiv 0 \peq .
\end{equation*}
Questo \`e dovuto al fatto che sia la somma degli scarti
$x_i - \bar x$ che quella degli scarti $y_i - \bar y$ sono
identicamente nulle in conseguenza della
\eqref{eq:4.mprop1}, ed \`e vero quindi indipendentemente
dal valore di $b$: questa propriet\`a vale insomma per
residui calcolati rispetto a \emph{qualunque} retta del
fascio di centro $(\bar x, \bar y)$ cui sappiamo che la
retta interpolante \emph{deve} appartenere.

Continuiamo osservando che i residui $\delta_i$ e le
coordinate $x_i$ hanno tra loro covarianza nulla:
\begin{align*}
  \cov (\delta, x) &= \frac{1}{N} \sum_{i=1}^N \left[ \left(
      \delta_i - \bar \delta \right) \left( x_i - \bar x
    \right) \right] \\[1ex]
  &= \frac{1}{N} \sum_{i=1}^N \left[ \left( y_i - \bar y
    \right) \left( x_i - \bar x \right) - b \left( x_i -
      \bar x \right)^2 \right] \\[1ex]
  &= \cov(x, y) - b \var(x) \\[1ex]
  &\equiv 0
\end{align*}
(si \`e sfruttata, alla fine, la \eqref{eq:c.valueb}).%
\index{residui|)}
Questa condizione non \`e sufficiente, come si sa, ad
assicurare l'indipendenza statistica tra i residui $\delta$
e le coordinate $x$ dei punti interpolati; in effetti queste
due variabili casuali \emph{non} sono tra loro indipendenti,
essendo ad esempio impossibile che i residui si presentino
in una sequenza crescente all'aumentare delle ascisse $x_i$.

Per\`o, quando il numero $N$ dei dati \`e grande, delle $N!$
sequenze possibili di residui assai poche sono quelle
escluse a priori dalla natura della loro origine; mentre la
probabilit\`a delle altre sequenze (che decresce in maniera
inversamente proporzionale a $N!$) \`e comunque assai
piccola: e si pu\`o in prima approssimazione assumere che
\emph{tutte} le sequenze di residui siano \emph{possibili ed
  equiprobabili}.

Tornando alla figura \ref{fig:c.intex}, i residui sono
(muovendosi nel senso delle $x$ crescenti) dapprima
positivi, poi negativi, poi ancora positivi; sono composti
insomma da una sequenza di tre sottoinsiemi di valori aventi
tutti lo stesso segno (o, con parola anglosassone, da una
sequenza di tre \emph{runs}).  Se possiamo assumere
equiprobabili tutte le possibili sequenze dei residui \`e
intuitivo capire come un numero cos\`\i\ basso di runs si
debba presentare con piccola probabilit\`a sulla base di
fluttuazioni unicamente casuali; per cui l'osservazione di
tale evento pu\`o essere attribuita invece alla falsit\`a
dell'ipotesi che ha prodotto i residui, ovvero alla non
linearit\`a della dipendenza funzionale $y = y(x)$.

Nell'ipotesi di avere un insieme composto da $N_+$ numeri
positivi (corrispondenti, nel caso dei residui, a punti al
di sopra della retta interpolante) e da $N_- = N - N_+$
numeri negativi (residui di punti al di sotto della retta
interpolante), \`e possibile calcolare quante delle loro
permutazioni producono un prefissato numero di runs $N_r$;
se \`e accettabile l'approssimazione cui abbiamo appena
accennato, il rapporto tra $N_r$ ed il numero totale di
permutazioni possibili ci dar\`a la probabilit\`a di
ottenere un certo $N_r$.

I calcoli dettagliati si possono trovare nel citato
paragrafo 8.3.2 del Barlow o, pi\`u in breve, nel paragrafo
11.3.1 di Eadie \emph{et al.} (testo sempre citato nella
bibliografia); qui ricordiamo solo come:
\begin{enumerate}
\item se $N_+ = 0$ o $N_- = 0$ (caso impossibile questo per
  i residui di un'interpolazione lineare) l'unico valore
  possibile \`e $N_r = 1$.
\item Se $N_+ > 0$ e $N_- > 0$, ed indicando con $m =
  \min(N_+, N_-)$ il pi\`u piccolo di questi due valori, il
  numero di runs $N_r$ \`e compreso tra i seguenti estremi:
  \begin{equation*}
    \begin{cases}
      2 \le N_r \le 2 m & \qquad \qquad \text{se}~~N_+ =
      N_- = m~; \\[1ex]
      2 \le N_r \le 2 m + 1 & \qquad \qquad \text{se}~~N_+
      \ne N_-~.
    \end{cases}
  \end{equation*}

  Il massimo valore di $N_r$ corrisponde, nel primo caso, ad
  un alternarsi di valori positivi e negativi; nel secondo,
  a singoli valori del segno che si \`e presentato con
  minore frequenza che separino sequenze di uno o pi\`u
  valori dell'altro segno.
\item Se $N_r$ \`e pari, con $N_r = 2s$, la probabilit\`a di
  avere $N_r$ runs \`e data da
  \begin{equation} \label{eq:c.pari}
    \Pr(N_r) = 2 \, \frac{ \displaystyle \binom{N_+ - 1}{s -
        1} \binom{N_- - 1}{s - 1} }{ \displaystyle
      \binom{N}{N_+} } \peq ;
  \end{equation}
\item se $N_r$ \`e dispari, con $N_r = 2s - 1$ (ed
  ovviamente $s \ge 2$ essendo $N_+>0$ e $N_->0$), la
  probabilit\`a di avere $N_r$ runs \`e data da
  \begin{equation} \label{eq:c.dispari}
    \Pr(N_r) = \frac{ \displaystyle \binom{N_+ - 1}{s - 2}
      \binom{N_- - 1}{s - 1} + \binom{N_+ - 1}{s - 1}
      \binom{N_- - 1}{ s - 2} }{ \displaystyle
      \binom{N}{N_+} } \peq .
  \end{equation}
\item In ogni caso, valore medio e varianza di $N_r$ valgono
  rispettivamente
  \begin{gather*}
    E( N_r ) = 1 + 2 \, \frac{N_+ N_-}{N} \\
    \intertext{e}
    \var( N_r ) = 2 \, \frac{N_+ N_- \left( 2 N_+ N_- - N
      \right)}{N^2 \, ( N - 1 )} \peq .
  \end{gather*}
\end{enumerate}

Nel caso della figura \ref{fig:c.intex} ($N_+ = N_- = 6$) la
probabilit\`a di ottenere casualmente $N_r \le 3$, calcolata
applicando direttamente le formule \eqref{eq:c.pari} e
\eqref{eq:c.dispari}, vale appena l'1.3\%; \`e insomma
lecito (almeno ad un livello di confidenza del 98.7\%)
rigettare l'ipotesi di un andamento lineare della $y$ in
funzione di $x$.%
\index{run test|)}%
\index{interpolazione lineare|)}

\section{Applicazioni alla stima di parametri}
Se la densit\`a di probabilit\`a $f(x;\theta)$ di una
variabile casuale $x$ dipende da un parametro di valore vero
ignoto $\theta^*$, abbiamo visto nel capitolo
\ref{ch:11.teldat} che una stima $\widehat \theta$ di tale
valore pu\`o essere ottenuta col metodo della massima
verosimiglianza; e che \`e possibile ricavare dalla derivata
seconda della funzione di verosimiglianza (che ne misura la
concavit\`a nel punto di massimo) l'errore di questa stima,
attraverso l'equazione \eqref{eq:11.varlik}.

Il metodo si pu\`o ovviamente estendere alla stima
contemporanea di pi\`u parametri (e lo abbiamo in effetti
usato, ad esempio, per ricavare i due coefficienti della
retta interpolante nel paragrafo \ref{ch:11.intlin}): se $x$
\`e la variabile misurata, di densit\`a di probabilit\`a
$f(x; \theta_1, \theta_2,\ldots,\theta_M)$ dipendente da $M$
parametri $\theta_k$, le stime di massima verosimiglianza
$\widehat \theta_k$ si ricavano risolvendo il sistema
ottenuto annullando contemporaneamente ognuna delle derivate
parziali prime (ed esaminando poi ognuna delle eventuali
soluzioni per controllare se corrisponde ad un massimo).

I punti enunciati nel paragrafo \ref{ch:11.maxver}
continuano a rimanere validi anche nel caso
multidimensionale: in particolare, ognuna delle stime
$\widehat \theta_k$ \`e asintoticamente normale; e si
troverebbe che le derivate seconde della funzione di
verosimiglianza nel punto di minimo sono legate all'inversa
della matrice delle covarianze delle $M$ stime attraverso la
\begin{equation} \label{eq:c.varlik}
  \left( \boldsymbol{V}^\mathbf{-1} \right)_{ij} = - N \cdot
    E \left\{ \frac{ \partial^2 \, \ln f( x; \theta_1,
    \theta_2,\ldots,\theta_M) }{\partial \theta_i \,
    \partial \theta_j } \right\}
\end{equation}
che si pu\`o pensare come la generalizzazione a pi\`u stime
contemporanee dell'equazione \eqref{eq:11.varlik}.

Come esempio, abbiamo gi\`a visto nel paragrafo
\ref{ch:11.exampl} come stimare \emph{contemporaneamente} i
due parametri $\mu$ e $\sigma$ di una popolazione normale da
un campione di $N$ determinazioni indipendenti col metodo
della massima verosimiglianza; e vogliamo ora ricavare gli
errori di quelle stime dalla \eqref{eq:c.varlik}.

Ricordiamo dal paragrafo \ref{ch:11.exampl} che il logaritmo
della densit\`a di probabilit\`a vale
\begin{equation*}
  \ln f ( x; \mu, \sigma ) \; = \; - \ln \sigma - \ln \sqrt{
    2 \pi } - \frac{1}{2} \left( \frac{x - \mu}{ \sigma }
    \right)^2 \peq ;
\end{equation*}
e che le due stime di massima verosimiglianza per $\mu$ e
$\sigma$ sono
\begin{align*}
  \widehat \mu &= \bar x = \frac{1}{N} \sum_{i=1}^N x_i
   &&\text{e} &
   \widehat \sigma^2 &= \frac{1}{N} \sum_{i=1}^N \left( x_i
     - \widehat \mu \right)^2
\end{align*}
rispettivamente.
Le derivate prime di $\ln f$ sono
\begin{align*}
  \frac{\partial}{\partial \mu} \ln f &= \frac{x -
    \mu}{\sigma^2}
  &
  \frac{\partial}{\partial \sigma} \ln f &= -
    \frac{1}{\sigma} + \frac{( x - \mu )^2}{\sigma^3}
\end{align*}
e le derivate seconde
\begin{align*}
  \frac{\partial^2}{\partial \mu^2} \ln f &= -
    \frac{1}{\sigma^2} &
  \frac{\partial^2}{\partial \mu \, \partial \sigma} \ln f
    &= - 2 \, \frac{x - \mu}{\sigma^3} \\[2ex]
  \frac{\partial^2}{\partial \sigma \, \partial \mu
    } \ln f &= - 2 \, \frac{x - \mu}{\sigma^3} &
  \frac{\partial^2}{\partial \sigma^2} \ln f &=
    \frac{1}{\sigma^2} - 3 \, \frac{( x - \mu )^2}{\sigma^4}
    \\
  \intertext{di valori medi}
  E \left( \frac{\partial^2}{\partial \mu^2} \ln f \right)
    &= - \frac{1}{\sigma^2} &
  E \left( \frac{\partial^2}{\partial \mu \, \partial
    \sigma} \ln f \right) &= 0 \\[2ex]
  E \left( \frac{\partial^2}{\partial \sigma \,
    \partial \mu } \ln f \right) &= 0 &
  E \left( \frac{\partial^2}{\partial \sigma^2} \ln f
    \right) &= - \frac{2}{\sigma^2} \peq ;
\end{align*}
per cui, dalla \eqref{eq:c.varlik}, l'inverso della matrice
delle covarianze (che \`e diagonale) \`e
\begin{gather*}
  \boldsymbol{V}^\mathbf{-1} = \left\|
    \begin{array}{cc}
      \tabtop \dfrac{N}{\widehat \sigma^2} & 0 \\[2ex]
      \tabbot 0 & \dfrac{2 N}{\widehat \sigma^2}
    \end{array}
  \right\| \\
  \intertext{e la matrice $\boldsymbol{V}$ stessa vale}
  \boldsymbol{V} = \left\|
    \begin{array}{cc}
      \tabtop \dfrac{\widehat \sigma^2}{N} & 0 \\[2ex]
      \tabbot 0 & \dfrac{\widehat \sigma^2}{2 N}
    \end{array}
  \right\| \peq .
\end{gather*}

Insomma, oltre alla consueta espressione della varianza
della media
\begin{equation*}
  \var( \widehat \mu ) = \frac{\widehat \sigma^2}{N}
\end{equation*}
abbiamo ottenuto quella della varianza di $\widehat \sigma$
\begin{equation*}
  \var( \widehat \sigma ) = \frac{\widehat \sigma^2}{2 N}
\end{equation*}
da confrontare con la \eqref{eq:b.errstd}, in cui per\`o
$\widehat \sigma$ era gi\`a stato corretto, moltiplicandolo
per un fattore $N/(N-1)$, per eliminare la distorsione della
stima; e la riconferma del fatto, gi\`a visto nel paragrafo
\ref{th:12.inmest} a pagina \pageref{th:12.inmest}, che
valore medio e varianza di un campione di stime indipendenti
sono variabili casuali statisticamente indipendenti tra loro
(le covarianze infatti sono nulle).%
\index{media!aritmetica!e varianza}%
\index{varianza!e media aritmetica}

\endinput

% $Id: chapterd.tex,v 1.1 2005/03/01 10:06:08 loreti Exp $

\chapter{Il modello di Laplace e la funzione di Gauss}%
\index{distribuzione!normale|(}%
\label{ch:d.applap}
Pensiamo di eseguire una misura di una grandezza fisica (il
cui valore vero indicheremo con il simbolo $x^*$), e sia $x$
il risultato ottenuto; in generale $x$ \`e diverso da $x^*$
per la presenza degli errori di misura, che supporremo siano
di natura puramente casuale.

\index{Laplace!modello di|(}%
Questi errori casuali di misura possono essere schematizzati
come un insieme estremamente grande, al limite infinito, di
disturbi contemporanei molto piccoli, al limite
infinitesimi, ognuno dei quali tende ad alterare di
pochissimo il risultato della misura; si considerino in
particolare le seguenti ipotesi (\emph{modello semplificato
  di Laplace\thinspace\footnote{Pierre Simon de Laplace
    visse in Francia dal 1749 al 1827; famoso matematico,
    fisico ed astronomo, prov\`o la stabilit\`a del sistema
    solare, svilupp\`o la teoria delle equazioni
    differenziali e dei potenziali, contribu\`\i\ allo
    studio del calore e dei fenomeni capillari oltre a
    gettare le basi matematiche per una teoria
    dell'elettromagnetismo.  Durante la rivoluzione francese
    fu uno degli ideatori del sistema metrico decimale; per
    quel che riguarda la statistica, nel 1812 pubblic\`o il
    trattato ``Th\'eorie Analytique des Probabilit\'es'' che
    contiene, tra l'altro, studi sulla distribuzione normale
    e la derivazione della regola dei minimi quadrati.}%
  \index{Laplace!Pierre Simon de|emidx}
  per gli errori di misura}):
\begin{enumerate}
\item \textit{Ognuna delle singole cause di disturbo
    presenti introdurr\`a nella misura una variazione
    rispetto al valore vero di modulo fisso $\epsilon$, con
    uguale probabilit\`a in difetto o in eccesso.}
\item \textit{Ognuna delle variazioni nella misura dovute a
    queste cause di disturbo \`e statisticamente
    indipendente dalle altre.}
\end{enumerate}

Ognuna delle $N$ cause indipendenti di disturbo produce
quindi la variazione $+\epsilon$ con probabilit\`a $p = 0.5$
oppure $-\epsilon$ con probabilit\`a $q = 1-p = 0.5$; se $M$
tra le $N$ perturbazioni sono positive (e le altre $N-M$
negative), il valore osservato sar\`a
\begin{equation*}
  x \; = \;
    x^* + M \epsilon - (N-M) \epsilon \; = \;
    x^* + (2M-N) \epsilon \peq .
\end{equation*}

La probabilit\`a di un dato valore di $M$ sulle $N$ prove
\`e data dalla distribuzione binomiale (vedi il paragrafo
\ref{ch:8.binom}, ed in particolare l'equazione
\eqref{eq:8.binom}), e vale
\begin{equation*}
  P(M,N) \; = \; \frac{N!}{M! \, (N-M)!} \,
    p^M q^{N-M} \peq .
\end{equation*}
Il valore medio di $M$ \`e dato da $Np$, e la sua varianza
da $Npq$; indichiamo poi con il simbolo $\lambda$ lo scarto
di $M$ dal suo valore medio
\begin{equation*}
  M \; = \; Np + \lambda \peq .
\end{equation*}

In corrispondenza al variare di $M$ tra 0 ed $N$, $\lambda$
varia tra i limiti $-Np$ e $+Nq$; risulta poi anche
\begin{gather*}
  N-M \; = \; N - Np - \lambda \; = \;
    Nq - \lambda \\
  \intertext{e la probabilit\`a di ottenere un
    certo valore di $\lambda$ su $N$ prove vale}
  P(\lambda,N) \; = \; \frac{N!}{(Np+\lambda)!
    \: (Nq-\lambda)!} \, p^{Np+\lambda} \,
    q^{Nq-\lambda} \peq . \\
  \intertext{Valore medio e varianza di $\lambda$
    valgono poi}
  E(\lambda) \; = \; E(M) - Np \; \equiv \; 0 \\
  \intertext{e}
  \var (\lambda) \; = \; \var (M) \; = \; Npq \peq .
\end{gather*}

L'andamento generale della probabilit\`a in funzione di $M$
si pu\`o trovare considerando il rapporto tra i valori di
$P$ che corrispondono a due valori successivi di $M $:
\begin{align*}
  \frac{P(M+1, N)}{P(M, N)} &=
    \frac{N! \, p^{M+1} \, q^{N-M-1}}{(M+1)!
    \, (N-M-1)!} \: \frac{M! \, (N-M)!}{N!
    \, p^M \, q^{N-M}} \\[1ex]
  &= \frac{N-M}{M+1} \, \frac{p}{q}
\end{align*}
e $P(M, N)$ risulter\`a minore, uguale o maggiore di $P(M+1,
N)$ a seconda che $(M+1)q$ risulti minore, uguale o maggiore
di $(N-M)p$; ossia, essendo $p+q=1$, a seconda che $M$ sia
minore, uguale o maggiore di $Np-q$.

Insomma, chiamato $\mu = \lceil Np-q \rceil$ il pi\`u
piccolo intero non minore di $Np-q$, la sequenza di valori
$P(0, N), P(1, N),\ldots, P(\mu, N)$ \`e crescente, mentre
quella dei valori $P(\mu+1, N), P(\mu+2, N),\ldots, P(N, N)$
\`e decrescente.  Il massimo valore della probabilit\`a si
ha in corrispondenza ad un intero $\mu$ che soddisfi la
\begin{equation*}
  Np-q \; \le \; \mu \; \le \; Np-q+1 \; = \; Np+p
\end{equation*}
e che \`e unico, salvo il caso che i due estremi
dell'intervallo siano entrambi numeri interi: in questo caso
si hanno due valori massimi, uguali, in corrispondenza di
entrambi.  Concludendo: il caso pi\`u probabile \`e che
l'evento $E$ si presenti in una sequenza di $N$ prove $Np$
volte, ed il valore di $\lambda$ con la massima
probabilit\`a di presentarsi \`e 0.

Cerchiamo ora di determinare se esiste e quanto vale il
limite della probabilit\`a di ottenere un certo risultato al
crescere indefinito del numero delle prove.  Per ottenere
questo, introduciamo la \emph{formula approssimata di de
  Moivre e Stirling}\thinspace\footnote{Per la
  dimostrazione, vedi ad esempio: G. Castelnuovo -- Calcolo
  delle probabilit\`a (Zanichelli), in appendice.  La
  formula \`e dovuta al solo Stirling,%
  \index{Stirling, James}
  che la pubblic\`o nel suo libro ``Methodus
  Differentialis'' del 1730; ma non divenne nota nel mondo
  scientifico fino a quando de Moivre%
  \index{de Moivre!Abraham}
  non la us\`o --- da qui il nome comunemente adottato.}
per il fattoriale:
\begin{gather*}%
  \index{de Moivre!e Stirling, formula di}
  N! \; = \; N^N e^{-N} \sqrt{2 \pi N} \, ( 1 +
    \epsilon_N ) \; \approx \; \sqrt{ 2 \pi} \,
    N^{\left( N+\frac{1}{2} \right)} e^{-N} \\
  \intertext{con}
  0 \; \le \; \epsilon_N \; < \; \frac{1}{11 \cdot N} \peq .
\end{gather*}

\`E lecito trascurare il resto $\epsilon_N$ quando
l'argomento del fattoriale \`e elevato: per $N=10$ l'errore
commesso \`e gi\`a inferiore all'1\%.  Per usare la formula
di de Moivre e Stirling nel nostro caso, sviluppiamo
\begin{align*}
  (Np+\lambda)! &\approx \sqrt{2 \pi}
    \, (Np+\lambda)^{\left( Np+\lambda + \frac{1}{2}
    \right)} e^{\left( -Np -\lambda \right)} \\[1ex]
  &= \sqrt{2 \pi}
    \left( 1 + \frac{\lambda}{Np} \right) ^{\left( Np
    +\lambda +\frac{1}{2} \right)} e^{\left( -Np -
    \lambda \right)} (Np)^{\left( Np +\lambda
    +\frac{1}{2} \right)}
\end{align*}
e, similmente,
\begin{equation*}
  (Nq-\lambda)! \approx \sqrt{2 \pi}
    \left( 1 - \frac{\lambda}{Nq} \right)
    ^{\left( Nq -\lambda +\frac{1}{2} \right)}
    e^{\left( -Nq +\lambda \right)}
    (Nq)^{\left( Nq -\lambda +\frac{1}{2} \right)} \peq .
\end{equation*}

Queste approssimazioni sono valide quando gli argomenti dei
fattoriali, $Np + \lambda$ e $Nq - \lambda$, sono abbastanza
grandi: cio\`e quando $\lambda$ non \`e vicino ai valori
limite $-Np$ e $Nq$; accettata la loro validit\`a (e
ritorneremo su questo punto tra poco), sostituendo si ha
\begin{equation*}
  P(\lambda, N) \; = \; \frac{1}{\sqrt{ 2 \pi N p q }}
    \left( 1 + \frac{\lambda}{Np} \right)
    ^{- \left( Np +\lambda +\frac{1}{2}
    \right)} \left( 1 - \frac{\lambda}{Nq} \right)
    ^{ - \left( Nq -\lambda +\frac{1}{2} \right)} \peq .
\end{equation*}

Questa espressione \`e certamente valida quando $|\lambda|$
non \`e troppo grande, e per $\lambda=0$ fornisce la
probabilit\`a del valore medio di $M$ ($M=Np$), che risulta
\begin{equation*}
  P(0, N) = \frac{1}{\sqrt{2 \pi N p q}} \peq .
\end{equation*}

Questa probabilit\`a tende a zero come $1/\sqrt{N}$ al
crescere di $N$; dato che la somma delle probabilit\`a
relative a tutti i casi possibili deve essere 1, si deve
concludere che il numero di valori di $\lambda$ per cui la
probabilit\`a non \`e trascurabile rispetto al suo massimo
deve divergere come $\sqrt{N}$ al crescere di $N$, sebbene
il numero di tutti i possibili valori (che \`e $N+1$)
diverga invece come $N$.

L'espressione approssimata di $P(\lambda, N)$ non \`e valida
per valori di $\lambda$ prossimi agli estremi $\lambda=-Np$
e $\lambda=Nq$ (\`e infatti divergente); tuttavia tali
valori hanno probabilit\`a infinitesime di presentarsi al
crescere di $N$.  Infatti $P(-Np, N) = q^N$ e $P(Nq, N) =
p^N$, ed entrambi tendono a zero quando $N$ tende
all'infinito essendo sia $p$ che $q$ inferiori all'unit\`a.

Concludendo: la formula approssimata da noi ricavata \`e
valida gi\`a per valori relativamente piccoli di $N$, e per
$N$ molto grande si pu\`o ritenere esatta per tutti i valori
dello scarto $\lambda$ con probabilit\`a non trascurabile di
presentarsi, valori che sono mediamente dell'ordine
dell'errore quadratico medio $\sqrt{Npq}$ e che quindi
divergono solo come $\sqrt{N}$.  Consideriamo ora il fattore
\begin{gather*}
  \kappa = \left( 1+ \frac{\lambda}{Np} \right) ^{- \left(
      Np +\lambda +\frac{1}{2} \right)} \left( 1
    -\frac{\lambda}{Nq} \right) ^{-
    \left( Nq -\lambda +\frac{1}{2} \right)} \\
  \intertext{che nell'espressione approssimata di
    $P(\lambda, N)$ moltiplica il valore massimo $P(0, N)$,
    e se ne prenda il logaritmo naturale:} \ln \kappa \; =
  \; - \left( Np +\lambda +\frac{1}{2} \right) \ln \left( 1
    +\frac{\lambda}{Np} \right) \; - \; \left( Nq -\lambda
    +\frac{1}{2} \right) \ln \left( 1 -\frac{\lambda}{Nq}
  \right) \peq .
\end{gather*}

Ora, poich\'e sia $\lambda / Np$ che $\lambda /Nq$ sono in
modulo minori dell'unit\`a (salvi i due casi estremi, di
probabilit\`a come sappiamo infinitesima), si possono
sviluppare i due logaritmi in serie di McLaurin:
\begin{equation*}
  \ln (1+x) \; = \; x -\frac{x^2}{2} + \frac{x^3}{3}
    - \frac{x^4}{4} + \cdots \peq .
\end{equation*}
Il primo termine di $\ln \kappa$ diventa
\begin{equation*}
  \begin{split}
    - \biggl( Np +\lambda &+ \frac{1}{2} \biggr)
      \left( \frac{\lambda}{Np} -
      \frac{\lambda^2}{2 \, N^2 p^2}
      +\frac{\lambda^3}{3 \, N^3 p^3} - \cdots
      \right) \; = \\[1ex]
    &= - \lambda + \left( \frac{\lambda^2}{2 \, Np}
      - \frac{\lambda^2}{Np} \right)
      - \left( \frac{\lambda^3}{3 \, N^2 p^2}
      - \frac{\lambda^3}{2 \, N^2 p^2}
      + \frac{\lambda}{2 \, Np} \right)
      + \cdots \\[1ex]
    &= - \, \lambda - \frac{\lambda^2}{2 \, Np} -
      \frac{\lambda}{2 \, Np} +
      \frac{\lambda^3}{6 \, N^2 p^2} + \cdots
  \end{split}
\end{equation*}
ed il secondo
\begin{multline*}
  - \left( Nq - \lambda + \frac{1}{2} \right)
    \left( - \frac{\lambda}{Nq} -
    \frac{\lambda^2}{2 \, N^2 q^2} -
    \frac{\lambda^3}{3 \, N^3 q^3} -
    \cdots \right) \; = \\[1ex]
  = \lambda - \frac{\lambda^2}{2 \, Nq} +
    \frac{\lambda}{2 \, Nq} -
    \frac{\lambda^3}{6 \, N^2 q^2} - \cdots
\end{multline*}
e sommando si ottiene
\begin{equation*}
  \ln \kappa = - \frac{\lambda^2}{2 \, Npq} -
    \frac{\lambda}{2 \, N} \left(
    \frac{1}{p} - \frac{1}{q} \right)
    + \frac{\lambda^3}{6 \, N^2} \left(
    \frac{1}{p^2} - \frac{1}{q^2} \right)
    + \cdots \peq .
\end{equation*}

Da questo sviluppo risulta che il solo termine che si
mantiene finito al divergere di $N$, e per valori di
$\lambda$ dell'ordine di $\sqrt{Npq}$, \`e il primo; gli
altri due scritti convergono a zero come $1/\sqrt{N}$, e
tutti gli altri omessi almeno come $1/N$.  In conclusione,
per valori dello scarto per cui la probabilit\`a non \`e
trascurabile (grosso modo $ |\lambda| < 3 \sqrt{Npq} $), al
divergere di $N$ il logaritmo di $\kappa$ \`e bene
approssimato da
\begin{gather*}
  \ln \kappa \; \approx \; - \,
    \frac{\lambda^2}{2 \, Npq} \\
  \intertext{e la probabilit\`a dello scarto dalla
    media $\lambda$ da}
  P (\lambda) \; \approx \; \frac{1}{\sqrt{ 2 \pi
    N p q}} \, e^{- \frac{1}{2}
    \frac{\lambda^2}{Npq} } \peq ; \\
  \intertext{per la variabile $M$ sar\`a invece}
  P(M) \; \approx  \;\frac{1}{\sqrt{2 \pi N p q}} \,
    e^{- \frac{1}{2} \frac{(M-Np)^2}{Npq} } \peq .
\end{gather*}

Nel caso particolare del modello semplificato di Laplace per
gli errori di misura, $p=q=0.5$ e pertanto i termini di
ordine $1/\sqrt{N}$ sono identicamente nulli:
l'approssimazione \`e gi\`a buona per $N \ge 25$; nel caso
generale $p \neq q$, essa \`e invece accettabile per $Npq
\ge 9$.  Introducendo lo scarto quadratico medio di $M$ e di
$\lambda$
\begin{gather*}
  \sigma = \sqrt{Npq} \\
  \intertext{l'espressione si pu\`o scrivere}
  P(\lambda) \; \approx \; \frac{1}{\sigma
    \sqrt{2 \pi}} \, e^{- \frac{\lambda^2}{2 \sigma^2} }
\end{gather*}
che \`e la celebre \emph{legge normale} o \emph{legge di
  Gauss}.

Tornando ancora al modello semplificato di Laplace per gli
errori di misura, il risultato $x$ ha uno scarto dal valore
vero che vale
\begin{gather*}
  x - x^* \; = \;
    \epsilon \, (2M-N) \; = \;
    \epsilon \, (2Np +2\lambda -N) \; = \;
    2 \epsilon \lambda \\
  \intertext{e possiede varianza}
  {\sigma_x}^2 \; \equiv \;
    \var \left( x - x^* \right) \; = \;
    4 \, \epsilon^2 \, \var (\lambda) \; = \;
    4 \epsilon^2 \sigma^2 \peq . \\
  \intertext{La probabilit\`a di un certo
    risultato $x = x^* + 2 \epsilon \lambda $
    vale infine}
  P(x) \; = \; P(\lambda) \; \approx \;
    \frac{1}{\sigma \sqrt{2 \pi} } \,
     e^{- \frac{1}{2}
    \frac{\lambda^2}{\sigma^2}} \; = \;
    \frac{2 \epsilon}{\sigma_x \sqrt{2 \pi}}
    \, e^{- \frac{1}{2} \bigl(
    \frac{x - x^*}{\sigma_x} \bigr) ^2 } \peq .
\end{gather*}%
\index{Laplace!modello di|)}

La $x$ \`e una grandezza discreta che varia per multipli di
$\epsilon$; nel limite su accennato diventa una variabile
continua, e $P(x)$ \`e infinitesima con $\epsilon$ perdendo
cos\`\i\ significato; si mantiene invece finita la densit\`a
di probabilit\`a, che si ottiene dividendo $P(x)$ per
l'ampiezza $2 \epsilon$ dell'intervallo che separa due
valori contigui di $x$:
\begin{equation*}
  f(x) \; = \;
    \frac{P(x)}{2 \epsilon} \; = \;
    \frac{1}{\sigma_x \sqrt{2 \pi}} \,
    e^{- \frac{1}{2} \bigl(
    \frac{x-x^*}{\sigma_x} \bigr) ^2 }
\end{equation*}
ed ha infatti le dimensioni fisiche di $1/\sigma_x$, ovvero
di $1/x$.

Al medesimo risultato per $f(x)$ si perverrebbe anche
nell'ipotesi pi\`u generale che gli errori elementari siano
distribuiti comunque, ed anche diversamente l'uno
dall'altro, purch\'e ciascuno abbia una varianza dello
stesso ordine di grandezza degli altri ed infinitesima al
divergere del numero delle cause di errore.%
\index{distribuzione!normale|)}

\endinput

% $Id: chaptere.tex,v 1.1 2005/03/01 10:06:08 loreti Exp $

\chapter{La funzione di verosimiglianza}%
\label{ch:e.maxlik}
Si supponga di aver compiuto $N$ osservazioni indipendenti
relative ad una grandezza fisica $x$, e di aver trovato i
valori $x_i$, con $i=1,2,\ldots,N$.  Ciascuna delle
variabili casuali $x_i$ abbia poi densit\`a di probabilit\`a
data da una funzione nota $ f_i (x_i ; \theta) $; funzione
che supponiamo dipenda da un parametro $\theta$ di valore
vero $\theta^*$ ignoto, e definita in un intervallo
dell'asse reale delle $x_i$ con estremi indipendenti da
$\theta$ (che potremo assumere essere $\pm \infty$ ponendo
eventualmente $ f_i (x_i ; \theta) \equiv 0$ esternamente
all'intervallo di definizione).

Una \emph{stima} di una generica funzione nota del
parametro, $\tau(\theta)$, che supporremo con derivata non
nulla, \`e una funzione dei soli valori osservati $t(x_1,
x_2,\ldots, x_N) $; dunque a sua volta una variabile
casuale, con associata una funzione densit\`a di
probabilit\`a che indicheremo con $g(t;\theta)$.  La stima
si dice \emph{imparziale}%
\index{stima!imparziale|emidx}
(o \emph{indistorta}) quando il suo valore medio
\begin{align*}
   E(t) &= \int_{-\infty}^{+\infty} \! t
     \: g(t;\theta) \, \de t \\[1ex]
   &= \int_{-\infty}^{+\infty} \! \de x_1
     \, f_1 (x_1;\theta) \cdots
     \int_{-\infty}^{+\infty} \! \de x_N \,
     f_N (x_N;\theta) \: t(x_1,x_2,\ldots,x_N)
\end{align*}
\`e uguale al rispettivo valore vero:
\begin{equation*}
  E(t) = \tau(\theta) \peq .
\end{equation*}
Il caso particolare della stima \emph{del parametro stesso}
corrisponde alla funzione $\tau(\theta) = \theta$, che
soddisfa evidentemente alla richiesta di possedere derivata
prima non nulla $\tau'(\theta) = 1$.

Una importante propriet\`a della stima $t$ \`e la sua
varianza, data (se essa \`e imparziale) da
\begin{multline*}
   {\sigma_t}^2 = \int_{-\infty}^{+\infty} \!
     \bigl[ t - \tau(\theta) \bigr] ^2
     g(t;\theta) \, \de t \\[1ex]
   = \int_{-\infty}^{+\infty} \! \de x_1 \,
     f_1 (x_1;\theta) \cdots
     \int_{-\infty}^{+\infty} \! \de x_N \,
     f_N (x_N;\theta) \:
     \bigl[ t(x_1,x_2,\ldots,x_N) -
     \tau(\theta) \bigr] ^2
\end{multline*}
perch\'e \emph{la minima varianza} sar\`a il nostro criterio
di scelta fra diverse stime di $\tau(\theta)$.

\index{Cram\'er--Rao, teorema di|(emidx}%
Il teorema che segue (\textbf{teorema di Cram\'er--Rao})
mostra che esiste un limite inferiore per la varianza di una
stima.  Osserviamo per prima cosa che la densit\`a di
probabilit\`a per la $N$-pla $(x_1, x_2,\ldots, x_N)$
risulta
\begin{equation*}
  \prod_{i=1}^N f_i (x_i;\theta^*)
\end{equation*}
per il teorema della probabilit\`a composta; se in luogo del
valore vero $\theta^*$ si pone il parametro variabile
$\theta$, si ottiene la \emph{funzione di verosimiglianza}
\begin{equation*}
  \mathcal{L} (x_1, x_2,\ldots, x_N;\theta) =
    \prod_{i=1}^N f_i (x_i;\theta) \peq .
\end{equation*}

La condizione di normalizzazione di ciascuna $f_i$ comporta
che l'integrale della verosimiglianza su tutti i domini
delle variabili $x_i$ valga 1:
\begin{equation*}
  \begin{split}
    \int_{-\infty}^{+\infty} \! \de x_1
      &\int_{-\infty}^{+\infty} \! \de x_2 \cdots
      \int_{-\infty}^{+\infty} \! \de x_N \:
      \mathcal{L} (x_1, x_2,\ldots, x_N;\theta)
      \; = \\[1ex]
    &= \; \int_{-\infty}^{+\infty} \! \de x_1 \,
      f_1(x_1;\theta)
      \int_{-\infty}^{+\infty} \! \de x_2 \,
      f_2(x_2;\theta) \cdots
      \int_{-\infty}^{+\infty} \! \de x_N \,
      f_N(x_N;\theta) \\[1ex]
    &= \; \prod_{i=1}^N \int_{-\infty}^{+\infty} \!
      \de x_i \, f_i(x_i;\theta) \\[1ex]
      &\equiv \; 1
  \end{split}
\end{equation*}
\emph{indipendentemente} dal valore di $\theta$.  Derivando
sotto il segno di integrale rispetto a $\theta$, dato che i
domini delle $f_i(x_i;\theta)$ non dipendono da detta
variabile si ottiene
\begin{equation*}
  \int_{-\infty}^{+\infty} \! \de x_1
    \int_{-\infty}^{+\infty} \! \de x_2 \cdots
    \int_{-\infty}^{+\infty} \! \de x_N \:
    \frac{\partial \mathcal{L}}{\partial \theta}
    = 0
\end{equation*}
da cui, dividendo e moltiplicando l'integrando per
$\mathcal{L}$, risulta
\begin{equation*}
  \begin{split}
    \int_{-\infty}^{+\infty} \! \de x_1
      \int_{-\infty}^{+\infty} \! \de x_2 &\cdots
      \int_{-\infty}^{+\infty} \! \de x_N \:
      \mathcal{L} \left( \frac{1}{\mathcal{L}} \,
      \frac{\partial \mathcal{L}}{\partial \theta}
      \right) \; = \\[1ex]
    &= \; \int_{-\infty}^{+\infty} \! \de x_1
      \int_{-\infty}^{+\infty} \! \de x_2 \cdots
      \int_{-\infty}^{+\infty} \! \de x_N \:
      \mathcal{L} \, \frac{\partial \left(
      \ln \mathcal{L} \right)}{\partial \theta}
      \\[1ex]
    &= \; \int_{-\infty}^{+\infty} \! \de x_1 \,
      f_1(x_1;\theta) \cdots
      \int_{-\infty}^{+\infty} \! \de x_N \:
      f_N(x_N;\theta) \, \frac{\partial \left(
      \ln \mathcal{L} \right)}{\partial \theta}
      \\[1ex]
    &= \; 0
  \end{split}
\end{equation*}
ossia
\begin{equation} \label{eq:e.crarao1}
  \boxed{ \rule[-6mm]{0mm}{14mm} \quad
    E \left\{ \dfrac{\partial \left(
    \ln \mathcal{L} \right)}{\partial \theta}
    \right\} = 0 \quad }
\end{equation}

Se $t$ \`e imparziale
\begin{gather*}
  E(t) \; = \; \int_{-\infty}^{+\infty} \!
    \de x_1 \cdots \int_{-\infty}^{+\infty} \!
    \de x_N \: t(x_1, x_2,\ldots, x_N) \,
    \mathcal{L}(x_1, x_2,\ldots, x_N ;\theta)
    \; = \; \tau(\theta) \\
  \intertext{da cui, derivando ambo i membri
    rispetto a $\theta$,}
  \int_{-\infty}^{+\infty} \! \de x_1
    \int_{-\infty}^{+\infty} \! \de x_2 \cdots
    \int_{-\infty}^{+\infty} \! \de x_N \:
    t \: \frac{\partial \mathcal{L}}{\partial
    \theta} = \tau ' (\theta) \peq .
\end{gather*}

Dividendo e moltiplicando poi l'integrando per la
verosimiglianza $\mathcal{L}$, risulta
\begin{equation*}
  \begin{split}
    \int_{-\infty}^{+\infty} \! \de x_1
      \int_{-\infty}^{+\infty} \! \de x_2 &\cdots
      \int_{-\infty}^{+\infty} \! \de x_N \:
      t \: \frac{\partial \mathcal{L}}{\partial
      \theta} \; = \\
    &= \; \int_{-\infty}^{+\infty} \! \de x_1
      \cdots \int_{-\infty}^{+\infty} \! \de x_N
      \: t \: \mathcal{L} \left( \frac{1}
      {\mathcal{L}} \, \frac{\partial
      \mathcal{L}}{\partial \theta} \right) \\
    &= \; \int_{-\infty}^{+\infty} \! \de x_1 \,
      f_1(x_1;\theta) \cdots
      \int_{-\infty}^{+\infty} \! \de x_N \:
      f_N(x_N;\theta) \: t \: \frac{\partial
      \left( \ln \mathcal{L} \right)} {\partial
      \theta} \\
    &= \; E \left\{ t \: \frac{\partial \left(
      \ln \mathcal{L} \right)}{\partial \theta}
      \right\}
  \end{split}
\end{equation*}
e, in definitiva,

\begin{equation*}
  \boxed{ \rule[-6mm]{0mm}{14mm} \quad
    E \left\{ t \: \frac{\partial
    \left( \ln \mathcal{L} \right)}{\partial
    \theta} \right\} = \tau ' (\theta) \quad }
\end{equation*}

Infine, sottraendo membro a membro da questa equazione la
precedente \eqref{eq:e.crarao1} moltiplicata per
$\tau(\theta)$, si ottiene
\begin{gather*}
  E \left\{ t \: \frac{\partial \left( \ln \mathcal{L}
    \right)}{\partial \theta} \right\} - \tau(\theta)
    \cdot E \left\{ \frac{\partial \left( \ln \mathcal{L}
    \right)}{\partial \theta} \right\} = \tau' (\theta)
    \\
  \intertext{ovvero}
  E \left \{ \bigl[ t - \tau(\theta) \bigr] \cdot
    \frac{\partial \left( \ln \mathcal{L} \right)}
    {\partial \theta} \right \}
    = \tau' (\theta) \peq .
\end{gather*}

Se ora si definiscono il rapporto
\begin{gather*}
  R(\theta) \; = \;
    \frac{ E \left \{ \bigl[ t -
    \tau(\theta) \bigr] \cdot
    \dfrac{\partial \left( \ln \mathcal{L} \right)}
    {\partial \theta} \right \} }
    {E \left\{ \left[ \dfrac{\partial \left( \ln
    \mathcal{L} \right)} {\partial \theta} \right]
    ^2 \right\} } \; = \; \frac{ \tau ' (\theta)}
    {E \left\{ \left[ \dfrac{\partial \left(
    \ln \mathcal{L} \right)} {\partial \theta}
    \right] ^2 \right\} } \\
  \intertext{(che \`e una costante dipendente
    da $\theta$; osserviamo anche che deve
    risultare $R(\theta) \neq 0$) e la
    variabile casuale}
  z = \bigl[ t - \tau(\theta) \bigr]
    - R(\theta) \,
    \frac{\partial \left( \ln \mathcal{L} \right)}
    {\partial \theta} \\
  \intertext{il cui quadrato risulta essere}
  z^2 = \bigl[ t - \tau(\theta) \bigr] ^2
    - 2 \, R(\theta) \cdot \bigl[ t - \tau(\theta)
    \bigr] \, \frac{\partial \left( \ln
    \mathcal{L} \right)}{\partial \theta} + R^2
    (\theta) \cdot \left[ \frac{\partial \left(
    \ln \mathcal{L} \right)}{\partial \theta}
    \right] ^2
\end{gather*}
prendendo il valore medio di $z^2$ si ottiene
\begin{multline*}
    E(z^2) \; = \; E \left \{ \bigl[ t -
      \tau(\theta) \bigr] ^2 \right \} -
      2 \, R(\theta) \cdot E \left \{ \bigl[
      t - \tau(\theta) \bigr] \cdot
      \frac{\partial \left( \ln \mathcal{L}
      \right)}{\partial \theta} \right \} +
      \\[1ex]
   + \; R^2 (\theta) \cdot E \left\{ \left[
    \frac{\partial \left( \ln \mathcal{L}
    \right)}{\partial \theta} \right] ^2
    \right\}
\end{multline*}
ossia
\begin{multline*}
  E(z^2) \; = \; {\sigma_t}^2 -
    2 \, \frac{\tau ' (\theta)}
    {E \left\{ \left[ \dfrac{\partial \left(
    \ln \mathcal{L} \right)}{\partial \theta}
    \right] ^2 \right\} }
    \, \tau ' (\theta) + \\[1ex]
  + \; \left \{ \frac{\tau ' (\theta)}
    {E \left\{ \left[ \dfrac{\partial \left(
    \ln \mathcal{L} \right)}{\partial \theta}
    \right] ^2 \right\} } \right\} ^2 E
    \left\{ \left[ \frac{\partial \left( \ln
    \mathcal{L} \right)}{\partial \theta}
    \right] ^2 \right\}
\end{multline*}
ed infine
\begin{align*}
  E(z^2) \; &= \; {\sigma_t}^2 -
    2 \, \frac{ \left[ \tau ' (\theta)
    \right] ^2 }{E \left\{ \left[
    \dfrac{\partial \left( \ln \mathcal{L}
    \right)}{\partial \theta} \right] ^2
    \right\} } +
    \frac{ \left[ \tau ' (\theta)
    \right] ^2 }{E \left\{ \left[
    \dfrac{\partial \left( \ln \mathcal{L}
    \right)}{\partial \theta} \right] ^2
    \right\} } \\[1ex]
  &= \; {\sigma_t}^2 -
    \frac{ \left[ \tau ' (\theta) \right] ^2 }
    {E \left\{ \left[ \dfrac{\partial \left(
    \ln \mathcal{L} \right)}{\partial \theta}
    \right] ^2 \right\} } \peq .
\end{align*}

Ma il valore medio del quadrato di una qualsiasi variabile
casuale non pu\`o essere negativo, e dunque
\begin{gather*}
  0 \; \le \;
    E(z^2) \; = \;
    {\sigma_t}^2 -
    \frac{ \left[ \tau ' (\theta) \right] ^2 }
    {E \left\{ \left[ \dfrac{\partial \left( \ln
    \mathcal{L} \right)}{\partial \theta} \right]
    ^2 \right\} } \\
  \intertext{ed infine}
  {\sigma_t}^2 \; \geq \;
    \frac{ \left[ \tau ' (\theta) \right] ^2 }
    {E \left\{ \left[ \dfrac{\partial \left( \ln
    \mathcal{L} \right)}{\partial \theta} \right]
    ^2 \right\} }
    \; = \; \left[ \tau ' (\theta) \right] ^2 \,
    \frac{R(\theta)}{\tau ' (\theta)}
    \; = \; \tau ' (\theta) \cdot R(\theta)
\end{gather*}
cio\`e:
\begin{quote}
  \textit{Nessuna funzione dei valori osservati
    $t(x_1,x_2,\ldots,x_N)$, che sia stima imparziale di una
    funzione del parametro $\tau(\theta)$, pu\`o avere
    varianza inferiore ad un limite determinato.}
\end{quote}

La varianza minima si raggiunge se e soltanto se $E(z^2)$
\`e nullo, il che \`e possibile solo se $z$ \`e nulla
ovunque, cio\`e se
\begin{equation*}
  z \; = \;
    t - \tau(\theta) - R(\theta) \,
    \frac{\partial \left( \ln \mathcal{L} \right)}
    {\partial \theta} \; \equiv \; 0
\end{equation*}
o, altrimenti detto, se la derivata logaritmica della
verosimiglianza \`e proporzionale alla variabile casuale $t
- \tau(\theta)$:

\begin{equation} \label{eq:e.condmin}
  \boxed{\rule[-6mm]{0mm}{14mm} \quad
    \frac{\partial \left( \ln \mathcal{L} \right)}
    {\partial \theta} \; = \;
    \frac{t - \tau(\theta)}
    {R(\theta)} \quad }
\end{equation}

Nel caso particolare che tutte le $x_i$ provengano dalla
stessa popolazione, e che quindi abbiano la stessa densit\`a
di probabilit\`a $f(x;\theta)$,
\begin{gather*}
  \frac{\partial (\ln  \mathcal{L})}{\partial \theta} \; =
    \; \frac{\partial}{\partial \theta} \sum_{i=1}^N \ln
    f(x_i; \theta) \; = \; \sum_{i=1}^N
    \frac{\partial}{\partial \theta} \ln f(x_i; \theta)
    \\[1ex]
  E \left\{ \frac{\partial (\ln \mathcal{L})}{\partial
    \theta} \right\} \; = \; \sum_{i=1}^N E \left\{
    \frac{\partial}{\partial \theta} \ln f(x_i; \theta)
    \right\} \; = \; N \cdot E \left\{
    \frac{\partial}{\partial \theta} \ln f(x; \theta)
    \right\}
\end{gather*}
e, tenuto conto della \eqref{eq:e.crarao1}, questo implica
che
\begin{equation} \label{eq:e.crarao2}
  E \left\{ \frac{\partial}{\partial \theta} \ln f(x;
    \theta) \right\} \; = \; 0 \peq .
\end{equation}
Ora
\begin{equation*}
  \begin{split}
    E &\left\{ \left[ \frac{\partial (\ln
      \mathcal{L})}{\partial \theta} \right]^2 \right\} = E
      \left\{ \left[ \sum_{i=1}^N \frac{\partial}{\partial
      \theta} \ln f(x_i; \theta) \right] \left[ \sum_{k=1}^N
      \frac{\partial}{\partial \theta} \ln f(x_k; \theta)
      \right] \right\} \\[1ex]
    &= \sum_{i=1}^N E \left\{ \left[
      \frac{\partial}{\partial \theta} \ln f(x_i; \theta)
      \right]^2 \right\} + \sum_{\substack{i,k\\ i \ne k}} E
      \left\{ \frac{\partial}{\partial \theta} \ln f(x_i;
      \theta) \cdot \frac{\partial}{\partial \theta} \ln
      f(x_k; \theta) \right\} \\[1ex]
    &= N \cdot E \left\{ \left[ \frac{\partial}{\partial
      \theta} \ln f(x; \theta) \right]^2 \right\} +
      \sum_{\substack{i,k\\ i \ne k}} E \left\{
      \frac{\partial}{\partial \theta} \ln f(x_i; \theta)
      \right\} \cdot E \left\{ \frac{\partial}{\partial
      \theta} \ln f(x_k; \theta) \right\}
  \end{split}
\end{equation*}
(tenendo conto del fatto che le $x_i$ sono indipendenti);
l'ultimo termine si annulla in conseguenza della
\eqref{eq:e.crarao2}, ed infine, in questo caso, il
minorante della varianza della stima si pu\`o scrivere
\begin{equation*}
  \boxed{ \rule[-12mm]{0mm}{20mm} \quad
    {\sigma_t}^2 \; \ge \; \frac{[ \tau'(\theta) ]^2}{N
    \cdot E \left\{ \left[ \dfrac{\partial}{\partial \theta}
    \ln f(x; \theta) \right]^2 \right\} }
  \quad}
\end{equation*}

\index{massima verosimiglianza, metodo della|(}%
Col metodo della massima verosimiglianza si assume, come
stima del valore vero $\theta^*$ del parametro $\theta$,
quel valore $\widehat \theta$ che rende massima la
verosimiglianza $\mathcal{L}$ per i valori osservati delle
variabili, $x_1, x_2,\ldots, x_N$.

Ora, \emph{nel caso esista una stima di minima varianza} $t$
per la funzione $\tau(\theta)$, tenendo conto della
\eqref{eq:e.condmin} la condizione perch\'e la funzione di
verosimiglianza abbia un estremante diviene
\begin{gather*}
  \frac{\partial \left( \ln \mathcal{L} \right)}
    {\partial \theta} \; = \;
    \frac{t - \tau(\theta)}{R(\theta)}
    \; = \; 0 \\
    \intertext{e le soluzioni $\widehat \theta$ sono
      tutte e sole quelle dell'equazione}
  \tau(\theta) = t(x_1,x_2,\ldots,x_N) \peq .
\end{gather*}

La derivata seconda di $\ln \mathcal{L}$ \`e in tal caso
\begin{align*}
  \frac{\partial^2 \left( \ln \mathcal{L} \right)}
    {\partial \theta^2} &=
    - \, \frac{\tau ' (\theta) \cdot
    R(\theta) + R'(\theta) \cdot
    \left[ t - \tau(\theta) \right] }
    { R^2(\theta) } \\[1ex]
  &= - \, \frac{ {\sigma_t}^2 +
    R'(\theta) \cdot \left[
    t - \tau(\theta) \right] }
    { R^2(\theta) }
\end{align*}
ma se $\theta = \widehat \theta$ \`e anche $t - \tau \bigl(
\widehat \theta \, \bigr) = 0$ e risulta
\begin{equation*}
  \left[ \frac{\partial^2 \left( \ln \mathcal{L}
     \right)}{\partial \theta^2}
     \right]_{\theta = \widehat \theta} \; = \;
     - \, \frac{ {\sigma_t}^2 }
     {R^2 \bigl ( \widehat \theta \, \bigr )}
     \; < \; 0 \peq ;
\end{equation*}
cio\`e \emph{per tutte le soluzioni $\theta = \widehat
  \theta$ la verosimiglianza \`e massima}.

Ora, se la funzione $\ln \mathcal{L}$ \`e regolare, tra due
massimi deve esistere un minimo; dato che non esistono
minimi, ne consegue che \emph{il massimo \`e unico} ed in
corrispondenza al valore della funzione $\tau^{-1}$ inversa
di $\tau(\theta)$ e calcolata in $t(x_1,x_2,\ldots,x_N)$:
\begin{equation*}
  \widehat \theta \; = \; \tau^{-1} \bigl [
    t(x_1,x_2,\ldots,x_N) \bigr ] \peq .
\end{equation*}%
\index{massima verosimiglianza, metodo della|)}%
\index{Cram\'er--Rao, teorema di|)}

La \emph{statistica} $t(x_1,x_2,\ldots,x_N)$ (come viene
anche indicata una funzione dei dati) di minima varianza \`e
un caso particolare di statistica \emph{sufficiente} per il
parametro $\theta$, come \`e chiamata una funzione dei
valori osservati, se esiste, che riassume in s\'e tutta
l'informazione che i dati possono fornire sul valore del
parametro.

\index{media!aritmetica!come stima del valore vero|(}%
Se $x_1,x_2,\ldots,x_N$ sono i valori osservati di $N$
variabili casuali \emph{normali} con lo stesso valore medio
$\lambda$ e varianze rispettive $\sigma_i$ supposte note, la
verosimiglianza \`e
\begin{gather*}
  \mathcal{L} \; = \; \prod_{i=1}^N \frac{1}
    {\sigma_i \, \sqrt{2 \pi}} \,
    e^{\textstyle -\frac{1}{2} \bigl(
    \frac{x_i - \lambda}{\sigma_i}
     \bigr) ^2 } \\
  \intertext{il suo logaritmo}
  \ln \mathcal{L} \; = \; - \,
    \frac{1}{2} \sum_{i=1}^N
    \frac{\left( x_i - \lambda \right) ^2}
    { {\sigma_i}^2 } \, - \,
    \sum_{i=1}^N \ln \left(
    \sigma_i \, \sqrt{2 \pi} \right) \\
  \intertext{e la sua derivata rispetto al
    parametro $\lambda$}
  \frac{\partial \left( \ln \mathcal{L} \right)}
    {\partial \lambda} \; = \;
    \sum_{i=1}^N \frac{x_i - \lambda}
    { {\sigma_i}^2 } \; = \;
    \left( \sum \nolimits_i \frac{1}
    { {\sigma_i}^2 } \right)
    \left( \frac{\displaystyle \sum\nolimits_i
    \frac{x_i} { {\sigma_i}^2 } }
    { \displaystyle \sum\nolimits_i \frac{1}{
    {\sigma_i}^2 } } \, - \lambda \right) \peq .
\end{gather*}

\emph{Pertanto la media dei dati, pesati con coefficienti
  inversamente proporzionali alle varianze, \`e una stima di
  minima varianza per $\lambda$}.  Se le $N$ varianze sono
poi tutte uguali tra loro e di valore $\sigma^2$, risulta
\begin{equation*}
  \frac{\partial \left( \ln \mathcal{L} \right)}
    {\partial \lambda} \; = \;
    \frac{1}{\sigma^2} \left[ \left( \,
    \sum_{i=1}^N x_i \right) -
    N \lambda \right] \; = \;
    \frac{N}{\sigma^2} \left( \bar x - \lambda \right) \; = \;
    \frac{\bar x - \lambda}{R}
\end{equation*}
ed in tal caso la \emph{media aritmetica} del campione \`e
una stima di minima varianza per $\lambda$.  Sempre in tal
caso \`e poi
\begin{equation*}
  R(\lambda) \; \equiv \; R \; = \;
    \frac{\sigma^2}{N}
\end{equation*}
con
\begin{equation*}
  \tau(\lambda) \; \equiv \; \lambda
    \makebox[50mm]{e}
   \tau'(\lambda) \; = \; 1
\end{equation*}
dunque
\begin{equation*}
  \var (\bar x) \; = \; \tau ' \, R \; = \;
    \frac{\sigma^2}{N}
\end{equation*}
come d'altra parte gi\`a si sapeva.

Qui la media del campione \`e un esempio di statistica
sufficiente per $\lambda$; infatti non ha alcuna importanza
quali siano i singoli valori $x_i$: ma se le medie di due
diversi campioni sono uguali, le conclusioni che si possono
trarre sul valore di $\lambda$ sono le medesime.%
\index{media!aritmetica!come stima del valore vero|)}

\index{varianza!della popolazione|(}%
Supponendo di conoscere il valore medio $\lambda$, la stima
della varianza $\sigma^2$ si ottiene cercando lo zero della
derivata logaritmica
\begin{gather*}
  \frac{\partial \left( \ln \mathcal{L} \right)}
    {\partial \sigma} \; = \;
    \frac{1}{\sigma^3} \left[ \,
    \sum_{i=1}^N (x_i - \lambda)^2
    \right] - \frac{N}{\sigma} \; = \;
    \frac{N}{\sigma^3} \, \left \{
    \left[ \frac{1}{N}
    \sum_{i=1}^N (x_i - \lambda)^2 \right] -
    \sigma^2 \right \} \\
  \intertext{la quale ha la forma richiesta
    perch\'e la soluzione}
  {\widehat \sigma}^2 = \frac{1}{N}
    \sum_{i=1}^N (x_i - \lambda)^2 \\
  \intertext{sia una stima di $\sigma^2$
    con minima varianza, data da}
  \var \left \{ \frac{1}{N} \sum_{i=1}^N
    (x_i - \lambda)^2
    \right \} \; = \; \tau' R \; = \;
    2 \, \sigma \, \frac{\sigma^3}{N}
    \; = \; \frac{2 \sigma^4}{N}
\end{gather*}
essendo $R(\sigma)=\sigma^3/N$, $\tau(\sigma)=\sigma^2$ e
$\tau'(\sigma)=2\sigma$: questo risultato \`e lo stesso
trovato nell'appendice \ref{ch:b.errvar}.

Il valore di $\lambda$ tuttavia non \`e generalmente noto, e
l'uso della media aritmetica del campione $\bar x$ comporta
una distorsione che si corregge, come si \`e visto, ponendo
$N-1$ in luogo di $N$.%
\index{varianza!della popolazione|)}

\endinput

% $Id: chapterf.tex,v 1.1 2005/04/13 08:28:24 loreti Exp $

\chapter{La licenza GNU GPL (General Public License)}
\label{ch:licgpl}
Questo capitolo contiene la licenza GNU GPL, sotto la quale
questo libro viene distribuito, sia nella versione originale
inglese\footnote{\texttt{http://www.gnu.org/licenses/gpl.html}
} (la sola dotata di valore legale) che in una traduzione
non ufficiale in italiano\footnote{La traduzione \`e dovuta
  al gruppo \emph{Pluto} (PLUTO Linux/Lumen Utentibus
  Terrarum Orbis, \texttt{http://www.pluto.linux.it/}). }
che aiuti chi ha difficolt\`a con l'inglese legale a
comprendere meglio il significato della licenza stessa.

\section{The GNU General Public License}
\selectlanguage{english}
\begin{center}
  \setlength{\parindent}{0in}
  Version 2, June 1991

  Copyright \copyright\ 1989, 1991 Free Software Foundation,
  Inc.

  \bigskip

  59 Temple Place - Suite 330, Boston, MA  02111-1307, USA

  \bigskip

  Everyone is permitted to copy and distribute verbatim
  copies of this \\
  license document, but changing it is not allowed.
\end{center}

\begin{quote}
  \begin{center} \textbf{Preamble} \end{center}

  The licenses for most software are designed to take away
  your freedom to share and change it.  By contrast, the GNU
  General Public License is intended to guarantee your
  freedom to share and change free software --- to make sure
  the software is free for all its users.  This General
  Public License applies to most of the Free Software
  Foundation's software and to any other program whose
  authors commit to using it.  (Some other Free Software
  Foundation software is covered by the GNU Library General
  Public License instead.)  You can apply it to your
  programs, too.

  When we speak of free software, we are referring to
  freedom, not price.  Our General Public Licenses are
  designed to make sure that you have the freedom to
  distribute copies of free software (and charge for this
  service if you wish), that you receive source code or can
  get it if you want it, that you can change the software or
  use pieces of it in new free programs; and that you know
  you can do these things.

  To protect your rights, we need to make restrictions that
  forbid anyone to deny you these rights or to ask you to
  surrender the rights.  These restrictions translate to
  certain responsibilities for you if you distribute copies
  of the software, or if you modify it.

  For example, if you distribute copies of such a program,
  whether gratis or for a fee, you must give the recipients
  all the rights that you have.  You must make sure that
  they, too, receive or can get the source code.  And you
  must show them these terms so they know their rights.

  We protect your rights with two steps: (1) copyright the
  software, and (2) offer you this license which gives you
  legal permission to copy, distribute and/or modify the
  software.

  Also, for each author's protection and ours, we want to
  make certain that everyone understands that there is no
  warranty for this free software.  If the software is
  modified by someone else and passed on, we want its
  recipients to know that what they have is not the
  original, so that any problems introduced by others will
  not reflect on the original authors' reputations.

  Finally, any free program is threatened constantly by
  software patents.  We wish to avoid the danger that
  redistributors of a free program will individually obtain
  patent licenses, in effect making the program proprietary.
  To prevent this, we have made it clear that any patent
  must be licensed for everyone's free use or not licensed
  at all.

  The precise terms and conditions for copying, distribution
  and modification follow.
\end{quote}

\begin{center}
  \Large \scshape GNU General Public License \\
  \vspace{3mm} Terms and Conditions For Copying,
  Distribution and Modification
\end{center}

\begin{enumerate}  \addtocounter{enumi}{-1}

\item This License applies to any program or other work
  which contains a notice placed by the copyright holder
  saying it may be distributed under the terms of this
  General Public License.  The ``Program'', below, refers to
  any such program or work, and a ``work based on the
  Program'' means either the Program or any derivative work
  under copyright law: that is to say, a work containing the
  Program or a portion of it, either verbatim or with
  modifications and/or translated into another language.
  (Hereinafter, translation is included without limitation
  in the term ``modification''.)  Each licensee is addressed
  as ``you''.

  Activities other than copying, distribution and
  modification are not covered by this License; they are
  outside its scope.  The act of running the Program is not
  restricted, and the output from the Program is covered
  only if its contents constitute a work based on the
  Program (independent of having been made by running the
  Program).  Whether that is true depends on what the
  Program does.

\item You may copy and distribute verbatim copies of the
  Program's source code as you receive it, in any medium,
  provided that you conspicuously and appropriately publish
  on each copy an appropriate copyright notice and
  disclaimer of warranty; keep intact all the notices that
  refer to this License and to the absence of any warranty;
  and give any other recipients of the Program a copy of
  this License along with the Program.

  You may charge a fee for the physical act of transferring
  a copy, and you may at your option offer warranty
  protection in exchange for a fee.

\item You may modify your copy or copies of the Program or
  any portion of it, thus forming a work based on the
  Program, and copy and distribute such modifications or
  work under the terms of Section 1 above, provided that you
  also meet all of these conditions:

  \begin{enumerate}
  \item You must cause the modified files to carry prominent
    notices stating that you changed the files and the date
    of any change.

  \item You must cause any work that you distribute or
    publish, that in whole or in part contains or is derived
    from the Program or any part thereof, to be licensed as
    a whole at no charge to all third parties under the
    terms of this License.

  \item If the modified program normally reads commands
    interactively when run, you must cause it, when started
    running for such interactive use in the most ordinary
    way, to print or display an announcement including an
    appropriate copyright notice and a notice that there is
    no warranty (or else, saying that you provide a
    warranty) and that users may redistribute the program
    under these conditions, and telling the user how to view
    a copy of this License.  (Exception: if the Program
    itself is interactive but does not normally print such
    an announcement, your work based on the Program is not
    required to print an announcement.)
  \end{enumerate}

  These requirements apply to the modified work as a whole.
  If identifiable sections of that work are not derived from
  the Program, and can be reasonably considered independent
  and separate works in themselves, then this License, and
  its terms, do not apply to those sections when you
  distribute them as separate works.  But when you
  distribute the same sections as part of a whole which is a
  work based on the Program, the distribution of the whole
  must be on the terms of this License, whose permissions
  for other licensees extend to the entire whole, and thus
  to each and every part regardless of who wrote it.

  Thus, it is not the intent of this section to claim rights
  or contest your rights to work written entirely by you;
  rather, the intent is to exercise the right to control the
  distribution of derivative or collective works based on
  the Program.

  In addition, mere aggregation of another work not based on
  the Program with the Program (or with a work based on the
  Program) on a volume of a storage or distribution medium
  does not bring the other work under the scope of this
  License.

\item You may copy and distribute the Program (or a work
  based on it, under Section 2) in object code or executable
  form under the terms of Sections 1 and 2 above provided
  that you also do one of the following:

  \begin{enumerate}
  \item Accompany it with the complete corresponding
    machine-readable source code, which must be distributed
    under the terms of Sections 1 and 2 above on a medium
    customarily used for software interchange; or,

  \item Accompany it with a written offer, valid for at
    least three years, to give any third party, for a charge
    no more than your cost of physically performing source
    distribution, a complete machine-readable copy of the
    corresponding source code, to be distributed under the
    terms of Sections 1 and 2 above on a medium customarily
    used for software interchange; or,

  \item Accompany it with the information you received as to
    the offer to distribute corresponding source code.
    (This alternative is allowed only for noncommercial
    distribution and only if you received the program in
    object code or executable form with such an offer, in
    accord with Subsection b above.)
  \end{enumerate}

  The source code for a work means the preferred form of the
  work for making modifications to it.  For an executable
  work, complete source code means all the source code for
  all modules it contains, plus any associated interface
  definition files, plus the scripts used to control
  compilation and installation of the executable.  However,
  as a special exception, the source code distributed need
  not include anything that is normally distributed (in
  either source or binary form) with the major components
  (compiler, kernel, and so on) of the operating system on
  which the executable runs, unless that component itself
  accompanies the executable.

  If distribution of executable or object code is made by
  offering access to copy from a designated place, then
  offering equivalent access to copy the source code from
  the same place counts as distribution of the source code,
  even though third parties are not compelled to copy the
  source along with the object code.

\item You may not copy, modify, sublicense, or distribute
  the Program except as expressly provided under this
  License.  Any attempt otherwise to copy, modify,
  sublicense or distribute the Program is void, and will
  automatically terminate your rights under this License.
  However, parties who have received copies, or rights, from
  you under this License will not have their licenses
  terminated so long as such parties remain in full
  compliance.

\item You are not required to accept this License, since you
  have not signed it.  However, nothing else grants you
  permission to modify or distribute the Program or its
  derivative works.  These actions are prohibited by law if
  you do not accept this License.  Therefore, by modifying
  or distributing the Program (or any work based on the
  Program), you indicate your acceptance of this License to
  do so, and all its terms and conditions for copying,
  distributing or modifying the Program or works based on
  it.

\item Each time you redistribute the Program (or any work
  based on the Program), the recipient automatically
  receives a license from the original licensor to copy,
  distribute or modify the Program subject to these terms
  and conditions.  You may not impose any further
  restrictions on the recipients' exercise of the rights
  granted herein.  You are not responsible for enforcing
  compliance by third parties to this License.

\item If, as a consequence of a court judgment or allegation
  of patent infringement or for any other reason (not
  limited to patent issues), conditions are imposed on you
  (whether by court order, agreement or otherwise) that
  contradict the conditions of this License, they do not
  excuse you from the conditions of this License.  If you
  cannot distribute so as to satisfy simultaneously your
  obligations under this License and any other pertinent
  obligations, then as a consequence you may not distribute
  the Program at all.  For example, if a patent license
  would not permit royalty-free redistribution of the
  Program by all those who receive copies directly or
  indirectly through you, then the only way you could
  satisfy both it and this License would be to refrain
  entirely from distribution of the Program.

  If any portion of this section is held invalid or
  unenforceable under any particular circumstance, the
  balance of the section is intended to apply and the
  section as a whole is intended to apply in other
  circumstances.

  It is not the purpose of this section to induce you to
  infringe any patents or other property right claims or to
  contest validity of any such claims; this section has the
  sole purpose of protecting the integrity of the free
  software distribution system, which is implemented by
  public license practices.  Many people have made generous
  contributions to the wide range of software distributed
  through that system in reliance on consistent application
  of that system; it is up to the author/donor to decide if
  he or she is willing to distribute software through any
  other system and a licensee cannot impose that choice.

  This section is intended to make thoroughly clear what is
  believed to be a consequence of the rest of this License.

\item If the distribution and/or use of the Program is
  restricted in certain countries either by patents or by
  copyrighted interfaces, the original copyright holder who
  places the Program under this License may add an explicit
  geographical distribution limitation excluding those
  countries, so that distribution is permitted only in or
  among countries not thus excluded.  In such case, this
  License incorporates the limitation as if written in the
  body of this License.

\item The Free Software Foundation may publish revised
  and/or new versions of the General Public License from
  time to time.  Such new versions will be similar in spirit
  to the present version, but may differ in detail to
  address new problems or concerns.

  Each version is given a distinguishing version number.  If
  the Program specifies a version number of this License
  which applies to it and ``any later version'', you have
  the option of following the terms and conditions either of
  that version or of any later version published by the Free
  Software Foundation.  If the Program does not specify a
  version number of this License, you may choose any version
  ever published by the Free Software Foundation.

\item If you wish to incorporate parts of the Program into
  other free programs whose distribution conditions are
  different, write to the author to ask for permission.  For
  software which is copyrighted by the Free Software
  Foundation, write to the Free Software Foundation; we
  sometimes make exceptions for this.  Our decision will be
  guided by the two goals of preserving the free status of
  all derivatives of our free software and of promoting the
  sharing and reuse of software generally.

  \begin{center}
    \Large \textsc{No Warranty}
  \end{center}

\item \textsc{Because the program is licensed free of
    charge, there is no warranty for the program, to the
    extent permitted by applicable law.  Except when
    otherwise stated in writing the copyright holders and/or
    other parties provide the program ``as is'' without
    warranty of any kind, either expressed or implied,
    including, but not limited to, the implied warranties of
    merchantability and fitness for a particular purpose.
    The entire risk as to the quality and performance of the
    program is with you.  Should the program prove
    defective, you assume the cost of all necessary
    servicing, repair or correction.}

\item \textsc{In no event unless required by applicable law
    or agreed to in writing will any copyright holder, or
    any other party who may modify and/or redistribute the
    program as permitted abo\-ve, be liable to you for
    damages, including any general, special, incidental or
    consequential damages arising out of the use or
    inability to use the program (including but not limited
    to loss of data or data being rendered inaccurate or
    losses sustained by you or third parties or a failure of
    the program to operate with any other programs), even if
    such holder or other party has been advised of the
    possibility of such damages.}
\end{enumerate}

\begin{center}
  \Large \textsc{End of Terms and Conditions}
\end{center}

\subsection*{Appendix: How to Apply These Terms to Your New
  Programs}
If you develop a new program, and you want it to be of the
greatest possible use to the public, the best way to achieve
this is to make it free software which everyone can
redistribute and change under these terms.

To do so, attach the following notices to the program.  It
is safest to attach them to the start of each source file to
most effectively convey the exclusion of warranty; and each
file should have at least the ``copyright'' line and a
pointer to where the full notice is found.

\begin{center}
  <one line to give the program's name and a brief idea of
  what it does.> \\
  Copyright (C) <year> <name of author>
\end{center}
\begin{quote}
  This program is free software; you can redistribute it
  and/or modify it under the terms of the GNU General Public
  License as published by the Free Software Foundation;
  either version 2 of the License, or (at your option) any
  later version.

  This program is distributed in the hope that it will be
  useful, but WITHOUT ANY WARRANTY; without even the implied
  warranty of MERCHANTABILITY or FITNESS FOR A PARTICULAR
  PURPOSE.  See the GNU General Public License for more
  details.

  You should have received a copy of the GNU General Public
  License along with this program; if not, write to the Free
  Software Foundation, Inc., 59 Temple Place - Suite 330,
  Boston, MA 02111-1307, USA.
\end{quote}

Also add information on how to contact you by electronic and
paper mail.  If the program is interactive, make it output a
short notice like this when it starts in an interactive
mode:

\begin{center}
  Gnomovision version 69, Copyright (C) <year> <name of
  author> \\
  Gnomovision comes with ABSOLUTELY NO WARRANTY; for details
  type `show w'.
\end{center}
\begin{quote}
  This is free software, and you are welcome to redistribute
  it under certain conditions; type `show c' for details.
\end{quote}

The hypothetical commands \texttt{show w} and \texttt{show
  c} should show the appropriate parts of the General Public
License.  Of course, the commands you use may be called
something other than \texttt{show w} and \texttt{show c};
they could even be mouse-clicks or menu items --- whatever
suits your program.

You should also get your employer (if you work as a
programmer) or your school, if any, to sign a ``copyright
disclaimer'' for the program, if necessary.  Here is a
sample; alter the names:

\begin{center}
  Yoyodyne, Inc., hereby disclaims all copyright interest in
  the program \\
  `Gnomovision' (which makes passes at compilers) written by
  James Hacker. \\[1ex]
  <signature of Ty Coon>, 1 April 1989 \\
  Ty Coon, President of Vice
\end{center}

This General Public License does not permit incorporating
your program into proprietary programs.  If your program is
a subroutine library, you may consider it more useful to
permit linking proprietary applications with the library.
If this is what you want to do, use the GNU Library General
Public License instead of this License.

\selectlanguage{italian}

\section{Licenza pubblica generica del progetto GNU}
\begin{center}
  \setlength{\parindent}{0in}
  Versione 2, Giugno 1991

  Copyright \copyright\ 1989, 1991 Free Software Foundation,
  Inc.

  \bigskip

  59 Temple Place - Suite 330, Boston, MA  02111-1307, USA

  \bigskip

  Tutti possono copiare e distribuire copie letterali di
  questo documento di licenza, ma non \`e permesso
  modificarlo.
\end{center}

\begin{quote}
  \begin{center} \textbf{Preambolo} \end{center}

  Le licenze per la maggioranza dei programmi hanno lo scopo
  di togliere all'utente la libert\`a di condividerlo e di
  modificarlo. Al contrario, la Licenza Pubblica Generica
  GNU \`e intesa a garantire la libert\`a di condividere e
  modificare il free software, al fine di assicurare che i
  programmi siano ``liberi'' per tutti i loro utenti. Questa
  Licenza si applica alla maggioranza dei programmi della
  Free Software Foundation e a ogni altro programma i cui
  autori hanno scelto questa Licenza. Alcuni altri programmi
  della Free Software Foundation sono invece coperti dalla
  Licenza Pubblica Generica per Librerie (LGPL). Chiunque
  pu\`o usare questa Licenza per i propri programmi.

  Quando si parla di free software, ci si riferisce alla
  libert\`a, non al prezzo. Le nostre Licenze (la GPL e la
  LGPL) sono progettate per assicurare che ciascuno abbia la
  libert\`a di distribuire copie del software libero (e
  farsi pagare per questo, se vuole), che ciascuno riceva il
  codice sorgente o che lo possa ottenere se lo desidera,
  che ciascuno possa modificare il programma o usarne delle
  parti in nuovi programmi liberi e che ciascuno sappia di
  potere fare queste cose.

  Per proteggere i diritti dell'utente, abbiamo bisogno di
  creare delle restrizioni che vietino a chiunque di negare
  questi diritti o di chiedere di rinunciarvi. Queste
  restrizioni si traducono in certe responsabilit\`a per chi
  distribuisce copie del software e per chi lo modifica.

  Per esempio, chi distribuisce copie di un Programma
  coperto da GPL, sia gratuitamente sia facendosi pagare,
  deve dare agli acquirenti tutti i diritti che ha ricevuto.
  Deve anche assicurarsi che gli acquirenti ricevano o
  possano ricevere il codice sorgente. E deve mostrar loro
  queste condizioni di Licenza, in modo che conoscano i loro
  diritti.

  Proteggiamo i diritti dell'utente attraverso due azioni:
  (1) proteggendo il software con un diritto d'autore (una
  nota di copyright), e (2) offrendo una Licenza che concede
  il permesso legale di copiare, distribuire e/o modificare
  il Programma.

  Infine, per proteggere ogni autore e noi stessi, vogliamo
  assicurarci che ognuno capisca che non ci sono garanzie
  per i programmi coperti da GPL. Se il Programma viene
  modificato da qualcun altro e ridistribuito, vogliamo che
  gli acquirenti sappiano che ci\`o che hanno non \`e
  l'originale, in modo che ogni problema introdotto da altri
  non si rifletta sulla reputazione degli autori originari.

  Infine, ogni programma libero \`e costantemente minacciato
  dai brevetti sui programmi. Vogliamo evitare il pericolo
  che chi ridistribuisce un Programma libero ottenga
  brevetti personali, rendendo perci\`o il Programma una
  cosa di sua propriet\`a. Per prevenire questo, abbiamo
  chiarito che ogni prodotto brevettato debba essere reso
  disponibile perch\'e tutti ne usufruiscano liberamente; se
  l'uso del prodotto deve sottostare a restrizioni allora
  tale prodotto non deve essere distribuito affatto.

  Seguono i termini e le condizioni precisi per la copia, la
  distribuzione e la modifica.
\end{quote}

\begin{center}
  \Large \scshape Licenza Pubblica Generica GNU \\
  \vspace{3mm} Termini e Condizioni per la Copia, la
  Distribuzione e la Modifica
\end{center}

\begin{enumerate}  \addtocounter{enumi}{-1}

\item Questa Licenza si applica a ogni Programma o altra
  opera che contenga una nota da parte del detentore del
  diritto d'autore che dica che tale opera pu\`o essere
  distribuita nei termini di questa Licenza Pubblica
  Generica. Il termine ``Programma'' nel seguito indica
  ognuno di questi programmi o lavori, e l'espressione
  ``lavoro basato sul Programma'' indica sia il Programma
  sia ogni opera considerata ``derivata'' in base alla legge
  sul diritto d'autore: cio\`e un lavoro contenente il
  Programma o una porzione di esso, sia letteralmente sia
  modificato e/o tradotto in un'altra lingua; da qui in
  avanti, la traduzione \`e in ogni caso considerata una
  ``modifica''.  Vengono ora elencati i diritti dei
  detentori di licenza.

  Attivit\`a diverse dalla copiatura, distribuzione e
  modifica non sono coperte da questa Licenza e sono al di
  fuori della sua influenza. L'atto di eseguire il programma
  non viene limitato, e l'output del programma \`e coperto
  da questa Licenza solo se il suo contenuto costituisce un
  lavoro basato sul Programma (indipendentemente dal fatto
  che sia stato creato eseguendo il Programma). In base alla
  natura del Programma il suo output pu\`o essere o meno
  coperto da questa Licenza.

\item \`E lecito copiare e distribuire copie letterali del
  codice sorgente del Programma cos\`\i\ come viene
  ricevuto, con qualsiasi mezzo, a condizione che venga
  riprodotta chiaramente su ogni copia un'appropriata nota
  di diritto d'autore e di assenza di garanzia; che si
  mantengano intatti tutti i riferimenti a questa Licenza e
  all'assenza di ogni garanzia; che si dia a ogni altro
  acquirente del Programma una copia di questa Licenza
  insieme al Programma.

  \`E possibile richiedere un pagamento per il trasferimento
  fisico di una copia del Programma, \`e anche possibile a
  propria discrezione richiedere un pagamento in cambio di
  una copertura assicurativa.

\item \`E lecito modificare la propria copia o copie del
  Programma, o parte di esso, creando perci\`o un lavoro
  basato sul Programma, e copiare o distribuire queste
  modifiche e questi lavori secondo i termini del precedente
  comma 1, a patto che vengano soddisfatte queste
  condizioni:

  \begin{enumerate}
  \item Bisogna indicare chiaramente nei file che si tratta
    di copie modificate e la data di ogni modifica.

  \item Bisogna fare in modo che ogni lavoro distribuito o
    pubblicato, che in parte o nella sua totalit\`a derivi
    dal Programma o da parti di esso, sia utilizzabile
    gratuitamente da terzi nella sua totalit\`a, secondo le
    condizioni di questa licenza.

  \item Se di solito il programma modificato legge comandi
    interattivamente quando viene eseguito, bisogna fare in
    modo che all'inizio dell'esecuzione interattiva usuale,
    stampi un messaggio contenente un'appropriata nota di
    diritto d'autore e di assenza di garanzia (oppure che
    specifichi che si offre una garanzia). Il messaggio deve
    inoltre specificare agli utenti che possono
    ridistribuire il programma alle condizioni qui descritte
    e deve indicare come consultare una copia di questa
    licenza. Se per\`o il programma di partenza \`e
    interattivo ma normalmente non stampa tale messaggio,
    non occorre che un lavoro derivato lo stampi.
  \end{enumerate}

  Questi requisiti si applicano al lavoro modificato nel suo
  complesso. Se sussistono parti identificabili del lavoro
  modificato che non siano derivate dal Programma e che
  possono essere ragionevolmente considerate lavori
  indipendenti, allora questa Licenza e i suoi termini non
  si applicano a queste parti quando vengono distribuite
  separatamente. Se per\`o queste parti vengono distribuite
  all'interno di un prodotto che \`e un lavoro basato sul
  Programma, la distribuzione di questo prodotto nel suo
  complesso deve avvenire nei termini di questa Licenza, le
  cui norme nei confronti di altri utenti si estendono a
  tutto il prodotto, e quindi a ogni sua parte, chiunque ne
  sia l'autore.

  Sia chiaro che non \`e nelle intenzioni di questa sezione
  accampare diritti su lavori scritti interamente da altri,
  l'intento \`e piuttosto quello di esercitare il diritto di
  controllare la distribuzione di lavori derivati o dal
  Programma o di cui esso sia parte.

  Inoltre, se il Programma o un lavoro derivato da esso
  viene aggregato a un altro lavoro non derivato dal
  Programma su di un mezzo di memorizzazione o di
  distribuzione, il lavoro non derivato non ricade nei
  termini di questa licenza.

\item \`E lecito copiare e distribuire il Programma (o un
  lavoro basato su di esso, come espresso al comma 2) sotto
  forma di codice oggetto o eseguibile secondo i termini dei
  precedenti commi 1 e 2, a patto che si applichi una delle
  seguenti condizioni:

  \begin{enumerate}
  \item Il Programma sia corredato dal codice sorgente
    completo, in una forma leggibile dal calcolatore e tale
    sorgente deve essere fornito secondo le regole dei
    precedenti commi 1 e 2 su di un mezzo comunemente usato
    per lo scambio di programmi.

  \item Il Programma sia accompagnato da un'offerta scritta,
    valida per almeno tre anni, di fornire a chiunque ne
    faccia richiesta una copia completa del codice sorgente,
    in una forma leggibile dal calcolatore, in cambio di un
    compenso non superiore al costo del trasferimento fisico
    di tale copia, che deve essere fornita secondo le regole
    dei precedenti commi 1 e 2 su di un mezzo comunemente
    usato per lo scambio di programmi.

  \item Il Programma sia accompagnato dalle informazioni che
    sono state ricevute riguardo alla possibilit\`a di
    ottenere il codice sorgente.  Questa alternativa \`e
    permessa solo in caso di distribuzioni non commerciali e
    solo se il programma \`e stato ricevuto sotto forma di
    codice oggetto o eseguibile in accordo al precedente
    punto b).
  \end{enumerate}

  Per ``codice sorgente completo'' di un lavoro si intende
  la forma preferenziale usata per modificare un lavoro. Per
  un programma eseguibile, ``codice sorgente completo''
  significa tutto il codice sorgente di tutti i moduli in
  esso contenuti, pi\`u ogni file associato che definisca le
  interfacce esterne del programma, pi\`u gli script usati
  per controllare la compilazione e l'installazione
  dell'eseguibile. In ogni caso non \`e necessario che il
  codice sorgente fornito includa nulla che sia normalmente
  distribuito (in forma sorgente o in formato binario) con i
  principali componenti del sistema operativo sotto cui
  viene eseguito il Programma (compilatore, kernel, e
  cos\`\i\ via), a meno che tali componenti accompagnino
  l'eseguibile.

  Se la distribuzione dell'eseguibile o del codice oggetto
  \`e effettuata indicando un luogo dal quale sia possibile
  copiarlo, permettere la copia del codice sorgente dallo
  stesso luogo \`e considerata una valida forma di
  distribuzione del codice sorgente, anche se copiare il
  sorgente \`e facoltativo per l'acquirente.

\item Non \`e lecito copiare, modificare, sublicenziare, o
  distribuire il Programma in modi diversi da quelli
  espressamente previsti da questa Licenza. Ogni tentativo
  contrario di copiare, modificare, sublicenziare o
  distribuire il Programma \`e legalmente nullo, e far\`a
  cessare automaticamente i diritti garantiti da questa
  Licenza. D'altra parte ogni acquirente che abbia ricevuto
  copie, o diritti, coperti da questa Licenza da parte di
  persone che violano la Licenza come qui indicato non
  vedranno invalidare la loro Licenza, purch\'e si
  comportino conformemente a essa.

\item L'acquirente non \`e obbligato ad accettare questa
  Licenza, poich\'e non l'ha firmata. D'altra parte nessun
  altro documento garantisce il permesso di modificare o
  distribuire il Programma o i lavori derivati da esso.
  Queste azioni sono proibite dalla legge per chi non
  accetta questa Licenza; perci\`o, modificando o
  distribuendo il Programma o un lavoro basato sul
  programma, si accetta implicitamente questa Licenza e
  quindi di tutti i suoi termini e le condizioni poste sulla
  copia, la distribuzione e la modifica del Programma o di
  lavori basati su di esso.

\item Ogni volta che il Programma o un lavoro basato su di
  esso vengono distribuiti, l'acquirente riceve
  automaticamente una licenza d'uso da parte del
  licenziatario originale. Tale licenza regola la copia, la
  distribuzione e la modifica del Programma secondo questi
  termini e queste condizioni. Non \`e lecito imporre
  restrizioni ulteriori all'acquirente nel suo esercizio dei
  diritti qui garantiti. Chi distribuisce programmi coperti
  da questa Licenza non \`e comunque responsabile per la
  conformit\`a alla Licenza da parte di terzi.

\item Se, come conseguenza del giudizio di un tribunale, o
  di un'imputazione per la violazione di un brevetto o per
  ogni altra ragione (anche non relativa a questioni di
  brevetti), vengono imposte condizioni che contraddicono le
  condizioni di questa licenza, che queste condizioni siano
  dettate dal tribunale, da accordi tra le parti o altro,
  queste condizioni non esimono nessuno dall'osservazione di
  questa Licenza. Se non \`e possibile distribuire un
  prodotto in un modo che soddisfi simultaneamente gli
  obblighi dettati da questa Licenza e altri obblighi
  pertinenti, il prodotto non pu\`o essere distribuito
  affatto. Per esempio, se un brevetto non permettesse a
  tutti quelli che lo ricevono di ridistribuire il Programma
  senza obbligare al pagamento di diritti, allora l'unico
  modo per soddisfare contemporaneamente il brevetto e
  questa Licenza \`e di non distribuire affatto il
  Programma.

  Se parti di questo comma sono ritenute non valide o
  inapplicabili per qualsiasi circostanza, deve comunque
  essere applicata l'idea espressa da questo comma; in ogni
  altra circostanza invece deve essere applicato il comma 7
  nel suo complesso.

  Non \`e nello scopo di questo comma indurre gli utenti a
  violare alcun brevetto n\'e ogni altra rivendicazione di
  diritti di propriet\`a, n\'e di contestare la validit\`a
  di alcuna di queste rivendicazioni; lo scopo di questo
  comma \`e solo quello di proteggere l'integrit\`a del
  sistema di distribuzione del software libero, che viene
  realizzato tramite l'uso della licenza pubblica. Molte
  persone hanno contribuito generosamente alla vasta gamma
  di programmi distribuiti attraverso questo sistema,
  basandosi sull'applicazione consistente di tale sistema.
  L'autore/donatore pu\`o decidere di sua volont\`a se
  preferisce distribuire il software avvalendosi di altri
  sistemi, e l'acquirente non pu\`o imporre la scelta del
  sistema di distribuzione.

  Questo comma serve a rendere il pi\`u chiaro possibile ci\`o
  che crediamo sia una conseguenza del resto di questa
  Licenza.

\item Se in alcuni paesi la distribuzione e/o l'uso del
  Programma sono limitati da brevetto o dall'uso di
  interfacce coperte da diritti d'autore, il detentore del
  copyright originale che pone il Programma sotto questa
  Licenza pu\`o aggiungere limiti geografici espliciti alla
  distribuzione, per escludere questi paesi dalla
  distribuzione stessa, in modo che il programma possa
  essere distribuito solo nei paesi non esclusi da questa
  regola. In questo caso i limiti geografici sono inclusi in
  questa Licenza e ne fanno parte a tutti gli effetti.

\item All'occorrenza la Free Software Foundation pu\`o
  pubblicare revisioni o nuove versioni di questa Licenza
  Pubblica Generica. Tali nuove versioni saranno simili a
  questa nello spirito, ma potranno differire nei dettagli
  al fine di coprire nuovi problemi e nuove situazioni.

  Ad ogni versione viene dato un numero identificativo. Se
  il Programma asserisce di essere coperto da una
  particolare versione di questa Licenza e ``da ogni
  versione successiva'', l'acquirente pu\`o scegliere se
  seguire le condizioni della versione specificata o di una
  successiva.  Se il Programma non specifica quale versione
  di questa Licenza deve applicarsi, l'acquirente pu\`o
  scegliere una qualsiasi versione tra quelle pubblicate
  dalla Free Software Foundation.

\item Se si desidera incorporare parti del Programma in
  altri programmi liberi le cui condizioni di distribuzione
  differiscano da queste, \`e possibile scrivere all'autore
  del Programma per chiederne l'autorizzazione. Per il
  software il cui copyright \`e detenuto dalla Free Software
  Foundation, si scriva alla Free Software Foundation;
  talvolta facciamo eccezioni alle regole di questa Licenza.
  La nostra decisione sar\`a guidata da due scopi:
  preservare la libert\`a di tutti i prodotti derivati dal
  nostro software libero e promuovere la condivisione e il
  riutilizzo del software in generale.

  \begin{center}
    \Large \textsc{Nessuna Garanzia}
  \end{center}

\item \textsc{Poich\'e il programma \`e concesso in uso
    gratuitamente, non c'\`e alcuna garanzia per il
    programma, nei limiti permessi dalle vigenti leggi. Se
    non indicato diversamente per iscritto, il detentore del
    Copyright e le altre parti forniscono il programma
    ``cosi` com'\`e'', senza alcun tipo di garanzia, n\'e
    esplicita n\'e implicita; ci\`o comprende, senza
    limitarsi a questo, la garanzia implicita di
    commerciabilit\`a e utilizzabilit\`a per un particolare
    scopo. L'intero rischio concernente la qualit\`a e le
    prestazioni del programma \`e dell'acquirente. Se il
    programma dovesse rivelarsi difettoso, l'acquirente si
    assume il costo di ogni manutenzione, riparazione o
    correzione necessaria.}

\item \textsc{N\'e il detentore del Copyright n\'e altre
    parti che possono modificare o ridistribuire il
    programma come permesso in questa licenza sono
    responsabili per danni nei confronti dell'acquirente, a
    meno che questo non sia richiesto dalle leggi vigenti o
    appaia in un accordo scritto.  Sono inclusi danni
    generici, speciali o incidentali, come pure i danni che
    conseguono dall'uso o dall'impossibilit\`a di usare il
    programma; ci\`o comprende, senza limitarsi a questo, la
    perdita di dati, la corruzione dei dati, le perdite
    sostenute dall'acquirente o da terze parti e
    l'inabilit\`a del programma a lavorare insieme ad altri
    programmi, anche se il detentore o altre parti sono
    state avvisate della possibilit\`a di questi danni.}
\end{enumerate}

\begin{center}
  \Large \textsc{Fine dei Termini e delle Condizioni}
\end{center}

\subsection*{Appendice: come applicare questi termini ai
  nuovi programmi}
Se si sviluppa un nuovo programma e lo si vuole rendere
della maggiore utilit\`a possibile per il pubblico, la cosa
migliore da fare \`e fare s\`\i\ che divenga software
libero, cosicch\'e ciascuno possa ridistribuirlo e
modificarlo secondo questi termini.

Per fare questo, si inserisca nel programma la seguente
nota. La cosa migliore da fare \`e mettere la nota
all`inizio di ogni file sorgente, per chiarire nel modo
pi\`u efficace possibile l'assenza di garanzia; ogni file
dovrebbe contenere almeno la nota di diritto d'autore e
l'indicazione di dove trovare l'intera nota.

\begin{center}
  <una riga per dire in breve il nome del programma e cosa
  fa> \\
  Copyright (C) <anno> <nome dell'autore>
\end{center}
\begin{quote}
  Questo programma \`e software libero; \`e lecito
  ridistribuirlo e/o modificarlo secondo i termini della
  Licenza Pubblica Generica GNU come pubblicata dalla Free
  Software Foundation; o la versione 2 della licenza o (a
  scelta) una versione successiva.

  Questo programma \`e distribuito nella speranza che sia
  utile, ma SENZA ALCUNA GARANZIA; senza neppure la garanzia
  implicita di COMMERCIABILIT\`A o di APPLICABILIT\`A PER UN
  PARTICOLARE SCOPO.  Si veda la Licenza Pubblica Generica
  GNU per avere maggiori dettagli.

  Ognuno dovrebbe avere ricevuto una copia della Licenza
  Pubblica Generica GNU insieme a questo programma; in caso
  contrario, la si pu\`o ottenere dalla Free Software
  Foundation, Inc., 59 Temple Place - Suite 330, Boston, MA
  02111-1307, Stati Uniti.
\end{quote}

Si aggiungano anche informazioni su come si pu\`o essere
contattati tramite posta elettronica e cartacea.

Se il programma \`e interattivo, si faccia in modo che
stampi una breve nota simile a questa quando viene usato
interattivamente:

\begin{center}
  Gnomovision versione 69, Copyright (C) <anno> <nome
  dell'autore> \\
  Gnomovision non ha ALCUNA GARANZIA; per i dettagli
  digitare `show w'.
\end{center}
\begin{quote}
  Questo \`e software libero, e ognuno \`e libero di
  ridistribuirlo sotto certe condizioni; digitare `show c'
  per dettagli.
\end{quote}

Gli ipotetici comandi \texttt{show w} e \texttt{show c}
mostreranno le parti appropriate della Licenza Pubblica
Generica. Chiaramente, i comandi usati possono essere
chiamati diversamente da \texttt{show w} e \texttt{show c} e
possono anche essere selezionati con il mouse o attraverso
un men\`u; in qualunque modo pertinente al programma.

Se necessario, si dovrebbe anche far firmare al proprio
datore di lavoro (se si lavora come programmatore) o alla
propria scuola, se si \`e studente, una ``rinuncia ai
diritti'' per il programma. Ecco un esempio con nomi
fittizi:

\begin{center}
  Yoyodyne, Inc. rinuncia con questo documento a ogni
  rivendicazione di diritti d'autore sul programma \\
  `Gnomovision' (che fa il primo passo con i compilatori)
  scritto da James Hacker. \\[1ex]
  <Firma di Ty Coon>, 1 Aprile 1989 \\
  Ty Coon, Presidente di Yoyodyne, Inc.
\end{center}

I programmi coperti da questa Licenza Pubblica Generica non
possono essere incorporati all'interno di programmi non
liberi. Se il proprio programma \`e una libreria di
funzioni, pu\`o essere pi\`u utile permettere di collegare
applicazioni proprietarie alla libreria. In questo caso
consigliamo di usare la Licenza Generica Pubblica GNU per
Librerie (LGPL) al posto di questa Licenza.

\endinput

% $Id: tabelle.tex,v 1.1 2005/03/01 10:06:08 loreti Exp $

\chapter{Tabelle}%
\label{ch:f.tabelle}
Nelle pagine seguenti sono riportati alcuni valori tabulati
relativi alle distribuzioni normale, di Student, del
$\chi^2$ e di Fisher.

Per le tabelle della distribuzione normale, per i valori
dell'ascissa compresi tra $0$ e $4$ sono state calcolate sia
l'ordinata della funzione di Gauss standardizzata
\begin{equation*}
  y = f(x) = \frac{1}{\sqrt{2\pi}} \,
    e^{-\frac{x^2}{2}}
\end{equation*}
che i valori $I_1$ ed $I_2$ di due differenti funzioni
integrali:
\begin{align*}
  I_1 &= \int_{-x}^x \! f(t) \, \de t &&\text{e} &
  I_2 &= \int_{-\infty}^x \! f(t) \, \de t \; .
\end{align*}

Per la distribuzione di Student, facendo variare il numero
di gradi di libert\`a $N$ (nelle righe della tabella) da $1$
a $40$, sono riportati i valori dell'ascissa $x$ che
corrispondono a differenti aree $P$ (nelle colonne della
tabella): in modo che, indicando con $S(t)$ la funzione di
frequenza di Student,
\begin{equation*}
  P = \int_{-\infty}^x \! S(t) \, \de t \; .
\end{equation*}

Per la distribuzione del $\chi^2$, poi, e sempre per diversi
valori di $N$, sono riportati i valori dell'ascissa $x$
corrispondenti ad aree determinate $P$, cos\`\i\ che
(indicando con $C(t)$ la funzione di frequenza del $\chi^2$)
risulti
\begin{equation*}
  P = \int_0^x \! C(t) \, \de t \; .
\end{equation*}

Per la distribuzione di Fisher, infine, per i soli due
valori 0.95 e 0.99 del livello di confidenza $P$, sono
riportati (per differenti gradi di libert\`a $M$ ed $N$) le
ascisse $x$ che corrispondono ad aree uguali al livello di
confidenza prescelto; ossia (indicando con $F(w)$ la
funzione di Fisher) tali che
\begin{equation} \label{eq:tabfish}
  P = \int_0^x \! F(w) \, \de w \; .
\end{equation}

Per calcolare i numeri riportati in queste tabelle si \`e
usato un programma in linguaggio \verb|C| che si serve delle
costanti matematiche e delle procedure di calcolo numerico
della \emph{GNU Scientific Library} (GSL); chi volesse
maggiori informazioni al riguardo le pu\`o trovare sul sito
web della \emph{Free Software Foundation}, sotto la URL
\verb|http://www.gnu.org/software/gsl/|.

La GSL contiene procedure per il calcolo numerico sia delle
funzioni di frequenza che di quelle cumulative per tutte le
funzioni considerate in questa appendice; e per tutte, meno
che per la funzione di Fisher, anche procedure per invertire
le distribuzioni cumulative.  Per trovare l'ascissa $x$ per
cui l'integrale \eqref{eq:tabfish} raggiunge un valore
prefissato si \`e usato il pacchetto della GSL che permette
di trovare gli zeri di una funzione definita dall'utente in
un intervallo arbitrario.

\clearpage
\input tabout
\endinput
               % Will be: Appendix G
%
% Appendix H: Bibliografia
%
\cleardoublepage
\chapter{Bibliografia}
\label{ch:g.biblio}
\noindent Per approfondire:
{ \setlength{\leftmargini}{\labelwidth}
  \begin{enumerate}
  \item Roger J.\ Barlow: \textit{Statistics: a guide to the
      use of statistical methods in the physical sciences}
    -- J.\ Wiley \& Sons, 1997
  \item G.\ Cowan: \textit{Statistical data analysis} --
    Oxford University Press, 1998 (ISBN 0-19-850155-2)
  \item H.\ Cram\'er: \textit{Mathematical methods of
      statistics} -- Princeton University Press, 1946
  \item W.T.\ Eadie, D.\ Drijard, F.E.\ James, M.\ Roos e
    B.\ Sadoulet: \textit{Statistical methods in
      experimental physics} -- North-Holland Publishing
    Company, 1971 (ISBN 0-7204-0239-5)
  \item W.\ Feller: \textit{An introduction to probability
      theory and its applications (3rd Ed.)} -- J.\ Wiley \&
    Sons, 1970 (ISBN 0-471-25711-7)
  \item R.A.\ Fisher: \textit{Statistical methods for
      research workers} -- Oliver \& Boyd, 1954
  \item H.\ Freeman: \textit{Introduction to statistical
      inference} -- Addison-Wesley, 1963
  \item M.G.\ Kendall e A.\ Stuart: \textit{The advanced
      theory of statistics} -- Griffin \& Co., 1958
  \item W.H.\ Press, S.A.\ Teukolsky, W.T.\ Vetterling e
    B.P.\ Flannery: \textit{Numerical recipes in C} --
    Cambridge University Press, 1992 (ISBN 0-521-43108-5)
  \item M.R.\ Spiegel: \textit{Statistica} -- Collana
    ``Schaum'' -- McGraw-Hill, 1961 (ISBN 88-386-5000-4)
  \item J.R.\ Taylor: \textit{Introduzione all'analisi degli
      errori} -- Zanichelli, 1986 (ISBN 88-08-03292-2)
  \item Particle Data Group: \textit{Review of particle
      physics: reviews, tables, and plots - Mathematical
      tools} -- \texttt{http://pdg.web.cern.ch/pdg/pdg.html}
  \end{enumerate}
}
%
% Postamble
%
\cleardoublepage
{\markboth{}{}
  \par\vspace*{50mm}\vfill
  \begin{flushright}
    \fontfamily{hlcn}\fontseries{m}\fontshape{it}
    \fontsize{12}{16}\selectfont
    In realt\`a un lavoro simile non termina mai. \\
    Lo si deve dichiarare concluso quando, \\
    a seconda del tempo e delle circostanze, \\
    si \`e fatto il possibile. \\[2ex]
    \fontsize{10}{12}\selectfont
    Johann Wolfgang von Goethe \\
    Italienische Reise (1789) \par
  \end{flushright}}
\cleardoublepage
%
% Index
%
\addcontentsline{toc}{chapter}{\numberline{}Indice analitico}
\printindex
\end{document}
