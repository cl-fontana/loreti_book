% $Id: tabelle.tex,v 1.1 2005/03/01 10:06:08 loreti Exp $

\chapter{Tabelle}%
\label{ch:f.tabelle}
Nelle pagine seguenti sono riportati alcuni valori tabulati
relativi alle distribuzioni normale, di Student, del
$\chi^2$ e di Fisher.

Per le tabelle della distribuzione normale, per i valori
dell'ascissa compresi tra $0$ e $4$ sono state calcolate sia
l'ordinata della funzione di Gauss standardizzata
\begin{equation*}
  y = f(x) = \frac{1}{\sqrt{2\pi}} \,
    e^{-\frac{x^2}{2}}
\end{equation*}
che i valori $I_1$ ed $I_2$ di due differenti funzioni
integrali:
\begin{align*}
  I_1 &= \int_{-x}^x \! f(t) \, \de t &&\text{e} &
  I_2 &= \int_{-\infty}^x \! f(t) \, \de t \; .
\end{align*}

Per la distribuzione di Student, facendo variare il numero
di gradi di libert\`a $N$ (nelle righe della tabella) da $1$
a $40$, sono riportati i valori dell'ascissa $x$ che
corrispondono a differenti aree $P$ (nelle colonne della
tabella): in modo che, indicando con $S(t)$ la funzione di
frequenza di Student,
\begin{equation*}
  P = \int_{-\infty}^x \! S(t) \, \de t \; .
\end{equation*}

Per la distribuzione del $\chi^2$, poi, e sempre per diversi
valori di $N$, sono riportati i valori dell'ascissa $x$
corrispondenti ad aree determinate $P$, cos\`\i\ che
(indicando con $C(t)$ la funzione di frequenza del $\chi^2$)
risulti
\begin{equation*}
  P = \int_0^x \! C(t) \, \de t \; .
\end{equation*}

Per la distribuzione di Fisher, infine, per i soli due
valori 0.95 e 0.99 del livello di confidenza $P$, sono
riportati (per differenti gradi di libert\`a $M$ ed $N$) le
ascisse $x$ che corrispondono ad aree uguali al livello di
confidenza prescelto; ossia (indicando con $F(w)$ la
funzione di Fisher) tali che
\begin{equation} \label{eq:tabfish}
  P = \int_0^x \! F(w) \, \de w \; .
\end{equation}

Per calcolare i numeri riportati in queste tabelle si \`e
usato un programma in linguaggio \verb|C| che si serve delle
costanti matematiche e delle procedure di calcolo numerico
della \emph{GNU Scientific Library} (GSL); chi volesse
maggiori informazioni al riguardo le pu\`o trovare sul sito
web della \emph{Free Software Foundation}, sotto la URL
\verb|http://www.gnu.org/software/gsl/|.

La GSL contiene procedure per il calcolo numerico sia delle
funzioni di frequenza che di quelle cumulative per tutte le
funzioni considerate in questa appendice; e per tutte, meno
che per la funzione di Fisher, anche procedure per invertire
le distribuzioni cumulative.  Per trovare l'ascissa $x$ per
cui l'integrale \eqref{eq:tabfish} raggiunge un valore
prefissato si \`e usato il pacchetto della GSL che permette
di trovare gli zeri di una funzione definita dall'utente in
un intervallo arbitrario.

\clearpage
\input tabout
\endinput
