% $Id: chapter1.tex,v 1.1 2005/03/01 10:06:08 loreti Exp $

\chapter{Introduzione}
Scopo della Fisica \`e lo studio dei fenomeni naturali, dei
quali essa cerca per prima cosa di dare una descrizione; che
deve essere non solo qualitativa, ma soprattutto
\emph{quantitativa}.  Questo richiede di individuare,
all'interno del fenomeno, quelle grandezze fisiche in grado
di caratterizzarlo univocamente; e di ottenere, per ognuna
di esse, i valori che si sono presentati in un insieme
significativo di casi reali.  Lo studio si estende poi
\emph{oltre} la semplice descrizione, e deve comprendere
l'indagine sulle relazioni reciproche tra pi\`u fenomeni,
sulle cause che li producono e su quelle che ne determinano
le modalit\`a di presentazione.  Fine ultimo di tale ricerca
\`e quello di formulare delle \emph{leggi fisiche} che siano
in grado di dare, del fenomeno in esame, una descrizione
razionale, quantitativa e (per quanto possibile) completa; e
che permettano di dedurre univocamente le caratteristiche
con cui esso si verificher\`a dalla conoscenza delle
caratteristiche degli altri fenomeni che lo hanno causato (o
che comunque con esso interagiscono).

Oggetto quindi della ricerca fisica devono essere delle
\emph{grandezze misurabili}; enti che cio\`e possano essere
caratterizzati dalla valutazione quantitativa di alcune loro
caratteristiche, suscettibili di variare da caso a caso a
seconda delle particolari modalit\`a con cui il fenomeno
studiato si svolge\/\footnote{Sono \emph{grandezze
    misurabili} anche quelle connesse a oggetti non
  direttamente osservabili, ma su cui possiamo indagare
  attraverso lo studio delle influenze prodotte dalla loro
  presenza sull'ambiente che li circonda.  Ad esempio i
  \emph{quarks}, costituenti delle particelle elementari
  dotate di interazione forte, secondo le attuali teorie per
  loro stessa natura non potrebbero esistere isolati allo
  stato libero; le loro caratteristiche (carica, spin etc.)
  non sono quindi direttamente suscettibili di misura: ma
  sono ugualmente oggetto della ricerca fisica, in quanto
  sono osservabili e misurabili i loro effetti al di fuori
  della particella entro la quale i quarks sono relegati.}.

\section{Il metodo scientifico}%
\index{metodo!scientifico|(}%
Il linguaggio usato dal ricercatore per la formulazione
delle leggi fisiche \`e il \emph{linguaggio matematico}, che
in modo naturale si presta a descrivere le relazioni tra i
dati numerici che individuano i fenomeni, le loro variazioni
ed i loro rapporti reciproci; il procedimento usato per
giungere a tale formulazione \`e il \emph{metodo
  scientifico}, la cui introduzione si fa storicamente
risalire a Galileo Galilei.  Esso pu\`o essere descritto
distinguendone alcune fasi successive:
\begin{itemize}
\item Una fase \emph{preliminare} in cui, basandosi sul
  bagaglio delle conoscenze precedentemente acquisite, si
  determinano sia le grandezze rilevanti per la descrizione
  del fenomeno che quelle che presumibilmente influenzano le
  modalit\`a con cui esso si presenter\`a.
\item Una fase \emph{sperimentale} in cui si compiono
  osservazioni accurate del fenomeno, controllando e
  misurando sia le grandezze che lo possono influenzare sia
  quelle caratteristiche quantitative che lo individuano e
  lo descrivono, mentre esso viene causato in maniera (per
  quanto possibile) esattamente riproducibile; ed in questo
  consiste specificatamente il lavoro dei fisici
  sperimentali.
\item Una fase di \emph{sintesi} o congettura in cui,
  partendo dai dati numerici raccolti nella fase precedente,
  si inducono delle relazioni matematiche tra le grandezze
  misurate che siano in grado di render conto delle
  osservazioni stesse; si formulano cio\`e delle leggi
  fisiche ipotetiche, controllando se esse sono in grado di
  spiegare il fenomeno.
\item Una fase \emph{deduttiva}, in cui dalle ipotesi
  formulate si traggono tutte le immaginabili conseguenze:
  particolarmente la previsione di fenomeni non ancora
  osservati (almeno non con la necessaria precisione); e
  questo \`e specificatamente il compito dei fisici teorici.
\item Infine una fase di \emph{verifica} delle ipotesi prima
  congetturate e poi sviluppate nei due passi precedenti, in
  cui si compiono ulteriori osservazioni sulle nuove
  speculazioni della teoria per accertarne l'esattezza.
\end{itemize}

Se nella fase di verifica si trova rispondenza con la
realt\`a, l'ipotesi diviene una legge fisica accettata; se
d'altra parte alcune conseguenze della teoria non risultano
confermate, e non si trovano spiegazioni delle discrepanze
tra quanto previsto e quanto osservato nell'ambito delle
conoscenze acquisite, la legge dovr\`a essere modificata in
parte, o rivoluzionata completamente per essere sostituita
da una nuova congettura; si ritorna cio\`e alla fase di
sintesi, e l'evoluzione della scienza comincia un nuovo
ciclo.

Naturalmente, anche se non si trovano contraddizioni con la
realt\`a ci\`o non vuol dire che la legge formulata sia
esatta: \`e possibile che esperimenti effettuati in
condizioni cui non si \`e pensato (o con strumenti di misura
pi\`u accurati di quelli di cui si dispone ora) dimostrino
in futuro che le nostre congetture erano in realt\`a
sbagliate; come esempio, basti pensare alla legge galileiana
del moto dei corpi ed alla moderna teoria della
relativit\`a.

\`E evidente quindi come le fasi di indagine e di verifica
sperimentale costituiscano parte fondamentale dello studio
dei fenomeni fisici; scopo di questo corso \`e quello di
presentare la teoria delle misure e degli
errori ad esse connessi.%
\index{metodo!scientifico|)}

\endinput
