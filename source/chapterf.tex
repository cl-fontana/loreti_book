% $Id: chapterf.tex,v 1.1 2005/04/13 08:28:24 loreti Exp $

\chapter{La licenza GNU GPL (General Public License)}
\label{ch:licgpl}
Questo capitolo contiene la licenza GNU GPL, sotto la quale
questo libro viene distribuito, sia nella versione originale
inglese\footnote{\texttt{http://www.gnu.org/licenses/gpl.html}
} (la sola dotata di valore legale) che in una traduzione
non ufficiale in italiano\footnote{La traduzione \`e dovuta
  al gruppo \emph{Pluto} (PLUTO Linux/Lumen Utentibus
  Terrarum Orbis, \texttt{http://www.pluto.linux.it/}). }
che aiuti chi ha difficolt\`a con l'inglese legale a
comprendere meglio il significato della licenza stessa.

\section{The GNU General Public License}
\selectlanguage{english}
\begin{center}
  \setlength{\parindent}{0in}
  Version 2, June 1991

  Copyright \copyright\ 1989, 1991 Free Software Foundation,
  Inc.

  \bigskip

  59 Temple Place - Suite 330, Boston, MA  02111-1307, USA

  \bigskip

  Everyone is permitted to copy and distribute verbatim
  copies of this \\
  license document, but changing it is not allowed.
\end{center}

\begin{quote}
  \begin{center} \textbf{Preamble} \end{center}

  The licenses for most software are designed to take away
  your freedom to share and change it.  By contrast, the GNU
  General Public License is intended to guarantee your
  freedom to share and change free software --- to make sure
  the software is free for all its users.  This General
  Public License applies to most of the Free Software
  Foundation's software and to any other program whose
  authors commit to using it.  (Some other Free Software
  Foundation software is covered by the GNU Library General
  Public License instead.)  You can apply it to your
  programs, too.

  When we speak of free software, we are referring to
  freedom, not price.  Our General Public Licenses are
  designed to make sure that you have the freedom to
  distribute copies of free software (and charge for this
  service if you wish), that you receive source code or can
  get it if you want it, that you can change the software or
  use pieces of it in new free programs; and that you know
  you can do these things.

  To protect your rights, we need to make restrictions that
  forbid anyone to deny you these rights or to ask you to
  surrender the rights.  These restrictions translate to
  certain responsibilities for you if you distribute copies
  of the software, or if you modify it.

  For example, if you distribute copies of such a program,
  whether gratis or for a fee, you must give the recipients
  all the rights that you have.  You must make sure that
  they, too, receive or can get the source code.  And you
  must show them these terms so they know their rights.

  We protect your rights with two steps: (1) copyright the
  software, and (2) offer you this license which gives you
  legal permission to copy, distribute and/or modify the
  software.

  Also, for each author's protection and ours, we want to
  make certain that everyone understands that there is no
  warranty for this free software.  If the software is
  modified by someone else and passed on, we want its
  recipients to know that what they have is not the
  original, so that any problems introduced by others will
  not reflect on the original authors' reputations.

  Finally, any free program is threatened constantly by
  software patents.  We wish to avoid the danger that
  redistributors of a free program will individually obtain
  patent licenses, in effect making the program proprietary.
  To prevent this, we have made it clear that any patent
  must be licensed for everyone's free use or not licensed
  at all.

  The precise terms and conditions for copying, distribution
  and modification follow.
\end{quote}

\begin{center}
  \Large \scshape GNU General Public License \\
  \vspace{3mm} Terms and Conditions For Copying,
  Distribution and Modification
\end{center}

\begin{enumerate}  \addtocounter{enumi}{-1}

\item This License applies to any program or other work
  which contains a notice placed by the copyright holder
  saying it may be distributed under the terms of this
  General Public License.  The ``Program'', below, refers to
  any such program or work, and a ``work based on the
  Program'' means either the Program or any derivative work
  under copyright law: that is to say, a work containing the
  Program or a portion of it, either verbatim or with
  modifications and/or translated into another language.
  (Hereinafter, translation is included without limitation
  in the term ``modification''.)  Each licensee is addressed
  as ``you''.

  Activities other than copying, distribution and
  modification are not covered by this License; they are
  outside its scope.  The act of running the Program is not
  restricted, and the output from the Program is covered
  only if its contents constitute a work based on the
  Program (independent of having been made by running the
  Program).  Whether that is true depends on what the
  Program does.

\item You may copy and distribute verbatim copies of the
  Program's source code as you receive it, in any medium,
  provided that you conspicuously and appropriately publish
  on each copy an appropriate copyright notice and
  disclaimer of warranty; keep intact all the notices that
  refer to this License and to the absence of any warranty;
  and give any other recipients of the Program a copy of
  this License along with the Program.

  You may charge a fee for the physical act of transferring
  a copy, and you may at your option offer warranty
  protection in exchange for a fee.

\item You may modify your copy or copies of the Program or
  any portion of it, thus forming a work based on the
  Program, and copy and distribute such modifications or
  work under the terms of Section 1 above, provided that you
  also meet all of these conditions:

  \begin{enumerate}
  \item You must cause the modified files to carry prominent
    notices stating that you changed the files and the date
    of any change.

  \item You must cause any work that you distribute or
    publish, that in whole or in part contains or is derived
    from the Program or any part thereof, to be licensed as
    a whole at no charge to all third parties under the
    terms of this License.

  \item If the modified program normally reads commands
    interactively when run, you must cause it, when started
    running for such interactive use in the most ordinary
    way, to print or display an announcement including an
    appropriate copyright notice and a notice that there is
    no warranty (or else, saying that you provide a
    warranty) and that users may redistribute the program
    under these conditions, and telling the user how to view
    a copy of this License.  (Exception: if the Program
    itself is interactive but does not normally print such
    an announcement, your work based on the Program is not
    required to print an announcement.)
  \end{enumerate}

  These requirements apply to the modified work as a whole.
  If identifiable sections of that work are not derived from
  the Program, and can be reasonably considered independent
  and separate works in themselves, then this License, and
  its terms, do not apply to those sections when you
  distribute them as separate works.  But when you
  distribute the same sections as part of a whole which is a
  work based on the Program, the distribution of the whole
  must be on the terms of this License, whose permissions
  for other licensees extend to the entire whole, and thus
  to each and every part regardless of who wrote it.

  Thus, it is not the intent of this section to claim rights
  or contest your rights to work written entirely by you;
  rather, the intent is to exercise the right to control the
  distribution of derivative or collective works based on
  the Program.

  In addition, mere aggregation of another work not based on
  the Program with the Program (or with a work based on the
  Program) on a volume of a storage or distribution medium
  does not bring the other work under the scope of this
  License.

\item You may copy and distribute the Program (or a work
  based on it, under Section 2) in object code or executable
  form under the terms of Sections 1 and 2 above provided
  that you also do one of the following:

  \begin{enumerate}
  \item Accompany it with the complete corresponding
    machine-readable source code, which must be distributed
    under the terms of Sections 1 and 2 above on a medium
    customarily used for software interchange; or,

  \item Accompany it with a written offer, valid for at
    least three years, to give any third party, for a charge
    no more than your cost of physically performing source
    distribution, a complete machine-readable copy of the
    corresponding source code, to be distributed under the
    terms of Sections 1 and 2 above on a medium customarily
    used for software interchange; or,

  \item Accompany it with the information you received as to
    the offer to distribute corresponding source code.
    (This alternative is allowed only for noncommercial
    distribution and only if you received the program in
    object code or executable form with such an offer, in
    accord with Subsection b above.)
  \end{enumerate}

  The source code for a work means the preferred form of the
  work for making modifications to it.  For an executable
  work, complete source code means all the source code for
  all modules it contains, plus any associated interface
  definition files, plus the scripts used to control
  compilation and installation of the executable.  However,
  as a special exception, the source code distributed need
  not include anything that is normally distributed (in
  either source or binary form) with the major components
  (compiler, kernel, and so on) of the operating system on
  which the executable runs, unless that component itself
  accompanies the executable.

  If distribution of executable or object code is made by
  offering access to copy from a designated place, then
  offering equivalent access to copy the source code from
  the same place counts as distribution of the source code,
  even though third parties are not compelled to copy the
  source along with the object code.

\item You may not copy, modify, sublicense, or distribute
  the Program except as expressly provided under this
  License.  Any attempt otherwise to copy, modify,
  sublicense or distribute the Program is void, and will
  automatically terminate your rights under this License.
  However, parties who have received copies, or rights, from
  you under this License will not have their licenses
  terminated so long as such parties remain in full
  compliance.

\item You are not required to accept this License, since you
  have not signed it.  However, nothing else grants you
  permission to modify or distribute the Program or its
  derivative works.  These actions are prohibited by law if
  you do not accept this License.  Therefore, by modifying
  or distributing the Program (or any work based on the
  Program), you indicate your acceptance of this License to
  do so, and all its terms and conditions for copying,
  distributing or modifying the Program or works based on
  it.

\item Each time you redistribute the Program (or any work
  based on the Program), the recipient automatically
  receives a license from the original licensor to copy,
  distribute or modify the Program subject to these terms
  and conditions.  You may not impose any further
  restrictions on the recipients' exercise of the rights
  granted herein.  You are not responsible for enforcing
  compliance by third parties to this License.

\item If, as a consequence of a court judgment or allegation
  of patent infringement or for any other reason (not
  limited to patent issues), conditions are imposed on you
  (whether by court order, agreement or otherwise) that
  contradict the conditions of this License, they do not
  excuse you from the conditions of this License.  If you
  cannot distribute so as to satisfy simultaneously your
  obligations under this License and any other pertinent
  obligations, then as a consequence you may not distribute
  the Program at all.  For example, if a patent license
  would not permit royalty-free redistribution of the
  Program by all those who receive copies directly or
  indirectly through you, then the only way you could
  satisfy both it and this License would be to refrain
  entirely from distribution of the Program.

  If any portion of this section is held invalid or
  unenforceable under any particular circumstance, the
  balance of the section is intended to apply and the
  section as a whole is intended to apply in other
  circumstances.

  It is not the purpose of this section to induce you to
  infringe any patents or other property right claims or to
  contest validity of any such claims; this section has the
  sole purpose of protecting the integrity of the free
  software distribution system, which is implemented by
  public license practices.  Many people have made generous
  contributions to the wide range of software distributed
  through that system in reliance on consistent application
  of that system; it is up to the author/donor to decide if
  he or she is willing to distribute software through any
  other system and a licensee cannot impose that choice.

  This section is intended to make thoroughly clear what is
  believed to be a consequence of the rest of this License.

\item If the distribution and/or use of the Program is
  restricted in certain countries either by patents or by
  copyrighted interfaces, the original copyright holder who
  places the Program under this License may add an explicit
  geographical distribution limitation excluding those
  countries, so that distribution is permitted only in or
  among countries not thus excluded.  In such case, this
  License incorporates the limitation as if written in the
  body of this License.

\item The Free Software Foundation may publish revised
  and/or new versions of the General Public License from
  time to time.  Such new versions will be similar in spirit
  to the present version, but may differ in detail to
  address new problems or concerns.

  Each version is given a distinguishing version number.  If
  the Program specifies a version number of this License
  which applies to it and ``any later version'', you have
  the option of following the terms and conditions either of
  that version or of any later version published by the Free
  Software Foundation.  If the Program does not specify a
  version number of this License, you may choose any version
  ever published by the Free Software Foundation.

\item If you wish to incorporate parts of the Program into
  other free programs whose distribution conditions are
  different, write to the author to ask for permission.  For
  software which is copyrighted by the Free Software
  Foundation, write to the Free Software Foundation; we
  sometimes make exceptions for this.  Our decision will be
  guided by the two goals of preserving the free status of
  all derivatives of our free software and of promoting the
  sharing and reuse of software generally.

  \begin{center}
    \Large \textsc{No Warranty}
  \end{center}

\item \textsc{Because the program is licensed free of
    charge, there is no warranty for the program, to the
    extent permitted by applicable law.  Except when
    otherwise stated in writing the copyright holders and/or
    other parties provide the program ``as is'' without
    warranty of any kind, either expressed or implied,
    including, but not limited to, the implied warranties of
    merchantability and fitness for a particular purpose.
    The entire risk as to the quality and performance of the
    program is with you.  Should the program prove
    defective, you assume the cost of all necessary
    servicing, repair or correction.}

\item \textsc{In no event unless required by applicable law
    or agreed to in writing will any copyright holder, or
    any other party who may modify and/or redistribute the
    program as permitted abo\-ve, be liable to you for
    damages, including any general, special, incidental or
    consequential damages arising out of the use or
    inability to use the program (including but not limited
    to loss of data or data being rendered inaccurate or
    losses sustained by you or third parties or a failure of
    the program to operate with any other programs), even if
    such holder or other party has been advised of the
    possibility of such damages.}
\end{enumerate}

\begin{center}
  \Large \textsc{End of Terms and Conditions}
\end{center}

\subsection*{Appendix: How to Apply These Terms to Your New
  Programs}
If you develop a new program, and you want it to be of the
greatest possible use to the public, the best way to achieve
this is to make it free software which everyone can
redistribute and change under these terms.

To do so, attach the following notices to the program.  It
is safest to attach them to the start of each source file to
most effectively convey the exclusion of warranty; and each
file should have at least the ``copyright'' line and a
pointer to where the full notice is found.

\begin{center}
  <one line to give the program's name and a brief idea of
  what it does.> \\
  Copyright (C) <year> <name of author>
\end{center}
\begin{quote}
  This program is free software; you can redistribute it
  and/or modify it under the terms of the GNU General Public
  License as published by the Free Software Foundation;
  either version 2 of the License, or (at your option) any
  later version.

  This program is distributed in the hope that it will be
  useful, but WITHOUT ANY WARRANTY; without even the implied
  warranty of MERCHANTABILITY or FITNESS FOR A PARTICULAR
  PURPOSE.  See the GNU General Public License for more
  details.

  You should have received a copy of the GNU General Public
  License along with this program; if not, write to the Free
  Software Foundation, Inc., 59 Temple Place - Suite 330,
  Boston, MA 02111-1307, USA.
\end{quote}

Also add information on how to contact you by electronic and
paper mail.  If the program is interactive, make it output a
short notice like this when it starts in an interactive
mode:

\begin{center}
  Gnomovision version 69, Copyright (C) <year> <name of
  author> \\
  Gnomovision comes with ABSOLUTELY NO WARRANTY; for details
  type `show w'.
\end{center}
\begin{quote}
  This is free software, and you are welcome to redistribute
  it under certain conditions; type `show c' for details.
\end{quote}

The hypothetical commands \texttt{show w} and \texttt{show
  c} should show the appropriate parts of the General Public
License.  Of course, the commands you use may be called
something other than \texttt{show w} and \texttt{show c};
they could even be mouse-clicks or menu items --- whatever
suits your program.

You should also get your employer (if you work as a
programmer) or your school, if any, to sign a ``copyright
disclaimer'' for the program, if necessary.  Here is a
sample; alter the names:

\begin{center}
  Yoyodyne, Inc., hereby disclaims all copyright interest in
  the program \\
  `Gnomovision' (which makes passes at compilers) written by
  James Hacker. \\[1ex]
  <signature of Ty Coon>, 1 April 1989 \\
  Ty Coon, President of Vice
\end{center}

This General Public License does not permit incorporating
your program into proprietary programs.  If your program is
a subroutine library, you may consider it more useful to
permit linking proprietary applications with the library.
If this is what you want to do, use the GNU Library General
Public License instead of this License.

\selectlanguage{italian}

\section{Licenza pubblica generica del progetto GNU}
\begin{center}
  \setlength{\parindent}{0in}
  Versione 2, Giugno 1991

  Copyright \copyright\ 1989, 1991 Free Software Foundation,
  Inc.

  \bigskip

  59 Temple Place - Suite 330, Boston, MA  02111-1307, USA

  \bigskip

  Tutti possono copiare e distribuire copie letterali di
  questo documento di licenza, ma non \`e permesso
  modificarlo.
\end{center}

\begin{quote}
  \begin{center} \textbf{Preambolo} \end{center}

  Le licenze per la maggioranza dei programmi hanno lo scopo
  di togliere all'utente la libert\`a di condividerlo e di
  modificarlo. Al contrario, la Licenza Pubblica Generica
  GNU \`e intesa a garantire la libert\`a di condividere e
  modificare il free software, al fine di assicurare che i
  programmi siano ``liberi'' per tutti i loro utenti. Questa
  Licenza si applica alla maggioranza dei programmi della
  Free Software Foundation e a ogni altro programma i cui
  autori hanno scelto questa Licenza. Alcuni altri programmi
  della Free Software Foundation sono invece coperti dalla
  Licenza Pubblica Generica per Librerie (LGPL). Chiunque
  pu\`o usare questa Licenza per i propri programmi.

  Quando si parla di free software, ci si riferisce alla
  libert\`a, non al prezzo. Le nostre Licenze (la GPL e la
  LGPL) sono progettate per assicurare che ciascuno abbia la
  libert\`a di distribuire copie del software libero (e
  farsi pagare per questo, se vuole), che ciascuno riceva il
  codice sorgente o che lo possa ottenere se lo desidera,
  che ciascuno possa modificare il programma o usarne delle
  parti in nuovi programmi liberi e che ciascuno sappia di
  potere fare queste cose.

  Per proteggere i diritti dell'utente, abbiamo bisogno di
  creare delle restrizioni che vietino a chiunque di negare
  questi diritti o di chiedere di rinunciarvi. Queste
  restrizioni si traducono in certe responsabilit\`a per chi
  distribuisce copie del software e per chi lo modifica.

  Per esempio, chi distribuisce copie di un Programma
  coperto da GPL, sia gratuitamente sia facendosi pagare,
  deve dare agli acquirenti tutti i diritti che ha ricevuto.
  Deve anche assicurarsi che gli acquirenti ricevano o
  possano ricevere il codice sorgente. E deve mostrar loro
  queste condizioni di Licenza, in modo che conoscano i loro
  diritti.

  Proteggiamo i diritti dell'utente attraverso due azioni:
  (1) proteggendo il software con un diritto d'autore (una
  nota di copyright), e (2) offrendo una Licenza che concede
  il permesso legale di copiare, distribuire e/o modificare
  il Programma.

  Infine, per proteggere ogni autore e noi stessi, vogliamo
  assicurarci che ognuno capisca che non ci sono garanzie
  per i programmi coperti da GPL. Se il Programma viene
  modificato da qualcun altro e ridistribuito, vogliamo che
  gli acquirenti sappiano che ci\`o che hanno non \`e
  l'originale, in modo che ogni problema introdotto da altri
  non si rifletta sulla reputazione degli autori originari.

  Infine, ogni programma libero \`e costantemente minacciato
  dai brevetti sui programmi. Vogliamo evitare il pericolo
  che chi ridistribuisce un Programma libero ottenga
  brevetti personali, rendendo perci\`o il Programma una
  cosa di sua propriet\`a. Per prevenire questo, abbiamo
  chiarito che ogni prodotto brevettato debba essere reso
  disponibile perch\'e tutti ne usufruiscano liberamente; se
  l'uso del prodotto deve sottostare a restrizioni allora
  tale prodotto non deve essere distribuito affatto.

  Seguono i termini e le condizioni precisi per la copia, la
  distribuzione e la modifica.
\end{quote}

\begin{center}
  \Large \scshape Licenza Pubblica Generica GNU \\
  \vspace{3mm} Termini e Condizioni per la Copia, la
  Distribuzione e la Modifica
\end{center}

\begin{enumerate}  \addtocounter{enumi}{-1}

\item Questa Licenza si applica a ogni Programma o altra
  opera che contenga una nota da parte del detentore del
  diritto d'autore che dica che tale opera pu\`o essere
  distribuita nei termini di questa Licenza Pubblica
  Generica. Il termine ``Programma'' nel seguito indica
  ognuno di questi programmi o lavori, e l'espressione
  ``lavoro basato sul Programma'' indica sia il Programma
  sia ogni opera considerata ``derivata'' in base alla legge
  sul diritto d'autore: cio\`e un lavoro contenente il
  Programma o una porzione di esso, sia letteralmente sia
  modificato e/o tradotto in un'altra lingua; da qui in
  avanti, la traduzione \`e in ogni caso considerata una
  ``modifica''.  Vengono ora elencati i diritti dei
  detentori di licenza.

  Attivit\`a diverse dalla copiatura, distribuzione e
  modifica non sono coperte da questa Licenza e sono al di
  fuori della sua influenza. L'atto di eseguire il programma
  non viene limitato, e l'output del programma \`e coperto
  da questa Licenza solo se il suo contenuto costituisce un
  lavoro basato sul Programma (indipendentemente dal fatto
  che sia stato creato eseguendo il Programma). In base alla
  natura del Programma il suo output pu\`o essere o meno
  coperto da questa Licenza.

\item \`E lecito copiare e distribuire copie letterali del
  codice sorgente del Programma cos\`\i\ come viene
  ricevuto, con qualsiasi mezzo, a condizione che venga
  riprodotta chiaramente su ogni copia un'appropriata nota
  di diritto d'autore e di assenza di garanzia; che si
  mantengano intatti tutti i riferimenti a questa Licenza e
  all'assenza di ogni garanzia; che si dia a ogni altro
  acquirente del Programma una copia di questa Licenza
  insieme al Programma.

  \`E possibile richiedere un pagamento per il trasferimento
  fisico di una copia del Programma, \`e anche possibile a
  propria discrezione richiedere un pagamento in cambio di
  una copertura assicurativa.

\item \`E lecito modificare la propria copia o copie del
  Programma, o parte di esso, creando perci\`o un lavoro
  basato sul Programma, e copiare o distribuire queste
  modifiche e questi lavori secondo i termini del precedente
  comma 1, a patto che vengano soddisfatte queste
  condizioni:

  \begin{enumerate}
  \item Bisogna indicare chiaramente nei file che si tratta
    di copie modificate e la data di ogni modifica.

  \item Bisogna fare in modo che ogni lavoro distribuito o
    pubblicato, che in parte o nella sua totalit\`a derivi
    dal Programma o da parti di esso, sia utilizzabile
    gratuitamente da terzi nella sua totalit\`a, secondo le
    condizioni di questa licenza.

  \item Se di solito il programma modificato legge comandi
    interattivamente quando viene eseguito, bisogna fare in
    modo che all'inizio dell'esecuzione interattiva usuale,
    stampi un messaggio contenente un'appropriata nota di
    diritto d'autore e di assenza di garanzia (oppure che
    specifichi che si offre una garanzia). Il messaggio deve
    inoltre specificare agli utenti che possono
    ridistribuire il programma alle condizioni qui descritte
    e deve indicare come consultare una copia di questa
    licenza. Se per\`o il programma di partenza \`e
    interattivo ma normalmente non stampa tale messaggio,
    non occorre che un lavoro derivato lo stampi.
  \end{enumerate}

  Questi requisiti si applicano al lavoro modificato nel suo
  complesso. Se sussistono parti identificabili del lavoro
  modificato che non siano derivate dal Programma e che
  possono essere ragionevolmente considerate lavori
  indipendenti, allora questa Licenza e i suoi termini non
  si applicano a queste parti quando vengono distribuite
  separatamente. Se per\`o queste parti vengono distribuite
  all'interno di un prodotto che \`e un lavoro basato sul
  Programma, la distribuzione di questo prodotto nel suo
  complesso deve avvenire nei termini di questa Licenza, le
  cui norme nei confronti di altri utenti si estendono a
  tutto il prodotto, e quindi a ogni sua parte, chiunque ne
  sia l'autore.

  Sia chiaro che non \`e nelle intenzioni di questa sezione
  accampare diritti su lavori scritti interamente da altri,
  l'intento \`e piuttosto quello di esercitare il diritto di
  controllare la distribuzione di lavori derivati o dal
  Programma o di cui esso sia parte.

  Inoltre, se il Programma o un lavoro derivato da esso
  viene aggregato a un altro lavoro non derivato dal
  Programma su di un mezzo di memorizzazione o di
  distribuzione, il lavoro non derivato non ricade nei
  termini di questa licenza.

\item \`E lecito copiare e distribuire il Programma (o un
  lavoro basato su di esso, come espresso al comma 2) sotto
  forma di codice oggetto o eseguibile secondo i termini dei
  precedenti commi 1 e 2, a patto che si applichi una delle
  seguenti condizioni:

  \begin{enumerate}
  \item Il Programma sia corredato dal codice sorgente
    completo, in una forma leggibile dal calcolatore e tale
    sorgente deve essere fornito secondo le regole dei
    precedenti commi 1 e 2 su di un mezzo comunemente usato
    per lo scambio di programmi.

  \item Il Programma sia accompagnato da un'offerta scritta,
    valida per almeno tre anni, di fornire a chiunque ne
    faccia richiesta una copia completa del codice sorgente,
    in una forma leggibile dal calcolatore, in cambio di un
    compenso non superiore al costo del trasferimento fisico
    di tale copia, che deve essere fornita secondo le regole
    dei precedenti commi 1 e 2 su di un mezzo comunemente
    usato per lo scambio di programmi.

  \item Il Programma sia accompagnato dalle informazioni che
    sono state ricevute riguardo alla possibilit\`a di
    ottenere il codice sorgente.  Questa alternativa \`e
    permessa solo in caso di distribuzioni non commerciali e
    solo se il programma \`e stato ricevuto sotto forma di
    codice oggetto o eseguibile in accordo al precedente
    punto b).
  \end{enumerate}

  Per ``codice sorgente completo'' di un lavoro si intende
  la forma preferenziale usata per modificare un lavoro. Per
  un programma eseguibile, ``codice sorgente completo''
  significa tutto il codice sorgente di tutti i moduli in
  esso contenuti, pi\`u ogni file associato che definisca le
  interfacce esterne del programma, pi\`u gli script usati
  per controllare la compilazione e l'installazione
  dell'eseguibile. In ogni caso non \`e necessario che il
  codice sorgente fornito includa nulla che sia normalmente
  distribuito (in forma sorgente o in formato binario) con i
  principali componenti del sistema operativo sotto cui
  viene eseguito il Programma (compilatore, kernel, e
  cos\`\i\ via), a meno che tali componenti accompagnino
  l'eseguibile.

  Se la distribuzione dell'eseguibile o del codice oggetto
  \`e effettuata indicando un luogo dal quale sia possibile
  copiarlo, permettere la copia del codice sorgente dallo
  stesso luogo \`e considerata una valida forma di
  distribuzione del codice sorgente, anche se copiare il
  sorgente \`e facoltativo per l'acquirente.

\item Non \`e lecito copiare, modificare, sublicenziare, o
  distribuire il Programma in modi diversi da quelli
  espressamente previsti da questa Licenza. Ogni tentativo
  contrario di copiare, modificare, sublicenziare o
  distribuire il Programma \`e legalmente nullo, e far\`a
  cessare automaticamente i diritti garantiti da questa
  Licenza. D'altra parte ogni acquirente che abbia ricevuto
  copie, o diritti, coperti da questa Licenza da parte di
  persone che violano la Licenza come qui indicato non
  vedranno invalidare la loro Licenza, purch\'e si
  comportino conformemente a essa.

\item L'acquirente non \`e obbligato ad accettare questa
  Licenza, poich\'e non l'ha firmata. D'altra parte nessun
  altro documento garantisce il permesso di modificare o
  distribuire il Programma o i lavori derivati da esso.
  Queste azioni sono proibite dalla legge per chi non
  accetta questa Licenza; perci\`o, modificando o
  distribuendo il Programma o un lavoro basato sul
  programma, si accetta implicitamente questa Licenza e
  quindi di tutti i suoi termini e le condizioni poste sulla
  copia, la distribuzione e la modifica del Programma o di
  lavori basati su di esso.

\item Ogni volta che il Programma o un lavoro basato su di
  esso vengono distribuiti, l'acquirente riceve
  automaticamente una licenza d'uso da parte del
  licenziatario originale. Tale licenza regola la copia, la
  distribuzione e la modifica del Programma secondo questi
  termini e queste condizioni. Non \`e lecito imporre
  restrizioni ulteriori all'acquirente nel suo esercizio dei
  diritti qui garantiti. Chi distribuisce programmi coperti
  da questa Licenza non \`e comunque responsabile per la
  conformit\`a alla Licenza da parte di terzi.

\item Se, come conseguenza del giudizio di un tribunale, o
  di un'imputazione per la violazione di un brevetto o per
  ogni altra ragione (anche non relativa a questioni di
  brevetti), vengono imposte condizioni che contraddicono le
  condizioni di questa licenza, che queste condizioni siano
  dettate dal tribunale, da accordi tra le parti o altro,
  queste condizioni non esimono nessuno dall'osservazione di
  questa Licenza. Se non \`e possibile distribuire un
  prodotto in un modo che soddisfi simultaneamente gli
  obblighi dettati da questa Licenza e altri obblighi
  pertinenti, il prodotto non pu\`o essere distribuito
  affatto. Per esempio, se un brevetto non permettesse a
  tutti quelli che lo ricevono di ridistribuire il Programma
  senza obbligare al pagamento di diritti, allora l'unico
  modo per soddisfare contemporaneamente il brevetto e
  questa Licenza \`e di non distribuire affatto il
  Programma.

  Se parti di questo comma sono ritenute non valide o
  inapplicabili per qualsiasi circostanza, deve comunque
  essere applicata l'idea espressa da questo comma; in ogni
  altra circostanza invece deve essere applicato il comma 7
  nel suo complesso.

  Non \`e nello scopo di questo comma indurre gli utenti a
  violare alcun brevetto n\'e ogni altra rivendicazione di
  diritti di propriet\`a, n\'e di contestare la validit\`a
  di alcuna di queste rivendicazioni; lo scopo di questo
  comma \`e solo quello di proteggere l'integrit\`a del
  sistema di distribuzione del software libero, che viene
  realizzato tramite l'uso della licenza pubblica. Molte
  persone hanno contribuito generosamente alla vasta gamma
  di programmi distribuiti attraverso questo sistema,
  basandosi sull'applicazione consistente di tale sistema.
  L'autore/donatore pu\`o decidere di sua volont\`a se
  preferisce distribuire il software avvalendosi di altri
  sistemi, e l'acquirente non pu\`o imporre la scelta del
  sistema di distribuzione.

  Questo comma serve a rendere il pi\`u chiaro possibile ci\`o
  che crediamo sia una conseguenza del resto di questa
  Licenza.

\item Se in alcuni paesi la distribuzione e/o l'uso del
  Programma sono limitati da brevetto o dall'uso di
  interfacce coperte da diritti d'autore, il detentore del
  copyright originale che pone il Programma sotto questa
  Licenza pu\`o aggiungere limiti geografici espliciti alla
  distribuzione, per escludere questi paesi dalla
  distribuzione stessa, in modo che il programma possa
  essere distribuito solo nei paesi non esclusi da questa
  regola. In questo caso i limiti geografici sono inclusi in
  questa Licenza e ne fanno parte a tutti gli effetti.

\item All'occorrenza la Free Software Foundation pu\`o
  pubblicare revisioni o nuove versioni di questa Licenza
  Pubblica Generica. Tali nuove versioni saranno simili a
  questa nello spirito, ma potranno differire nei dettagli
  al fine di coprire nuovi problemi e nuove situazioni.

  Ad ogni versione viene dato un numero identificativo. Se
  il Programma asserisce di essere coperto da una
  particolare versione di questa Licenza e ``da ogni
  versione successiva'', l'acquirente pu\`o scegliere se
  seguire le condizioni della versione specificata o di una
  successiva.  Se il Programma non specifica quale versione
  di questa Licenza deve applicarsi, l'acquirente pu\`o
  scegliere una qualsiasi versione tra quelle pubblicate
  dalla Free Software Foundation.

\item Se si desidera incorporare parti del Programma in
  altri programmi liberi le cui condizioni di distribuzione
  differiscano da queste, \`e possibile scrivere all'autore
  del Programma per chiederne l'autorizzazione. Per il
  software il cui copyright \`e detenuto dalla Free Software
  Foundation, si scriva alla Free Software Foundation;
  talvolta facciamo eccezioni alle regole di questa Licenza.
  La nostra decisione sar\`a guidata da due scopi:
  preservare la libert\`a di tutti i prodotti derivati dal
  nostro software libero e promuovere la condivisione e il
  riutilizzo del software in generale.

  \begin{center}
    \Large \textsc{Nessuna Garanzia}
  \end{center}

\item \textsc{Poich\'e il programma \`e concesso in uso
    gratuitamente, non c'\`e alcuna garanzia per il
    programma, nei limiti permessi dalle vigenti leggi. Se
    non indicato diversamente per iscritto, il detentore del
    Copyright e le altre parti forniscono il programma
    ``cosi` com'\`e'', senza alcun tipo di garanzia, n\'e
    esplicita n\'e implicita; ci\`o comprende, senza
    limitarsi a questo, la garanzia implicita di
    commerciabilit\`a e utilizzabilit\`a per un particolare
    scopo. L'intero rischio concernente la qualit\`a e le
    prestazioni del programma \`e dell'acquirente. Se il
    programma dovesse rivelarsi difettoso, l'acquirente si
    assume il costo di ogni manutenzione, riparazione o
    correzione necessaria.}

\item \textsc{N\'e il detentore del Copyright n\'e altre
    parti che possono modificare o ridistribuire il
    programma come permesso in questa licenza sono
    responsabili per danni nei confronti dell'acquirente, a
    meno che questo non sia richiesto dalle leggi vigenti o
    appaia in un accordo scritto.  Sono inclusi danni
    generici, speciali o incidentali, come pure i danni che
    conseguono dall'uso o dall'impossibilit\`a di usare il
    programma; ci\`o comprende, senza limitarsi a questo, la
    perdita di dati, la corruzione dei dati, le perdite
    sostenute dall'acquirente o da terze parti e
    l'inabilit\`a del programma a lavorare insieme ad altri
    programmi, anche se il detentore o altre parti sono
    state avvisate della possibilit\`a di questi danni.}
\end{enumerate}

\begin{center}
  \Large \textsc{Fine dei Termini e delle Condizioni}
\end{center}

\subsection*{Appendice: come applicare questi termini ai
  nuovi programmi}
Se si sviluppa un nuovo programma e lo si vuole rendere
della maggiore utilit\`a possibile per il pubblico, la cosa
migliore da fare \`e fare s\`\i\ che divenga software
libero, cosicch\'e ciascuno possa ridistribuirlo e
modificarlo secondo questi termini.

Per fare questo, si inserisca nel programma la seguente
nota. La cosa migliore da fare \`e mettere la nota
all`inizio di ogni file sorgente, per chiarire nel modo
pi\`u efficace possibile l'assenza di garanzia; ogni file
dovrebbe contenere almeno la nota di diritto d'autore e
l'indicazione di dove trovare l'intera nota.

\begin{center}
  <una riga per dire in breve il nome del programma e cosa
  fa> \\
  Copyright (C) <anno> <nome dell'autore>
\end{center}
\begin{quote}
  Questo programma \`e software libero; \`e lecito
  ridistribuirlo e/o modificarlo secondo i termini della
  Licenza Pubblica Generica GNU come pubblicata dalla Free
  Software Foundation; o la versione 2 della licenza o (a
  scelta) una versione successiva.

  Questo programma \`e distribuito nella speranza che sia
  utile, ma SENZA ALCUNA GARANZIA; senza neppure la garanzia
  implicita di COMMERCIABILIT\`A o di APPLICABILIT\`A PER UN
  PARTICOLARE SCOPO.  Si veda la Licenza Pubblica Generica
  GNU per avere maggiori dettagli.

  Ognuno dovrebbe avere ricevuto una copia della Licenza
  Pubblica Generica GNU insieme a questo programma; in caso
  contrario, la si pu\`o ottenere dalla Free Software
  Foundation, Inc., 59 Temple Place - Suite 330, Boston, MA
  02111-1307, Stati Uniti.
\end{quote}

Si aggiungano anche informazioni su come si pu\`o essere
contattati tramite posta elettronica e cartacea.

Se il programma \`e interattivo, si faccia in modo che
stampi una breve nota simile a questa quando viene usato
interattivamente:

\begin{center}
  Gnomovision versione 69, Copyright (C) <anno> <nome
  dell'autore> \\
  Gnomovision non ha ALCUNA GARANZIA; per i dettagli
  digitare `show w'.
\end{center}
\begin{quote}
  Questo \`e software libero, e ognuno \`e libero di
  ridistribuirlo sotto certe condizioni; digitare `show c'
  per dettagli.
\end{quote}

Gli ipotetici comandi \texttt{show w} e \texttt{show c}
mostreranno le parti appropriate della Licenza Pubblica
Generica. Chiaramente, i comandi usati possono essere
chiamati diversamente da \texttt{show w} e \texttt{show c} e
possono anche essere selezionati con il mouse o attraverso
un men\`u; in qualunque modo pertinente al programma.

Se necessario, si dovrebbe anche far firmare al proprio
datore di lavoro (se si lavora come programmatore) o alla
propria scuola, se si \`e studente, una ``rinuncia ai
diritti'' per il programma. Ecco un esempio con nomi
fittizi:

\begin{center}
  Yoyodyne, Inc. rinuncia con questo documento a ogni
  rivendicazione di diritti d'autore sul programma \\
  `Gnomovision' (che fa il primo passo con i compilatori)
  scritto da James Hacker. \\[1ex]
  <Firma di Ty Coon>, 1 Aprile 1989 \\
  Ty Coon, Presidente di Yoyodyne, Inc.
\end{center}

I programmi coperti da questa Licenza Pubblica Generica non
possono essere incorporati all'interno di programmi non
liberi. Se il proprio programma \`e una libreria di
funzioni, pu\`o essere pi\`u utile permettere di collegare
applicazioni proprietarie alla libreria. In questo caso
consigliamo di usare la Licenza Generica Pubblica GNU per
Librerie (LGPL) al posto di questa Licenza.

\endinput
