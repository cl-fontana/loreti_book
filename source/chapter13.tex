% $Id: chapter13.tex,v 1.1 2005/03/01 10:06:08 loreti Exp $

\chapter{La verifica delle ipotesi (II)}
Nel precedente capitolo \ref{ch:12.veripo} abbiamo esaminato
varie tecniche che ci permettono di decidere se una
caratteristica del processo fisico che ha prodotto un
campione di dati \`e o non \`e confermata dai dati stessi;
tutte queste tecniche non sono che casi particolari di una
teoria generale, di cui ora ci occuperemo, senza per\`o
scendere in profondit\`a nei dettagli.

In sostanza, nei vari casi del capitolo \ref{ch:12.veripo},
abbiamo formulato una certa ipotesi $H_0$ sulla natura di un
fenomeno casuale; e, ammesso per assurdo che questa ipotesi
fosse vera, abbiamo associato un ben definito valore della
densit\`a di probabilit\`a ad ogni punto $E$ dello spazio
$\mathcal{S}$ degli eventi.

Se indichiamo con $K$ un valore (arbitrariamente scelto)
della probabilit\`a, \emph{livello di confidenza} nel
linguaggio statistico, abbiamo in sostanza diviso
$\mathcal{S}$ in due sottoinsiemi esclusivi ed esaurienti:
uno $\mathcal{R}$ di eventi con probabilit\`a complessiva $1
- K$, ed uno $\mathcal{A} = \mathcal{S} - \mathcal{R}$ di
eventi con probabilit\`a complessiva $K$.

Per verificare l'ipotesi $H_0$ occorre scegliere a priori un
valore di $K$ da assumere come il confine che separi, da una
parte, eventi che riteniamo ragionevole si possano
presentare nell'ambito di pure fluttuazioni casuali se \`e
vera $H_0$; e, dall'altra, eventi cos\`\i\ improbabili
(sempre ammesso che $H_0$ sia vera) da far s\`\i\ che la
loro effettiva realizzazione debba implicare la falsit\`a
dell'ipotesi.

Normalmente si sceglie $K = 0.95$ o $K = 0.997$, i
valori della probabilit\`a che corrispondono a scarti di due
o tre errori quadratici medi per la distribuzione di Gauss,
anche se altri valori (come ad esempio $K = 0.999$ o
$K=0.99$) sono abbastanza comuni; e, una volta fatto
questo, si \emph{rigetta l'ipotesi} $H_0$ se il dato a
disposizione (un evento $E$ ottenuto dall'effettivo studio
del fenomeno in esame) appartiene ad $\mathcal{R}$; e la si
accetta se appartiene ad $\mathcal{A}$.

In realt\`a nella pratica si presenta in generale la
necessit\`a di discriminare tra \emph{due} ipotesi, sempre
mutuamente esclusive, che indicheremo con i simboli $H_0$ ed
$H_a$ e che, usando la terminologia della statistica, si
chiamano rispettivamente \emph{ipotesi nulla}%
\index{ipotesi!nulla}
ed \emph{ipotesi alternativa};%
\index{ipotesi!alternativa}
i casi precedenti corrispondono al caso particolare in cui
l'ipotesi alternativa coincida con il \emph{non realizzarsi}
di $H_0$.

Ipotesi nulla ed ipotesi alternativa possono essere entrambe
eventi semplici, oppure composti (ossia somma logica di
pi\`u eventualit\`a semplici); e lo scopo di questo capitolo
\`e quello di mostrare dei criteri sulla base dei quali si
possa opportunamente definire nello spazio degli eventi una
\emph{regione di rigetto} $\mathcal{R}$ per l'ipotesi nulla
(e, in corrispondenza, ovviamente, una regione $\mathcal{A}
= \mathcal{S} - \mathcal{R}$ nella quale tale ipotesi viene
accettata).

\`E chiaro che si corre sempre il rischio di sbagliare: o
rigettando erroneamente ipotesi in realt\`a vere
(\emph{errori di prima specie})%
\index{errore!di prima specie}
o accettando invece ipotesi in realt\`a false (\emph{errori
di seconda specie});%
\index{errore!di seconda specie}
e che, allargando o restringendo la regione di rigetto, si
pu\`o diminuire la probabilit\`a di uno di questi due tipi
di errori solo per aumentare la probabilit\`a di quelli
dell'altra categoria.  Se indichiamo con $P_I$ e $P_{II}$ le
probabilit\`a degli errori di prima e seconda specie
rispettivamente, sulla base della definizione risulta
\begin{align*}
  P_I \; &= \; \Pr( E \in \mathcal{R} | H_0 )
  &&\text{e} &
  P_{II} \; &= \; \Pr( E \in \mathcal{A} | H_a ) \peq .
\end{align*}

Quello che abbiamo finora chiamato ``livello di confidenza''
non \`e altro che $1 - P_I$; $P_I$ viene anche indicato col
simbolo $\alpha$ e chiamato \emph{significanza}%
\index{significanza}
del criterio adottato.  Infine, la probabilit\`a di
\emph{non} commettere un errore di seconda specie, ovvero la
probabilit\`a di rigettare $H_0$ quando l'ipotesi nulla \`e
falsa (e quindi quella alternativa \`e vera) si indica col
simbolo $\beta$ e si chiama \emph{potenza}%
\index{potenza}
del criterio adottato; essa vale quindi
\begin{equation*}
  \beta \; = \; \Pr( E \in \mathcal{R} | H_a ) \; = \; 1 -
  P_{II} \peq .
\end{equation*}

Per fare un esempio concreto, il fisico si trova spesso ad
esaminare ``eventi'' sperimentali e deve decidere se essi
sono del tipo desiderato (segnale) o no (fondo): in questo
caso l'ipotesi nulla $H_0$ consiste nell'appartenenza di un
evento al segnale, mentre l'ipotesi alternativa $H_a$
corrisponde invece all'appartenenza dello stesso evento al
fondo; che in genere non \`e l'intero insieme di eventi
complementare all'ipotesi nulla, $\ob{H}_0$, ma si sa
restringere ad una classe ben definita di fenomeni.

Gli errori di prima specie consistono in questo caso nello
scartare eventi buoni (errori di \emph{impoverimento} del
segnale), e quelli di seconda specie nell'introduzione nel
segnale di eventi di fondo (errori di
\emph{contaminazione}).

I criteri da seguire per definire una regione $\mathcal{R}$
nella quale rigettare $H_0$ sono dettati dalle
caratteristiche del processo di generazione: se gli eventi
di fondo sono preponderanti rispetto al segnale, ad esempio,
bisogner\`a evitare gli errori di seconda specie per quanto
possibile; anche al prezzo di scartare in questo modo una
parte consistente del segnale.

Estendendo al caso generale il metodo seguito nei vari casi
del capitolo \ref{ch:12.veripo} e prima delineato, se si \`e
in grado di associare ad ogni punto dello spazio degli
eventi \emph{due} valori della probabilit\`a (o della
densit\`a di probabilit\`a nel caso di variabili continue),
sia ammessa vera l'ipotesi nulla che ammessa invece vera
l'ipotesi alternativa, si pu\`o pensare di usare \emph{il
  loro rapporto} per definire la regione di rigetto.

Limitandoci al caso delle variabili continue, insomma, dopo
aver definito una nuova variabile casuale $\lambda$
attraverso la
\begin{equation*}
  \lambda = \frac{\mathcal{L} (\boldsymbol{x} |
    H_0)}{\mathcal{L} (\boldsymbol{x} | H_a)} \peq ,
\end{equation*}
possiamo scegliere arbitrariamente un numero reale $k$ e
decidere di accettare l'ipotesi $H_0$ se $\lambda \ge k$ o
di rifiutarla se $\lambda < k$; in definitiva ad ogni $k$
ammissibile \`e associata una differente regione di rigetto
$\mathcal{R}_k$ definita da
\begin{equation*}
  \mathcal{R}_k \; \equiv \; \left\{ \lambda =
    \frac{\mathcal{L} (\boldsymbol{x} | H_0)}{\mathcal{L}
    (\boldsymbol{x} | H_a)} < k \right\} \peq .
\end{equation*}

$\mathcal{L}$, nelle espressioni precedenti, \`e la funzione
di verosimiglianza;%
\index{funzione!di verosimiglianza|(}
che rappresenta appunto la densit\`a di probabilit\`a
corrispondente all'ottenere (sotto una certa ipotesi) un
campione di $N$ valori $x_1, x_2,\ldots,x_N$ (qui indicato
sinteticamente come un vettore $\boldsymbol{x}$ a $N$
componenti).  Ma in base a quale criterio dobbiamo scegliere
$k$?

\section{Un primo esempio}%
\label{ch:13.es1}
Cominciamo con un esempio didattico: supponiamo che i valori
$x_i$ si sappiano provenienti da una popolazione normale
$N(x; \mu, \sigma)$ di varianza $\sigma^2$ nota: e che il
nostro scopo consista nel discriminare tra due possibili
valori $\mu_1$ e $\mu_2$ per $\mu$; valori che, senza
perdere in generalit\`a, supponiamo siano 0 e 1 (potendosi
sempre effettuare un opportuno cambiamento di variabile
casuale che ci porti in questa situazione).  Riassumendo:
siano
\begin{equation*}
  \begin{cases}
    \setbox0=\hbox{$H_0$}
    \makebox[\wd0]{$x$} \sim N(x; \mu, \sigma)
    &\text{\hspace{2cm}(con $\sigma > 0$ noto)} \\[1ex]
    H_0 \equiv \{ \mu = 0 \} \\[1ex]
    H_a \equiv \{ \mu = 1 \}
  \end{cases}
\end{equation*}
le nostre ipotesi.

La densit\`a di probabilit\`a della $x$ vale
\begin{gather*}
  N(x; \mu, \sigma) = \frac{1}{\sigma \sqrt{2 \pi}} \, e^{
    - \frac{1}{2} \left( \frac{x - \mu}{\sigma}
    \right)^2 } \\
  \intertext{e, quindi, la funzione di verosimiglianza ed il
    suo logaritmo valgono}
  \mathcal{L} (\boldsymbol{x}; \mu, \sigma) =
    \frac{1}{\bigl( \sigma \sqrt{2 \pi} \bigr)^N}
    \prod_{i=1}^N e^{- \frac{1}{2 \sigma^2}
    \left( x_i - \mu \right)^2 }
\end{gather*}
e, rispettivamente,
\begin{align*}
  \ln \mathcal{L} (\boldsymbol{x}; \mu, \sigma) &= - N \,
    \ln \bigl( \sigma \sqrt{2 \pi} \bigr) - \frac{1}{2
    \sigma^2} \sum_{i=1}^N \left( x_i - \mu \right)^2
    \\[1ex]
  &= - N \, \ln \bigl( \sigma \sqrt{2 \pi} \bigr) -
    \frac{1}{2 \sigma^2} \sum_{i=1}^N \left( {x_i}^2 - 2 \mu
    x_i + \mu^2 \right) \peq ; \\
\end{align*}
per cui
\begin{equation} \label{eq:13.lnlino}
  \ln \mathcal{L} (\boldsymbol{x}; \mu, \sigma) = - N \, \ln
    \bigl( \sigma \sqrt{2 \pi} \bigr) - \frac{1}{2
    \sigma^2} \left( \sum_{i=1}^N {x_i}^2 - 2 N \bar x \mu +
    N \mu^2 \right) \peq .
\end{equation}
Dalla \eqref{eq:13.lnlino} si ricava immediatamente
\begin{equation*}
  \ln \lambda \; = \; \ln \mathcal{L} (\boldsymbol{x}; 0,
    \sigma) - \ln \mathcal{L} (\boldsymbol{x}; 1, \sigma) \;
    = \; \frac{N}{2 \sigma^2} \left( 1 - 2 \bar x \right)
\end{equation*}
e la regione di rigetto $\mathcal{R}_k$ \`e definita dalla
\begin{gather*}
  \mathcal{R}_k \; \equiv \; \left\{ \; \ln \lambda =
    \frac{N}{2 \sigma^2} \left( 1 - 2 \bar x \right) <
    \ln k \; \right\} \\
  \intertext{da cui consegue, con facili passaggi,}
  \mathcal{R}_k \; \equiv \; \left\{ \; \bar x > \frac{N -
    2 \sigma^2 \ln k}{2 N} = c \; \right\} \peq ;
\end{gather*}
ed insomma $H_0$ va rigettata se la media aritmetica del
campione $\bar x$ risulta superiore a $c$; ed accettata
altrimenti.

\begin{figure}[htbp]
  \vspace*{2ex}
  \begin{center} {
    \input{duegau.pstex_t}
  } \end{center}
  \caption[Un esempio: errori di prima e seconda specie]
    {L'esempio del paragrafo \ref{ch:13.es1}, con delineate
    (in corrispondenza ad un particolare valore di $c$) le
    probabilit\`a degli errori di prima e seconda specie; le
    due curve sono $N(0, \sigma / \sqrt{N})$ e $N(1, \sigma
    / \sqrt{N})$.}
  \label{fig:13.duegau}
\end{figure}

Come si pu\`o scegliere un valore opportuno di $k$ (e quindi
di $c$)?  Gli errori di prima specie (si faccia riferimento
anche alla figura \ref{fig:13.duegau}) hanno probabilit\`a
\begin{gather}
  P_I \; = \; 1 - \alpha \; = \; \Pr \bigl( \bar x > c | H_0
    \bigr) \; = \; \int_c^{+\infty} \! N \left( x; 0,
    \frac{\sigma}{\sqrt{N}} \right) \, \de x
    \label{eq:13.es1_1} \\
  \intertext{e quelli di seconda specie}
  P_{II} \; = \; 1 - \beta \; = \; \Pr \bigl( \bar x < c | H_a
    \bigr) \; = \; \int_{-\infty}^c \! N \left( x; 1,
    \frac{\sigma}{\sqrt{N}} \right) \, \de x
    \label{eq:13.es1_2}
\end{gather}
per cui si hanno svariate possibilit\`a: ad esempio, se
interessa contenere gli errori di prima specie e la
dimensione del campione \`e nota, si fissa un valore
opportunamente grande per $\alpha$ e dalla
\eqref{eq:13.es1_1} si ricava $c$; o, se interessa contenere
gli errori di seconda specie e la dimensione del campione
\`e nota, si fissa $\beta$ e si ricava $c$ dalla
\eqref{eq:13.es1_2}; o, infine, se si vogliono contenere gli
errori di entrambi i tipi, si fissano sia $\alpha$ che
$\beta$ e si ricava la dimensione minima del campione
necessaria per raggiungere lo scopo utilizzando entrambe le
equazioni \eqref{eq:13.es1_1} e \eqref{eq:13.es1_2}.

\section{Il lemma di Neyman--Pearson}%
\index{Neyman--Pearson, lemma di|(}
L'essere ricorsi per la definizione della regione di rigetto
$\mathcal{R}$ al calcolo del \emph{rapporto} delle funzioni
di verosimiglianza non \`e stato casuale; esiste infatti un
teorema (il cosiddetto \emph{lemma di Neyman--Pearson}) il
quale afferma che
\begin{quote}
  \textit{Se si ha a disposizione un campione di $N$ valori
    indipendenti $x_i$ da utilizzare per discriminare tra
    un'ipotesi nulla ed un'ipotesi alternativa entrambe
    semplici, e se \`e richiesto un livello fisso $\alpha$
    di significanza, la massima potenza del test (ovvero la
    minima probabilit\`a di errori di seconda specie) si
    raggiunge definendo la regione di rigetto
    $\mathcal{R}_k$ attraverso una relazione del tipo}
\end{quote}
\begin{equation} \label{eq:13.lambdak}
  \mathcal{R}_k \; \equiv \; \left\{ \; \lambda =
    \frac{\mathcal{L} (\boldsymbol{x} | H_0)}{\mathcal{L}
    (\boldsymbol{x} | H_a)} < k \; \right\} \peq .
\end{equation}

Infatti, indicando con $f = f(x; \theta)$ la densit\`a di
probabilit\`a della variabile $x$ (che supponiamo dipenda da
un solo parametro $\theta$), siano $H_0 \equiv \{ \theta =
\theta_0 \}$ e $H_a \equiv \{ \theta = \theta_a \}$ le due
ipotesi (semplici) tra cui decidere; la funzione di
verosimiglianza vale, come sappiamo,
\begin{equation*}
  \mathcal{L} (\boldsymbol{x}; \theta) = \prod_{i=1}^N
    f(x_i; \theta) \peq .
\end{equation*}
Indichiamo con $\Re$ l'insieme di tutte le regioni
$\mathcal{R}$ per le quali risulti
\begin{equation} \label{eq:13.explainx}
  P_I \; = \; \int_\mathcal{R} \! \mathcal{L}
    (\boldsymbol{x}; \theta_0) \, \de \boldsymbol{x} \; = \;
    1 - \alpha
\end{equation}
con $\alpha$ costante prefissata (nella
\eqref{eq:13.explainx} abbiamo sinteticamente indicato con
$\de \boldsymbol{x}$ il prodotto $\de x_1 \, \de x_2 \cdots
\de x_N$).
Vogliamo trovare quale di queste regioni rende massima
\begin{equation*}
  \beta \; = \; 1 - P_{II} \; = \; \int_\mathcal{R} \!
    \mathcal{L} (\boldsymbol{x}; \theta_a) \, \de
    \boldsymbol{x} \peq .
\end{equation*}

\noindent Ora, per una qualsiasi regione $\mathcal{R} \ne
\mathcal{R}_k$, valgono sia la
\begin{gather*}
  \mathcal{R}_k \; = \; (\mathcal{R}_k \cap \mathcal{R})
    \cup (\mathcal{R}_k \cap \overbar{\mathcal{R}}\,) \\
  \intertext{che la}
  \mathcal{R} \; = \; (\mathcal{R} \cap \mathcal{R}_k) \cup
    (\mathcal{R} \cap \overbar{\mathcal{R}}_k) \peq ; \\
  \intertext{e quindi, per una qualsiasi funzione
    $\phi(\boldsymbol{x})$, risulta sia}
  \int_{\mathcal{R}_k} \! \phi(\boldsymbol{x}) \, \de
    \boldsymbol{x} \; = \; \int_{\mathcal{R}_k \cap
    \mathcal{R}} \phi(\boldsymbol{x}) \, \de \boldsymbol{x}
    + \int_{\mathcal{R}_k \cap \overbar{\mathcal{R}}}
    \phi(\boldsymbol{x}) \, \de \boldsymbol{x} \\
  \intertext{che}
  \int_{\mathcal{R}} \! \phi(\boldsymbol{x}) \, \de
    \boldsymbol{x} \; = \; \int_{\mathcal{R} \cap
    \mathcal{R}_k} \phi(\boldsymbol{x}) \, \de
    \boldsymbol{x} + \int_{\mathcal{R} \cap
    \overbar{\mathcal{R}}_k} \phi(\boldsymbol{x}) \, \de
    \boldsymbol{x}
\end{gather*}
e, sottraendo membro a membro,
\begin{equation} \label{eq:13.neypea2}
  \int_{\mathcal{R}_k} \! \phi(\boldsymbol{x}) \, \de
    \boldsymbol{x} - \int_{\mathcal{R}} \!
    \phi(\boldsymbol{x}) \, \de \boldsymbol{x} \; = \;
    \int_{\mathcal{R}_k \cap \overbar{\mathcal{R}}}
    \phi(\boldsymbol{x}) \, \de \boldsymbol{x} -
    \int_{\mathcal{R} \cap \overbar{\mathcal{R}}_k}
    \phi(\boldsymbol{x}) \, \de \boldsymbol{x} \peq .
\end{equation}

\noindent Applicando la \eqref{eq:13.neypea2} alla funzione
$\mathcal{L}(\boldsymbol{x} | \theta_a)$ otteniamo:
\begin{multline} \label{eq:13.neypea3}
  \int_{\mathcal{R}_k} \! \mathcal{L}(\boldsymbol{x} |
    \theta_a) \, \de \boldsymbol{x} -\int_{\mathcal{R}} \!
    \mathcal{L}(\boldsymbol{x} | \theta_a) \, \de
    \boldsymbol{x} \; = \\
  = \; \int_{\mathcal{R}_k \cap \overbar{\mathcal{R}}}
    \mathcal{L}(\boldsymbol{x} | \theta_a) \, \de
    \boldsymbol{x} - \int_{\mathcal{R} \cap
    \overbar{\mathcal{R}}_k} \mathcal{L}(\boldsymbol{x}
    | \theta_a) \, \de \boldsymbol{x} \peq ; \qquad
\end{multline}
ma, nel primo integrale del secondo membro, essendo la
regione di integrazione contenuta in $\mathcal{R}_k$, deve
valere la \eqref{eq:13.lambdak}; e quindi risultare ovunque
\begin{gather*}
  \mathcal{L}(\boldsymbol{x} | \theta_a) \; > \; \frac{1}{k}
    \cdot \mathcal{L}(\boldsymbol{x} | \theta_0) \\
  \intertext{mentre, per lo stesso motivo, nel secondo
    integrale}
  \mathcal{L}(\boldsymbol{x} | \theta_a) \; \le \;
    \frac{1}{k} \cdot \mathcal{L}(\boldsymbol{x} | \theta_0)
\end{gather*}
e quindi la \eqref{eq:13.neypea3} implica che
\begin{multline*}
  \int_{\mathcal{R}_k} \! \mathcal{L}(\boldsymbol{x} |
    \theta_a) \, \de \boldsymbol{x} -\int_{\mathcal{R}} \!
    \mathcal{L}(\boldsymbol{x} | \theta_a) \, \de
    \boldsymbol{x} \; > \\
  > \; \frac{1}{k} \cdot \left[ \int_{\mathcal{R}_k \cap
    \overbar{\mathcal{R}}} \mathcal{L}(\boldsymbol{x} |
    \theta_0) \, \de \boldsymbol{x} - \int_{\mathcal{R}
    \cap \overbar{\mathcal{R}}_k}
    \mathcal{L}(\boldsymbol{x} | \theta_0) \, \de
    \boldsymbol{x} \right] \peq .
\end{multline*}
Ricordando la \eqref{eq:13.neypea2},
\begin{equation*}
  \int_{\mathcal{R}_k} \! \mathcal{L}(\boldsymbol{x} |
    \theta_a) \, \de \boldsymbol{x} -\int_{\mathcal{R}} \!
    \mathcal{L}(\boldsymbol{x} | \theta_a) \, \de
    \boldsymbol{x} \; > \; \frac{1}{k} \cdot \left[
    \int_{\mathcal{R}_k} \! \mathcal{L}(\boldsymbol{x} |
    \theta_0) \, \de \boldsymbol{x} -\int_{\mathcal{R}} \!
    \mathcal{L}(\boldsymbol{x} | \theta_0) \, \de
    \boldsymbol{x} \right]
\end{equation*}
e, se $\mathcal{R} \in \Re$ e quindi soddisfa anch'essa alla
\eqref{eq:13.explainx},
\begin{equation*}
  \int_{\mathcal{R}_k} \! \mathcal{L}(\boldsymbol{x} |
    \theta_a) \, \de \boldsymbol{x} -\int_{\mathcal{R}} \!
    \mathcal{L}(\boldsymbol{x} | \theta_a) \, \de
    \boldsymbol{x} \; > \; \frac{1}{k} \cdot \left[ P_I -
    P_I \right] \; = \; 0
\end{equation*}
che era la nostra tesi.%
\index{Neyman--Pearson, lemma di|)}

\section{Tests di massima potenza uniforme}
Consideriamo ora un esempio del tipo di quello del paragrafo
\ref{ch:13.es1}; e sia sempre disponibile un campione di $N$
misure indipendenti derivante da una popolazione normale di
varianza nota.  Assumiamo ancora come ipotesi nulla quella
che la popolazione abbia un certo valore medio, che
supponiamo essere 0, ma sostituiamo alla vecchia ipotesi
alternativa $H_a$ una nuova ipotesi \emph{composta}; ovvero
quella che il valore medio della popolazione sia positivo:
\begin{equation*}
  \begin{cases}
    \setbox0=\hbox{$H_0$}
    \makebox[\wd0]{$x$} \sim N(x; \mu, \sigma)
      &\text{\hspace{2cm}(con $\sigma > 0$ noto)} \\[1ex]
    H_0 \equiv \{ \mu = 0 \} \\[1ex]
    H_a \equiv \{ \mu > 0 \}
  \end{cases}
\end{equation*}
(l'ipotesi alternativa \`e dunque somma logica di infinite
ipotesi semplici del tipo $\mu = \mu_a$ con $\mu_a > 0$).

Dalla \eqref{eq:13.lnlino} ricaviamo immediatamente le
\begin{gather*}
  \mathcal{L} ( \boldsymbol{x}; 0, \sigma ) \; = \; - N \,
    \ln \bigl( \sigma \sqrt{2 \pi} \bigr) - \frac{1}{2
    \sigma^2} \sum_{i=1}^N {x_i}^2 \\
  \intertext{e}
  \mathcal{L} ( \boldsymbol{x}; \mu_a, \sigma ) \; = \; - N
    \, \ln \bigl( \sigma \sqrt{2 \pi} \bigr) - \frac{1}{2
    \sigma^2} \left( \sum_{i=1}^N {x_i}^2 - 2 N \bar x \mu_a
    + N {\mu_a}^2 \right)
  \intertext{(sempre con $\mu_a > 0$); e, sostituendole
    nella \eqref{eq:13.lambdak}, che definisce la generica
    regione di rigetto $\mathcal{R}_k$, otteniamo}
  \ln \lambda \; = \; \ln \mathcal{L} ( \boldsymbol{x}; 0,
    \sigma ) - \mathcal{L} ( \boldsymbol{x}; \mu_a, \sigma )
    \; = \; \frac{N \mu_a}{2 \sigma^2} \left( \mu_a - 2 \bar
    x \right) \; < \; \ln k \\
  \intertext{equivalente alla}
  \mathcal{R}_k \; \equiv \; \left\{ \; \bar x > \frac{N
      {\mu_a}^2 - 2 \sigma^2 \ln k}{2 N \mu_a} = c \;
  \right\} \peq .
\end{gather*}

Si rigetta quindi $H_0$ se la media aritmetica del campione
\`e superiore a $c$ e la si accetta altrimenti: la
probabilit\`a di commettere errori di prima specie vale
\begin{equation*}
  P_I \; = \; 1 - \alpha \; = \; \int_c^{+\infty} \! N
    \left( x; 0, \frac{\sigma}{\sqrt{N}} \right) \, \de x
\end{equation*}
ed \`e ben definita; ma, al contrario, la probabilit\`a di
commettere errori di seconda specie dipende dal particolare
valore di $\mu_a$, e non pu\`o quindi essere calcolata.

Se interessa solo contenere gli errori di prima specie e la
dimensione del campione \`e nota, si fissa $\alpha$ e si
ricava il corrispondente valore di $c$ dall'equazione
precedente; altrimenti occorre fare delle ulteriori ipotesi
sulla funzione di frequenza dei differenti valori di
$\mu_a$, e, ad esempio, calcolare la probabilit\`a degli
errori di seconda specie con tecniche di Montecarlo.

In ogni caso, per\`o, osserviamo che la regione di rigetto
\`e sempre dello stesso tipo \eqref{eq:13.lambdak} per
\emph{qualsiasi} $\mu_a > 0$; e quindi un confronto separato
tra $H_0$ ed ognuna delle differenti ipotesi semplici che
costituiscono $H_a$ \`e comunque del tipo per cui il lemma
di Neyman--Pearson garantisce la massima potenza.

Tests di questo tipo, per i quali \emph{la significanza \`e
  costante e la potenza \`e massima per ognuno dei casi
  semplici che costituiscono l'ipotesi alternativa}, si
dicono ``tests di massima potenza uniforme''.

\section{Il rapporto delle massime verosimiglianze}%
\index{metodo!del rapporto delle massime verosimiglianze|(}
Nel caso generale in cui sia l'ipotesi nulla che quella
alternativa siano composte, la situazione \`e pi\`u
complicata: non esiste normalmente un test di massima
potenza uniforme, e, tra i vari criteri possibili per
decidere tra le due ipotesi, bisogna capire quali abbiano
caratteristiche (significanza e potenza) adeguate; un metodo
adatto a costruire una regione di rigetto dotata
asintoticamente (per grandi campioni) di caratteristiche,
appunto, desiderabili, \`e quello seguente (\emph{metodo del
  rapporto delle massime verosimiglianze}).

Sia una variabile casuale $x$, la cui densit\`a di
probabilit\`a supponiamo sia una funzione $f(x; \theta_1,
\theta_2,\ldots, \theta_M )$ dipendente da $M$ parametri:
indicando sinteticamente la $M$-pla dei valori dei parametri
come un vettore $\boldsymbol{\theta}$ in uno spazio a $M$
dimensioni (\emph{spazio dei parametri}), consista $H_0$
nell'essere $\boldsymbol{\theta}$ compreso all'interno di
una certa regione $\Omega_0$ di tale spazio; mentre $H_a$
consista nell'appartenere $\boldsymbol{\theta}$ alla
regione $\Omega_a$ complementare a $\Omega_0$: $\Omega_a
\equiv \ob{H}_0$, cos\`\i\ che $(\Omega_0 \cup
\Omega_a)$ coincida con l'intero spazio dei parametri
$\mathcal{S}$.

In particolare, $\Omega_0$ pu\`o estendersi, in alcune delle
dimensioni dello spazio dei parametri, da $-\infty$ a
$+\infty$; e, in tal caso, il vincolo sulle $\theta_i$ cui
corrisponde l'ipotesi nulla riguarder\`a un numero di
parametri minore di $M$.

Scritta la funzione di verosimiglianza,
\begin{equation} \label{eq:13.genlik}
  \mathcal{L} ( \boldsymbol{x}; \boldsymbol{\theta} ) =
    \prod_{i=1}^N f(x_i; \boldsymbol{\theta} )
\end{equation}
indichiamo con $\mathcal{L} (\widehat S)$ il suo massimo
valore nell'intero spazio dei parametri; e con $\mathcal{L}
(\widehat R)$ il massimo valore assunto sempre della
\eqref{eq:13.genlik}, ma con i parametri vincolati a
trovarsi nella regione $\Omega_0$ (quindi limitatamente a
quei casi nei quali $H_0$ \`e vera).  Il rapporto
\begin{equation} \label{eq:13.gelira}
  \lambda = \frac{\mathcal{L} (\widehat R)}{\mathcal{L} (\widehat
    S)}
\end{equation}
deve essere un numero appartenente all'intervallo $[0,1]$;
se si fissa un arbitrario valore $k$ ($0<k<1$), esso
definisce una generica regione di rigetto, $\mathcal{R}_k$,
attraverso la
\begin{equation*}
  \mathcal{R}_k \; \equiv \; \left\{ \; \lambda =
    \frac{\mathcal{L} (\widehat R)}{\mathcal{L} (\widehat
      S)} < k \; \right\}
\end{equation*}
(ovvero si accetta $H_0$ quando $\lambda \ge k$ e la si
rigetta quando $\lambda < k$).  Nel caso si sappia
determinare la densit\`a di probabilit\`a di $\lambda$
condizionata all'assunzione che $H_0$ sia vera, $g(\lambda |
H_0)$, la probabilit\`a di un errore di prima specie \`e
data ovviamente da
\begin{equation*}
  P_I \; = \; \alpha \; = \; \Pr \bigl( \lambda \in [0,k] |
    H_0 \bigr) \; = \; \int_0^k \! g(\lambda | H_0) \, \de
    \lambda \peq .
\end{equation*}

L'importanza del metodo sta nel fatto che si pu\`o
dimostrare il seguente
\begin{quote}
  \textsc{Teorema:} \textit{se l'ipotesi nulla $H_0$
    consiste nell'appartenenza di un insieme di $P \le M$
    dei parametri $\theta_i$ ad una determinata regione
    $\Omega_0$, e se l'ipotesi alternativa $H_a$ consiste
    nel fatto che essi non vi appartengano ($H_a \equiv
    \ob{H}_0$), allora $- 2 \ln \lambda$, ove $\lambda$ \`e
    definito dalla \eqref{eq:13.gelira}, ha densit\`a di
    probabilit\`a che, ammessa vera l'ipotesi nulla,
    converge in probabilit\`a (all'aumentare di $N$) alla
    distribuzione del $\chi^2$ a $P$ gradi di libert\`a.}
\end{quote}%
\index{metodo!del rapporto delle massime verosimiglianze|)}
che, dicendoci quale \`e (almeno nel limite di grandi
campioni) la forma di $g(\lambda | H_0)$, ci mette comunque
in grado di calcolare la significanza del test.

Illustriamo il metodo con un esempio: disponendo ancora di
un campione di $N$ determinazioni indipendenti, provenienti
da una popolazione normale di varianza nota, vogliamo
applicarlo per discriminare tra l'ipotesi nulla che il valore
medio abbia valore 0 ($H_0 \equiv \{ \mu = 0 \}$) e quella
che esso abbia valore differente ($H_a \equiv \{ \mu \ne 0
\}$).

Il logaritmo della funzione di verosimiglianza \`e ancora
dato dalla \eqref{eq:13.lnlino}; e gi\`a sappiamo, dal
paragrafo \ref{ch:11.mepeted}, che $\mathcal{L}$ assume il
suo massimo valore quando $\mu = \bar x$, per cui
\begin{gather*}
  \ln \mathcal{L} (\widehat S) = - N \, \ln \bigl( \sigma
    \sqrt{2 \pi} \bigr) - \frac{1}{2 \sigma^2} \left(
    \sum_{i=1}^N {x_i}^2 - N \bar x^2 \right) \peq . \\
  \intertext{Inoltre $\Omega_0$ si riduce ad un unico punto,
    $\mu = 0$; per cui}
  \ln \mathcal{L} (\widehat R) = - N \, \ln \bigl( \sigma
    \sqrt{2 \pi} \bigr) - \frac{1}{2 \sigma^2} \sum_{i=1}^N
    {x_i}^2 \peq . \\
  \intertext{Dalla \eqref{eq:13.gelira} si ricava}
  \ln \lambda \; = \; \ln \mathcal{L} (\widehat R) - \ln
    \mathcal{L} (\widehat S) \; = \; - \frac{1}{2 \sigma^2} \, N
    \bar x^2 \\
  \intertext{e la regione di rigetto \`e definita dalla $\ln
    \lambda < \ln k$; ovvero (ricordando che $\ln k < 0$) da}
  \mathcal{R}_k \; \equiv \; \left\{ \; \bar x^2 > - \frac{2
    \sigma^2 \ln k}{N} \; \right\} \\
  \intertext{e, posto}
  c = \sigma \sqrt{- \frac{2 \ln k}{N}}
\end{gather*}
si accetter\`a $H_0$ se $| \bar x | \le c$ (e la si
rigetter\`a se $| \bar x | > c$).

In questo caso il teorema precedentemente citato afferma che
\begin{equation*}
  - 2 \ln \lambda = \frac{\phantom{M} \bar x^2
    \phantom{M}}{\dfrac{\sigma^2}{N}}
\end{equation*}
\`e distribuito asintoticamente come il $\chi^2$ ad un grado
di libert\`a (cosa che del resto gi\`a sapevamo, vista
l'espressione di $- 2 \ln \lambda$); per cui, indicando con
$F(t; N)$ la densit\`a di probabilit\`a della distribuzione
del $\chi^2$ a $N$ gradi di libert\`a, avremo
\begin{equation*}
  P_I \; = \; \alpha \; = \; \int_0^k \! g( \lambda | H_0 )
    \, \de \lambda \; = \; \int_{- 2 \ln k}^{+\infty} \!
    F(t; 1) \, \de t
\end{equation*}
della quale ci possiamo servire per ricavare $k$ se vogliamo
che la significanza del test abbia un certo valore: ad
esempio un livello di confidenza del 95\% corrisponde ad
$\alpha = 0.05$ e, dalle tabelle della distribuzione del
$\chi^2$, ricaviamo
\begin{align*}
  - 2 \ln k &= 3.84 &&\text{e quindi} & c &= 1.96 \,
    \frac{\sigma}{\sqrt{N}} \peq .
\end{align*}

Anche senza dover ricorrere al teorema sul comportamento
asintotico di $- 2 \ln \lambda$, allo stesso risultato si
pu\`o pervenire per altra via: in questo caso si conosce
infatti esattamente $\alpha$, che vale
\begin{equation*}
  P_I \; = \; \alpha \; = \; \Pr \Bigl( | \bar x | > c
    \bigl| H_0 \bigr. \Bigr) \; = \; 2 \int_c^{+\infty} \!
    N \left( t; 0, \frac{\sigma}{\sqrt{N}} \right) \, \de t
\end{equation*}
e, dalle tabelle della distribuzione normale standardizzata,
si ricava che un'area two-tailed del 5\% corrisponde ad un
valore assoluto dello scarto normalizzato $t_0 = 1.96$; per
cui, ancora, si ricaverebbe $| \bar x | > 1.96 ( \sigma /
\sqrt{N} )$ come test per un livello di confidenza del
95\%.

\section{Applicazione: ipotesi sulle probabilit\`a}
Nel paragrafo \ref{ch:11.exampl} abbiamo preso in
considerazione il caso di un evento casuale che si pu\`o
manifestare in un numero finito $M$ di modalit\`a, aventi
ognuna probabilit\`a incognita $p_i$; la stima di massima
verosimiglianza delle $p_i$ \`e data dal rapporto tra la
frequenza assoluta di ogni modalit\`a, $n_i$, ed il numero
totale di prove, $N$.

Vogliamo ora applicare il metodo del rapporto delle massime
verosimiglianze per discriminare, sulla base di un campione
di determinazioni indipendenti, l'ipotesi nulla che le
probabilit\`a abbiano valori noti a priori e l'ipotesi
alternativa complementare, $H_a \equiv \ob{H}_0$:
\begin{equation*}
  \begin{cases}
    H_0 \; \equiv \; \left\{ p_i = \pi_i \right\} &
    \hspace{2cm} (\forall  i \in \{ 1, 2,\ldots, M \})
    \\[1ex]
    H_a \; \equiv \; \left\{ p_i \ne \pi_i \right\} &
    \hspace{2cm} (\exists i \in \{ 1, 2,\ldots, M \})
  \end{cases}
\end{equation*}

Ricordiamo che la funzione di verosimiglianza, a meno di un
fattore moltiplicativo costante, \`e data da
\begin{gather*}
  \mathcal{L} ( \boldsymbol{n}; \boldsymbol{p} ) =
    \prod_{i=1}^M {p_i}^{n_i} \\
  \intertext{e che, essendo la stima di massima
    verosimiglianza data da}
  \widehat p_i = \frac{n_i}{N} \\
  \intertext{il massimo assoluto di $\mathcal{L}$ \`e}
  \mathcal{L} (\widehat S) \; = \; \prod_{i=1}^M \left(
    \frac{n_i}{N} \right)^{n_i} \; = \; \frac{1}{N^N}
    \prod_{i=1}^M {n_i}^{n_i} \peq . \\
  \intertext{Inoltre, nell'unico punto dello spazio dei
    parametri che corrisponde ad $H_0$,}
  \mathcal{L} (\widehat R) = \prod_{i=1}^M {\pi_i}^{n_i} \\
  \intertext{per cui}
  \lambda \; = \; \frac{\mathcal{L} (\widehat R)}{\mathcal{L}
    (\widehat S)} \; = \; N^N \prod_{i=1}^M \left(
    \frac{\pi_i}{n_i} \right)^{n_i}
\end{gather*}
dalla quale si pu\`o, come sappiamo, derivare una generica
regione di rigetto attraverso la consueta $\mathcal{R}_k
\equiv \{ \lambda < k \}$.

\begin{equation*}
  -2 \, \ln \lambda \; = \; -2 \left[ N \, \ln N +
    \sum_{i=1}^M n_i \left( \ln \pi_i - \ln n_i \right)
    \right]
\end{equation*}
\`e inoltre asintoticamente distribuita come il $\chi^2$ a
$M-1$ gradi di libert\`a (c'\`e un vincolo: che le $n_i$
abbiano somma $N$), e questo pu\`o servire a scegliere un
$k$ opportuno (nota la dimensione del campione) una volta
fissata $\alpha$.

Il criterio di verifica dell'ipotesi dato in precedenza
consisteva nel calcolo del valore della variabile casuale
\begin{equation*}
  X = \sum_{i=1}^M \frac{ \left( n_i - N \pi_i \right)^2 }{
    N \pi_i }
\end{equation*}
e nel suo successivo confronto con la distribuzione del
$\chi^2$ a $M-1$ gradi di libert\`a; lo studio del rapporto
delle massime verosimiglianze porta dunque ad un criterio
\emph{differente} e, senza sapere nulla della probabilit\`a
di commettere errori di seconda specie, non \`e possibile
dire quale dei due risulti migliore (a parit\`a di
significanza).

\section{Applicazione: valore medio di una
  popolazione normale}
Ancora un esempio: sia una popolazione normale $N(x; \mu,
\sigma)$ dalla quale vengano ottenuti $N$ valori
indipendenti $x_i$, ma questa volta \emph{la varianza}
$\sigma$ \emph{sia ignota}; vogliamo discriminare, sulla
base del campione, tra l'ipotesi nulla che il valore medio
della popolazione abbia un valore prefissato e l'ipotesi
alternativa complementare,
\begin{equation*}
  \begin{cases}
    H_0 \; \equiv \; \left\{ \mu = \mu_0 \right\} \\[1ex]
    H_a \; \equiv \; \left\{ \mu \ne \mu_0 \right\}
  \end{cases}
\end{equation*}

Il logaritmo della funzione di verosimiglianza \`e
\begin{equation} \label{eq:13.likgau}
  \ln \mathcal{L} ( \boldsymbol{x}; \mu, \sigma ) \; = \;
    - N \, \ln \sigma - \frac{N}{2} \, \ln ( 2 \pi )
    - \frac{1}{2 \sigma^2} \sum_{i=1}^N ( x_i - \mu)^2
\end{equation}
ed essendo le stime di massima verosimiglianza date, come
avevamo trovato nel paragrafo \ref{ch:11.exampl}, da
\begin{align*}
  \widehat \mu &= \bar x = \frac{1}{N} \sum_{i=1}^N x_i
   &&\text{e} &
   \widehat \sigma^2 &= \frac{1}{N} \sum_{i=1}^N \left( x_i
     - \widehat \mu \right)^2
\end{align*}
ne deriva, sostituendo nella \eqref{eq:13.likgau}, che
\begin{equation*}
  \ln \mathcal{L} (\widehat S) \; = \; - \frac{N}{2} \ln \left[
    \sum_{i=1}^N \left( x_i - \bar x \right)^2 \right] +
    \frac{N}{2} \, \ln N - \frac{N}{2} \, \ln ( 2 \pi ) -
    \frac{N}{2} \peq .
\end{equation*}

D'altra parte, ammessa vera $H_0$, abbiamo che
\begin{gather*}
  \ln \mathcal{L} ( \boldsymbol{x} | H_0 ) = - N \, \ln
    \sigma - \frac{N}{2} \, \ln ( 2 \pi ) - \frac{1}{2
    \sigma^2} \, \sum_{i=1}^N \left( x_i - \mu_0 \right)^2
    \\
  \intertext{e, derivando rispetto a $\sigma$,}
  \frac{\de}{\de \sigma} \, \ln \mathcal{L} ( \boldsymbol{x}
    | H_0 ) =  - \frac{N}{\sigma} + \frac{1}{\sigma^3}
    \sum_{i=1}^N \left( x_i - \mu_0 \right)^2 \peq . \\
  \intertext{Annullando la derivata prima, si trova che
    l'unico estremante di $\mathcal{L} ( \boldsymbol{x} |
    H_0 )$ si ha per}
  \sigma_0 = \frac{1}{N} \sum_{i=1}^N \left( x_i - \mu_0
    \right)^2 \\
  \intertext{mentre la derivata seconda vale}
  \frac{\de^2}{\de \sigma^2} \, \ln \mathcal{L} (
    \boldsymbol{x} | H_0 ) = \frac{N}{\sigma^2} -
    \frac{3}{\sigma^4} \sum_{i=1}^N \left( x_i - \mu_0
    \right)^2 \\
  \intertext{e, calcolata per $\sigma = \sigma_0$,}
  \left. \frac{\de^2 (\ln \mathcal{L}) }{\de \sigma^2}
    \right|_{\sigma = \sigma_0} \; = \; - \frac{2 N^2}{
    \sum_i \left( x_i - \mu_0 \right)^2 } \; < \; 0 \\
\end{gather*}
per cui l'estremante \`e effettivamente un massimo.
Sostituendo,
\begin{gather*}
  \ln \mathcal{L} (\widehat R) \; = \; - \frac{N}{2} \ln \left[
    \sum_{i=1}^N \left( x_i - \mu_0 \right)^2 \right] +
    \frac{N}{2} \, \ln N - \frac{N}{2} \, \ln ( 2 \pi ) -
    \frac{N}{2} \\[1ex]
  \ln \lambda \; = \; \ln \mathcal{L} (\widehat R) - \ln
    \mathcal{L} (\widehat S) \; = \; - \frac{N}{2} \left\{ \ln
    \left[ \sum_{i=1}^N \left( x_i - \mu_0 \right)^2 \right]
    - \ln \left[ \sum_{i=1}^N \left( x_i - \bar x \right)^2
    \right] \right\}
\end{gather*}
ed infine
\begin{align*}
  \ln \lambda &= - \frac{N}{2} \, \ln \left[ \frac{ \sum_i
    \left( x_i - \mu_0 \right)^2 }{ \sum_i \left( x_i - \bar
    x \right)^2 } \right] \\[1.5ex]
  &= - \frac{N}{2} \, \ln \left[ 1 + \frac{N \left( \bar x -
    \mu_0 \right)^2 }{ \sum_i \left( x_i - \bar x \right)^2
    } \right] \\[1.5ex]
  &= - \frac{N}{2} \, \ln \left( 1 + \frac{t^2}{N - 1}
    \right)
\end{align*}
tenendo conto dapprima del fatto che $\sum_i ( x_i - \mu_0
)^2 = \sum_i ( x_i - \bar x )^2 + N (\bar x - \mu_0 )^2$, e
definendo poi una nuova variabile casuale
\begin{equation*}
  t \; = \; \left( \bar x - \mu_0 \right) \sqrt{ \frac{N (N
    - 1)}{\sum_i \left( x_i - \bar x \right)^2} } \; = \;
    \frac{\phantom{t} \bar x - \mu_0 \phantom{t}}{
    \dfrac{s}{\sqrt{N}} } \peq .
\end{equation*}

Un qualunque metodo per il rigetto di $H_0$ definito
confrontando $\lambda$ con un prefissato valore $k$ si
traduce, in sostanza, in un corrispondente confronto da
eseguire per $t$:
\begin{gather*}
  \mathcal{R}_k \; \equiv \; \bigl\{ \ln \lambda < \ln k
    \bigr\} \\
  \intertext{che porta alla}
  - \frac{N}{2} \, \ln \left( 1 + \frac{t^2}{N - 1}
    \right) \; < \; \ln k \\
  \intertext{ed alla condizione}
  t^2 \; > \; (N - 1) \, \left( k^{- \tfrac{2}{N}} - 1
    \right) \peq ;
\end{gather*}
ovvero si rigetta l'ipotesi nulla se $|t|$ \`e maggiore di
un certo $t_0$ (derivabile dall'equazione precedente), e la
si accetta altrimenti.

Ma $t$ (vedi anche l'equazione \eqref{eq:12.xsecond}) segue
la distribuzione di Student a $N-1$ gradi di libert\`a, e
quindi accettare o rigettare $H_0$ sotto queste ipotesi si
riduce ad un test relativo a quella distribuzione: come
gi\`a si era concluso nel capitolo \ref{ch:12.veripo}.
Il livello di significanza $\alpha$ \`e legato a $t_0$ dalla
\begin{equation*}
  \frac{\alpha}{2} = \int_{t_0}^{+\infty} \! F(t; N-1)
    \, \de t
\end{equation*}
(indicando con $F(t;N)$ la funzione di frequenza di Student
a $N$ gradi di libert\`a), tenendo conto che abbiamo a che
fare con un two-tailed test ($\mathcal{R}_k \equiv \bigl\{
|t| > t_0 \bigr\}$).

Insomma non c'\`e differenza, in questo caso, tra quanto
esposto nel capitolo precedente e la teoria generale
discussa in quello presente: nel senso che i due criteri di
verifica dell'ipotesi portano per questo problema allo
stesso metodo di decisione (ma, come abbiamo visto nel
paragrafo precedente, non \`e sempre cos\`\i).%
\index{funzione!di verosimiglianza|)}

\endinput
