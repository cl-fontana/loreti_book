% $Id: chapterd.tex,v 1.1 2005/03/01 10:06:08 loreti Exp $

\chapter{Il modello di Laplace e la funzione di Gauss}%
\index{distribuzione!normale|(}%
\label{ch:d.applap}
Pensiamo di eseguire una misura di una grandezza fisica (il
cui valore vero indicheremo con il simbolo $x^*$), e sia $x$
il risultato ottenuto; in generale $x$ \`e diverso da $x^*$
per la presenza degli errori di misura, che supporremo siano
di natura puramente casuale.

\index{Laplace!modello di|(}%
Questi errori casuali di misura possono essere schematizzati
come un insieme estremamente grande, al limite infinito, di
disturbi contemporanei molto piccoli, al limite
infinitesimi, ognuno dei quali tende ad alterare di
pochissimo il risultato della misura; si considerino in
particolare le seguenti ipotesi (\emph{modello semplificato
  di Laplace\thinspace\footnote{Pierre Simon de Laplace
    visse in Francia dal 1749 al 1827; famoso matematico,
    fisico ed astronomo, prov\`o la stabilit\`a del sistema
    solare, svilupp\`o la teoria delle equazioni
    differenziali e dei potenziali, contribu\`\i\ allo
    studio del calore e dei fenomeni capillari oltre a
    gettare le basi matematiche per una teoria
    dell'elettromagnetismo.  Durante la rivoluzione francese
    fu uno degli ideatori del sistema metrico decimale; per
    quel che riguarda la statistica, nel 1812 pubblic\`o il
    trattato ``Th\'eorie Analytique des Probabilit\'es'' che
    contiene, tra l'altro, studi sulla distribuzione normale
    e la derivazione della regola dei minimi quadrati.}%
  \index{Laplace!Pierre Simon de|emidx}
  per gli errori di misura}):
\begin{enumerate}
\item \textit{Ognuna delle singole cause di disturbo
    presenti introdurr\`a nella misura una variazione
    rispetto al valore vero di modulo fisso $\epsilon$, con
    uguale probabilit\`a in difetto o in eccesso.}
\item \textit{Ognuna delle variazioni nella misura dovute a
    queste cause di disturbo \`e statisticamente
    indipendente dalle altre.}
\end{enumerate}

Ognuna delle $N$ cause indipendenti di disturbo produce
quindi la variazione $+\epsilon$ con probabilit\`a $p = 0.5$
oppure $-\epsilon$ con probabilit\`a $q = 1-p = 0.5$; se $M$
tra le $N$ perturbazioni sono positive (e le altre $N-M$
negative), il valore osservato sar\`a
\begin{equation*}
  x \; = \;
    x^* + M \epsilon - (N-M) \epsilon \; = \;
    x^* + (2M-N) \epsilon \peq .
\end{equation*}

La probabilit\`a di un dato valore di $M$ sulle $N$ prove
\`e data dalla distribuzione binomiale (vedi il paragrafo
\ref{ch:8.binom}, ed in particolare l'equazione
\eqref{eq:8.binom}), e vale
\begin{equation*}
  P(M,N) \; = \; \frac{N!}{M! \, (N-M)!} \,
    p^M q^{N-M} \peq .
\end{equation*}
Il valore medio di $M$ \`e dato da $Np$, e la sua varianza
da $Npq$; indichiamo poi con il simbolo $\lambda$ lo scarto
di $M$ dal suo valore medio
\begin{equation*}
  M \; = \; Np + \lambda \peq .
\end{equation*}

In corrispondenza al variare di $M$ tra 0 ed $N$, $\lambda$
varia tra i limiti $-Np$ e $+Nq$; risulta poi anche
\begin{gather*}
  N-M \; = \; N - Np - \lambda \; = \;
    Nq - \lambda \\
  \intertext{e la probabilit\`a di ottenere un
    certo valore di $\lambda$ su $N$ prove vale}
  P(\lambda,N) \; = \; \frac{N!}{(Np+\lambda)!
    \: (Nq-\lambda)!} \, p^{Np+\lambda} \,
    q^{Nq-\lambda} \peq . \\
  \intertext{Valore medio e varianza di $\lambda$
    valgono poi}
  E(\lambda) \; = \; E(M) - Np \; \equiv \; 0 \\
  \intertext{e}
  \var (\lambda) \; = \; \var (M) \; = \; Npq \peq .
\end{gather*}

L'andamento generale della probabilit\`a in funzione di $M$
si pu\`o trovare considerando il rapporto tra i valori di
$P$ che corrispondono a due valori successivi di $M $:
\begin{align*}
  \frac{P(M+1, N)}{P(M, N)} &=
    \frac{N! \, p^{M+1} \, q^{N-M-1}}{(M+1)!
    \, (N-M-1)!} \: \frac{M! \, (N-M)!}{N!
    \, p^M \, q^{N-M}} \\[1ex]
  &= \frac{N-M}{M+1} \, \frac{p}{q}
\end{align*}
e $P(M, N)$ risulter\`a minore, uguale o maggiore di $P(M+1,
N)$ a seconda che $(M+1)q$ risulti minore, uguale o maggiore
di $(N-M)p$; ossia, essendo $p+q=1$, a seconda che $M$ sia
minore, uguale o maggiore di $Np-q$.

Insomma, chiamato $\mu = \lceil Np-q \rceil$ il pi\`u
piccolo intero non minore di $Np-q$, la sequenza di valori
$P(0, N), P(1, N),\ldots, P(\mu, N)$ \`e crescente, mentre
quella dei valori $P(\mu+1, N), P(\mu+2, N),\ldots, P(N, N)$
\`e decrescente.  Il massimo valore della probabilit\`a si
ha in corrispondenza ad un intero $\mu$ che soddisfi la
\begin{equation*}
  Np-q \; \le \; \mu \; \le \; Np-q+1 \; = \; Np+p
\end{equation*}
e che \`e unico, salvo il caso che i due estremi
dell'intervallo siano entrambi numeri interi: in questo caso
si hanno due valori massimi, uguali, in corrispondenza di
entrambi.  Concludendo: il caso pi\`u probabile \`e che
l'evento $E$ si presenti in una sequenza di $N$ prove $Np$
volte, ed il valore di $\lambda$ con la massima
probabilit\`a di presentarsi \`e 0.

Cerchiamo ora di determinare se esiste e quanto vale il
limite della probabilit\`a di ottenere un certo risultato al
crescere indefinito del numero delle prove.  Per ottenere
questo, introduciamo la \emph{formula approssimata di de
  Moivre e Stirling}\thinspace\footnote{Per la
  dimostrazione, vedi ad esempio: G. Castelnuovo -- Calcolo
  delle probabilit\`a (Zanichelli), in appendice.  La
  formula \`e dovuta al solo Stirling,%
  \index{Stirling, James}
  che la pubblic\`o nel suo libro ``Methodus
  Differentialis'' del 1730; ma non divenne nota nel mondo
  scientifico fino a quando de Moivre%
  \index{de Moivre!Abraham}
  non la us\`o --- da qui il nome comunemente adottato.}
per il fattoriale:
\begin{gather*}%
  \index{de Moivre!e Stirling, formula di}
  N! \; = \; N^N e^{-N} \sqrt{2 \pi N} \, ( 1 +
    \epsilon_N ) \; \approx \; \sqrt{ 2 \pi} \,
    N^{\left( N+\frac{1}{2} \right)} e^{-N} \\
  \intertext{con}
  0 \; \le \; \epsilon_N \; < \; \frac{1}{11 \cdot N} \peq .
\end{gather*}

\`E lecito trascurare il resto $\epsilon_N$ quando
l'argomento del fattoriale \`e elevato: per $N=10$ l'errore
commesso \`e gi\`a inferiore all'1\%.  Per usare la formula
di de Moivre e Stirling nel nostro caso, sviluppiamo
\begin{align*}
  (Np+\lambda)! &\approx \sqrt{2 \pi}
    \, (Np+\lambda)^{\left( Np+\lambda + \frac{1}{2}
    \right)} e^{\left( -Np -\lambda \right)} \\[1ex]
  &= \sqrt{2 \pi}
    \left( 1 + \frac{\lambda}{Np} \right) ^{\left( Np
    +\lambda +\frac{1}{2} \right)} e^{\left( -Np -
    \lambda \right)} (Np)^{\left( Np +\lambda
    +\frac{1}{2} \right)}
\end{align*}
e, similmente,
\begin{equation*}
  (Nq-\lambda)! \approx \sqrt{2 \pi}
    \left( 1 - \frac{\lambda}{Nq} \right)
    ^{\left( Nq -\lambda +\frac{1}{2} \right)}
    e^{\left( -Nq +\lambda \right)}
    (Nq)^{\left( Nq -\lambda +\frac{1}{2} \right)} \peq .
\end{equation*}

Queste approssimazioni sono valide quando gli argomenti dei
fattoriali, $Np + \lambda$ e $Nq - \lambda$, sono abbastanza
grandi: cio\`e quando $\lambda$ non \`e vicino ai valori
limite $-Np$ e $Nq$; accettata la loro validit\`a (e
ritorneremo su questo punto tra poco), sostituendo si ha
\begin{equation*}
  P(\lambda, N) \; = \; \frac{1}{\sqrt{ 2 \pi N p q }}
    \left( 1 + \frac{\lambda}{Np} \right)
    ^{- \left( Np +\lambda +\frac{1}{2}
    \right)} \left( 1 - \frac{\lambda}{Nq} \right)
    ^{ - \left( Nq -\lambda +\frac{1}{2} \right)} \peq .
\end{equation*}

Questa espressione \`e certamente valida quando $|\lambda|$
non \`e troppo grande, e per $\lambda=0$ fornisce la
probabilit\`a del valore medio di $M$ ($M=Np$), che risulta
\begin{equation*}
  P(0, N) = \frac{1}{\sqrt{2 \pi N p q}} \peq .
\end{equation*}

Questa probabilit\`a tende a zero come $1/\sqrt{N}$ al
crescere di $N$; dato che la somma delle probabilit\`a
relative a tutti i casi possibili deve essere 1, si deve
concludere che il numero di valori di $\lambda$ per cui la
probabilit\`a non \`e trascurabile rispetto al suo massimo
deve divergere come $\sqrt{N}$ al crescere di $N$, sebbene
il numero di tutti i possibili valori (che \`e $N+1$)
diverga invece come $N$.

L'espressione approssimata di $P(\lambda, N)$ non \`e valida
per valori di $\lambda$ prossimi agli estremi $\lambda=-Np$
e $\lambda=Nq$ (\`e infatti divergente); tuttavia tali
valori hanno probabilit\`a infinitesime di presentarsi al
crescere di $N$.  Infatti $P(-Np, N) = q^N$ e $P(Nq, N) =
p^N$, ed entrambi tendono a zero quando $N$ tende
all'infinito essendo sia $p$ che $q$ inferiori all'unit\`a.

Concludendo: la formula approssimata da noi ricavata \`e
valida gi\`a per valori relativamente piccoli di $N$, e per
$N$ molto grande si pu\`o ritenere esatta per tutti i valori
dello scarto $\lambda$ con probabilit\`a non trascurabile di
presentarsi, valori che sono mediamente dell'ordine
dell'errore quadratico medio $\sqrt{Npq}$ e che quindi
divergono solo come $\sqrt{N}$.  Consideriamo ora il fattore
\begin{gather*}
  \kappa = \left( 1+ \frac{\lambda}{Np} \right) ^{- \left(
      Np +\lambda +\frac{1}{2} \right)} \left( 1
    -\frac{\lambda}{Nq} \right) ^{-
    \left( Nq -\lambda +\frac{1}{2} \right)} \\
  \intertext{che nell'espressione approssimata di
    $P(\lambda, N)$ moltiplica il valore massimo $P(0, N)$,
    e se ne prenda il logaritmo naturale:} \ln \kappa \; =
  \; - \left( Np +\lambda +\frac{1}{2} \right) \ln \left( 1
    +\frac{\lambda}{Np} \right) \; - \; \left( Nq -\lambda
    +\frac{1}{2} \right) \ln \left( 1 -\frac{\lambda}{Nq}
  \right) \peq .
\end{gather*}

Ora, poich\'e sia $\lambda / Np$ che $\lambda /Nq$ sono in
modulo minori dell'unit\`a (salvi i due casi estremi, di
probabilit\`a come sappiamo infinitesima), si possono
sviluppare i due logaritmi in serie di McLaurin:
\begin{equation*}
  \ln (1+x) \; = \; x -\frac{x^2}{2} + \frac{x^3}{3}
    - \frac{x^4}{4} + \cdots \peq .
\end{equation*}
Il primo termine di $\ln \kappa$ diventa
\begin{equation*}
  \begin{split}
    - \biggl( Np +\lambda &+ \frac{1}{2} \biggr)
      \left( \frac{\lambda}{Np} -
      \frac{\lambda^2}{2 \, N^2 p^2}
      +\frac{\lambda^3}{3 \, N^3 p^3} - \cdots
      \right) \; = \\[1ex]
    &= - \lambda + \left( \frac{\lambda^2}{2 \, Np}
      - \frac{\lambda^2}{Np} \right)
      - \left( \frac{\lambda^3}{3 \, N^2 p^2}
      - \frac{\lambda^3}{2 \, N^2 p^2}
      + \frac{\lambda}{2 \, Np} \right)
      + \cdots \\[1ex]
    &= - \, \lambda - \frac{\lambda^2}{2 \, Np} -
      \frac{\lambda}{2 \, Np} +
      \frac{\lambda^3}{6 \, N^2 p^2} + \cdots
  \end{split}
\end{equation*}
ed il secondo
\begin{multline*}
  - \left( Nq - \lambda + \frac{1}{2} \right)
    \left( - \frac{\lambda}{Nq} -
    \frac{\lambda^2}{2 \, N^2 q^2} -
    \frac{\lambda^3}{3 \, N^3 q^3} -
    \cdots \right) \; = \\[1ex]
  = \lambda - \frac{\lambda^2}{2 \, Nq} +
    \frac{\lambda}{2 \, Nq} -
    \frac{\lambda^3}{6 \, N^2 q^2} - \cdots
\end{multline*}
e sommando si ottiene
\begin{equation*}
  \ln \kappa = - \frac{\lambda^2}{2 \, Npq} -
    \frac{\lambda}{2 \, N} \left(
    \frac{1}{p} - \frac{1}{q} \right)
    + \frac{\lambda^3}{6 \, N^2} \left(
    \frac{1}{p^2} - \frac{1}{q^2} \right)
    + \cdots \peq .
\end{equation*}

Da questo sviluppo risulta che il solo termine che si
mantiene finito al divergere di $N$, e per valori di
$\lambda$ dell'ordine di $\sqrt{Npq}$, \`e il primo; gli
altri due scritti convergono a zero come $1/\sqrt{N}$, e
tutti gli altri omessi almeno come $1/N$.  In conclusione,
per valori dello scarto per cui la probabilit\`a non \`e
trascurabile (grosso modo $ |\lambda| < 3 \sqrt{Npq} $), al
divergere di $N$ il logaritmo di $\kappa$ \`e bene
approssimato da
\begin{gather*}
  \ln \kappa \; \approx \; - \,
    \frac{\lambda^2}{2 \, Npq} \\
  \intertext{e la probabilit\`a dello scarto dalla
    media $\lambda$ da}
  P (\lambda) \; \approx \; \frac{1}{\sqrt{ 2 \pi
    N p q}} \, e^{- \frac{1}{2}
    \frac{\lambda^2}{Npq} } \peq ; \\
  \intertext{per la variabile $M$ sar\`a invece}
  P(M) \; \approx  \;\frac{1}{\sqrt{2 \pi N p q}} \,
    e^{- \frac{1}{2} \frac{(M-Np)^2}{Npq} } \peq .
\end{gather*}

Nel caso particolare del modello semplificato di Laplace per
gli errori di misura, $p=q=0.5$ e pertanto i termini di
ordine $1/\sqrt{N}$ sono identicamente nulli:
l'approssimazione \`e gi\`a buona per $N \ge 25$; nel caso
generale $p \neq q$, essa \`e invece accettabile per $Npq
\ge 9$.  Introducendo lo scarto quadratico medio di $M$ e di
$\lambda$
\begin{gather*}
  \sigma = \sqrt{Npq} \\
  \intertext{l'espressione si pu\`o scrivere}
  P(\lambda) \; \approx \; \frac{1}{\sigma
    \sqrt{2 \pi}} \, e^{- \frac{\lambda^2}{2 \sigma^2} }
\end{gather*}
che \`e la celebre \emph{legge normale} o \emph{legge di
  Gauss}.

Tornando ancora al modello semplificato di Laplace per gli
errori di misura, il risultato $x$ ha uno scarto dal valore
vero che vale
\begin{gather*}
  x - x^* \; = \;
    \epsilon \, (2M-N) \; = \;
    \epsilon \, (2Np +2\lambda -N) \; = \;
    2 \epsilon \lambda \\
  \intertext{e possiede varianza}
  {\sigma_x}^2 \; \equiv \;
    \var \left( x - x^* \right) \; = \;
    4 \, \epsilon^2 \, \var (\lambda) \; = \;
    4 \epsilon^2 \sigma^2 \peq . \\
  \intertext{La probabilit\`a di un certo
    risultato $x = x^* + 2 \epsilon \lambda $
    vale infine}
  P(x) \; = \; P(\lambda) \; \approx \;
    \frac{1}{\sigma \sqrt{2 \pi} } \,
     e^{- \frac{1}{2}
    \frac{\lambda^2}{\sigma^2}} \; = \;
    \frac{2 \epsilon}{\sigma_x \sqrt{2 \pi}}
    \, e^{- \frac{1}{2} \bigl(
    \frac{x - x^*}{\sigma_x} \bigr) ^2 } \peq .
\end{gather*}%
\index{Laplace!modello di|)}

La $x$ \`e una grandezza discreta che varia per multipli di
$\epsilon$; nel limite su accennato diventa una variabile
continua, e $P(x)$ \`e infinitesima con $\epsilon$ perdendo
cos\`\i\ significato; si mantiene invece finita la densit\`a
di probabilit\`a, che si ottiene dividendo $P(x)$ per
l'ampiezza $2 \epsilon$ dell'intervallo che separa due
valori contigui di $x$:
\begin{equation*}
  f(x) \; = \;
    \frac{P(x)}{2 \epsilon} \; = \;
    \frac{1}{\sigma_x \sqrt{2 \pi}} \,
    e^{- \frac{1}{2} \bigl(
    \frac{x-x^*}{\sigma_x} \bigr) ^2 }
\end{equation*}
ed ha infatti le dimensioni fisiche di $1/\sigma_x$, ovvero
di $1/x$.

Al medesimo risultato per $f(x)$ si perverrebbe anche
nell'ipotesi pi\`u generale che gli errori elementari siano
distribuiti comunque, ed anche diversamente l'uno
dall'altro, purch\'e ciascuno abbia una varianza dello
stesso ordine di grandezza degli altri ed infinitesima al
divergere del numero delle cause di errore.%
\index{distribuzione!normale|)}

\endinput
