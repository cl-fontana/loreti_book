% $Id: prefazione.tex,v 1.2 2005/04/13 08:28:24 loreti Exp $

\chapter*{Prefazione}
\addcontentsline{toc}{chapter}{\numberline{}Prefazione}
\chaptermark{Prefazione}

{\itshape Quando ho cominciato a tenere uno dei due corsi di
  Esperimentazioni di Fisica I (per il primo anno del Corso
  di Laurea in Fisica), ormai molti anni fa, non sono
  riuscito a trovare un libro di testo in cui fosse
  contenuto, della materia, tutto quello che io ritenevo
  fosse necessario per la formazione di uno studente che si
  supponeva destinato alla carriera del fisico; e, dopo aver
  usato per qualche tempo varie dispense manoscritte, mi
  sono deciso a riunirle in un testo completo in cui la
  materia fosse organicamente esposta.

  Per giungere a questo \`e stato fondamentale l'aiuto
  datomi dal docente dell'altro corso parallelo di
  Esperimentazioni, il Prof.\ Sergio Ciampolillo, senza il
  quale questo libro non sarebbe probabilmente mai venuto
  alla luce; e tanto pi\`u caro mi \`e stato questo suo
  aiuto in quanto lui stesso mi aveva insegnato nel passato
  la statistica, completa di tutti i crismi del rigore
  matematico ma esposta con la mentalit\`a di un fisico e
  mirata ai problemi dei fisici, nei lontani anni in cui
  frequentavo la Scuola di Perfezionamento (a quel tempo il
  Dottorato di Ricerca non era ancora nato).

  Assieme abbiamo deciso l'impostazione da dare al testo e,
  nel 1987, preparata la prima edizione; che era
  ciclostilata, e che veniva stampata e distribuita a prezzo
  di costo agli studenti a cura dello stesso Dipartimento di
  Fisica.  Il contenuto \`e stato poi pi\`u volte ampliato e
  rimaneggiato da me (e, all'inizio, ancora dal Prof.\
  Ciampolillo: che ha partecipato a tutte le successive
  edizioni fino alla quarta compresa); dalla seconda alla
  quarta edizione, poi, il testo \`e stato edito a cura
  della Libreria Progetto.

  L'esposizione della materia \`e vincolata dalla struttura
  del corso: un testo organico dovrebbe ad esempio
  presentare dapprima la probabilit\`a e poi la statistica;
  ma gli studenti entrano in laboratorio sin dal primo
  giorno, e fanno delle misure che devono pur sapere come
  organizzare e come trattare per estrarne delle
  informazioni significative.  Cos\`\i\ si \`e preferito
  parlare subito degli errori di misura e
  dell'organizzazione dei dati, per poi dimostrare soltanto
  alla fine (quando tutta la matematica necessaria \`e stata
  alfine esposta) alcune delle assunzioni fatte; si veda a
  tale proposito l'esempio della media aritmetica, che gli
  studenti adoperano fin dal primo giorno ma il cui uso
  viene del tutto giustificato soltanto nel paragrafo
  \ref{ch:11.mepeted} di questo libro.

  Questo testo non contiene soltanto materia oggetto di
  studio nel primo anno del Corso di Laurea: su richiesta di
  docenti degli anni successivi, nel passato erano state
  aggiunte alcune parti (collocate tra le appendici) che
  potessero loro servire come riferimento.  Ho poi
  largamente approfittato, sia dell'occasione offertami
  dall'uscita di questa quinta edizione che del fatto di
  dover tenere anche un corso di Statistica per la Scuola di
  Dottorato in Fisica, per includere nel testo degli
  argomenti di teoria assiomatica della probabilit\`a e di
  statistica teorica che vanno assai pi\`u avanti delle
  esigenze di uno studente del primo anno: questo perch\'e
  se, negli anni successivi, le necessit\`a dell'analisi
  spingeranno dei fisici ad approfondire dei particolari
  argomenti di statistica, queste aggiunte sono sicuramente
  le basi da cui partire.

  Ho cercato insomma di trasformare un testo ad uso
  specifico del corso di Esperimentazioni di Fisica I in una
  specie di riferimento base per la statistica e l'analisi
  dei dati: per questo anche il titolo \`e cambiato, e
  ``Introduzione alle Esperimentazioni di Fisica I'' \`e
  diventato un pi\`u ambizioso ``Introduzione alla Fisica
  Sperimentale''; e ``Teoria degli errori e analisi dei
  dati'' un pi\`u veritiero ``Teoria degli Errori e
  Fondamenti di Statistica''.  Ma, anche se le nuove
  aggiunte (addirittura per un raddoppio complessivo del
  contenuto originale) sono mescolate alle parti che
  utilizzo nel corso, ho cercato di far s\`\i\ che queste
  ultime possano essere svolte indipendentemente dalla
  conoscenza delle prime.

  Pu\`o stupire invece che manchi una parte di descrizione e
  discussione organica delle esperienze svolte e degli
  strumenti usati: ma gli studenti, come parte integrante
  del corso, devono stendere delle relazioni scritte sul
  loro operato che contengano appunto questi argomenti; e si
  \`e preferito evitare che trovassero gi\`a pronto un
  riferimento a cui potessero, per cos\`\i\ dire, ispirarsi.

  \vfill\begin{flushright}
    \mbox{
      \begin{tabular}{c}
        Maurizio Loreti \\[3mm]
        Gennaio 1998 \\
        (Quinta edizione)
      \end{tabular}
      }\quad
  \end{flushright}
}

\chapter*{Prefazione alla sesta edizione}
\addcontentsline{toc}{chapter}{\numberline{}Prefazione alla
  sesta edizione}
\chaptermark{Prefazione alla sesta edizione}

{\itshape Gli anni sono passati, e non ho mai smesso di
  modificare questo testo: per prima cosa ho corretto gli
  errori segnalati dai colleghi e dagli studenti, ed ho
  sfruttato parecchi dei loro consigli.  Poi ho aggiunto
  nuovi paragrafi: sia per descrivere distribuzioni teoriche
  che, sia pur poco usate nella Fisica, ogni tanto vi
  compaiono; sia per introdurre piccoli esempi ed
  applicazioni giocattolo usati nel tenere per la seconda
  volta un corso di Statistica alla Scuola di Dottorato in
  Fisica.

  Come conseguenza, questo testo \`e a parer mio parecchio
  migliorato rispetto a quello pubblicato nel 1998 dalla
  Decibel-Zanichelli; purtroppo le scarse vendite hanno
  indotto la casa editrice a rinunciare ad una seconda
  edizione che seguisse la prima.  Conseguentemente ho
  deciso di mettere nella sua forma attuale (la ``sesta
  edizione'') questo libro a disposizione della comunit\`a
  su Internet: sperando che possa ancora servire a qualcuno.
  La licenza \`e quella GPL, inclusa nella appendice
  \ref{ch:licgpl}; in sostanza \`e permesso modificare a
  piacimento e ridistribuire in qualunque forma questo
  libro, purch\'e la ridistribuzione comprenda il sorgente.

  Un'ultima considerazione personale: adesso il Corso di
  Laurea in Fisica \`e articolato in ``tre pi\`u due'' anni
  di studio; ed \`e con un certo piacere che ho visto come
  tutti gli studenti che hanno preparato con noi, sui dati
  dell'esperimento CDF, le loro tesi di laurea di primo
  livello abbiano potuto trovare aiuto in queste pagine per
  completarle, usando ad esempio sia metodi di verifica
  delle ipotesi basati sul rapporto delle funzioni di
  verosimiglianza che il test di Kolmogorov e Smirnov.

  \vfill\begin{flushright}
    \mbox{
      \begin{tabular}{c}
        Maurizio Loreti \\[3mm]
        \thismonth\ \number\year \\
        (Sesta edizione)
      \end{tabular}
    }\quad
  \end{flushright}
}

\endinput
