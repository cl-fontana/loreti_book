% $Id: chapter12.tex,v 1.3 2006/05/16 11:10:55 loreti Exp $

\chapter{La verifica delle ipotesi (I)}%
\label{ch:12.veripo}
Una volta eseguita una misura, si pu\`o voler controllare se
i nostri risultati possono confermare o rigettare una
determinata ipotesi riguardante il fenomeno fisico che li ha
prodotti; naturalmente, visto che risultati di una misura
comunque lontani dal valore vero sono sempre possibili
(anche se con probabilit\`a sempre pi\`u piccole al crescere
dello scarto), una qualunque ipotesi sulla grandezza fisica
misurata potr\`a essere confermata o rigettata dai dati solo
ad un certo livello di probabilit\`a.

Qui ci occuperemo inoltre di alcune funzioni di frequenza
collegate a quella di Gauss, ossia della distribuzione del
$\chi^2$, di quella di Student\/\footnote{``Student'' \`e lo
  pseudonimo con cui vennero pubblicati i lavori statistici
  di William Gosset, scienziato inglese vissuto dal 1876 al
  1937.  Uno dei pionieri di questo ramo della matematica,
  svolse le sue ricerche essendo dipendente (prima come
  chimico, poi come dirigente) della Guinness Brewery di
  Dublino.}%
\index{Gosset, William (``Student'')}%
\index{Student|see{Gosset, William}}
e di quella di Fisher; e dell'uso che di esse si pu\`o fare
per la verifica di ipotesi statistiche: quali ad esempio
quella che un campione di dati sperimentali provenga da una
popolazione descritta da una densit\`a di probabilit\`a nota
a priori; o quella che il valore vero della grandezza
misurata coincida con un valore determinato, noto anch'esso
a priori.

\section[La distribuzione del $\chi^2$]
{La distribuzione del
  $\boldsymbol{\chi}^{\boldsymbol{2}}$}%
\index{distribuzione!del $\chi^2$|(}
Se le $N$ variabili casuali $x_i$, tra loro statisticamente
indipendenti, sono variabili normali standardizzate
(ovverosia distribuite secondo la legge normale con media 0
e varianza 1), si pu\`o dimostrare che la nuova variabile
casuale
\begin{gather}
  X = \sum_{i=1}^N {x_i}^2 \notag \\
  \intertext{(ovviamente non negativa) \`e
    distribuita con una densit\`a di probabilit\`a
    data dalla}
  \frac{\de p}{\de X} \; = \; f(X ; N) \; = \;
    K_N \, X^{\left(\frac{N}{2}-1 \right)}
    e^{-\frac{X}{2}} \label{eq:12.denchi}
\end{gather}
(\emph{distribuzione del chi quadro}); la costante $K_N$
viene fissata dalla condizione di normalizzazione, ed il
parametro $N$ prende il nome di \emph{numero di gradi di
  libert\`a} della distribuzione.

\begin{figure}[hbtp]
  \vspace*{2ex}
  \begin{center} {
    \input{chi.pstex_t}
  } \end{center}
  \caption[La distribuzione del $\chi^2$]
    {La distribuzione del $\chi^2$ per alcuni
    valori del parametro $N$.}
\end{figure}

La funzione caratteristica della $X$ si pu\`o trovare
facilmente considerando che, se la $x$ \`e una variabile
normale standardizzata, il suo quadrato $y = x^2$ ha una
funzione caratteristica
\begin{align*}
  \phi_y(t) &= E \bigl( e^{ity} \bigr) \\[1ex]
  &= E \left( e^{itx^2} \right) \\[1ex]
  &= \int_{-\infty}^{+\infty} \! e^{itx^2}
    \frac{1}{\sqrt{2\pi}} \, e^{- \frac{x^2}{2}} \, \de
    x \\[1ex]
  &= \int_{-\infty}^{+\infty} \frac{1}{\sqrt{2\pi}}
    \, e^{- \frac{x^2}{2} (1-2it)} \, \de x \\[1ex]
  &= \frac{1}{\sqrt{1-2it}} \int_{-\infty}^{+\infty}
    \frac{1}{\sqrt{2\pi}} \, e^{- \frac{u^2}{2}} \, \de
    u \\[1ex]
  &= (1-2it)^{- \frac{1}{2}}
\end{align*}
(si \`e eseguita la sostituzione di variabile $u = x
\sqrt{1-2it}$; l'integrale definito \`e quello di una
distribuzione normale $N(u; 0,1)$ e vale dunque 1).  Di
conseguenza, applicando l'equazione \eqref{eq:6.fucacl}, la
funzione caratteristica della $X$ vale
\begin{equation} \label{eq:12.fucachi}
  \phi_X(t) = (1 - 2it)^{- \frac{N}{2}} \peq .
\end{equation}

Per dimostrare che la funzione di frequenza della $X$ \`e
effettivamente la \eqref{eq:12.denchi}, si parte poi
dall'espressione \eqref{eq:12.fucachi} della funzione
caratteristica e le si applica la trasformazione inversa di
Fourier%
\index{Fourier, trasformata di}
gi\`a definita nella \eqref{eq:6.trinfo}.

Con simili passaggi si potrebbe ricavare la funzione
generatrice dei momenti, che vale
\begin{equation*}
  M_X(t) = (1 - 2t)^{- \frac{N}{2}}
\end{equation*}
e, da queste, si ottiene infine che il valore medio e la
varianza di una variabile casuale distribuita come il
$\chi^2$ a $N$ gradi di libert\`a sono
\begin{gather*}
  E(X) \; = \; N \makebox[40mm]{\mbox{e}} \var(X) \; =
  \; 2N \\
  \intertext{mentre i coefficienti di asimmetria e di
    curtosi valgono}
  \gamma_1 \; = \; 2 \sqrt{\frac{2}{N}}
  \makebox[40mm]{\mbox{e}} \gamma_2 \; = \; \frac{12}{N} \peq
  .
\end{gather*}

\index{distribuzione!del $\chi^2$!e distribuzione normale|(}%
La distribuzione del $\chi^2$ tende asintoticamente ad una
distribuzione normale con la stessa media $N$ e la stessa
varianza $2N$; infatti la funzione caratteristica della
variabile standardizzata
\begin{gather*}
  y \; = \; \frac{X - N}{\sqrt{2N}} \; = \;
    \frac{X}{\sqrt{2N}} - \frac{N}{\sqrt{2N}} \\
  \intertext{vale, ricordando la \eqref{eq:6.fuccav},}
  \phi_y(t) = e^{- \frac{i N t}{\sqrt{2N}}} \left[ 1 -
    \frac{2 i t}{\sqrt{2N}} \right]^{- \frac{N}{2}} \peq . \\
  \intertext{Passando ai logaritmi naturali,}
  \ln \phi_y(t) = - \, \frac{i N t}{\sqrt{2N}} -
    \frac{N}{2} \ln \left( 1 - \frac{2 i t}{\sqrt{2N}}
    \right)
\end{gather*}
e, sviluppando in serie di McLaurin il logaritmo,
\begin{align*}
  \ln \phi_y(t) &= - \, \frac{i N t}{\sqrt{2N}} -
    \frac{N}{2} \left[ - \, \frac{2 i t}{\sqrt{2N}} -
    \frac{1}{2} \left( \frac{2 i t}{\sqrt{2N}}
    \right)^2 + \mathcal{O} \left( N^{-\frac{3}{2}} \right)
  \right] \\[1ex]
  &= - \, \frac{t^2}{2} + \mathcal{O} \left(
    N^{-\frac{1}{2}} \right)
\end{align*}
da cui
\begin{equation*}
  \lim_{N \to \infty} \phi_y(t) = e^{- \frac{t^2}{2}}
\end{equation*}
che \`e appunto la funzione caratteristica di una
distribuzione normale standardizzata.

In definitiva:
\begin{itemize}
\item Quando $N$ assume valori sufficientemente grandi, la
  distribuzione del $\chi^2$ \`e ben approssimata da una
  distribuzione normale avente la stessa media $N$ e la
  stessa varianza $2N$; tale approssimazione si pu\`o
  ritenere in pratica gi\`a buona quando $N$ \`e superiore a
  30.
\item Inoltre si potrebbe analogamente dimostrare che la
  variabile casuale $\sqrt{2 X}$, anche per valori
  relativamente piccoli di $N$, ha una distribuzione che \`e
  assai bene approssimata da una funzione normale con media
  $\sqrt{2N-1}$ e varianza 1; l'approssimazione \`e gi\`a
  buona per $N\gtrsim 8$.
\end{itemize}%
\index{distribuzione!del $\chi^2$!e distribuzione normale|)}

Dalla definizione (o dalla funzione caratteristica
\eqref{eq:12.fucachi}) discende immediatamente la cosiddetta
\emph{regola di somma del $\chi^2$}\label{th:12.resochi}:%
\index{distribuzione!del $\chi^2$!regola di somma} ossia
che, se $X$ ed $Y$ sono due variabili casuali
statisticamente indipendenti entrambe distribuite come il
$\chi^2$, con $N$ ed $M$ gradi di libert\`a rispettivamente,
la loro somma $Z=X+Y$ \`e una variabile casuale ancora
distribuita come il $\chi^2$; per\`o con $N+M$ gradi di
libert\`a.

Ovviamente, se le $x_i$ (con $i=1,\ldots,N$) sono $N$
variabili casuali statisticamente indipendenti tra loro e
provenienti da una stessa distribuzione normale con media
$\mu$ e varianza $\sigma^2$, discende da quanto detto che la
nuova variabile casuale
\begin{equation*}
  X' = \sum_{i=1}^N \left( \frac{x_i - \mu}{\sigma}
   \right)^2
\end{equation*}
\`e distribuita come il $\chi^2$ a $N$ gradi di libert\`a.
Indichiamo ora, al solito, con $\bar x$ la media aritmetica
delle $x_i$: vogliamo dimostrare che la variabile casuale
\begin{equation*}
  X'' = \sum_{i=1}^N \left( \frac{x_i - \bar x}{\sigma}
  \right)^2
\end{equation*}
\`e distribuita \emph{ancora come il $\chi^2$, ma con $N -
  1$ gradi di libert\`a}.

A questo scopo facciamo dapprima alcune considerazioni,
indipendenti dalle ipotesi prima fatte sulle $x_i$ e che
risultano quindi valide per variabili casuali qualunque:
supponiamo di definire $N$ nuove variabili $y_i$ come
generiche combinazioni lineari delle $x_j$, con coefficienti
che indicheremo col simbolo $A_{ij}$; in modo insomma che
risulti
\begin{equation*}
  y_i = \sum_{j=1}^N A_{ij} \, x_j \peq .
\end{equation*}

La somma dei quadrati delle $y_i$ \`e data da
\begin{equation*}
  \sum_{i=1}^N {y_i}^2 = \sum_{i=1}^N \left(
  \sum_{j=1}^N A_{ij} \, x_j \right) \left( \sum_{k=1}^N
  A_{ik} \, x_k \right) = \sum\nolimits_{jk} x_j \, x_k
  \sum\nolimits_i A_{ij} \, A_{ik} \peq ;
\end{equation*}
\`e possibile che questa somma risulti uguale alla somma dei
quadrati delle $x_i$ \emph{qualunque} sia il valore di
queste ultime?  Ovviamente questo avviene se e solo se vale
la
\begin{equation} \label{eq:12.conort}
  \sum\nolimits_i A_{ij} \, A_{ik} \; = \; \delta_{jk} \;
  = \;
  \begin{cases}
    0 & \text{per $j \ne k$} \\[2ex]
    1 & \text{per $j = k$}
  \end{cases}
\end{equation}
(il simbolo $\delta_{jk}$, che assume il valore 1 quando gli
indici sono uguali e 0 quando sono invece diversi, si chiama
\emph{simbolo di Kronecker} o \emph{delta di Kronecker}).%
\index{Kronecker, delta di}

Consideriamo gli $A_{ij}$ come gli elementi di una matrice
quadrata $\boldsymbol{A}$ di ordine $N$; gli $x_j$ e le
$y_i$ si possono invece considerare come le componenti di
due \emph{vettori} $\boldsymbol{X}$ ed $\boldsymbol{Y}$
definiti in uno spazio $N$-dimensionale --- ossia come gli
elementi di due matrici rettangolari con $N$ righe ed 1
colonna.

La trasformazione che muta $\boldsymbol{X}$ in
$\boldsymbol{Y}$ si pu\`o scrivere, in forma matriciale,
come $\boldsymbol{Y} = \boldsymbol{A} \boldsymbol{X}$; la
somma dei quadrati delle $x_j$ o delle $y_i$ altro non \`e
se non il prodotto scalare, di $\boldsymbol{X}$ ed
$\boldsymbol{Y}$ rispettivamente, per loro stessi: ovverosia
la loro \emph{norma}, il quadrato della loro lunghezza nello
spazio a $N$ dimensioni.  Quella che abbiamo ricavato adesso
\`e la condizione perch\'e una \emph{trasformazione lineare}
applicata ad un vettore ne conservi la lunghezza: occorre e
basta che la matrice $\boldsymbol{A}$ sia \emph{ortogonale}.
Infatti la \eqref{eq:12.conort} si pu\`o scrivere
\begin{align*}
  \boldsymbol{\widetilde A} \boldsymbol{A} &=
  \boldsymbol{1} && \text{ossia}
  & \boldsymbol{\widetilde A} &= \boldsymbol{A}^{-1}
\end{align*}
($\boldsymbol{\widetilde A}$ \`e la matrice trasposta di
$\boldsymbol{A}$, di elementi $\boldsymbol{\widetilde
  A}_{ij} = A_{ji}$; $\boldsymbol{1}$ \`e la matrice
unit\`a, di elementi $\boldsymbol{1}_{ij} = \delta_{ij}$;
$\boldsymbol{A}^{-1}$ \`e la matrice inversa di
$\boldsymbol{A}$; ed una matrice per cui
$\boldsymbol{\widetilde A} = \boldsymbol{A}^{-1}$ si dice,
appunto, ortogonale).

Consideriamo adesso una trasformazione lineare definita
dalle seguenti relazioni:
\begin{equation} \label{eq:12.hack}
  \begin{cases}
    y_1 = \displaystyle \frac{1}{\sqrt{N}} \, (x_1 +
      x_2 +\cdots+ x_N) \\[2ex]
    y_2 = \displaystyle \frac{1}{\sqrt{2}} \, (x_1 -
      x_2) \\[2ex]
    y_3 = \displaystyle \frac{1}{\sqrt{6}} \, (x_1 +
      x_2 - 2 x_3) \\[2ex]
    \cdots \\[1ex]
    y_N = \displaystyle \frac{1}{\sqrt{N(N-1)}} \,
      \bigl[ x_1 + x_2 +\cdots+ x_{N-1} - (N-1) x_N
      \bigr]
  \end{cases}
\end{equation}
e per la quale la matrice di trasformazione abbia, insomma,
elementi $A_{ij}$ definiti come
\begin{equation*}
  A_{ij} \; \equiv \;
  \begin{cases}
    \text{$i=1$:} &  \displaystyle \frac{1}{\sqrt{N}}
      \\[4ex]
    \text{$i>1$:} &
    \begin{cases}
      \text{$j<i$:} &  \displaystyle
        \frac{1}{\sqrt{i(i-1)}} \\[2ex]
      \text{$j=i$:} &  \displaystyle - \,
        \frac{i-1}{\sqrt{i(i-1)}} \\[2ex]
      \text{$j>i$:} &  \displaystyle 0
    \end{cases}
  \end{cases}
\end{equation*}
Non \`e difficile controllare che la matrice
$\boldsymbol{A}$ \`e ortogonale; inoltre la prima riga \`e
stata scelta in modo tale che
\begin{gather}
  y_1 \; = \; \sum_{i=1}^N \frac{1}{\sqrt{N}} \, x_i \;
    = \; \frac{1}{\sqrt{N}} \cdot N \bar x \; = \;
    \sqrt{N} \, \bar x \notag \\
  \intertext{e quindi}
  \sum_{i=1}^N {x_i}^2 \; = \; \sum_{i=1}^N {y_i}^2 \;
    = \; N {\bar x}^2 + \sum_{i=2}^N {y_i}^2 \peq . \notag \\
  \intertext{Inoltre risulta (per $i > 1$)}
  \sum_{j=1}^N A_{ij} \; = \; \sum_{j=1}^{i-1}
    \frac{1}{\sqrt{i(i-1)}} \, - \,
    \frac{i-1}{\sqrt{i(i-1)}} \; = \; 0
    \label{eq:12.hackmean} \\
  \intertext{e, per ogni $i$,}
  \sum_{j=1}^N {A_{ij}}^2 \; = \; \left( \boldsymbol{A}
    \boldsymbol{\widetilde A} \right)_{ii} \; = \;
    \delta_{ii} \; = \; 1 \peq . \label{eq:12.hackstd}
\end{gather}

Tornando al nostro problema, supponiamo ora che tutte le
$x_j$ siano variabili aventi distribuzione normale; che
abbiano tutte valore medio $\mu$ e varianza $\sigma^2$; ed
inoltre che siano tra loro tutte statisticamente
indipendenti.  Una qualsiasi loro combinazione lineare,
quindi anche ognuna delle $y_i$ legate alle $x_j$ da quella
particolare matrice di trasformazione \eqref{eq:12.hack} che
abbiamo prima definita, \`e anch'essa distribuita secondo la
legge normale; inoltre risulta

\begin{align*}
  \frac{1}{\sigma^2} \sum_{i=1}^N \left( x_i - \bar x
    \right)^2 &= \frac{1}{\sigma^2} \left(
    \sum_{i=1}^N {x_i}^2 - N {\bar x}^2 \right) \\[1ex]
  &= \frac{1}{\sigma^2} \left( N {\bar x}^2 +
    \sum_{i=2}^N {y_i}^2 - N {\bar x}^2  \right)
    \\[1ex]
  &= \sum_{i=2}^N \frac{{y_i}^2}{\sigma^2} \peq .
\end{align*}

Applicando alle $y_i = \sum_j A_{ij} x_j$ le formule per la
media e la varianza delle combinazioni lineari di variabili
casuali statisticamente indipendenti gi\`a ricavate nel
capitolo \ref{ch:5.varcun}, si trova facilmente (tenendo
presenti la \eqref{eq:12.hackmean} e la
\eqref{eq:12.hackstd}) che la varianza di ognuna di esse \`e
ancora $\sigma^2$; e che, per $i \ne 1$, il loro valore
medio \`e 0.  Di conseguenza, per $i \geq 2$ le $y_i /
\sigma$ sono variabili casuali normali aventi media 0 e
varianza 1: e questo implica che
\begin{equation} \label{eq:12.xii}
  X'' = \sum_{i=1}^N \left( \frac{x_i - \bar x}{\sigma}
    \right)^2
\end{equation}
sia effettivamente distribuita come il $\chi^2$ a
$N - 1$ gradi di libert\`a.

\`E interessante confrontare questo risultato con quello
precedentemente ricavato, e riguardante la stessa
espressione --- in cui per\`o gli scarti erano calcolati
rispetto alla media della popolazione $\mu$.  Nel primo caso
la distribuzione era ancora quella del $\chi^2$, ma con $N$
gradi di libert\`a: riferendoci invece alla media aritmetica
del campione, i gradi di libert\`a diminuiscono di una
unit\`a.  Questo \`e conseguenza di una legge generale,
secondo la quale il numero di gradi di libert\`a da
associare a variabili che seguono la distribuzione del
$\chi^2$ \`e dato dal numero di contributi
\emph{indipendenti}: ovvero il numero di termini con
distribuzione normale sommati in quadratura (qui $N$, uno
per ogni determinazione $x_i$) diminuito del numero di
parametri che compaiono nella formula e che sono stati
stimati dai dati stessi (qui uno: appunto la media della
popolazione, stimata usando la media aritmetica delle
misure).

Un'ultima notevole conseguenza del fatto che la variabile
casuale $X''$ definita dalla \eqref{eq:12.xii} sia
distribuita come il $\chi^2$ a $N - 1$ gradi di libert\`a
\`e la seguente: la stima della varianza della popolazione
ottenuta dal campione, $s^2$, vale
\begin{equation} \label{eq:12.xiis2}
  s^2 = X'' \, \frac{\sigma^2}{N-1}
\end{equation}
e, essendo proporzionale a $X''$, \`e anch'essa distribuita
come il $\chi^2$ a $N - 1$ gradi di libert\`a; quindi la sua
densit\`a di probabilit\`a \`e data dalla
\eqref{eq:12.denchi} e dipende solamente da $N$; non
dipende, in particolare, dalla media del campione $\bar x$.
Quindi:
\begin{quote}
  \index{media!aritmetica!e varianza|emidx}%
  \index{varianza!e media aritmetica|emidx}%
  \label{th:12.inmest}
  \textit{Il valore medio $\bar x$ e la varianza campionaria
    $s^2$, calcolati su valori estratti a caso da una stessa
    popolazione normale, sono due variabili casuali
    \textbf{statisticamente indipendenti} tra loro.}
\end{quote}%
\index{distribuzione!del $\chi^2$|)}

Questo risulta anche intuitivamente comprensibile; se
infatti ci \`e noto che un certo campione di dati ha una
dispersione pi\`u o meno grande, questo non deve alterare la
probabilit\`a che il suo valore medio abbia un valore
piuttosto che un altro; n\'e, viceversa, il fatto che il
campione sia centrato attorno ad un certo valore deve
permetterci di prevedere in qualche modo la sua dispersione.

\section[Verifiche basate sulla distribuzione del $\chi^2$]
{Verifiche basate sulla distribuzione del
  $\boldsymbol{\chi}^{\boldsymbol{2}}$}
\subsection{Compatibilit\`a dei dati con una distribuzione}%
\index{compatibilit\`a!con una distribuzione|(}%
\label{ch:12.comdadis}
Supponiamo di avere dei dati raccolti in un istogramma, e di
voler verificare l'ipotesi che i dati provengano da una
certa distribuzione; ad esempio, dalla distribuzione
normale.  Ora, per una misura, la probabilit\`a $p_i$ di
cadere nell'intervallo $i$-esimo (di ampiezza prefissata
$\Delta x$ e corrispondente alla generica classe di
frequenza usata per la realizzazione dell'istogramma) \`e
data dal valore medio della funzione densit\`a di
probabilit\`a nell'intervallo stesso moltiplicato per
$\Delta x$.

Il numero di misure effettivamente ottenute in una classe di
frequenza su $N$ prove deve obbedire poi alla distribuzione
binomiale: il loro valore medio \`e quindi $N p_i$, e la
loro varianza $N \, p_i \, (1 - p_i)$; quest'ultimo termine
si pu\`o approssimare ancora con $N p_i$ se si ammette che
le classi di frequenza%
\index{classi di frequenza|(}
siano sufficientemente ristrette da poter trascurare i
termini in ${p_i}^2$ rispetto a quelli in $p_i$ (cio\`e se
$p_i \ll 1$).

In questo caso il numero di misure in ciascuna classe segue
approssimativamente la distribuzione di Poisson; questa \`e
infatti la funzione di frequenza che governa il presentarsi,
su un grande numero di osservazioni, di eventi aventi
probabilit\`a trascurabile di verificarsi singolarmente in
ognuna: distribuzione nella quale l'errore quadratico medio
\`e effettivamente dato dalla radice quadrata del valore
medio, $\sigma = \sqrt{N \, p_i \, (1 - p_i)} \simeq \sqrt{N
  p_i}$.

Nei limiti in cui il numero di misure attese in una classe
\`e sufficientemente elevato da poter confondere la relativa
funzione di distribuzione con la funzione normale, la
quantit\`a
\begin{equation} \label{eq:12.chi2fit}
  X \; = \;\sum_{i=1}^M
    \frac{(n_i - N p_i)^2}{N p_i} \; = \; \sum_{i=1}^M
    \frac{(O_i - A_i)^2}{A_i}
\end{equation}
cio\`e la somma, su tutte le classi di frequenza (il cui
numero abbiamo supposto sia $M$), del quadrato della
differenza tra il numero di misure ivi \emph{attese} ($A_i =
N p_i$) ed ivi \emph{effettivamente osservate} ($O_i =
n_i$), diviso per la varianza del numero di misure attese
(approssimata da $N p_i = A_i$), ha
\emph{approssimativamente} la distribuzione del $\chi^2$,
con $M - 1$ gradi di libert\`a; il motivo di quest'ultima
affermazione \`e che esiste un vincolo sulle $O_i$, quello
di avere per somma il numero totale di misure effettuate $N$
(che viene usato nella formula \eqref{eq:12.chi2fit},
mediante la quale abbiamo definito $X$, per calcolare il
numero $A_i$ di misure attese in ogni intervallo).

La condizione enunciata si pu\`o in pratica supporre
verificata se le $A_i$ in ogni intervallo sono almeno pari a
5; o, meglio, se il numero di classi di frequenza%
\index{classi di frequenza|)}
in cui ci si aspetta un numero di misure minore di 5 \`e
trascurabile rispetto al totale (meno del 10\%).  In
realt\`a, se le classi di frequenza si possono scegliere
arbitrariamente, la cosa migliore consiste nel definirle di
ampiezze differenti: in modo tale che quegli intervalli dove
cadono poche misure vengano riuniti assieme in un'unica
classe pi\`u ampia, ove $n_i$ valga almeno 5 (ma nemmeno
troppo ampia, per soddisfare al vincolo di avere ${p_i}^2
\ll p_i$; in genere si cerca di riunire assieme pi\`u classi
in modo da avere degli $n_i \sim 5\div 10$).

Tornando al problema iniziale, per la verifica dell'ipotesi
statistica che i dati vengano dalla distribuzione usata per
il calcolo delle $A_i$ basta:
\begin{itemize}
\item fissare arbitrariamente un livello di probabilit\`a
  che rappresenti il confine tra eventi ammissibili
  nell'ipotesi della pura casualit\`a ed eventi invece tanto
  improbabili da far supporre che il loro verificarsi sia
  dovuto non a fluttuazioni statistiche, ma al non essere
  verificate le ipotesi fatte in partenza (il provenire i
  dati dalla distribuzione nota a priori): ad esempio il
  95\% o il 99\%.
\item Cercare nelle apposite tabelle\/\footnote{Alcuni
    valori numerici di questo tipo sono tabulati
    nell'appendice \ref{ch:f.tabelle}.  \`E bene anche
    ricordare che quando il numero di gradi di libert\`a $N$
    \`e superiore a 30 si pu\`o far riferimento alla
    distribuzione normale con media $N$ ed errore quadratico
    medio $\sqrt{2N}$; e che, gi\`a per piccoli $N$,
    $\sqrt{2\chi^2}$ \`e approssimativamente normale con
    media $\sqrt{2N-1}$ e varianza 1.} il valore di taglio
  corrispondente alla coda superiore della distribuzione del
  $\chi^2$ ad $M - 1$ gradi di libert\`a avente area pari al
  livello di confidenza desiderato; ossia quell'ascissa
  $\xi$ che lascia alla propria sinistra, sotto la curva
  della distribuzione del $\chi^2$ ad $M - 1$ gradi di
  libert\`a, un'area pari a tale valore.
\item Calcolare $X$; ed infine rigettare l'ipotesi (al
  livello di confidenza prescelto) perch\'e incompatibile
  con i dati raccolti, se $X$ risultasse superiore a $\xi$
  (o, altrimenti, considerare l'ipotesi compatibile con i
  dati al livello di confidenza prescelto e quindi
  accettarla).
\end{itemize}

Quanto detto a proposito della particolare distribuzione del
$\chi^2$ da usare per il la verifica della nostra ipotesi,
per\`o, \`e valido solo se le caratteristiche della
distribuzione teorica con cui confrontare i nostri dati sono
note a priori; se, invece, $R$ parametri da cui essa dipende
fossero stati stimati a partire dai dati, il numero di gradi
di libert\`a sarebbe inferiore e pari ad $M - R - 1$.

Cos\`\i\ se le $p_i$ sono state ricavate integrando sulle
classi di frequenza una distribuzione normale la cui media e
la cui varianza siano state a loro volta ottenute dal
campione istogrammato, il numero di gradi di libert\`a,
essendo $R=2$, sarebbe pari a $M - 3$.

\begin{figure}[hbtp]
  \vspace*{2ex}
  \begin{center} {
    \input{chicdf.pstex_t}
  } \end{center}
  \caption[La funzione di distribuzione del $\chi^2$]
  {L'integrale da $x$ a $+\infty$ della funzione di
    frequenza del $\chi^2$, per alcuni valori del parametro
    $N$.}
  \label{fig:chicdf}
\end{figure}

\begin{figure}[hbtp]
  \vspace*{2ex}
  \begin{center} {
    \input{chirid.pstex_t}
  } \end{center}
  \caption[La funzione di distribuzione del $\chi^2$
  ridotto]
  {I valori del $\chi^2$ ridotto ($\chi^2/N$) che
    corrispondono, per differenti gradi di libert\`a $N$, ad
    un certo livello di confidenza.}
  \label{fig:chirid}
\end{figure}

Per dare un'idea dei valori del $\chi^2$ che corrispondono
al rigetto di una ipotesi (ad un certo livello di
confidenza), e senza ricorrere alle tabelle numeriche, nella
figura \ref{fig:chicdf} sono riportati in grafico i valori
$P$ dell'integrale da $x$ a $+\infty$ della funzione di
frequenza del $\chi^2$ (ovvero il complemento ad uno della
funzione di distribuzione), per alcuni valori del parametro
$N$.

Le curve della figura \ref{fig:chirid} permettono invece di
identificare (per differenti scelte del livello di
confidenza $\varepsilon$) i corrispondenti valori di taglio
del $\chi^2$ \emph{ridotto} --- ovvero del rapporto
$\chi^2/N$ tra esso ed il numero di gradi di libert\`a $N$.
Insomma, ogni punto di queste curve al di sopra di
un'ascissa (intera) $N$ ha come ordinata un numero $X/N$
tale che l'integrale da $X$ a $+\infty$ della funzione di
frequenza del $\chi^2$ ad $N$ gradi di libert\`a sia uguale
ad $\varepsilon$.%
\index{compatibilit\`a!con una distribuzione|)}

\subsection[Il metodo del minimo $\chi^2$]{Il metodo
  del minimo $\boldsymbol{\chi}^{\boldsymbol{2}}$}%
\index{metodo!del minimo $\chi^2$|(}
Supponiamo di sapere a priori che i nostri dati istogrammati
debbano seguire una data distribuzione, ma che essa dipenda
da $R$ parametri incogniti che dobbiamo stimare a partire
dai dati stessi; visto che l'accordo tra i dati e la
distribuzione \`e dato dalla $X$ definita nella
\eqref{eq:12.chi2fit}, ed \`e tanto migliore quanto pi\`u il
valore ottenuto per essa \`e basso, un metodo plausibile di
stima potrebbe essere quello di trovare per quali valori dei
parametri stessi la $X$ \`e minima (\emph{metodo del minimo}
$\chi^2$).

Indicando con $\alpha_k$ ($k=1,\ldots,R$) i parametri da
stimare, ognuna delle $p_i$ sar\`a esprimibile in funzione
delle $\alpha_k$; ed imponendo che le derivate prime della
$X$ rispetto ad ognuna delle $\alpha_k$ siano tutte nulle
contemporaneamente, otteniamo
\begin{gather}
  \frac{\partial X}{\partial \alpha_k} \; = \;
    \sum_{i=1}^M \frac{-2 \left(n_i - N p_i \right) N^2
    p_i - N \left( n_i - N p_i \right)^2}{N^2 {p_i}^2}
    \, \frac{\partial p_i}{\partial \alpha_k}
    \; = \; 0 \peq , \notag \\
  \intertext{ossia}
  - \frac{1}{2} \, \frac{\partial X}{\partial \alpha_k}
    \; = \; \sum_{i=1}^M \left[ \frac{n_i - N p_i}{p_i}
    + \frac{\left( n_i - N p_i \right)^2}{2 N {p_i}^2}
    \right] \frac{\partial p_i}{\partial \alpha_k} \; =
    \; 0 \peq . \label{eq:12.michi1}
\end{gather}

L'insieme delle \eqref{eq:12.michi1} costituisce un sistema
di $R$ equazioni, nelle $R$ incognite $\alpha_k$, che ci
permetter\`a di stimarne i valori (salvo poi, nel caso il
sistema delle \eqref{eq:12.michi1} abbia pi\`u di una
soluzione, controllare quali di esse corrispondono in
effetti ad un minimo e quale tra queste ultime corrisponde
al minimo assoluto); le condizioni sotto le quali il metodo
\`e applicabile sono quelle gi\`a enunciate in
precedenza\/\footnote{Se la prima di esse non si pu\`o
  ritenere accettabile, delle equazioni ancora valide ma
  pi\`u complesse si possono ottenere dalla
  \eqref{eq:12.chi2fit} sostituendo $N p_i (1 - p_i)$ al
  posto di $N p_i$ nel denominatore.}, ossia ${p_i}^2 \ll
p_i$ e $n_i \gtrsim 5$.

In genere per\`o si preferisce servirsi, in luogo delle
equazioni \eqref{eq:12.michi1}, di una forma semplificata,
ottenuta trascurando il secondo termine nella parentesi
quadra: che, si pu\`o dimostrare, \`e molto inferiore al
primo per grandi $N$ (infatti il rapporto tra i due termini
vale
\begin{equation*}
  \frac{ \left( n_i - N p_i \right)^2 }{ 2 N {p_i}^2 } \,
    \frac{ p_i }{ n_i - N p_i } \; = \; \frac{ n_i - N p_i
    }{ 2 N p_i } \; = \; \frac{1}{2 p_i} \left(
    \frac{n_i}{N} - p_i \right)
\end{equation*}
e converge ovviamente a zero all'aumentare di $N$); e
risolvere, insomma, il sistema delle
\begin{equation} \label{eq:12.michi2}
  \sum_{i=1}^M \left( \frac{n_i - N p_i}{p_i} \right)
    \frac{\partial p_i}{\partial \alpha_k} = 0
\end{equation}
(metodo \emph{semplificato} del minimo $\chi^2$).

Si pu\`o dimostrare che le soluzioni $\bar \alpha_k$ del
sistema delle \eqref{eq:12.michi2} tendono stocasticamente
ai valori veri $\alpha_k^*$ (in assenza di errori
sistematici) al crescere di $N$; inoltre il valore di $X$
calcolato in corrispondenza dei valori ricavati
$\bar \alpha_k$ d\`a, se rapportato alla distribuzione del
$\chi^2$ con $M - R - 1$ gradi di libert\`a, una misura
della bont\`a della soluzione stessa.

Ora, le equazioni \eqref{eq:12.michi2} si possono scrivere
anche
\begin{gather*}
  \sum_{i=1}^M \left( \frac{n_i - N p_i}{p_i} \right)
    \frac{\partial p_i}{\partial \alpha_k} =
    \sum_{i=1}^M \frac{n_i}{p_i} \, \frac{\partial
    p_i}{\partial \alpha_k} - N \sum_{i=1}^M
    \frac{\partial p_i}{\partial \alpha_k} \\
  \intertext{e si possono ulteriormente semplificare,
    visto che l'ultimo termine si annulla, essendo}
  \sum_{i=1}^M \frac{\partial p_i}{\partial \alpha_k}
    \; = \; \frac{\partial}{\partial \alpha_k}
    \sum_{i=1}^M p_i \; = \; \frac{\partial}{\partial
    \alpha_k} \, 1 \; \equiv \; 0
\end{gather*}
se si fa l'ulteriore ipotesi che l'intervallo dei valori
indagati copra, anche approssimativamente, tutti quelli in
pratica permessi; per cui il sistema di equazioni da
risolvere \`e in questo caso quello delle
\begin{equation} \label{eq:12.michi4}
  \sum_{i=1}^M \frac{n_i}{p_i} \, \frac{\partial
    p_i}{\partial \alpha_k} = 0 \peq .
\end{equation}

\index{massima verosimiglianza, metodo della|(}%
Per la stima di parametri incogniti a partire da dati
misurati abbiamo gi\`a affermato che teoricamente \`e da
preferire il metodo della massima verosimiglianza, le cui
soluzioni sono quelle affette, come sappiamo, dal minimo
errore casuale (almeno asintoticamente); in questo caso
particolare (dati in istogramma), come lo si dovrebbe
applicare?  Se le misure sono indipendenti, la probabilit\`a
di avere $n_i$ eventi nella generica classe di frequenza \`e
data da $p_i^{n_i}$; la funzione di
verosimiglianza\/\footnote{Per essere precisi, la
  probabilit\`a che $n_1$ misure si trovino nella prima
  classe di frequenza, $n_2$ nella seconda e cos\`\i\ via,
  \`e dato dalla espressione \eqref{eq:12.michi3}
  moltiplicata per il numero di modi differenti in cui $N$
  oggetti possono essere suddivisi in $M$ gruppi composti da
  $n_1, n_2,\ldots,n_M$ oggetti rispettivamente (numero
  delle \emph{partizioni ordinate});%
  \index{partizioni ordinate}
  questo vale, come mostrato nel paragrafo
  \ref{ch:a.parord}, $N! / (n_1!\, n_2!\cdots n_M!)$, e
  rappresenta un fattore costante che non incide nella
  ricerca del massimo della \eqref{eq:12.michi3}.}  da
\begin{gather}
  \mathcal{L}(\alpha_1,\ldots,\alpha_R) = \prod_{i=1}^M
    p_i^{n_i} \label{eq:12.michi3} \\
  \intertext{ed il suo logaritmo da}
  \ln \mathcal{L} = \sum_{i=1}^M \left( n_i \cdot \ln
    p_i \right) \peq . \notag
\end{gather}

La soluzione di massima verosimiglianza (e quindi di minima
varianza) si trova cercando il massimo di $\ln \mathcal{L}$:
e risolvendo quindi il sistema delle
\begin{equation*}
  \frac{\partial}{\partial \alpha_k} \, \ln \mathcal{L}
    \; = \; \sum_{i=1}^M n_i \, \frac{1}{p_i} \,
    \frac{\partial p_i}{\partial \alpha_k} \; = \; 0 \peq ;
\end{equation*}
in questo caso, vista l'equazione \eqref{eq:12.michi4} in
precedenza ricavata, i due metodi (della massima
verosimiglianza e del minimo $\chi^2$ semplificato)
conducono dunque \emph{alla stessa soluzione}.%
\index{massima verosimiglianza, metodo della|)}%
\index{metodo!del minimo $\chi^2$|)}

\subsection{Test di omogeneit\`a per dati raggruppati}%
\index{compatibilit\`a!tra dati sperimentali|(}%
\index{omogeneit\`a, test di|see{compatibilit\`a tra dati sperimentali}}
Supponiamo di avere a disposizione $Q$ campioni di dati,
indipendenti l'uno dall'altro e composti da $n_1,
n_2,\ldots, n_Q$ elementi rispettivamente; e, all'interno di
ognuno di tali campioni, i dati siano suddivisi nei medesimi
$P$ gruppi: indichiamo infine col simbolo $\nu_{ij}$ il
numero di dati appartenenti al gruppo $i$-esimo all'interno
del campione $j$-esimo.

Per fare un esempio, i campioni si potrebbero riferire alle
regioni italiane e i gruppi al livello di istruzione
(licenza elementare, media, superiore, laurea): cos\`\i\ che
i $\nu_{ij}$ rappresentino il numero di persone, per ogni
livello di istruzione, residenti in ogni data regione;
oppure (e questo \`e un caso che si presenta frequentemente
nelle analisi fisiche) si abbiano vari istogrammi
all'interno di ognuno dei quali i dati siano stati
raggruppati secondo le medesime classi di frequenza:%
\index{classi di frequenza} allora i $\nu_{ij}$ saranno il
numero di osservazioni che cadono in una determinata classe
in ogni istogramma.

Il problema che ci poniamo \`e quello di verificare
l'ipotesi che tutti i campioni provengano dalla stessa
popolazione e siano perci\`o compatibili tra loro
(\emph{test di omogeneit\`a}).  Indichiamo con il simbolo
$N$ il numero totale di dati a disposizione; e con $m_i$
(con $i=1,\ldots,P$) il numero totale di dati che cadono
nell'$i$-esimo gruppo in tutti i campioni a disposizione.

\begin{table}[htbp]
  \vspace*{2ex}
  \begin{center}
    \begin{tabular}{|r|ccccc|c|}
      \cline{2-6}
      \multicolumn{1}{c}{\tabtop\tabbot} &
        \multicolumn{5}{|c|}{Campioni} \\
      \hline
      & $\nu_{11}$ & $\nu_{12}$ & $\nu_{13}$ & $\cdots$
        & $\nu_{1Q}$ & $m_1$\tabtop \\
      & $\nu_{21}$ & $\nu_{22}$ & $\cdots$ & $\cdots$ &
        $\nu_{2Q}$ & $m_2$ \\
      Gruppi & $\nu_{31}$ & $\cdots$ & $\cdots$ &
        $\cdots$ & $\cdots$ & $m_3$ \\
      & $\cdots$ & $\cdots$ & $\cdots$ & $\cdots$ &
        $\cdots$ & $\cdots$ \\
      & $\nu_{P1}$ & $\nu_{P2}$ & $\cdots$ & $\cdots$ &
        $\nu_{PQ}$ & $m_P$\tabbot \\
      \hline
      \multicolumn{1}{c|}{} & $n_1$ & $n_2$ & $n_3$ &
        $\cdots$ & $n_Q$ & $N$\tabtop\tabbot \\
      \cline{2-7}
    \end{tabular}
  \end{center}
  \caption{Un esempio delle cosiddette \emph{tabelle
    delle contingenze}.}
  \label{tab:12.contin}
\end{table}

\`E consuetudine che dati di questo genere siano
rappresentati in una tabella del tipo della
\ref{tab:12.contin}, che si chiama \emph{tabella delle
  contingenze};%
\index{contingenze, tabella delle}
e risulta ovviamente
\begin{align*}
  n_j &= \sum_{i=1}^P \nu_{ij} && \qquad \qquad
    (j=1,2,\ldots,Q) \peq ; \\[1ex]
  m_i &= \sum_{j=1}^Q \nu_{ij} && \qquad \qquad
    (i=1,2,\ldots,P) \peq ; \\[1ex]
  N &= \sum_{j=1}^Q n_j = \sum_{i=1}^P m_i = \sum_{i,j}
    \nu_{ij} \peq .
\end{align*}

Vogliamo ora dimostrare che la variabile casuale
\begin{equation} \label{eq:12.chiomo}
  X = N \left[ \sum_{i,j} \frac{\left( \nu_{ij}
  \right)^2}{m_i \, n_j} - 1 \right]
\end{equation}
\`e distribuita come il $\chi^2$ a $(P-1)(Q-1)$ gradi di
libert\`a: a questo scopo supponiamo innanzi tutto sia
valida l'ipotesi che i dati provengano tutti dalla medesima
popolazione, ed indichiamo con i simboli $p_i$ e $q_j$ le
probabilit\`a che un componente di tale popolazione scelto a
caso cada rispettivamente nel gruppo $i$-esimo o nel
campione $j$-esimo; e sappiamo inoltre che (ammessa per\`o
vera l'ipotesi che \emph{tutti} i campioni provengano dalla
stessa distribuzione) questi due eventi sono statisticamente
indipendenti: per cui ognuno dei dati ha probabilit\`a
complessiva $p_i q_j$ di cadere in una delle caselle della
tabella delle contingenze.

Possiamo stimare i $P$ valori $p_i$ a partire dai dati
sperimentali: si tratta in realt\`a solo di $P-1$ stime
\emph{indipendenti}, perch\'e, una volta ricavate le prime
$P-1$ probabilit\`a, l'ultima di esse risulter\`a
univocamente determinata dalla condizione che la somma
complessiva valga 1.  Analogamente possiamo anche stimare i
$Q$ valori $q_j$ dai dati sperimentali, e si tratter\`a in
questo caso di effettuare $Q-1$ stime indipendenti.

Le stime di cui abbiamo parlato sono ovviamente
\begin{align} \label{eq:12.piqj}
  p_i &= \frac{m_i}{N} &&\text{e} & q_j &=
    \frac{n_j}{N}
\end{align}
e, applicando le conclusioni del paragrafo precedente
(l'equazione \eqref{eq:12.chi2fit}), la variabile
\begin{align*}
  X &= \sum_{i,j} \frac{\left( \nu_{ij} - N p_i q_j
    \right)^2}{N p_i q_j} \\[1ex]
  &= \sum_{i,j} \left[ \frac{\left( \nu_{ij}
    \right)^2}{N p_i q_j} - 2 \nu_{ij} + N p_i q_j
    \right] \\[1ex]
  &= \sum_{i,j} \frac{\left( \nu_{ij} \right)^2}{N p_i
    q_j} -2N + N \\[1ex]
  &= \sum_{i,j} \frac{\left( \nu_{ij} \right)^2}{N p_i
    q_j} - N
\end{align*}
deve essere distribuita come il $\chi^2$.

Sostituendo in quest'ultima espressione i valori
\eqref{eq:12.piqj} per $p_i$ e $q_j$, essa si riduce alla
\eqref{eq:12.chiomo}; il numero di gradi di libert\`a \`e
pari al numero di contributi sperimentali indipendenti, $PQ
- 1$ (c'\`e il vincolo che la somma totale sia $N$),
diminuito del numero $(P-1) + (Q-1)$ di parametri stimato
sulla base dei dati: ovverosia proprio $(P-1) (Q-1)$ come
anticipato.%
\index{compatibilit\`a!tra dati sperimentali|)}

\subsection{Un esempio: diffusione elastica protone-protone}
\begin{figure}[hbtp]
  \vspace*{2ex}
  \begin{center} {
    \input{scat.pstex_t}
  } \end{center}
  \caption{Urto elastico protone-protone.}
  \label{fig:12.scat}
\end{figure}
Nella figura \ref{fig:12.scat} \`e schematicamente
rappresentato un processo di urto elastico tra due
particelle, una delle quali sia inizialmente ferma; dopo
l'urto esse si muoveranno lungo traiettorie rettilinee ad
angoli $\vartheta_1$ e $\vartheta_2$ rispetto alla direzione
originale della particella urtante.

Gli angoli $\vartheta_i$ vengono misurati; supponendo che il
processo di misura introduca errori che seguono la
distribuzione normale ed abbiano una entit\`a che (per
semplificare le cose) assumiamo sia costante, nota ed
indipendente dall'ampiezza dell'angolo, vogliamo verificare
l'ipotesi che le due particelle coinvolte nel processo
d'urto siano di massa uguale (ad esempio che siano entrambe
dei protoni).

La prima cosa da fare \`e quella di ricavare dai dati
misurati $\vartheta_i$, che per ipotesi hanno una funzione
di frequenza
\begin{equation*}
  f(\vartheta; \vartheta^*, \sigma) \; = \; \frac{1}{\sigma
    \sqrt{2 \pi}} \, e^{- \frac{1}{2} \left( \frac{\vartheta
        - \vartheta^*}{\sigma} \right)^2}
\end{equation*}
una stima dei valori veri $\vartheta^*$.  Il logaritmo della
funzione di verosimiglianza \`e dato da
\begin{equation*}
  \ln \mathcal{L} \; = \; - 2 \ln \left( \sigma \sqrt{2 \pi}
    \right) - \frac{1}{2} \left( \frac{\vartheta_1 -
        \vartheta_1^*}{\sigma} \right)^2  - \frac{1}{2}
    \left( \frac{\vartheta_2 - \vartheta_2^*}{\sigma}
    \right)^2 \peq ;
\end{equation*}
ma le variabili $\vartheta_1$ e $\vartheta_2$ \emph{non sono
  indipendenti}, visto che il processo deve conservare sia
energia che quantit\`a di moto.  Ammessa vera l'ipotesi che
le due particelle abbiano uguale massa (e restando nel
limite non-relativistico), le leggi di conservazione
impongono il vincolo che l'angolo tra le due particelle dopo
l'urto sia di 90\gra (o, in radianti, $\pi / 2$); usando il
metodo dei moltiplicatori di Lagrange, la funzione da
massimizzare \`e
\begin{equation*}
  \varphi( \vartheta_1^*, \vartheta_2^*, \lambda) \; =
  \; - \frac{1}{2} \left( \frac{\vartheta_1 -
      \vartheta_1^*}{\sigma} \right)^2 - \frac{1}{2}
  \left( \frac{\vartheta_2 - \vartheta_2^*}{\sigma}
  \right)^2 + \lambda \left( \vartheta_1^* +
    \vartheta_2^* - \frac{\pi}{2} \right)
\end{equation*}
e, annullando contemporaneamente le sue derivate rispetto
alle tre variabili, si giunge al sistema
\begin{equation*}
  \left\{
    \begin{array}{cclcc}
      \dfrac{\partial \varphi}{\partial \lambda} & = &
      \vartheta_1^* + \vartheta_2^* - \dfrac{\pi}{2} & = &
      0 \\[2.5ex]
      \dfrac{\partial \varphi}{\partial \vartheta_1^*} & = &
      \dfrac{1}{\sigma^2} \left( \vartheta_1 - \vartheta_1^*
      \right) + \lambda & = & 0 \\[2.5ex]
      \dfrac{\partial \varphi}{\partial \vartheta_2^*} & = &
      \dfrac{1}{\sigma^2} \left( \vartheta_2 - \vartheta_2^*
      \right) + \lambda & = & 0
    \end{array}
  \right.
\end{equation*}
Eliminando $\lambda$ dalle ultime due equazioni otteniamo
\begin{gather*}
  \vartheta_1 - \vartheta_1^* \; = \; \vartheta_2 -
  \vartheta_2^* \\
  \intertext{e, sostituendo l'espressione per
    $\vartheta_2^*$ ricavata dalla prima equazione,}
  \vartheta_1 - \vartheta_1^* \; = \; \vartheta_2 - \left(
    \frac{\pi}{2} - \vartheta_1^* \right)
\end{gather*}
per cui le due stime di massima verosimiglianza sono
\begin{equation*}
  \left\{
    \begin{array}{ccl}
      \hat \vartheta_1^* & = & \vartheta_1 + \dfrac{1}{2}
      \left( \dfrac{\pi}{2} - \vartheta_1 - \vartheta_2
      \right) \\[2.5ex]
      \hat \vartheta_2^* & = & \vartheta_2 + \dfrac{1}{2}
      \left( \dfrac{\pi}{2} - \vartheta_1 - \vartheta_2
      \right)
    \end{array}
  \right.
\end{equation*}

Ammesso che queste soluzioni siano buone stime dei valori
veri, la variabile casuale
\begin{equation*}
  X \; = \; \left( \frac{ \vartheta_1 -
      \vartheta_1^*}{\sigma} \right)^2 + \left( \frac{
      \vartheta_2 - \vartheta_2^*}{\sigma} \right)^2 \;
  = \; \frac{1}{2 \sigma^2} \left( \frac{\pi}{2} -
    \vartheta_1 - \vartheta_2 \right)^2
\end{equation*}
\`e distribuita come il $\chi^2$ ad un grado di libert\`a
(due contributi, un vincolo); ed il valore di $X$
confrontato con le tabelle del $\chi^2$ pu\`o essere usato
per la verifica dell'ipotesi.

\section{Compatibilit\`a con un valore prefissato}%
\index{compatibilit\`a!con un valore|(}
Un altro caso che frequentemente si presenta \`e il
seguente: si vuole controllare se un determinato valore
numerico, a priori attribuibile alla grandezza fisica in
esame, \`e o non \`e confermato dai risultati della misura;
cio\`e se quel valore \`e o non \`e \emph{compatibile} con i
nostri risultati --- pi\`u precisamente, a che livello di
probabilit\`a (o, per usare la terminologia statistica, a
che \emph{livello di confidenza}) \`e con essi compatibile.

Ammettiamo che gli errori di misura seguano la legge
normale; sappiamo che la probabilit\`a per il risultato di
cadere in un qualunque intervallo prefissato dell'asse reale
si pu\`o calcolare integrando la funzione di Gauss fra gli
estremi dell'intervallo stesso.  Riferiamoci per comodit\`a
alla variabile \emph{scarto normalizzato}
\begin{equation*}
  t = \frac{x - E(x)}{\sigma}
\end{equation*}
che sappiamo gi\`a dal paragrafo \ref{ch:9.scanor} essere
distribuita secondo una legge che \`e indipendente
dall'entit\`a degli errori di misura.

Se fissiamo arbitrariamente un numero positivo $\tau$,
possiamo calcolare la probabilit\`a che si verifichi
l'evento casuale consistente nell'ottenere, in una
particolare misura, un valore di $t$ che in modulo
superi $\tau$; come esempio particolare, le condizioni
$|t| > 1$ o $|t|> 2$ gi\`a sappiamo che si verificano
con probabilit\`a rispettivamente del 31.73\% e del
4.55\%, visto che l'intervallo $-1 \leq t \leq 1$
corrisponde al 68.27\% dell'area della curva normale, e
quello $-2 \leq t \leq 2$ al 95.45\% .

Se consideriamo poi un campione di $N$ misure indipendenti,
avente valore medio $\bar x$ e proveniente da questa stessa
popolazione di varianza $\sigma^2$, \`e immediato capire
come la variabile
\begin{equation*}
  t = \frac{\bar x - E(x)}{\dfrac{\sigma}{\sqrt{N}}}
\end{equation*}
soddisfer\`a a queste stesse condizioni: accadr\`a cio\`e
nel 31.73\% dei casi che $|t|$ sia maggiore di $\tau=1$, e
nel 4.55\% dei casi che $|t|$ sia superiore a $\tau=2$ .

Per converso, se fissiamo arbitrariamente un qualunque
valore ammissibile $P$ per la probabilit\`a, possiamo
calcolare in conseguenza un numero $\tau$, tale che la
probabilit\`a di ottenere effettivamente da un particolare
campione un valore dello scarto normalizzato $t$ superiore
ad esso (in modulo) sia data dal numero $P$.  Ad esempio,
fissato un valore del 5\% per $P$, il limite per $t$ che se
ne ricava \`e $\tau = 1.96$: insomma
\begin{equation*}
  \int_{-1.96}^{+1.96} \frac{1}{\sqrt{2 \pi}} \,
    e^{- \frac{1}{2} t^2} \de t
    \; = \; 0.95
\end{equation*}
e solo nel cinque per cento dei casi si ottiene un valore di
$t$ che supera (in modulo) 1.96.

Se si fissa per convenzione un valore della probabilit\`a
che indichi il confine tra un avvenimento accettabile ed uno
inaccettabile nei limiti della pura casualit\`a, possiamo
dire che l'ipotesi consistente nell'essere un certo numero
$\xi$ il valore vero della grandezza misurata sar\`a
compatibile o incompatibile con i nostri dati a seconda che
lo scarto normalizzato
\begin{equation*}
  t = \frac{\bar x - \xi}{\dfrac{\sigma}{\sqrt{N}}}
\end{equation*}
relativo a tale numero sia, in valore assoluto, inferiore o
superiore al valore di $\tau$ che a quella probabilit\`a
corrisponde; e diremo che la compatibilit\`a (o
incompatibilit\`a) \`e riferita a quel certo livello di
confidenza prescelto.

La difficolt\`a \`e che tutti questi ragionamenti
coinvolgono una quantit\`a numerica (lo scarto quadratico
medio) relativa \emph{alla popolazione} e per ci\`o stesso
in generale ignota; in tal caso, per calcolare lo scarto
normalizzato relativo ad un certo valore numerico $\xi$ non
possiamo che servirci, in luogo di $\sigma$, della
corrispondente stima ricavata dal campione, $s$:
\begin{equation*}
  t = \frac{\bar x - \xi}{\dfrac{s}{\sqrt{N}}}
\end{equation*}
e quindi si deve presupporre di avere un campione di
dimensioni tali che questa stima si possa ritenere
ragionevole, ossia sufficientemente vicina ai corrispondenti
valori relativi alla popolazione a meno di fluttuazioni
casuali abbastanza poco probabili.

In generale si ammette che \emph{almeno 30 dati} siano
necessari perch\'e questo avvenga: in corrispondenza a tale
dimensione del campione, l'errore della media \`e circa 5.5
volte inferiore a quello dei dati; e l'errore relativo di
$s$ \`e approssimativamente del 13\%.

Bisogna anche porre attenzione alla esatta natura
dell'ipotesi che si intende verificare.  Per un valore
limite di $\tau = 1.96$ abbiamo visto che il 95\% dell'area
della curva normale \`e compreso tra $-\tau$ e $+\tau$:
superiormente a $+\tau$ si trova il 2.5\% di tale area; ed
anche inferiormente a $-\tau$ se ne trova un'altra porzione
pari al 2.5\%.

\begin{table}[htbp]
  \vspace*{2ex}
  \begin{center}
    \begin{tabular}{rcc}
      \toprule
      \multicolumn{1}{c}{$P$ (\%)} & $\tau_2$ & $\tau_1$ \\
      \midrule
      10.0 & 1.64485 & 1.28155 \\
      5.0  & 1.95996 & 1.64485 \\
      2.0  & 2.32635 & 2.05375 \\
      1.0  & 2.57583 & 2.32635 \\
      0.5  & 2.81297 & 2.57583 \\
      0.2  & 3.09023 & 2.87816 \\
      0.1  & 3.29053 & 3.09023 \\
      \bottomrule
    \end{tabular}
  \end{center}
  \caption{Alcuni valori della
    probabilit\`a $P$ e dei corrispondenti limiti
    $\tau$ sullo scarto normalizzato, per verifiche
    two-tailed ($\tau_2$) o one-tailed ($\tau_1$).}
  \label{tab:12.conlev1}
\end{table}

\begin{table}[hbtp]
  \vspace*{2ex}
  \begin{center}
    \begin{tabular}{crr}
      \toprule
      $\tau$ & \multicolumn{1}{c}{$P_2$ (\%)} &
      \multicolumn{1}{c}{$P_1$ (\%)} \\
      \midrule
      0.5 & 61.708 & 30.854 \\
      1.0 & 31.731 & 15.866 \\
      1.5 & 13.361 &  6.681 \\
      2.0 &  4.550 &  2.275 \\
      2.5 &  1.242 &  0.621 \\
      3.0 &  0.270 &  0.135 \\
      \bottomrule
    \end{tabular}
  \end{center}
  \caption{I valori della
    probabilit\`a per verifiche two-tailed ($P_2$) ed
    one-tailed ($P_1$) che corrispondono a valori
    prefissati dello scarto normalizzato $\tau$.}
  \label{tab:12.conlev2}
\end{table}

Se l'ipotesi da verificare riguarda l'\emph{essere
  differenti tra loro} due entit\`a (il presupposto valore
vero della grandezza misurata e la media aritmetica dei
nostri dati, nell'esempio precedente) quel valore di $\tau$
corrisponde in effetti ad una verifica relativa ad un
livello di confidenza del 5\% (usando il termine inglese,
stiamo effettuando un \emph{two-tailed test});%
\index{two-tailed test}
ma se l'ipotesi riguarda l'essere un valore numerico
\emph{superiore} (od \emph{inferiore}) alla nostra media
aritmetica (ad esempio, i dati misurati potrebbero essere
relativi al rendimento di una macchina, e si vuole
verificare l'ipotesi che tale rendimento misurato sia
superiore ad un valore prefissato), allora un limite $\tau =
1.96$ corrisponde in effetti ad un livello di confidenza del
2.5\% (\emph{one-tailed test}):%
\index{one-tailed test}
nell'esempio fatto, soltanto l'intervallo $[-\infty, -\tau]$
deve essere preso in considerazione per il calcolo della
probabilit\`a.  Alcuni limiti relativi a diversi livelli di
confidenza si possono trovare nelle tabelle
\ref{tab:12.conlev1} e \ref{tab:12.conlev2}; altri si
possono facilmente ricavare dalle tabelle dell'appendice
\ref{ch:f.tabelle}.%
\index{compatibilit\`a!con un valore|)}

\section{I piccoli campioni e la distribuzione di
  Student}%
\index{piccoli campioni|(}
Cosa si pu\`o fare riguardo alla verifica di ipotesi
statistiche come quella (considerata nel paragrafo
precedente) della compatibilit\`a del risultato delle misure
con un valore noto a priori, quando si abbiano a
disposizione solamente piccoli campioni?  Ci riferiamo,
pi\`u esattamente, a campioni costituiti da un numero di
dati cos\`\i\ esiguo da farci ritenere che non si possa
ottenere da essi con ragionevole probabilit\`a una buona
stima delle varianze delle rispettive popolazioni (sempre
per\`o supposte normali).

\index{distribuzione!di Student|(emidx}%
Sia $X$ una variabile casuale distribuita come il $\chi^2$
ad $N$ gradi di libert\`a, ed $u$ una seconda variabile
casuale, indipendente dalla prima, e avente distribuzione
normale standardizzata $N(u;0,1)$; consideriamo la nuova
variabile casuale $t$ definita attraverso la
\begin{equation} \label{eq:12.stuvar}
  t = \frac{u}{\sqrt{\dfrac{X}{N}}} \peq .
\end{equation}

Si pu\`o dimostrare che la funzione densit\`a di
probabilit\`a relativa alla variabile casuale $t$ \`e data
dalla
\begin{equation*}
  f(t;N) = \frac{T_N}{\left( 1 + \frac{t^2}{N}
    \right)^{\frac{N + 1}{2}}}
\end{equation*}
che si chiama \emph{distribuzione di Student ad $N$ gradi di
  libert\`a}.

\begin{figure}[hbtp]
  \vspace*{2ex}
  \begin{center} {
    \input{student.pstex_t}
  } \end{center}
  \caption[La distribuzione di Student]
    {La distribuzione di Student per $N=2$ ed $N=4$,
    confrontata con la funzione normale.}
  \label{fig:12.student}
\end{figure}

Il coefficiente $T_N$ \`e una costante che viene fissata
dalla condizione di normalizzazione; se $N$ viene poi fatto
tendere all'infinito il denominatore della funzione (come si
potrebbe facilmente provare partendo dal limite notevole
\eqref{eq:9.linote}) tende a $e^{t^2/2}$, e dunque la
distribuzione di Student tende alla distribuzione normale
(con media 0 e varianza 1).%
\index{distribuzione!di Student!e distribuzione normale}
Anche la forma della funzione di Student ricorda molto
quella della funzione di Gauss, come appare evidente dalla
figura \ref{fig:12.student}; soltanto, rispetto a dati che
seguano la distribuzione normale, valori elevati dello
scarto sono relativamente pi\`u probabili\/\footnote{Per
  valori di $N \gtrsim 35$ la distribuzione di Student si
  pu\`o approssimare con la distribuzione normale a media 0
  e varianza 1.}.

La distribuzione di Student \`e simmetrica, quindi tutti i
momenti di ordine dispari (compreso il valore medio
$\lambda_1$) sono nulli; mentre la varianza della
distribuzione \`e
\begin{gather*}
  \var(t) = \frac{N}{N-2} \\
  \intertext{(se $N > 2$); ed il coefficiente di
    curtosi vale}
  \gamma_2 = \frac{6}{N-4}
\end{gather*}
(se $N > 4$).

Indicando con $\bar x$ la media aritmetica di un campione di
dimensione $N$, estratto a caso da una popolazione normale
avente valore medio $E(x)$ e varianza $\sigma^2$; e con $s$
la stima della deviazione standard della popolazione
ottenuta dal campione stesso, cio\`e
\begin{gather}
  s^2 = \frac{\sum_i (x_i - \bar x)^2}{N-1} \notag \\
  \intertext{sappiamo, ricordando l'equazione
    \eqref{eq:12.xiis2}, che la variabile casuale}
  X'' = (N - 1) \, \frac{s^2}{\sigma^2} \notag \\
  \intertext{\`e distribuita come il $\chi^2$ ad
    $N - 1$ gradi di libert\`a; inoltre, ovviamente,}
  u = \frac{\bar x - E(x)}{\dfrac{\sigma}{\sqrt{N}}}
    \notag \\
  \intertext{segue la legge normale, con media 0 e
    varianza 1. Di conseguenza la variabile casuale}
  t \; = \; \frac{u}{\sqrt{\dfrac{X''}{N - 1}}} \; = \;
    \frac{\bar x - E(x)}{\dfrac{s}{\sqrt{N}}}
    \label{eq:12.xsecond}
\end{gather}
\emph{segue la distribuzione di Student ad $N - 1$
  gradi di libert\`a}.

Insomma: se i campioni a disposizione non hanno dimensioni
accettabili, una volta calcolato lo scarto normalizzato
relativo alla differenza tra la media di un campione ed un
valore prefissato occorrer\`a confrontare il suo valore con
i limiti degli intervalli di confidenza relativi alla
distribuzione di Student e non alla distribuzione
normale\/\footnote{Per taluni pi\`u usati valori del livello
  di confidenza, i limiti rilevanti si possono trovare
  tabulati anche nell'appendice \ref{ch:f.tabelle}.}.%
\index{distribuzione!di Student|)}%
\index{piccoli campioni|)}

\section{La compatibilit\`a di due valori misurati}%
\index{compatibilit\`a!tra valori misurati|(}
Un altro caso frequente \`e quello in cui si hanno a
disposizione due campioni di misure, e si vuole verificare
l'ipotesi statistica che essi provengano da popolazioni
aventi lo stesso valore medio: un caso particolare \`e
quello dell'ipotesi consistente nell'essere i due campioni
composti da misure della stessa grandezza fisica, che hanno
prodotto differenti stime come effetto della presenza in
entrambi degli errori; errori che assumiamo ancora seguire
la legge normale.

Siano ad esempio un primo campione di $N$ misure $x_i$, ed
un secondo campione di $M$ misure $y_j$; indichiamo con
$\bar x$ e $\bar y$ le medie dei due campioni, con
${\sigma_x}^2$ e ${\sigma_y}^2$ le varianze delle
popolazioni da cui tali campioni provengono, e con $\delta =
\bar x - \bar y$ la differenza tra le due medie.

Sappiamo gi\`a che i valori medi e le varianze delle medie
dei campioni sono legati ai corrispondenti valori relativi
alle popolazioni dalle
\begin{equation*}
  E(\bar x) \; = \; E(x)
    \makebox[4cm]{\mbox{,}}
    E(\bar y) \; = \; E(y)
\end{equation*}
e
\begin{equation*}
  \var (\bar x) \; = \; \frac{{\sigma_x}^2}{N}
    \makebox[4cm]{\mbox{,}}
    \var (\bar y) \; = \; \frac{{\sigma_y}^2}{M}
\end{equation*}
per cui risulter\`a, se i campioni sono tra loro
statisticamente indipendenti e se si ammette valida
l'ipotesi (da verificare) che abbiano la stessa media,
\begin{gather*}
  E(\delta) \; = \; E(\bar x - \bar y)
    \; =  \;E(x) - E(y) = 0 \\
  \intertext{e}
  \var (\delta) \; = \;
    \var (\bar x - \bar y) \; = \;
    \frac{{\sigma_x}^2}{N} + \frac{{\sigma_y}^2}{M} \peq .
\end{gather*}

Inoltre, essendo $\bar x$, $\bar y$ (e quindi $\delta$)
combinazioni lineari di variabili normali, seguiranno
anch'esse la legge normale; e la verifica dell'ipotesi che i
campioni provengano da popolazioni aventi la stessa media si
traduce nella verifica dell'ipotesi che $\delta$ abbia
valore vero nullo.

Tale verifica, essendo $\delta$ distribuita secondo la legge
normale, si esegue come abbiamo visto nel paragrafo
precedente: si fissa arbitrariamente un valore del livello
di confidenza, si determina il corrispondente valore limite
degli scarti normalizzati, e lo si confronta con il valore
di
\begin{equation*}
  \frac{\delta - E(\delta)}{\sigma_\delta} \; = \;
    \frac{\bar x - \bar y}{\sqrt{
    \dfrac{{\sigma_x}^2}{N} +
    \dfrac{{\sigma_y}^2}{M}}} \peq .
\end{equation*}

Ovviamente vale anche qui l'osservazione fatta nel paragrafo
precedente: non conoscendo le deviazioni standard delle
popolazioni, $\sigma_x$ e $\sigma_y$, siamo costretti ad
usare in loro vece le stime ottenute dai campioni, $s_x$ ed
$s_y$; e questo si ammette generalmente lecito quando la
dimensione di entrambi i campioni \`e almeno pari a 30.

\index{piccoli campioni|(}%
In caso contrario, presupponendo cio\`e di avere a
disposizione piccoli campioni per almeno una delle due
variabili, limitiamo la nostra analisi al caso in cui si
sappia con sicurezza che le due popolazioni $x$ ed $y$
\emph{abbiano la stessa varianza},
\begin{equation*}
  {\sigma_x}^2 \; = \; {\sigma_y}^2 \; \equiv \;
    \sigma^2
\end{equation*}
e definiamo la grandezza $S^2$ (\emph{varianza globale} dei
campioni) come
\begin{align*}
  S^2 &= \frac{1}{N + M - 2} \cdot \left\{ \sum_{i=1}^N
    \left[ x_i - \bar x \right]^2 + \sum_{j=1}^M \left[
    y_j - \bar y \right]^2 \right\} \\[2ex]
  &= \frac{(N-1) \, {s_x}^2 + (M-1) \, {s_y}^2}{N + M -
    2} \peq .
\end{align*}

Sapendo, dall'equazione \eqref{eq:12.xiis2}, che le due
variabili
\begin{align*}
  &(N-1) \, \frac{{s_x}^2}{\sigma^2} &&\text{e}
    &&(M-1)\, \frac{{s_y}^2}{\sigma^2}
\end{align*}
sono entrambe distribuite come il $\chi^2$, con $N - 1$ ed
$M - 1$ gradi di libert\`a rispettivamente, sfruttando la
regola di somma enunciata a pagina \pageref{th:12.resochi}
si ricava che la variabile casuale
\begin{equation*}
  X \; = \; \frac{(N-1) \, {s_x}^2 + (M-1) \,
    {s_y}^2}{\sigma^2} \; = \; (N + M - 2) \,
    \frac{S^2}{\sigma^2}
\end{equation*}
\`e distribuita come il $\chi^2$ ad $N + M - 2$ gradi di
libert\`a; essendo inoltre $\delta = \bar x - \bar y$ una
variabile normale con media e varianza date da
\begin{align*}
  E(\delta) &= E(x) - E(y) &&\text{e}
    &{\sigma_\delta}^2 &= \frac{\sigma^2}{N} +
    \frac{\sigma^2}{M}
\end{align*}
la variabile casuale
\begin{equation*}
  u \; = \; \frac{\delta - E(\delta)}{\sigma_\delta} \;
    = \; \frac{(\bar x - \bar y) - \left[ E(x) - E(y)
    \right]}{\sqrt{{\sigma}^2 \left( \dfrac{1}{N} +
    \dfrac{1}{M} \right)}}
\end{equation*}
\`e normale con media 0 e varianza 1.  Per concludere,
\begin{equation} \label{eq:12.tdif2c}
  t \; = \; \frac{u}{\sqrt{\dfrac{X}{N+M-2}}} \; = \;
    \frac{(\bar x - \bar y) - \left[ E(x) - E(y)
    \right]}{\sqrt{S^2 \left( \dfrac{1}{N} +
    \dfrac{1}{M} \right)}}
\end{equation}
deve seguire la distribuzione di Student%
\index{distribuzione!di Student}
con $N + M - 2$ gradi di libert\`a; di conseguenza, per
verificare l'ipotesi che le due popolazioni \emph{normali}
da cui i campioni provengono abbiano la stessa media
\emph{ammesso gi\`a che posseggano la stessa varianza},
basta confrontare con le apposite tabelle della
distribuzione di Student il valore della $t$ ottenuta dalla
\eqref{eq:12.tdif2c} ponendovi $E(x) - E(y) = 0$:
\begin{equation*}
  t = \frac{\bar x - \bar y}{\sqrt{S^2 \left(
    \dfrac{1}{N} + \dfrac{1}{M} \right)} } \peq .
\end{equation*}%
\index{piccoli campioni|)}%
\index{compatibilit\`a!tra valori misurati|)}

\section{La distribuzione di Fisher}%
\index{distribuzione!di Fisher|(}
Sia $X$ una variabile casuale distribuita come il $\chi^2$
ad $M$ gradi di libert\`a; ed $Y$ una seconda variabile
casuale, indipendente dalla prima, distribuita ancora come
il $\chi^2$, ma con $N$ gradi di libert\`a.

La variabile casuale $w$ (sempre positiva) definita in
funzione di esse attraverso la relazione
\begin{equation*}
  w \; = \; \frac{\phantom{t} \dfrac{X}{M}
    \phantom{t}}{\dfrac{Y}{N}}
\end{equation*}
ha una densit\`a di probabilit\`a che segue la cosiddetta
\emph{funzione di frequenza di Fisher} con $M$ ed $N$ gradi
di libert\`a.  La forma analitica della funzione di Fisher
\`e data dalla
\begin{equation} \label{eq:12.fisher}
  F(w;M,N) = K_{MN} \, \frac{w^{\frac{M}{2} - 1}}{\left( M
    w + N \right)^{\frac{M + N}{2}}}
\end{equation}
(nella quale $K_{MN}$ \`e un fattore costante determinato
dalla condizione di normalizzazione).

Il valore medio e la varianza della funzione di frequenza di
Fisher sono dati poi rispettivamente da
\begin{align*}
  E(F) &= \frac{N}{N-2} &&\text{(se $N>2$)} \\
  \intertext{e da}
  \var(F) &= \frac{2 \, N^2 (M + N - 2)}{M (N - 2)^2
    (N - 4)} &&\text{(se $N>4$)} \peq .
\end{align*}

Si pu\`o dimostrare che, se $Y$ \`e una variabile casuale
distribuita come il $\chi^2$ ad $N$ gradi di libert\`a,
\begin{equation*}
  \lim_{N \to +\infty} \frac{Y}{N} = 1
\end{equation*}
in senso statistico (ovverosia la probabilit\`a che il
rapporto $Y/N$ sia differente da 1 tende a zero quando $N$
viene reso arbitrariamente grande); per cui, indicando con
$f(x;M)$ la funzione di frequenza del $\chi^2$ ad $M$ gradi
di libert\`a,
\begin{gather*}
  F(w; M, \infty) \; \equiv \; \lim_{N \to +\infty}
    F(w; M, N) \; = \; \frac{f(w; M)}{M} \peq . \\
  \intertext{Allo stesso modo}
  F(w; \infty, N) \; \equiv \; \lim_{M \to +\infty}
    F(w; M, N) \; = \; \frac{N}{f(w;N)}
\end{gather*}
e quindi esiste una stretta relazione tra le distribuzioni
di Fisher e del chi quadro.

Inoltre, ricordando che, se $u$ \`e una variabile casuale
distribuita secondo la legge normale standardizzata
$N(u;0,1)$, l'altra variabile casuale $u^2$ \`e distribuita
come il $\chi^2$ ad un grado di libert\`a, il rapporto
\begin{gather*}
  w \; = \; \frac{\phantom{t} u^2
    \phantom{t}}{\dfrac{Y}{N}} \\
  \intertext{deve essere distribuito secondo
    $F(w;1,N)$; ma, se definiamo}
  t \; = \; \frac{\phantom{t} u
    \phantom{t}}{\sqrt{\dfrac{Y}{N}}}
\end{gather*}
sappiamo anche dalla \eqref{eq:12.stuvar} che la $t$ segue
la distribuzione di Student ad $N$ gradi di libert\`a.  La
conclusione \`e che il quadrato di una variabile $t$ che
segua la distribuzione di Student ad $N$ gradi di libert\`a
\`e a sua volta distribuito con una densit\`a di
probabilit\`a data da $F(t^2; 1, N)$.

Per terminare, quando i due parametri $M$ ed $N$ (da cui la
funzione di frequenza di Fisher \eqref{eq:12.fisher}
dipende) vengono resi arbitrariamente grandi, essa tende ad
una distribuzione normale; ma la convergenza \`e lenta, e
l'approssimazione normale alla distribuzione di Fisher si
pu\`o pensare in pratica usabile quando sia $M$ che $N$ sono
superiori a 50.%
\index{distribuzione!di Fisher|)}

\subsection{Confronto tra varianze}%
\index{compatibilit\`a!tra varianze|(}
Supponiamo di avere a disposizione due campioni di misure,
che ipotizziamo provenire da due differenti popolazioni che
seguano delle distribuzioni normali.

Siano $M$ ed $N$ le dimensioni di tali campioni, e siano
${\sigma_1}^2$ e ${\sigma_2}^2$ le varianze delle rispettive
popolazioni di provenienza; indichiamo poi con ${s_1}^2$ ed
${s_2}^2$ le due stime delle varianze delle popolazioni
ricavate dai campioni.  Vogliamo ora capire come si pu\`o
verificare l'ipotesi statistica che le due popolazioni
abbiano \emph{la stessa varianza}, ossia che $\sigma_1 =
\sigma_2$.

Ora sappiamo gi\`a dalla equazione \eqref{eq:12.xiis2}
che le due variabili casuali
\begin{align*}
  X &= (M - 1) \, \frac{{s_1}^2}{{\sigma_1}^2}
    &&\text{e} & Y &= (N - 1) \,
    \frac{{s_2}^2}{{\sigma_2}^2}
\end{align*}
sono entrambe distribuite come il $\chi^2$, con $M - 1$
ed $N - 1$ gradi di libert\`a rispettivamente; quindi
la quantit\`a
\begin{equation*}
  w \; = \; \frac{X}{M - 1} \, \frac{N - 1}{Y} \; = \;
    \frac{{s_1}^2}{{\sigma_1}^2} \,
    \frac{{\sigma_2}^2}{{s_2}^2}
\end{equation*}
ha densit\`a di probabilit\`a data dalla funzione di
Fisher con $M - 1$ ed $N - 1$ gradi di libert\`a.

Assunta a priori vera l'ipotesi statistica $\sigma_1 =
\sigma_2$, la variabile casuale
\begin{equation*}
  w = \frac{{s_1}^2}{{s_2}^2}
\end{equation*}
ha densit\`a di probabilit\`a data dalla funzione di Fisher
prima menzionata, $F(w; M-1, N-1)$; per cui, fissato un
livello di confidenza al di l\`a del quale rigettare
l'ipotesi, e ricavato dalle apposite tabelle\/\footnote{Per
  un livello di confidenza pari a 0.95 o 0.99, e per alcuni
  valori dei due parametri $M$ ed $N$, ci si pu\`o riferire
  ancora alle tabelle dell'appendice \ref{ch:f.tabelle}; in
  esse si assume che sia $s_1 > s_2$, e quindi $w > 1$.}  il
valore $W$ che lascia alla propria sinistra, al di sotto
della funzione $F(w; M-1, N-1)$, un'area pari al livello di
confidenza prescelto, si pu\`o escludere che i due campioni
provengano da popolazioni con la stessa varianza se $w >
W$.%
\index{compatibilit\`a!tra varianze|)}

\section{Il metodo di Kolmogorov e Smirnov}%
\index{Kolmogorov e Smirnov, test di|(}%
\index{compatibilit\`a!con una distribuzione|(}%
\index{compatibilit\`a!tra dati sperimentali|(}%
\index{Kolmogorov, Andrei Nikolaevich}
Il \emph{test di Kolmogorov e Smirnov} \`e un metodo di
analisi statistica che permette di confrontare tra loro un
campione di dati ed una distribuzione teorica (oppure due
campioni di dati) allo scopo di verificare l'ipotesi
statistica che la popolazione da cui i dati provengono sia
quella in esame (oppure l'ipotesi che entrambi i campioni
provengano dalla stessa popolazione).

Una caratteristica interessante di questo metodo \`e che
esso non richiede la preventiva, e pi\`u o meno arbitraria,
suddivisione dei dati in classi di frequenza; definendo
queste ultime in modo diverso si ottengono ovviamente, dal
metodo del $\chi^2$, differenti risultati per gli stessi
campioni.

Il test di Kolmogorov e Smirnov si basa infatti sulla
\emph{frequenza cumulativa relativa} dei dati, introdotta
nel paragrafo \ref{def:4.frcure} a pagina
\pageref{def:4.frcure}; e sull'analogo concetto di
\emph{funzione di distribuzione} di una variabile continua
definito nel paragrafo \ref{def:6.fundis} a pagina
\pageref{def:6.fundis}.  Per la compatibilit\`a tra un
campione ed una ipotetica legge che si ritiene possa
descriverne la popolazione di provenienza, e collegata ad
una funzione di distribuzione $\Phi(x)$, bisogna confrontare
la frequenza cumulativa relativa $F(x)$ del campione con
$\Phi(x)$ per ricavare \emph{il valore assoluto del massimo
  scarto tra esse},
\begin{equation*}
  \delta = \max \Bigl\{ \bigl| F(x) - \Phi(x) \bigr|
    \Bigr\} \peq .
\end{equation*}

Si pu\`o dimostrare che, se l'ipotesi da verificare fosse
vera, la probabilit\`a di ottenere casualmente un valore di
$\delta$ non inferiore ad una prefissata quantit\`a
(positiva) $\delta_0$ sarebbe data da
\begin{gather}
  \Pr \left( \delta \ge \delta_0 \right) = F_{\mathrm{KS}}
    \left( \delta'_0 \right) \notag \\
  \intertext{ove $F_{\mathrm{KS}}$ \`e la serie}
  F_{\mathrm{KS}} (x) = 2 \sum_{k=1}^\infty ( -1 )^{k-1}
    e^{-2 \, k^2 x^2} \label{eq:12.kosmfu} \\
  \intertext{e $\delta'_0$ vale}
  \delta'_0 = \left( \sqrt{N} + 0.12 +
    \frac{0.11}{\sqrt{N}} \right) \delta_0 \peq .
    \label{eq:12.kosmva}
\end{gather}

La legge ora enunciata \`e approssimata, ma il test di
Kolmogorov e Smirnov pu\`o essere usato gi\`a per dimensioni
del campione $N$ uguali a 5.  Attenzione per\`o che, se
qualche parametro da cui la distribuzione teorica dipende
\`e stato stimato sulla base dei dati, l'integrale della
densit\`a di probabilit\`a per la variabile $\delta$ di
Kolmogorov e Smirnov \emph{non segue pi\`u} la legge
\eqref{eq:12.kosmfu}: non solo, ma non \`e pi\`u possibile
ricavare teoricamente una funzione che ne descriva il
comportamento in generale (in questi casi, nella pratica, la
distribuzione di $\delta$ viene studiata usando metodi di
Montecarlo).

Se si vogliono invece confrontare tra loro due campioni
indipendenti per verificarne la compatibilit\`a, bisogna
ricavare dai dati il massimo scarto (in valore assoluto),
$\delta$, tra le due frequenze cumulative relative; e
ricavare ancora dalla \eqref{eq:12.kosmfu} la probabilit\`a
che questo possa essere avvenuto (ammessa vera l'ipotesi)
per motivi puramente casuali.  L'unica differenza \`e che la
funzione \eqref{eq:12.kosmfu} va calcolata in un'ascissa
$\delta'_0$ data dalla \eqref{eq:12.kosmva}, nella quale $N$
vale
\begin{equation*}
  N \; = \; \frac{1}{\frac{1}{N_1} + \frac{1}{N_2}} \; = \;
  \frac{N_1 \, N_2}{N_1 + N_2}
\end{equation*}
($N_1$ ed $N_2$ sono le dimensioni dei due campioni).

Oltre al gi\`a citato vantaggio di non richiedere la
creazione di pi\`u o meno arbitrarie classi di frequenza per
raggrupparvi i dati, un'altra caratteristica utile del test
di Kolmogorov e Smirnov \`e quella di essere, entro certi
limiti, \emph{indipendente dalla variabile usata} nella
misura: se al posto di $x$ si usasse, per caratterizzare il
campione, $\ln(x)$ o $\sqrt{x}$, il massimo scarto tra
frequenza cumulativa e funzione di distribuzione rimarrebbe
invariato.

Un altrettanto ovvio svantaggio \`e collegato al fatto che
per valori molto piccoli (o molto grandi) della variabile
casuale usata, qualsiasi essa sia, \emph{tutte} le funzioni
di distribuzione e tutte le frequenze cumulative \emph{hanno
  lo stesso valore} (0, o 1 rispettivamente).  Per questo
motivo il test di Kolmogorov e Smirnov \`e assai sensibile a
differenze nella zona centrale dei dati (attorno al valore
medio), mentre non \`e affatto efficace per discriminare tra
due distribuzioni che differiscano significativamente tra
loro solo nelle code; ad esempio che abbiano lo stesso
valore medio e differente ampiezza.%
\index{compatibilit\`a!tra dati sperimentali|)}%
\index{compatibilit\`a!con una distribuzione|)}%
\index{Kolmogorov e Smirnov, test di|)}

\endinput
