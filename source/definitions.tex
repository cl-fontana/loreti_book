% $Id: definitions.tex,v 1.2 2005/04/27 10:08:51 loreti Exp $

\hyphenation{}
\setlength{\unitlength}{1mm}

\newcommand{\copyleft}{\reflectbox{\textcopyright}}

% \ldots (with the other *dots macros) is redefined from amsmath.sty
% in a way I don't like; here follows the old definition, from
% latex.ltx

\DeclareRobustCommand{\ldots}{\ifmmode\mathellipsis\else\textellipsis\fi}

% "Right length and right position" overlined symbols: works for
% symbols used in subscripts, superscripts or sizes different from
% textstyle.
%
% - A first command \ob[size]{symbol} is intended for use in textstyle
%   and displaystyle sizes only, both in math and text mode: the first
%   parameter (the height of the overbar line) is optional and
%   defaults to 0.48 pt; the second parameter is the (math) symbol to
%   be overlined.
% - A second command \overbar{symbol} is intended for use in math mode
%   only; and scales appropriately for every text size and in sub- and
%   superscripts.

\newcommand{\ob}[2][0.48pt]{\setbox0=\hbox{\ensuremath{#2}}%
  \dimen0=\ht0\advance\dimen0 by0.2ex%
  \dimen3=0.17\ht0%
  \dimen1=\wd0\advance\dimen1 by-\dimen3\advance\dimen1 by-0.02em%
  \dimen2=\dimen1\advance\dimen2 by-0.03em%
  \box0%
  \kern-\dimen1\hbox{\rule[\dimen0]{\dimen2}{#1}}%
}

\newcommand{\overbar}[1]{{\mathchoice
  {\displaystyle\ob{\displaystyle#1}}%
  {\textstyle\ob{\textstyle#1}}%
  {\scriptstyle\ob[0.36pt]{\scriptstyle#1}}%
  {\scriptscriptstyle\ob[0.24pt]{\scriptscriptstyle#1}}}}

% Definition of a command for "emphasis in the index"

%\newcommand{\emidx}[1]{\textbf{\textit{#1}}}
\newcommand{\emidx}[1]{\textbf{#1}}

% Definition of a command that takes care of the "spacing for
% punctuation at the end of equations"

\newcommand{\peq}{\;\;}

% Miscellaneous commands

\DeclareMathOperator{\var}{Var}
\DeclareMathOperator{\cov}{Cov}
\DeclareMathOperator{\cor}{Corr}
\newcommand{\updot}{{\raisebox{1.2ex}{.}}}
\newcommand{\gra}{\ensuremath{^\circ}}
\newcommand{\de}{\mathrm{d}}

% To insert space at top/bottom of tabular lines (written by Claudio
% Beccari, beccari@polito.it)

\newcommand{\tabtop}{\rule{0pt}{2.6ex}}
\newcommand{\tabbot}{\rule[-1.2ex]{0pt}{0pt}}

% (Code written by Donald Arseneau, asnd@reg.triumf.ca)
% \un Put units after a number; e.g., 0.0821\un{l\,atm\,mol^{-1}\,K^{-1}}
%     Use \, or space to put thin spaces between units (but spaces only
%     work if they are not buried in inner braces).  Works in math and
%     text mode. \un is fragile.

\makeatletter
\newmuskip\unitmuskip \unitmuskip=3mu plus 1mu minus 1mu
\def\unitmskip{\penalty\@highpenalty \mskip\unitmuskip}
\newcommand\un[1]{\leavevmode\unskip\penalty\@highpenalty
  \ensuremath{\mskip\medmuskip
  \begingroup \fam\z@ % limit \fam but not style changes
  \let\,\unitmskip \unit@PreserveSpaces\@empty #1 \unit@PreserveSpaces
  \endgroup}}
\def\unit@PreserveSpaces#1 {#1\@ifnextchar\unit@PreserveSpaces{\@gobble}%
  {\unitmskip \unit@PreserveSpaces\@empty}}
\makeatother

% Two commands, "daytime" and "thismonth", used for the preliminary
% copies (in "preambolo.tex")

\newcount\hour \newcount\minute
\hour=\time \divide \hour by 60
\minute=\time
\count99=\hour \multiply \count99 by -60
\advance \minute by \count99
\newcommand{\daytime}{%
  \ifnum\hour=0 00\else\ifnum\hour<10 0\fi\number\hour\fi:%
  \ifnum\minute<10 0\fi\number\minute%
}

\newcommand{\thismonth}{%
  \ifcase\month\or Gennaio\or Febbraio\or Marzo\or Aprile\or
  Maggio\or Giugno\or Luglio\or Agosto\or Settembre\or Ottobre\or
  Novembre\or Dicembre\fi}

\endinput
