% $Id: chaptera.tex,v 1.1 2005/03/01 10:06:08 loreti Exp $

\chapter{Cenni di calcolo combinatorio}%
\index{calcolo combinatorio|(}
Il \emph{calcolo combinatorio} \`e una branca della
matematica orientata alla discussione ed allo sviluppo di
formule che permettano di ottenere il numero di casi
distinti che si possono presentare in un esperimento, od il
numero di elementi che compongono un insieme, senza
ricorrere alla loro enumerazione esplicita.

Il calcolo combinatorio trova importanti applicazioni nella
teoria della probabilit\`a e nella statistica: alcune
formule, specificatamente quelle per le permutazioni e le
combinazioni, vengono usate nel corso del testo; qui se ne
d\`a una breve giustificazione.

\section{Il lemma fondamentale del calcolo combinatorio}%
\index{calcolo combinatorio!lemma fondamentale|(}
\begin{quote}
  \textsc{Lemma fondamentale del calcolo combinatorio}:
  \textit{dati due insiemi $I_1$ ed $I_2$, composti da $N_1$
    ed $N_2$ elementi distinti rispettivamente, l'insieme $I
    = I_1 \otimes I_2$ di tutte le coppie ordinate che si
    possono costruire associando un elemento di $I_1$ con un
    elemento di $I_2$ \`e composto da $N_1 \cdot N_2$
    elementi.}
\end{quote}

Questo lemma si pu\`o immediatamente generalizzare (per
induzione completa) a $K$ insiemi $I_1,\ldots, I_K$ composti
da $N_1,\ldots, N_K$ elementi distinti rispettivamente:
l'insieme $I = I_1 \otimes I_2 \otimes \cdots \otimes I_K$,
costituito da tutte le possibili associazioni ordinate di
$K$ elementi ognuno dei quali provenga da un differente
insieme $I_j$, con $j=1,\ldots, K$, \`e composto da $N_1
\cdot N_2 \cdots N_K$ elementi.%
\index{calcolo combinatorio!lemma fondamentale|)}

\section{Fattoriale di un numero intero}
Si definisce come \emph{fattoriale} di un numero intero
positivo $N$, e si indica con il simbolo $N!$, il prodotto
dei primi $N$ numeri interi:
\begin{equation*}
  N! = 1 \cdot 2 \cdot 3 \cdots N \peq ;
\end{equation*}
per motivi che appariranno chiari pi\`u avanti\footnote{La
  ``definizione'' $0!  = 1$ non \`e cos\`\i\ arbitraria come
  pu\`o sembrare: in realt\`a si comincia definendo una
  certa funzione di variabile complessa $\Gamma(z)$ che,
  quando l'argomento $z$ \`e un numero intero positivo,
  coincide con il suo fattoriale; e per la quale si vede che
  $\Gamma(0) = 1$.}, si definisce poi il fattoriale di zero
come $0!  = 1$.

\section{Disposizioni}%
\index{disposizioni|(}
Se $N$ e $K$ sono due numeri interi positivi tali che sia $K
\le N$, si definisce come numero delle disposizioni di $N$
oggetti di classe $K$ (che si indica con il simbolo $D^N_K$)
il numero dei gruppi \emph{distinti} di $K$ oggetti che \`e
possibile formare a partire dagli $N$ originali; definendo
come \emph{distinti} due gruppi se essi differiscono o per
qualche elemento o per l'ordine.

Come esempio, le disposizioni di classe 2 che si possono
formare con le 21 lettere dell'alfabeto italiano sono le
seguenti:
\begin{equation*}
  \left\{ \begin{array}{lllllll}
       & AB & AC & AD & \cdots & AV & AZ \\
    BA &    & BC & BD & \cdots & BV & BZ \\
       &    &    &    & \cdots &    &    \\
    ZA & ZB & ZC & ZD & \cdots & ZV &    \\
  \end{array} \right.
\end{equation*}

Il valore di $D^N_K$ si pu\`o facilmente trovare sfruttando
il lemma fondamentale del calcolo combinatorio: il primo
elemento di una disposizione si pu\`o infatti scegliere in
$N$ modi distinti, il secondo in $N-1$, e cos\`\i\ via.  Di
conseguenza $D^N_K$ \`e il prodotto di $K$ numeri interi
decrescenti a partire da $N$:
\begin{equation} \label{eq:a.dnk}
  D^N_K \; = \; N \cdot (N-1) \cdot (N-2) \cdots
    (N-K+1) \; = \; \frac{N!}{(N-K)!}
\end{equation}
(nel caso dell'esempio fatto, le disposizioni sono $D^{21}_2
= 21 \cdot 20 = 420$; nella tabella in cui sono state
elencate vi sono 21 righe di 20 elementi ciascuna).

L'espressione \eqref{eq:a.dnk} \`e verificata anche se
$K=N$, per\`o purch\'e (come prima detto) si ponga $0!=1$.%
\index{disposizioni|)}

\section{Permutazioni}%
\index{permutazioni|(}
Se $N$ \`e un numero intero positivo, si definisce come
numero delle permutazioni di $N$ oggetti, e si indica con
$P_N$, il numero di maniere distinte in cui si possono
ordinare gli $N$ oggetti stessi.  Evidentemente risulta
\begin{equation*}
  P_N \; \equiv \; D^N_N \; = \; N! \peq .
\end{equation*}%
\index{permutazioni|)}

\section{Permutazioni con ripetizione}%
\index{permutazioni!con ripetizione|(}
Se gli $N$ oggetti che si hanno a disposizione sono tali da
poter essere divisi in $M$ gruppi (composti da $N_1,
N_2,\ldots,N_M$ oggetti rispettivamente; ovviamente $N_1 +
N_2 + \cdots+N_M = N$), tali che gli oggetti in ognuno di
questi gruppi siano \emph{indistinguibili} tra loro, il
numero di permutazioni che con essi si possono realizzare
\`e inferiore a $P_N$; pi\`u precisamente, visto che gli
oggetti di ogni gruppo si possono scambiare tra loro in
qualsiasi modo senza per questo dare luogo a una sequenza
distinta, il numero di \emph{permutazioni con ripetizione}
\`e dato da
\begin{equation} \label{eq:a.perip}
  \frac{N!}{N_1! \cdot N_2! \cdots N_M!} \peq .
\end{equation}%
\index{permutazioni!con ripetizione|)}

\section{Combinazioni}%
\index{combinazioni|(}%
\label{ch:a.combina}
Se $N$ e $K$ sono due numeri interi positivi tali che sia $K
\le N$, si definisce come numero delle combinazioni di
classe $K$ di $N$ oggetti il numero dei sottoinsiemi
\emph{distinti} composti da $K$ oggetti che \`e possibile
formare a partire dagli $N$ originali; definendo come
\emph{distinti} due sottoinsiemi se essi differiscono per
qualche elemento.  Il numero delle combinazioni di classe
$K$ di $N$ oggetti si indica con uno dei due simboli
\begin{equation*}
  C^N_K \makebox[20mm]{o} \binom{N}{K}
\end{equation*}%
\index{coefficienti binomiali}
(l'ultimo dei quali si chiama \emph{coefficiente
  binomiale}).

Consideriamo l'insieme composto da tutte le disposizioni di
classe $K$ di $N$ oggetti, e pensiamo di raggruppare i suoi
elementi in sottoinsiemi in modo che ciascuno di essi
contenga tutte e sole quelle disposizioni che differiscano
esclusivamente per l'ordine ma siano composte dagli stessi
oggetti; ovviamente il numero di questi sottoinsiemi \`e
$C^N_K$: ed ognuno di essi contiene un numero di elementi
che \`e $P_K$.

Da qui ricaviamo
\begin{equation} \label{eq:a.binom}
  C^N_K \; \equiv \; \binom{N}{K} \; = \;
    \frac{D^N_K}{P_K} \; = \;
    \frac{N \cdot (N-1) \cdots (N-K+1)}{K \cdot (K-1)
    \cdots 1} \; = \; \frac{N!}{K! \, (N-K)!}
\end{equation}
O, in altre parole, il numero di combinazioni di classe $K$
di $N$ oggetti \`e uguale al rapporto tra il prodotto di $K$
numeri interi decrescenti a partire da $N$ ed il prodotto di
$K$ numeri interi crescenti a partire dall'unit\`a.

\index{coefficienti binomiali|(}%
Si dimostrano poi facilmente, a partire dalla definizione,
due importanti propriet\`a dei coefficienti binomiali:
\begin{gather*}
  \binom{N}{K} = \binom{N}{N-K} \\
  \intertext{e}
  \binom{N+1}{K} = \binom{N}{K-1} + \binom{N}{K} \peq .
\end{gather*}

\`E da osservare che, cos\`\i\ come sono stati ricavati
(dalla definizione delle possibili combinazioni di $N$
oggetti), i coefficienti binomiali hanno senso solo se $N$ e
$K$ sono numeri interi; ed inoltre se risulta sia $N > 0$
che $0 \le K \le N$.  La definizione \eqref{eq:a.binom}
pu\`o comunque essere estesa a valori interi qualunque, ed
anche a valori reali di $N$ --- ma questo esula dal nostro
interesse.%
\index{coefficienti binomiali|)}%
\index{combinazioni|)}

\section{Partizioni ordinate}%
\index{partizioni ordinate|(emidx}%
\label{ch:a.parord}
Consideriamo un insieme di $N$ oggetti; vogliamo calcolare
il numero di maniere in cui essi possono essere divisi in
$M$ gruppi che siano composti da $N_1, N_2,\ldots,N_M$
oggetti rispettivamente (essendo $N_1 + N_2 + \cdots + N_M =
N$).

Gli $N_1$ oggetti che compongono il primo gruppo possono
essere scelti in $C^N_{N_1}$ modi differenti; quelli del
secondo gruppo in $C^{N-N_1}_{N_2}$ modi; e cos\`\i\ via.
Per il lemma fondamentale del calcolo combinatorio, il
numero delle \emph{partizioni ordinate} deve essere uguale a
\begin{multline*}
  \binom{N}{N_1} \binom{N-N_1}{N_2}
    \binom{N-N_1-N_2}{N_3} \cdots
    \binom{N-N_1-\cdots-N_{M-1}}{N_M} = \\[1.5ex]
  = \frac{N!}{N_1! \, (N-N_1)!} \cdot
    \frac{(N-N_1)!}{N_2! \, (N-N_1-N_2)!} \cdots
    \frac{(N-N_1-\cdots-N_{M-1})!}{N_M! \,
    (N-N_1-\cdots-N_M)!} = \\[1.5ex]
  = \frac{N!}{N_1! \, N_2! \cdots N_M!}
\end{multline*}
(sfruttando il fatto che tutti i numeratori dei termini dal
secondo in poi si semplificano con uno dei fattori del
denominatore del termine precedente; inoltre, nell'ultimo
termine, $N-N_1-\cdots-N_M \equiv 0$).  Si pu\`o notare che
l'ultimo termine della prima espressione, essendo
$N-N_1-\cdots-N_{M-1} = N_M$, vale sempre uno; cosa non
sorprendente visto che, quando i primi $M-1$ gruppi sono
stati scelti, anche l'ultimo risulta univocamente
determinato.

Insomma il numero delle partizioni ordinate \`e uguale al
numero delle permutazioni con ripetizione di $N$ oggetti
raggruppabili in $M$ insiemi, composti rispettivamente da
$N_1, N_2,\ldots,N_M$ oggetti indistinguibili tra loro, dato
dalla formula \eqref{eq:a.perip}%
\index{partizioni ordinate|)}%
\index{calcolo combinatorio|)}

\endinput
